

\documentclass[a4paper,11pt]{article}

% Pakete für Zeichencodierung und Spracheinstellungen
\usepackage[utf8]{inputenc}    % UTF-8 Zeichenkodierung
\usepackage[T1]{fontenc}       % Korrekte Darstellung von Umlauten

\usepackage[ngerman]{babel}    % Deutsche Spracheinstellungen
\usepackage{amsmath}
\usepackage{amssymb}
% Pakete für Layout und Formatierung
\usepackage{graphicx}          % Einbindung von Bildern
\usepackage{svg}               % Einbinden von .svg
\usepackage{geometry}          % Anpassung der Seitenränder
\usepackage{float}             % Exakte Positionierung von Abbildungen
\usepackage{fancyhdr}          % Anpassung von Kopf- und Fußzeilen
\usepackage{titlesec}          % Anpassung der Überschriftenformatierung
\usepackage{tocloft}           % Anpassung des Inhaltsverzeichnisses
\usepackage{hyperref}
\usepackage{gensymb}
\usepackage{tabularx}

\usepackage{listings}
\usepackage{xcolor}

\lstset{
  language=Python,
  basicstyle=\ttfamily\small,
  keywordstyle=\color{blue},
  commentstyle=\color{gray},
  stringstyle=\color{red},
  showstringspaces=false,
  breaklines=true
}

% Paket für das Literaturverzeichnis
\usepackage{csquotes}
\usepackage[backend=biber, style=ieee]{biblatex} % IEEE-Zitierstil mit Biber als Backend
\addbibresource{Literaturverzeichnis.bib}       % Einbindung der Literaturdatenbank

% Paket zur Nutzung der letzten Seitenzahl
\usepackage{lastpage} % Ermöglicht „Seite X von Y“-Format

% Seitenlayout
\geometry{
  top=2cm,    % Abstand zum oberen Rand
  bottom=3cm, % Abstand zum unteren Rand
  left=2.5cm, % Abstand zum linken Rand
  right=2.5cm % Abstand zum rechten Rand
}

% Kein Absatzeinzug
\setlength\parindent{0pt}

% Kopf- und Fußzeilen
\setlength\headheight{26pt}  % Höhe der Kopfzeile
\setlength\headsep{35pt}     % Abstand zwischen Kopfzeile und Text
\pagestyle{fancy}
\fancyhf{}
\lhead{Daniel Albinger (5183249)}
\chead{Bachelor Thesis}
\rhead{\includegraphics[width=4cm]{Logo_HSB_Hochschule_Bremen.png}}
\cfoot{} % Keine Seitenzahl auf Titelseite und Inhaltsverzeichnis

\begin{document}

% Titelseite
\begin{titlepage}
    \centering
    \includegraphics[width=0.6\textwidth]{Logo_HSB_Hochschule_Bremen.png}\\[1cm]
    {\scshape\LARGE Hochschule Bremen\\}
    {\scshape\Large Fakultät 4: Elektrotechnik und Informatik\\[1.5cm]}
    {\huge\bfseries Bachelor Thesis\\[0.5cm]}
    {\vspace{20mm}}
    {\Large\bfseries Systems zum Empfang von Geostationären Satelliten durch die IAT Bodenstation\\[0.5cm]}
    {\Large\bfseries Daniel Albinger (5183249)\\[2cm]}
    {\Large\bfseries Prüfer\\[0.5cm]}
    \begin{tabular}{l}
    1. Prüfer: Prof.\ Dr.\ Peik \\ 
    2. Prüfer: Prof.\ Dr.\ García \\
    \end{tabular}\\[2cm]
    {\Large\bfseries Abgabe: 03. Februar 2026}\\[2cm]
    \vfill
\end{titlepage}


\newpage
\section*{Abstrakt}
Das Ziel der Arbeit ist die Planung, Entwicklung und Umsetzung eines Empfangssystem für geostationäre Satelliten im X-Band. Als Satellite bietet sich dabei der Amateurfunksatellit Es'Hail-2 (QO-100) an. Dieser besitzt einen Schmal- und Breitbandtransponder, welche beide ihren Downlink im X-Band haben.\newline
Zur Planung und Entwicklung des Empfangssystem gehören neben der Untersuchung des Satelliten und seiner Empfangsparameter auch die Überprüfung einer an der Hochschule bereits vorhandenen Satellitenschüssel und der Auswahl eines geeigneten Antennenfeeds für den Empfang des Downlink vom Schmalbandtransponders auf Es'Hail-2 (QO-100). Ebenso muss ein geeigneter Abwärtsmischer, um die Signale in einen für Software-Defined-Radios üblichen Frequenzbereich zu bringen, sowie ein rauscharmer Verstärker zur Verstärkung der schwachen empfangenen Signale ausgewählt werden. Ebenfalls müssen geeigneten Koaxialleitungen für die verschiedenen Bereiche des Empfangssystems ausgewählt und eine SDR-Software zur Dekodierung der empfangenen Signale entwickelt werden.\newline
Im Zuge der Planung und Entwicklung des Empfangssystem wird eine theoretische Betrachtung des Downlink vorgenommen. Besonderer Fokus wird dabei auf die Übertragungsstrecke zwischen dem Satelliten und der Empfangsstation am IAT (Institut für Aerospace Technologie) gesetzt. Dabei werden verschiedene auftretenden Dämpfungen ermittelt und Einflüsse durch verschiedene Wetterbedingungen auf die Qualität und das Link Budget des Downlink untersucht. Erwähnenswert sind dabei die Dämpfung in der Atmosphäre bei einem klaren Himmel $L_\mathrm{ATklarerHimmel}=0.547\,\text{dB}$, bei leichten Niederschlägen $L_\mathrm{ATleichterRegen}=0.947\,\text{dB}$ und bei starken Niederschlägen $L_\mathrm{ATRegen}=9.61\,\text{dB}$.\newline
Im Zuge der theoretischen Betrachtung wird auch das Rauschen des Empfangssystems untersucht. Dabei wird eine äquivalente Rauschtemperatur des Empfangssystems $T_\mathrm{e,sys}=336.63\,\text{K}$ und eine Rauschzahl von $F_\mathrm{sys}=3.34\,\text{dB}$ ermittelt. Auch wird die Rauschleistung im und der Signal-zu-Rauschabstand am Ein- und Ausgang des Empfangssystems bei den unterschiedlichen Wetterbedingungen klarer Himmel, leichter Regen und Regen betrachtet. Mit dem Signal-zu-Rauschabstand kann eine Aussage drüber getroffen werden, ob der Downlink bei den unterschiedlichen Wetterbedienungen aufrechterhalten werden kann. Bei einem klaren Himmel und leichten Regen kann der Downlink bei einer reduzierten Bandbreite von $B=25\,\text{kHz}$ aufrecht erhalten werden. Bei starken Niederschlägen muss die Bandbreite auf $B=2.7\,\text{kHz}$ reduziert werden, um den Downlink aufrecht erhalten zu können.\newline
Zum Schluss werden noch die mit dem Empfangssystem aufgenommen Werte mit der Goonhilly Bodenstation in Cornwall, England verglichen.



\newpage
\section*{Eidesstattliche Erklärung}
Ich versichere, die vorliegende Arbeit selbstständig und nur unter Benutzung der angegebenen Hilfsmittel angefertigt zu haben.
\vspace{25mm}  % Abstand oben

\begin{minipage}[t]{10cm}
\flushleft
Bremen, \today \\
\centering
\dotfill \\  % Gepunktete Linie
Daniel Albinger (5183249)
\end{minipage}
\newpage
\section*{Danksagung}
\newpage 
\section*{Aufgabenstellung}


% Inhaltsverzeichnis ohne Seitenzahl
\pagenumbering{gobble}  % Keine Seitenzahl auf Inhaltsverzeichnis
\tableofcontents

\newpage  

% Seitenzahlen beginnen hier im Format „Seite X von Y“





\pagenumbering{arabic}  
\setcounter{page}{1}  
\cfoot{Seite \thepage\ von \pageref{LastPage}}
\section{Einleitung}
Im heutigen Zeitalter sind Satelliten nicht mehr aus unseren Leben wegzudenken. Sie werden für die verschiedenste Dinge benötigt, wie Navigation zur Arbeit oder nächsten Urlaubsziel, für Rundfunk und Fernsehn, für Internet und für vieles mehr. Damit aus den Satelliten auch ein Nutzen gezogen werden kann, muss eine Kommunikationsmöglichkeit mit diesen bestehen. Hier für werden sogenannte Bodenstationen eingesetzt. Bodenstation werden zur Steuerung und Verfolgen der Satelliten, sowie zum Senden und Empfangen von Daten und Informationen von Satelliten eingesetzt.\newline
Zurzeit entsteht am Institut für Aerospace Technologie (IAT) eine Satellitenbodenstation. Für die ersten Versuche und Test eignen sich der Empfang von Satelliten in einer geostationären Umlaufbahn. Eine akktraktive Möglichkeit bietet dabei der Kommunikations- und Amateurfunksatellit Es'Hail-2, auch bekannt unter den Rufname Qatar Oscar 100 (QO-100). Neben dem Equipment zur kommerziellen Nutzung des Satelliten, befinden sich ebenfalls zwei Amateurfunktransponder an Bord von Es'Hail-2 (QO-100). Bei den beiden Transponder handelt es sich um einen Schmal- und Breitbandtransponder, welche von Amateurfunkern aus verschiedenen Länder für verschiedene Zwecke frei verwendet werden können.
\subsection{Motivation}
Schon seit längerer Zeit interessiere ich mich privat für Kommunikations- und Hochfrequenztechnik, sowie für Satelliten und Raumfahrt im Allgemeinen. In den letzten zwei Jahren hat sich dieses Interesse zunehmend auf den Bereich der Satellitenkommunikation konzentriert. Erste eigene Versuche, Signale und Bilder der NOAA-Wettersatelliten zu empfangen, haben mein technisches Interesse weiter vertieft.\newline
Daher war es mir ein besonderes Anliegen, für meine Bachelorarbeit ein Thema im Bereich der Kommunikations- und Hochfrequenztechnik beziehungsweise der Satellitenkommunikation zu wählen. Über Prof. Dr. Peik und Prof. Dr. García ergab sich schließlich die Möglichkeit, im Rahmen der entstehenden Bodenstation des IAT ein Empfangssystem für geostationäre Satelliten im X‑Band zu planen, zu entwickeln und aufzubauen.

\subsection{Zielsetzung und Vorgehen}
Diese Arbeit beschäftigt sich mit der Planung, Entwicklung und dem Aufbau einer Empfangsstation für den Downlink des Schmalbandtransponder auf Es'Hail-2 (QO-100), der sich im X-Band befindet. Ziel ist der Aufbau und die Inbetriebnahme eines funktionsfähigen Empfangssystems, sowie das erfolgreiche Empfangen und Dekodieren von Amateurfunksignalen.\newline
Zur Erreichung des Ziels werden im ersten Schritt der Schmalbandtransponder und der 
Satellit Es'Hail-2 (QO-100) genauer untersucht. Dabei werden die technischen Eigenschaften des Satelliten und des Downlink, sowie relevante Empfangsparameter recherchiert.\newline
Im zweiten Schritt wird der Downlink theoretisch betrachtet. Besonderer Fokus liegt dabei auf der Übertragungsstrecke zwischen dem Satelliten und der Bodenstation am IAT. Untersucht werden die verschiedenen Dämpfungen, welche in den unterschiedlichen Abschnitten der Übertragungsstrecke auftreten. Von besonderem Interesse sind dabei die Dämpfungen in der Atmosphäre bei verschiedenen Wetterbedingungen und deren Auswirkungen auf den Downlink. Auf Basis dieser Informationen kann dann anschließend ein geeignetes Empfangssystem geplant und entwickelt werden.\newline 
Hierzu wird zunächst eine an der Hochschule bereits vorhandene Satellitenschüssel auf ihre Eignung für den Empfang des Downlinks überprüft. Weiterhin muss ein geeigneter Antennenfeed zum empfangen des Downlink, ein Verstärker zur Anhebung der empfangenen Signale auf ein verarbeitbaren Pegel, ein Abwärtsmischer zur Umsetzung der empfangenen Signale vom X-Band in einen niedrigeren Frequenzbereich, sowie geeignete Koaxialleitungen für die unterschiedlichen Abschnitte des Empfangssystems gewählt werden. Die ausgewählten Komponenten müssen dabei mit dem bereits vorhandenen Equipment - einer RF-Schaltmatrix von Mini-Circuits, einem Patchfeld und einem USRP X310 Software Defined Radio - kompatibel sein.\newline
Im nächsten Schritt werden die Eigenschaften des geplanten Empfangssystems ermittelt. Dazu gehören die im RF-Bereich des Empfangssystems auftretenden Dämpfungen und Verluste, die erzielten Verstärkungen, sowie das Eigenrauschen der Komponenten und damit des gesamten Empfangssystems. Mit diesen Werten kann die Empfangsgüte und die theoretischen Empfangsparameter, wie empfangene Leistung, empfangenes Rauschen, Rauschleistung im Empfangssystem, Leistung am Ausgang, das Link Budget und die Qualität des Downlinks für unterschiedliche Wetterbedingungen ermittelt und miteinander verglichen werden. Auf dieser Grundlage kann eine Aussage über die Leistungsfähigkeit des entwickelten Empfangssystems getroffen werden.\newline
Abschließend wird der Aufbau des geplanten und entwickelten Empfangssystems dokumentiert und eine Software zur Steuerung des Software Defined Radio, sowie zur Dekodierung der unterschiedlichen Signale im Downlink mithilfe von GNU Radio entwickelt. Zu dem werden die in der Praxis erreichten mit den in der Theorie bestimmten Empfangsparameter verglichen, sowie das Link Budget des Empfangssystem am IAT mit dem Link Budget des Empfangssystem an der Goonhilly in Cornwall (England) Bodenstation verglichen.





\section{Geschichte von Satelliten}
\subsection{Was ist ein Satellit}
Bei einem Satelliten handelt es sich im allgemeinen Verständnis um ein Objekt, welches sich in einer Umlaufbahn um einen Himmelskörper, wie z.B ein Planet, Mond, Stern oder ähnliches befindet. Dabei kann der Satellit natürlichem oder künstlichen Ursprung sein.\cite{Satellite_Technology}\newline
Im weiteren Verlauf handelt es sich bei einem Satelliten um ein künstliches Objekt, welches sich in einer Umlaufbahn um die Erde befindet.\newline
Die erste Idee für einen Satelliten in einer geostationären Umlaufbahn stammt aus dem Jahren 1945. In diesem Jahr veröffentlichte der Autor Arthur C. Clarke im Magazin Wireless World einen Artikel, in welchem er die Bedeutung des geostationären Orbits beschreibt und die Idee eines Kommunikationssatelliten im geostationären Orbit vorstellt. Mit dem richtigen Equipment könnte solch ein Satellit interkontinentalen Datenaustausch ermöglichen.\cite{Satellite_Technology}\newline
\begin{figure}[H]
    \centering
    \includegraphics[width=0.5\linewidth]{Bilder/Sputnik 1.jpeg}
    \caption{Das Bild zeigt den ersten Satelliten Sputnik 1 vor dem Start. Sputnik 1 ist eine Aluminiumkugel mit einem Durchmesser $0.58 \text{ m}$ und einem Gewicht von $58 \text{ kg}$.\cite{Sputnik_1}\cite{Bild_Sputnik}}
    \label{Sputnik 1}
\end{figure}
Der erste Satellit starte am 04. Oktober 1957 von der damaligen UdSSR. Der Satellit mit Namen Sputnik 1, was so viel wie Begleiter oder Trabant bedeutet, umkreiste die Erde alle 98 min. Ausgerüstet war Sputnik 1 mit zwei Antennenpaaren und Telekommunikationsequipment, mit welchem er über $20.005$ MHz und $40.002$ MHz kurze Signale aussendete. Diese Signale konnten auf der ganzen Welt empfangen werden. Nach etwa 92 Tage verglühte Sputnik 1 beim Wiedereintritt in die Atmosphäre. \cite{Sputnik_1}\newline
Der erste amerikanische Satellit starte am 31. Januar 1958 mit dem Namen Explorer 1. Explorer 1 ist der erste Satellit gewesen, welcher wissenschaftliches Equipment ab Bord hatte.\cite{NASA_Explorer1}
\begin{figure}[H]
    \centering
    \includegraphics[width=0.5\linewidth]{Bilder/1958_january_explorer_01_team_0.jpg}
    \caption{Vorführung des ersten amerikanischen Satelliten Explorer 1\cite{NASA_Explorer1} }
    \label{Explorer 1}
\end{figure}
An Bord von Explorer 1 befanden sich wissenschaftliches Equipment, unter anderem auch ein Messgerät für kosmische Strahlung. Mit diesem Messgerät sollte die Strahlung in der Atmosphäre der Erde gemessen werden. Explorer 1 erbrachte den Nachweis für des Van-Allen Strahlungsgürtels. Der Satellit umrundete die Erde alle 114 min in einer kreisförmigen Umlaufbahn, wobei diese den Satelliten bis auf 354 km an die Erde ran und 2515 km entfernt brachte. Explorer 1 war 203 cm lang, hatte einen Durchmesser von 15.9 cm und wog 14 kg. Am 23. Mai 1958 machte die Explorer 1 ihre letzte Übertragung und verglühte am 31. März 1970 beim Wiedereintritt in der Atmosphäre.\cite{NASA_Explorer1}\newline
In der heutigen Zeit gibt es viele verschiedene Arten an Satelliten, welche sich in ihrem Verwendungszweck und ihrem damit verbunden Equipment und Umlaufbahn unterscheiden.\newline
Ein paar Beispiele wären dabei:
\begin{itemize}
    \item Erdbeobachtungssatelliten: Diese werden zur Beobachtung und Analyse der Erdoberfläche und Atmosphäre eingesetzt. Zu dieser Gruppe an Satelliten gehören unter anderem Wettersatelliten. Ausgerüstet mit verschiedenen Kameras und Messequipment nehmen sie Bilder von Wolkenformationen und Daten der Atmosphäre auf. Diese Daten und Bilder bilden dann die Grundlage für die Wettervorhersage. Einige Beispiele für Wettersatelliten sind unter andrem die NOAA und GOES Reihe der Amerikaner, die METOP und METOSAT Reihe der Europäer und die METEOR und Electro-L Reihe der Russen.\cite{DWD}
    \item Kommunikations- und Rundfunksatelliten: Dieser Art der Satelliten stellen verschiedenste Service im Bereich der Telekommunikation und Rundfunk bereit. Sie werden unter anderem zur Übertragung von Fernsehsignalen, Telefonie und Internet verwendet. Sie sind in den unterschiedlichsten Umlaufbahnen anzutreffen. Einige Beispiele wäre dabei die Starlink Reihe von SpaceX, Iridium von Iridium und Inmarsat von Inmarsat oder Astra von SES S.A. ASTRA.\cite{Satellite_Technology}\cite{Satellitenkommunikation}\cite{N2YO_SATLIST}
    
    \item Navigationssatelliten: Diese werden zur genauen Positionsbestimmung verwendet. Dafür bilden diese ganze Satellitenkonstellationen, welche die gesamte Erde umspannen können. Die Positionsbestimmung basiert dabei auf der der Triangulierung und Einwegentfernungsmessung. Zur Bestimmung werden Signale von drei oder mehreren Satelliten empfangen. Die Signale enthalten neben den genauen Koordinaten des Satelliten auch den genauen Zeitpunkt, an welchem die Signale versendet werden. Grundlage für den genauen Zeitbestimmung bilden Atomuhren, welche sich auf den Satelliten befinden. Beispiele für solche Sternenkonstellationen sind das GPS der Amerikaner, das russische GLONASS und das europäische Galileo.\cite{Satellitenkommunikation}
    
    \item Amateurfunksatelliten: Amateurfunksatelliten bilden eine besondere Untergruppe der Kommunikationssatelliten. Sie werden meistens von Universitäten, Vereinigungen von Amateurfunkern oder ähnlichen Vereinen geplant, entwickelt, gebaut und betrieben. Dabei sind die engen Budgets und technologischen Innovationen bewundernswert.\cite{Satellitenkommunikation}\newline
    Eine solche Vereinigung ist AMSAT, welche mehrere Ableger weltweit hat. In Deutschland gibt es die AMSAT-DL, welche sich aus Funkamateuren, Ingenieuren, Wissenschaftlern, Studenten und Raumfahrtenthusiasten zusammensetzt. Seit über 50 Jahren plant, entwickelt, baut und betreibt die AMSAT-DL verschiedenste Satelliten, welche von Funkamateuren frei verwendet werden dürfen.\cite{AMSAT-DL}\newline
    Der erste Amateurfunksatellit OSCAR-I (Orbital Satellite Carrying Amateur Radio) starte am 12. Dezember 1961, vier Jahre vor dem ersten kommerziellen Kommunikationssatelliten "Early Bird". Die ersten OSCAR-I,-II und -III Satelliten funktionierten nur wenige Tage. Erst OSCAR-VI von der deutschen AMSAT (AMSAT-DL) schaffte es 4,5 Jahren lang zu arbeiten. Es folgten  weitere deutsche OSCARS, weltweit insgesamt mehr als 50 Stück.\cite{Satellitenkommunikation}\newline
    Weitere Meilensteine von AMSAT-DL sind die sogenannten Phase 3 Satelliten. Die Entwicklung dieser Satelliten begann in den 1970er Jahren. Das Ziel der Phase 3 Satelliten ist eine Generation von Erdsatelliten in einer hoch elliptischen Umlaufbahn zu erschaffen. Gegenüber der bisherigen Satelliten würden diese einen weltweiten Benutzerkreis erschließen. Von den bisher vier gestarteten Phase 3 Satelliten, mit der Bezeichnung P3-A bis P3-D, sind nur noch P3-B und P3-D im Orbit. \cite{AMSAT-DL}\cite{Satellitenkommunikation}\newline
    Ein weiterer Meilenstein ist der erste Phase 4 Satellit. Bei dem Satelliten handelt es sich um den katarischen Rundfunk- und Kommunikationssatelliten Es'Hail-2. Dieser trägt den Rufname QO-100 und hat neben dem Equipment zur kommerziellen Nutzung auch zwei Amateurfunktransponder an Bord hat, welche die ersten im geostationären Orbit sind.\cite{AMSAT-DL}\newline
\end{itemize}



\section{Theorie und Grundlagen}
\subsection{Umlaufbahnen für Satelliten}
\begin{figure}[H]
    \centering
    \includegraphics[width=0.5\linewidth]{Bilder/Umlaufbahnen.png}
    \caption{Die verschiedenen Umlaufbahnen von Satelliten im Überblick \cite{Satellitenkommunikation}}
    \label{fig:Umlaufbahnen}
\end{figure}
Die Abbildung \ref{fig:Umlaufbahnen} zeigt die verschiedenen Umlaufbahnen, welche von verschiedensten Satelliten verwendet werden. Die Umlaufbahnen unterscheiden sich dabei in Form (Kreis oder Ellipse), Höhe und Inklination zum Äquator. \cite{Satellitenkommunikation}\newline
In welcher Umlaufbahn ein Satellit eingesetzt wird, hängt von seiner Aufgabe und geplanten Lebensdauer ab.\cite{Satellitenkommunikation}\newline
Wettersatelliten zum Beispiel werden unter anderem möglichst nah an die Erdoberfläche platziert, um so den Detailgrad der Bilder zu erhöhen. Allerdings können Sie auch in weit entfernteren Umlaufbahnen angesiedelt werden. So kann mit geringer Anzahl an Satelliten ein Großteil der Erde abgedeckt werden.
Mögliche Umlaufbahnen sind können die Erdnahe Umlaufbahn (engl. low earth orbit) LEO, die Polare Umlaufbahn (engl. polar earth Orbit) oder die geostationäre Umlaufbahn (engl. geostationary Orbit) GEO sein.\cite{Satellitenkommunikation}\newline
Kommunikations- und Rundfunksatelliten sind meistens, bis auf wenige Ausnahmen (Starlink), in höheren Umlaufbahnen untergebracht. So kann mit wenigen Satelliten eine globale Abdeckung erreicht werden. Eine mögliche Umlaufbahn wäre dabei die geostationäre Umlaufbahn.\cite{Satellitenkommunikation}\newline
Die geostationäre Kreisbahn ist eine besondere Umlaufbahn. Die Umlaufzeit eines Satelliten in der geostationäre Umlaufbahn entspricht der Dauer einer Rotation der Erde. So erscheint für einem Beobachter auf Erde der Satellit immer am gleichen Punkt im Himmel.\cite{Satellitenkommunikation}\newline

\subsection{Kommunikation mit einem Satelliten}
Sei es zum senden oder empfangen von Daten und Informationen oder zum steuern eines Satelliten. Damit ein Satellit einen Nutzen hat, muss die Möglichkeit bestehen mit diesem auch kommunizieren zu können. Eingesetzt werden dafür sogenannten Bodenstationen (engl. Groundstations).\newline
Bei den Bodenstation handelt es sich um auf der Erdoberfläche befindliche, meistens ortsfeste, Stationen, welche zur Beobachtung, Überwachung, Kommunikation oder Steuerung von Flugkörpern inner- und außerhalb der Erdatmosphäre eingesetzt werden.\cite{Bodensationen}\newline
Zur Erhöhung der Abdeckung und Kapazität werden meistens einzelne Bodenstation für mehrere verschiedene Anwendungen und Frequenzbänder ausgelegt und zu großen globalen Netzwerken miteinander verbunden\cite{Bodensationen}. Beispiele für solche Netzwerke sind das Estrack von ESA oder das Deep Space Network der NASA.\newline
Die Kommunikation mit Satelliten findet in den unterschiedlichsten Frequenzbänder statt. Eingeteilt und benannt werden diese durch die IEEE.

\begin{table}[H]
    \centering
    \begin{tabular}{c|c|p{6.5cm}}
    \hline
       Bezeichnung  & Frequenzbereich & Beispiele für das Frequenzband\\
       HF  & $3-30\,\text{MHz}$ & Kurzwellen Radiosender \cite{FrequenzplanBundesnetzagentur}\\
       VHF  & $30-300\,\text{MHz}$ & UKW Radio, BOS-Funk\cite{FrequenzplanBundesnetzagentur}\\
       UHF  & $300-1000\,\text{MHz}$ & LoRa, DVB-T, LTE\cite{FrequenzplanBundesnetzagentur}\\
       L  & $1-2\,\text{GHz}$ & Navigationsdienste (GPS, GLONASS, Galileo)\cite{FrequenzplanBundesnetzagentur} \\
       S  & $2-4\,\text{GHz}$ & Erderkundungsfunkdienst, Radioastronomie, WLAN\cite{FrequenzplanBundesnetzagentur} \\
       C  & $4-8\,\text{GHz}$ & Mobilfunk, Satellitenkommunikationsdienste, WLAN \cite{FrequenzplanBundesnetzagentur} \\
       X  & $8-12\,\text{GHz}$ & Satellitenkommunikationsdienste, Amateurfunk\cite{FrequenzplanBundesnetzagentur}\\
       Ku  & $12-18\,\text{GHz}$ & Satellitenrundfunk \cite{FrequenzplanBundesnetzagentur} \\
       K  & $18-27\,\text{GHz}$ & Funkdienste \cite{FrequenzplanBundesnetzagentur} \\
       Ka  & $27-40\,\text{GHz}$ &  Satellitenkommunikationsdienste\cite{FrequenzplanBundesnetzagentur}\\
       V  & $40-75\,\text{GHz}$ & Militärische Navigationsdienste \cite{FrequenzplanBundesnetzagentur}\\
       W  & $75-110\,\text{GHz}$ &Amateurfunkdienst über Satelliten \cite{FrequenzplanBundesnetzagentur} \\
       mm  & $110-300\,\text{GHz}$ & Radioastronomie \cite{FrequenzplanBundesnetzagentur}\\
    \end{tabular}
    \caption{Einteilung der Radarbänder nach IEEE 521-2002 (R2009)\cite{Frequenzbänder}}
    \label{tab:Einteilung-der-Frequenzbänder}
\end{table}
In der Tabelle \ref{tab:Einteilung-der-Frequenzbänder} sind die einzelnen Bezeichnungen der Frequenzbänder und ihr jeweiliger Frequenzbereich nach IEEE IEEE 521-2002 (R2009) aufgeführt. Ebenso sind einige Funkdienste des jeweiligen Frequenzbandes als Beispiel aufgeführt. Diese stammen aus dem Frequenznutzungsplan der Bundesnetzagentur (Stand 2022).

\subsection{Positionsbestimmung von Satelliten}
Um Informationen und Daten von einem Satelliten empfangen zu können wird eine Antenne benötigt. Diese muss gegebenenfalls direkt auf den Satelliten ausgerichtet werden. Aus diesem Grund muss die Position des Satelliten im Bezug auf die Position der Antenne in einem geeigneten Koordinatensystem angegeben werden.\newline
Verwendet werden dafür sogenannte astronomische Koordinatensysteme. Es gibt dabei mehrere verschiedene, welche sich dabei in ihrem Ursprung und in der Ermittlung der Koordinaten unterscheiden.

\begin{itemize}
    \item Horizontales System: Der Bezugspunkt in diesem System ist der Standort der Antenne. Die Position des Satelliten wird also relativ auf den Standpunkt der Antenne beschrieben. Dafür werden zwei Koordinaten, der Höhenwinkel $\varepsilon$ (Elvation) und der Kurs $\varphi$ (Azimut), verwendet.\cite{astronomischeKoordinatensysteme}\cite{Satellitenkommunikation}
    \item Äquatoriales System: Anders als bei horizontalen Koordinatensystem wird die Position des Satelliten beim äquatorialen Koordinatensystem im Bezug auf den Himmelsäquator beschrieben. Die beiden verwendeten Hauptkoordinaten sind die Deklination $\delta$ und der Stundenwinkel $t$.\cite{astronomischeKoordinatensysteme}
    \item Ekliptales System: Im ekliptikalen System wird als Bezugspunkt die Bahnebene der Erde um die Sonne, die Ekliptik, verwendet. Das Koordinatensystem verwendet dafür die beiden Hauptkoordinaten ekliptikale Länge und ekliptikale Breite.\cite{astronomischeKoordinatensysteme}
\end{itemize}
Zum Ausrichten von Antennen auf Satelliten kann am besten das horizontale Koordinatensystem verwendet werden, da es die Postion des Satelliten relativ zum Standort der Antenne beschreibt.\newline
\begin{figure}[H]
    \centering
    \includegraphics[width=0.5\linewidth]{Bilder/Horizontales Koordiantensystem.png}
    \caption{Darstellung des horizontalen Koordinatensystems\cite{astronomischeKoordinatensysteme}}
    \label{fig:horizontales-Koordinatensystem}
\end{figure}
In der Abbildung \ref{fig:horizontales-Koordinatensystem} ist das horizontale Koordinatensystem dargestellt. Der Ausgangspunkt ist dabei der Standort der Antenne. Der Punkt senkrecht über der Antenne ($\varepsilon=90\degree$) wird Zenit und der Punkt senkrecht unter der Antenne $(\varepsilon=-90\degree)$ wird Nadir genannt.\cite{astronomischeKoordinatensysteme}\newline
Mit der Azimut $\varphi$ wird die Position des Satelliten entlang des Horizontes angegeben und entspricht dem Winkel zwischen dem Satelliten und einem Ausgangspunkt. Als Ausgangspunkt kann entweder der Nord oder Südpunkt angegeben werden.\cite{Sternwarte}\cite{TU-Dresden}\newline
Im Bereich der Astronomie wird der Südpunkt als Bezugspunkt verwendet. Der Azimut $\varphi$ wird dann in Richtung Westen zählend angegeben. Wird der Nordpunk als Bezugspunkt verwendet, wird der Azimut $\varphi$ in Richtung Osten zählend angegeben. Beide Methoden sind $180\degree$ versetzt zueinander.\cite{astronomischeKoordinatensysteme}\cite{Sternwarte}\cite{TU-Dresden}\newline
Bestimmt werden kann der Azimut $\varphi$ mit der Differenz $\Delta long$ zwischen dem Längengrad der Bodenstation und dem Längengrad des Satelliten, sowie des Breitengrades $lat_\mathrm{BS}$ der Bodenstation.\cite{rfwireless-poiting}
\begin{equation}
    \varphi=\arctan\left(\frac{\tan(\Delta long)}{\sin (lat_\mathrm{BS})}\right)
    \label{eq:Azimut}
\end{equation}
Der Höhenwinkel (Elevation) $\varepsilon$ ist der Winkel zwischen dem Horizont und dem Satelliten. Dieser kann Werte zwischen $-90\degree$ (Nadir) und $90\degree$ (Zenit) annehmen.\cite{Sternwarte}\cite{TU-Dresden}\newline
Für die Bestimmung der Elevation $\varepsilon$ wird der Radius der Erde $r_\mathrm{0}$, die Flughöhe des Satelliten $r$, sowie die Differenz $\Delta long$ zwischen dem Längengrad $long_\mathrm{BS}$ der Bodenstation und dem Längengrad $long_\mathrm{SAT}$ des Satelliten und den Breitengrad $lat_\mathrm{BS}$ der Bodenstation.\cite{rfwireless-poiting}\cite{easycalculation-satellite-antenna}
\begin{equation}
    \varepsilon=\arctan\left( \frac{\cos(lat_\mathrm{BS})\cdot \cos(\Delta long)-\frac{r_\mathrm{0}}{r_\mathrm{0}+r}}{\sqrt{1-\cos^2(lat_\mathrm{BS})\cdot \cos^2(\mathrm{\Delta long)}}}\right)
    \label{eq:Elevation}
\end{equation}
Neben der Azimut $\varphi$ und der Elevation $\varepsilon$ ist auch die Neigung $Skew$ der Antenne wichtig. Die Neigung der Antenne ist wichtig, da sich die Polarisationsebene der vom Satelliten elektromagnetischen Wellen ,je nach Standpunkt auf der Erde, dreht. Um die Polarisationsebene der Antenne ideal auf die Polarisationsebene der eintreffenden EM-Welle auszurichten wird die Antenne um ihre eigene Achse gedreht. Eine nicht optimale Ausrichtung führt zum Leistungsverlust.\cite{skew}\newline
Zur Bestimmung des $Skew$ wird die Differenz $\Delta long$ zwischen dem Längengrad $long_\mathrm{BS}$ der Bodenstation und dem Längengrad $long_\mathrm{SAT}$ des Satelliten und den Breitengrad $lat_\mathrm{BS}$ der Bodenstation, sowie das Offset der Antenne benötigt.\cite{skew}
\begin{equation}
    Skew = \arctan\left(\frac{\sin(\Delta long)}{\tan(lat_\mathrm{BS})}\right)-Offset
    \label{eq:Skew}
\end{equation}











\subsection{Es’Hail-2 (QO-100)}
Bei dem Satelliten Es'Hail-2 (QO-100) handelt es sich um einen Kommunikationssatelliten, welcher von dem katarischen Unternehmen Es'hailSat betrieben wird.\cite{EsHail2}\newline
Basieren tut der Satellit auf Melco DS-200 Plattform, welche von der Japanische Firma Melco (Mitsubishi Electric Company) entwicklet wurde.\cite{EsHail2}
Am 15.11.2018 startete der Satellit an Bord einer Falcon 9 Rakete vom Cape Canaveral Space Center in seinen geostationären Testorbit, welcher bei $24\degree\text{E}$ liegt. Nach einer Testphase ist Es'Hail-2 in seine endgültigen geostationäre Umlaufbahn bei $25.9\degree\text{E}$ transferiert worden. Die geplante Lebenszeit beträgt 15 Jahre.\cite{EsHail2}\newline
Auf Es'Hail-2 befinden sich insgesamt 72 verschiedene Transponder, welche für die L-,S-,X-, Ku- und Ka-Bänder vorgesehen sind. Die Hauptaufgabe des Satelliten ist, die Regionen Nordafrika und den mittleren Osten mit TV- und Telekommunikationsdienste versorgen. Die Nutzer sind neben privaten Haushalten auch Unternehmen und Regierungsorganisationen.\cite{EsHail2}\newline
Neben den Transpondern für die kommerzielle Nutzung befinden sich auch zwei Transponder für Amateurfunk an Bord von Es'Hail-2. Bei den Transpondern handelt es sich um ein Schmalbandtransponder (engl. Narrowbandtransponder) für den Amateurfunk und einen Breitbandtransponder (engl. Widebandtransponder) für Amateurfernsehen. Diese beiden Transponder sind die ersten Amateurfunk Transponder im geostationären Orbit und gehören zur P4-A Reihe von AMSAT. Sie sind in einer Zusammenarbeit zwischen Es'hailSat, dem Qatar Amateur Radio Club (QARS) und der AMSAT Deutschland (AMSAT-DL) entstanden. Die Transponder tragen den Rufnamen Qatar Oscar 100 (QO-100), woher der Name Es'Hail-2 (QO-100) stammt.\cite{EsHail2}

\begin{figure}[H]
    \centering
    \includegraphics[width=0.5\linewidth]{Bilder/EsHail-2 Coverage.png}
    \caption{Abdeckungsbereich der Amateurfunktransponder von Es'Hail-2 (QO-100)\cite{CoverageEsHail2Amateur}}
    \label{fig:CoverageEsHail2Amateur}
\end{figure}
Die Karte in Abbildung \ref{fig:CoverageEsHail2Amateur} zeigt den abgedeckten Bereich der beiden Amateurfunktransponder auf Es'Hail-2 (QO-100). Abgedeckt sind Regionen bis zu einer Antennenelevationswinkel von $\varepsilon=5\degree$, in einigen Regionen auch bis $\varepsilon=0\degree$. Die Abdeckung reicht von Brasilien, über Afrika, Europa und Teile Grönlands und der Antarktis bis nach Thailand.\cite{EsHail2} 
Im Zuge dieser Arbeit wird nur der Schmalbandtransponder von Interesse sein. Auf den Breitbandtransponder wird nicht weiter eingegangen.

\subsubsection*{Technische Daten und Voraussetzungen für den Schmalbandtransponder Schmalbandtransponde}
Bei dem Schmalbandtransponder auf Es'Hail-2 (QO-100) handelt es sich um einen linearen Transponder. Ein linearer Transponder empfängt ein gesamtes Frequenzband, welches Uplink genannt wird, und versendet dieses empfangene Frequenzband wieder in einem anderen Frequenzbereich, welches Downlink genannt wird. Dabei hält der lineare Transponder den relative Position des Signals im empfangene Frequenzband und versendet dieses Signal wieder auf der gleichen relativen Positionen im Downlink. Lineare Transponder werden häufig in der Satellitenkommunikation und Amateurfunk eingesetzt.\cite{EsHail2}\cite{linearTransponder} 
\begin{figure}[H]
    \centering
    \includegraphics[width=0.75\linewidth]{Bilder/AMSAT-QO-100-NB-Transponder-Bandplan-Graph.png}
    \caption{Vorgeschriebener Bandplan von AMSAT-DL des Schmalbandtransponder auf Es'Hail-2 (QO-100)\cite{EsHail2NarrowbandBandplan}}
    \label{fig:NB-Bandplan}
\end{figure}
Die Abbildung \ref{fig:NB-Bandplan} zeigt den von AMSAT-DL veröffentlichen Bandplan für den Schmalbandtransponder auf Es'Hail-2 (QO-100). Dieser Bandplan ist verpflichtend für die Nutzung des Schmalbandtransponder.\newline
Im Falle von Es'Hail-2 (QO-100) liegt der Uplink im S-Band zwischen $2400.005\,\text{MHz}$ und $2400.490\,\text{MHz}$, was zu einer Bandbreite von $500\,\text{kHz}$ führt. Die Bandbreite reicht theoretisch für 100 gleichzeitige Nutzer\cite{EsHail2}\newline 
Technische Details zum Uplink zum Schmalbandtransponder auf Es'Hail-2 (QO-100):\cite{EsHail2}
\begin{itemize}
    \item Mittenfrequenz: $f_\mathrm{center}$ = $2400.250\,\text{MHz}$ (S-Band)
    \item Bandbreite des zugelassenen Bandes: $B=500\,\text{kHz}$ 
    \item Polarisation: RHCP (Rechtshändig kreisförmig Polarisiert)
    \item Maximale Sendeleistung: $2-5\,\text{W PEP} $ bei einem Antennengewinn von $G=22.5\,\text{dBi}$
    \item Maximale zugelassene Bandbreite zum Senden: $B=2.7\,\text{kHz}$
    \item Zugelassene Modulationen: Einseitenband-AM, CW, Schmalbandige Digitale Modulationen wie PSK oder BPSK. Keine Frequenzmodulation.
\end{itemize}
Der Downlink von Es'Hail-2 (QO-100) liegt im X-Band zwischen $10489.500\,\text{MHz}$ und $10490\,\text{MHz}$\cite{EsHail2}. Wird ein Signal im Uplink, zum Beispiel auf $2400.1\,\text{MHz}$, vom Schmalbandtransponder empfangen, wird dieses im Downlink $10489.650\,\text{MHz}$. So kann die Funktion des linearen Transponder am besten erklärt werden.\newline
Technische Details zum Downlink vom Schmalbandtransponder auf Es'Hail-2 (QO-100):\cite{EsHail2}
\begin{itemize}
    \item Mittenfrequenz: $f_\mathrm{center}=10489.750\,\text{MHz}$ (X-Band)
    \item Bandbreite des Downlink: $B=500\,\text{kHz}$
    \item Polarisation: V (Vertikal linear)
    \item Empfohlene größe der Parabolantenne: $90\,\text{cm}$ in Regenreichen Regionen und am Rand des Abgedeckten Bereiches (Thailand, Brasilien, etc.), $60\,\text{cm}$ im Zentrum des abgedeckten Bereiches, $75\,\text{cm}$ in Regionen bis zur $-2\,\text{dB}$ Grenze. Dazu gehören Teile Afrikas und Europa, sowie der mittlere Osten. 
\end{itemize}
Das $500\,\text{kHz}$ breite Frequenzband ist mehrere Bereiche unterteilt, welche für verschiedene Anwendungen vorgesehen sind.\newline
\begin{table}[H]
    \centering
    \begin{tabular}{c|c|c|p{4cm}}
        Uplink $\text{[MHz]}$ & Downlink $\text{[MHz]}$ & Bandbreite $\text{[kHz]}$ & Verwendung  \\
        \hline
         -& $10489.5\,\text{bis}\,10489.505$ & $5$ & Untere Funkbake mit CW-Modulation. Begrenzt das zugelassene Band  \\
        $2400.005\,\text{bis}\,2400.04$ & $10489.505\,\text{bis}\,10489.54$ & $35$& Nur für Signale mit CW Modulation  \\
        $2400.04\,\text{bis}\,2400.08$ & $10489.54\,\text{bis}\,10489.58$ & $40$& Nur für Signale mit digitaler Modulation und max. $B=0.5\,\text{kHz}$    \\
        $2400.08\,\text{bis}\,2400.15$ & $10489.58\,\text{bis}\,10489.65$ & $70$& Nur für Signale mit digitaler Modulation und max. $B=2.7\,\text{kHz}$    \\
        $2400.15\,\text{bis}\,2400.245$ & $10489.65\,\text{bis}\,10489.745$ & $95$& Nur für Signale mit Einseitenband-AM und max. $B=2.7\,\text{kHz}$    \\
        -& $10489.745\,\text{bis}\,10489.755$ & $10$& Mittlere Funkbake mit $400\,\text{Bit/S}$ BSPK    \\
        $2400.255\,\text{bis}\,2400.350$ & $10489.755\,\text{bis}\,10489.85$ & $95$& Nur für Signale mit Einseitenband-AM und max. $B=2.7\,\text{kHz}$    \\
        $2400.35\,\text{bis}\,2400.495$ & $10489.85\,\text{bis}\,10489.995$ & $70$& Alle Arten an Modulation mit max $B=2.7\,\text{kHz}$ und für spezielle Events    \\
        - & $10489.995\,\text{bis}\,10489$ & $5$& Obere experimentelle Funkbake. CW und andere Modulationen. Begrenzt das zugelassene Band   \\
    \end{tabular}
    \caption{Bandbelegung und Vorgesehene Verwendung des Schmalbandtransponder\cite{EsHail2}}
    \label{tab:NB-Bandplan}
\end{table}
Die Tabelle \ref{tab:NB-Bandplan} beinhaltet den vorgesehenen Nutzungsplan für den Schmalbandtransponder auf Es'Hail-2 (QO-100). Für die Nutzung des Schmalbandtransponders gelten mehrere Regeln:
\begin{enumerate}
    \item Für eine gerechte und faire Nutzung für alle Amateurfunker zu ermöglichen, ist die maximale Bandbreite pro Nutzer auf $B=2.7\,\text{kHz}$ begrenzt.\cite{EsHail2}
    \item Es darf keine FM-Modulation verwendet werden. Im Vergleich zu anderen Modulationen benötigt die FM-Modulation eine größere Bandbreite und mehr Sendeleistung. Da beide Faktoren begrenzt sind, ist auf eine FM-Modulation zu verzichten.\cite{EsHail2}\cite{BarkerFM}
    \item Das zugelassene Band soll eingehalten werden. Im Bereich der Funkbaken darf nicht gesendet werden.\cite{EsHail2}
    \item Eine Full-Duplex Kommunikation des Schmalbandtransponder ist vorgeschrieben. Jeder Nutzer muss zu jederzeit in der Lage sein, sein gesendetes Signal gleichzeitig auch zu empfangen.\cite{EsHail2}
    \item AMSAT-DL empfiehlt die Sendeleistung auf dem gleichen Level der Funkbaken zuhalten. Zu starke Signale werden mit einer LEILA-Sirene (Leistungs Limit Anzeige) gekennzeichnet. Der jeweilige Nutzer muss daraufhin seine Sendeleistung reduzieren.\cite{EsHail2}
\end{enumerate}
Allgemein wird der Schmalbandtransponder für eine Reihe an verschiedenen Kommunikationsarten verwendet. Normale Sprachübertragungen mittels Einseitenband-AM, Digitale Kommunikationen und auch Morse Code mit CW-Modulation werden über den Schmalbandtransponder versendet. Wichtig ist nur, dass immer der dafür vorgesehene Bereich benutzt wird.

\subsubsection*{Umlaufbahn von Es'Hail-2 (QO-100)}
Der Satellit Es'Hail-2 (QO-100) befindet sich in einer einer geostationären Umlaufbahn um die Erde. Die Flughöhe von Es'Hail-2 (QO-100), gemessen vom Äquator der Erde, beträgt $r=35790\,\text{km}$\cite{EsHail2}. Für spätere Berechnung ist die Entfernung $D_\mathrm{SAT}$ von der Bodenstation am IAT zum Satelliten Es'Hail-2 (QO-100) von Bedeutung.
\begin{figure}[H]
    \centering
    \includesvg[width=0.75\linewidth]{Bilder/Entfernung Eshail2}
    \caption{Skizze zeigt die Umlaufbahn und Entfernung zu Es'Hail-2}
    \label{fig:EntfernungEsHail2}
\end{figure}
Die Abbildung \ref{fig:EntfernungEsHail2} zeigt eine Skizze der Umlaufbahn und der Entfernungen von der Bodenstation $D_\mathrm{SAT}$, sowie vom Äquator $r$ zum Satelliten Es'Hail-2 (QO-100). Für die Bestimmung der Entfernung $D_\mathrm{SAT}$ sind zunächst die Koordinaten der Bodenstation von Bedeutung. Diese können mittels Onlinekarten, wie z.B. Google Maps oder OpenStreetMap, ermittelt werden.
\begin{figure}[H]
    \centering
    \includegraphics[width=0.75\linewidth]{Bilder/Position Bodenstation.png}
    \caption{Koordinaten der Bodenstation\cite{KoordinatenBodensation}}
    \label{fig:Koordinaten der Bodenstation}
\end{figure}
Die Bodenstation vom IAT befindet sich an den Koordinaten $53.055\degree, 8.78\degree$\cite{KoordinatenBodensation}, wobei die erste Zahl den Breitengrad und die zweite Zahl den Längengrad angibt.\newline
Zur Bestimmung der Entfernung $D_\mathrm{SAT}$zwischen der Bodenstation am IAT und Es'Hail-2 muss zunächst die senkrechte Höhe $h$ der Bodenstation zum Äquator bestimmt werden. Diese kann mithilfe des Radius der Erde $r_\mathrm{0}$ und dem Winkel $\alpha$, welcher dem Breitengrad entspricht, bestimmt werden. Der Radius der Erde beträgt $r_\mathrm{0}=6378\space\text{km}$\cite{Satellitenkommunikation}
\begin{equation*}
h=r_\mathrm{0}\cdot\sin(\alpha)=6378\space\text{km}\cdot\sin(53.055\degree)=5100.39\space\text{km}
\end{equation*}
Mithilfe der Höhe $h$ kann über den Satz des Pythagoras der Teilradius $r_\mathrm{01}$ bestimmt werden, welcher benötigt wird um den Teilradius $r_\mathrm{02}$ zu bestimmen.
\begin{equation*}
    r_\mathrm{01}=\sqrt{r_\mathrm{0}^2-h^2}=\sqrt{(6378\space\text{km})^2-(5100.39\space\text{km})^2}=3829.49\space\text{km}
\end{equation*}
Damit beträgt dann der Teilradius $r_\mathrm{02}$
\begin{equation*}
    r_\mathrm{02} = r_\mathrm{0}-r_\mathrm{01}=6378\space\text{km}-3829.49\space\text{km}=2548.22\space\text{km}
\end{equation*}
Schlussendlich kann über den Satz des Pythagoras die Entfernung $D_\mathrm{SAT}$ zwischen der Bodenstation und Es'Hail-2, mithilfe der Höhe $h$ und den zusammengesetzten Radius $r+r_\mathrm{02}$ bestimmt werden.
\begin{equation}
\begin{split}
    D_\mathrm{SAT}&=\sqrt{h^2+(r+r_\mathrm{02})^2}\\
    &=\sqrt{(5100.39\space\text{km})^2+(35790\space\text{km}+2548.22\space\text{km})^2}\\
    &=38676\space\text{km}
\end{split}
    \label{eq:EntfernungEsHail2}
\end{equation}




\subsection{Mischer}
Ein Mischer ist ein elektrisches Bauteil, welches verwendet wird um ein elektrisches Signal von seinem ursprünglichen Frequenzband in ein höheres oder niedrigeres Frequenzband umzusetzen. Beim Umsetzen in ein höheres Frequenzband handelt es sich um einen Aufwärtsmischer (engl. Upconverter) und beim umsetzen in ein niedrigeres Frequenzband um einen Abwärtsmischer (engl. Downconverter).\cite{HEUERMANN_2018}\newline
Anwendung findet der Mischer häufig im Bereich der HF-Technik und der Telekommunikation.

\subsubsection*{Funktionsweise von Mischer}
\begin{figure}[H]
    \centering
    \includesvg[width=0.75\linewidth]{Bilder/Mischer}
    \caption{Darstellung der beiden Anwendungsarten von Mischern}
    \label{fig:Theoretische-Mischer}
\end{figure}

Die Abbildung \ref{fig:Theoretische-Mischer} zeigt die Verschaltung eines Mischers als Aufwärts- (links) und Abwärtsmischer (rechts). Ein idealer Mischer ist ein Dreitor Bauelement, wovon zwei als Eingang und eins als Ausgang fungieren. Die Beschaltung der Eingänge hängt von der gewollten Anwendungsart des Mischers ab.\cite{Microwave_Wiley}\newline
Ein Mischer besteht aus nichtlinearen Bauelementen, wie z.B. Dioden oder Transistoren. Die Nichtlinearität dieser Bauelemente wird für die Frequenzumsetzung benötigt. Für die Frequenzumsetzung wird das Signal $s_\mathrm{IN}(t)$ am Eingang $\text{IN}_1$  mit dem Signal des lokalen Oszillator $s_\mathrm{LO}(t)$ multipliziert. Das Signal am Eingang $s_\mathrm{IN}(t)$ kann mit einer Kosinusfunktion definiert werden.\cite{Thiede_2013}
\begin{equation}
    s_\mathrm{IN}(t)=\hat{u}_\mathrm{IN}\cdot\cos(2\pi\cdot f_\mathrm{IN}\cdot t)
    \label{def:Eingangssignal}
\end{equation}
Für die Durchführung der Multiplikation ist die nichtlineare Kennlinie erforderlich. Allerdings führt die nichtlineare Kennlinie zu einer Vielzahl an Oberwelle und harmonischen Schwingungen. Um diese zu reduzieren, sollte ein Mischer in einem möglichst linearen Arbeitspunkt betrieben werden. \cite{Thiede_2013}\newline
Das Signal $s_\mathrm{LO}(t)$ des lokalen Oszillator (LO) kann folgend definiert werden.
\begin{equation}
    s_\mathrm{LO}(t)=\hat{u}_\mathrm{LO}\cdot\cos(2\pi\cdot f_\mathrm{LO}\cdot t)
    \label{def:LO-Signal}
\end{equation}
Das LO-Signal $s_\mathrm{LO}(t)$ sollte eine stabile Frequenz $f_\mathrm{LO}$ und stabilen Pegel aufweisen. Schwankungen im Pegel können zu einer Verschiebung des Arbeitspunktes führen, was wiederum zu mehr Oberwellen im Mischprodukt $s_\mathrm{out}(t)$ führen kann. Bei Schwankungen in der Frequenz $f_\mathrm{LO}$ verschiebt sich die Frequenz des Mischproduktes $s_\mathrm{out}(t)$.\newline
Das Mischprodukt $s_\mathrm{out}(t)$ am Ausgang des Mischers wird durch die Multiplikation des Eingangssignals $s_\mathrm{IN}(t)$ mit dem LO-Signal $s_\mathrm{LO}(t)$ bestimmt.\cite{Thiede_2013} 
\begin{equation}
    \begin{split}
    s_\mathrm{out}(t)
        &=s_\mathrm{IN}(t)\cdot s_\mathrm{LO}(t)\\
        &=\frac{\hat{u}_\mathrm{IN}\cdot \hat{u}_\mathrm{LO}}{2}\left(\cos(2\pi\cdot( f_\mathrm{IN}+ f_\mathrm{LO})\cdot t)+ \cos(2\pi\cdot( f_\mathrm{IN}- f_\mathrm{LO})\cdot t)\right)
    \end{split}
    \label{eq:Multiplikation-Mischer}
\end{equation}
Nach der Beziehung in \ref{eq:Multiplikation-Mischer} besteht das Mischprodukt $ s_\mathrm{out}(t)$ aus mehreren Frequenzkomponenten. Diese Entsprechen der Summe und der Differenz der Frequenz des Eingangssignals $f_\mathrm{in}$ und der Frequenz $f_\mathrm{LO}$ lokalen Oszillator.\cite{Thiede_2013} 
\begin{equation}
    f_\mathrm{out} = |f_\mathrm{in} \pm f_\mathrm{LO}|
    \label{eq:Frequenz-des-Mischproduktes}
\end{equation}
Je nach Anwendungsart des Mischers ist nur eins der beiden Mischprodukte erwünscht. Die zweite Frequenzkomponente kann mithilfe eines Filters entfernt werden.\cite{Microwave_Wiley}\cite{Thiede_2013}

\subsubsection*{Anwendung als Aufwärtsmischer}
Bei der Anwendung des Mischers als Aufwärtsmischer wird ein Signal aus dem niedrigen Frequenzband, dem Zwischenfrequenzbereich (ZF), in ein höheres Frequenzband verschoben. Das ZF-Signal $s_\mathrm{ZF}(t)$ kann zum Beispiel ein Datensignal oder ähnliches sein, welches mithilfe des LO-Signal aus Gleichung \ref{def:LO-Signal} in ein höheres Frequenzband verschoben wird, in welchem es zum Beispiel über eine Antenne abgestrahlt werden kann. Das LO-Signal dient dabei als Trägersignal für das LO-Signal und der Mischer kann als Modulator angesehen werden.\cite{Thiede_2013}\cite{Microwave_Wiley}. Das ZF-Signal kann folgend definiert werden.\newline
\begin{equation*}
    s_\mathrm{ZF}(t)=\hat{u}_\mathrm{ZF}\cdot\cos(2\pi\cdot f_\mathrm{ZF}\cdot t)
\end{equation*}
Die Verschaltung des Mischers als Aufwärtsmischer ist in Abbildung \ref{fig:Theoretische-Mischer} auf der linken Seite dargestellt. Das ZF-Signal $s_\mathrm{ZF}(t)$ liegt an Eingang $\text{IN}_1$ des Mischers an. Am zweiten Eingang des Mischers wird das LO-Signal
$s_\mathrm{LO}(t)$ angeschlossen, wobei $f_\mathrm{LO}>> f_\mathrm{ZF}$ ist.\cite{Thiede_2013}\newline
Am Ausgang des Mischers liegt das hochfrequente Signal $s_\mathrm{HF}(t)$ an, welches das Mischprodukt aus dem ZF-Signal und LO-Signal ist. Das HF-Signal kann mithilfe der Gleichung \ref{eq:Multiplikation-Mischer} bestimmt werden.\cite{Microwave_Wiley}
\begin{equation*}
    \begin{split}
    s_\mathrm{HF}(t)
        &=s_\mathrm{ZF}(t)\cdot s_\mathrm{LO}(t)\\
        &=\frac{\hat{u}_\mathrm{ZF}\cdot \hat{u}_\mathrm{LO}}{2}\left(\cos(2\pi\cdot( f_\mathrm{ZF}+ f_\mathrm{LO})\cdot t)+ \cos(2\pi\cdot( f_\mathrm{ZF}- f_\mathrm{LO})\cdot t)\right)
    \end{split}
\end{equation*}
Bei der Aufwärtsmischung ist nur die Summe der beiden Eingangsfrequenzen von Bedeutung. 
\begin{equation*}
    f_\mathrm{HF} = f_\mathrm{ZF} + f_\mathrm{LO}
\end{equation*}
Die Differenz kann mithilfe eines Filters entfernt werden.\cite{Thiede_2013}
\begin{figure}[H]
    \centering
    \includegraphics[width=0.75\linewidth]{Bilder/Aufwärtsmischer.png}
    \caption{Darstellung der Aufwärtsmischung im Frequenzspektrum}
    \label{fig:Spektrum-von-shf}
\end{figure}
Die Abbildung \ref{fig:Spektrum-von-shf} zeigt die Aufwärtsmischung im Frequenzspektrum. Im oberen Plot sind das ZF-Signal bei $f_\mathrm{ZF}=50\,\text{MHz}$ und das LO-Signal bei $f_\mathrm{LO}=400\,\text{MHz}$ zu sehen. Der untere Plot zeigt das Mischprodukt $s_\mathrm{HF}(t)$, welches am Ausgang des Mischers anliegt. Zu erkennen ist die modulierende Wirkung des Mischers. Das ZF-Signal wird um die Frequenz $f_\mathrm{LO}$ des LO-Signal verschoben und weißt nun zwei Signale um die Frequenz des LO-Signals auf. Das zweite Signal entsteht aufgrund der Spiegelung des Fourierspektrum um $0\,\mathrm{Hz}$. Bei der Verschiebung des ZF-Signal mit der Frequenz $f_\mathrm{ZF}$ um die Frequenz $f_\mathrm{LO}$ des LO-Signal, wird das gespiegelte ZF-Signal mit der Frequenz $-f_\mathrm{ZF}$ ebenfalls um die Frequenz $f_\mathrm{LO}$ des LO-Signal verschoben.\newline
Die beiden Signale um die Frequenz $f_\mathrm{LO}$ des LO-Signal werden Seitenbänder genannt. Das Seitenband bei $f_\mathrm{LO}+f_\mathrm{ZF}$ wird oberes Seitenband (engl. upper side band) USB und das Seitenband bei $f_\mathrm{LO}-f_\mathrm{ZF}$ wird unteres Seitenband (engl. lower side band) LSB genannt\cite{Microwave_Wiley}. Die Leistung beider Seitenbänder ist geringer als die Leistung des ursprünglichen ZF-Signal, da sich die Leistung auf zwei Signale aufteilt. Hinzu kommt aber auch die Leistung des LO-Signal, weshalb die Leistung der Seitenbänder nicht ganz halbiert ist.\newline
Wie bereits erwähnt ist nur die $f_\mathrm{LO}+f_\mathrm{ZF}$ Frequenzkomponente, also das USB, bei der Aufwärtsmischung von Interesse, weshalb das LSB auch mit einem Hochpassfilter entfernt werden kann. \cite{Thiede_2013}


\subsubsection*{Anwendung als Abwärtsmischer}
Bei der Anwendung des Mischers als Abwärtsmischer wird ein hochfrequentes Signal $s_\mathrm{HF}(t)$ aus dem HF-Bereich in ein niedrigeres Frequenzband, dem ZF-Bereich, umgesetzt. Beim HF-Signal $s_\mathrm{HF}(t)$ kann es sich zum Beispiel um ein Datensignal handeln, welches mithilfe von einer Antenne empfangen wird und an einen Empfänger weitergegeben wird. Das HF-Signal $ s_\mathrm{HF}(t)$ kann folgend definiert werden.
\begin{equation*}
    s_\mathrm{HF}(t)=\hat{u}_\mathrm{HF}\cdot\cos(2\pi\cdot f_\mathrm{HF}\cdot t)
\end{equation*}
Dabei ist die Frequenz $f_\mathrm{HF}$ des HF-Signals $f_\mathrm{HF}>>f_\mathrm{ZF}$.Die Umsetzung des HF-Signals in den niedrigen ZF-Bereich hat mehrere Gründe. Zu einem kann der Empfänger unter Umständen nicht in der Lage sein, das HF-Signal mit seiner hohen Frequenz $f_\mathrm{HF}$ direkt zu verarbeiten, weshalb das HF-Signal erst in den niedrigeren ZF-Bereich umgesetzt werden muss. Auch können eventuell verwendete Filter, Verstärker oder andere Komponenten frequenzabhängige Eigenschaften besitzen. Um die optimale Leistungsfähigkeit der Komponenten zu erreichen, wird das HF-Signal in einem für die Komponenten vorteilhaften ZF-Bereich herabgesetzt. Auch können hohe Kosten ein Grund für die Umsetzung des HF-Signals in einen niedrigeren ZF-Bereich sein. Empfänger, welche sehr hohe Frequenzen direkt verarbeiten können, und Komponeten, welche für entsprechend hohe Frequenzen optimiert sind, können hohe Anschaffungskosten mit sich bringen.\newline
Die Verschaltung des Mischers als Abwärtsmischer ist in der Abbildung \ref{fig:Theoretische-Mischer} auf der rechten Seite dargestellt. Am Eingang $\text{IN}_1$ liegt das HF-Signal $s_\mathrm{HF}(t)$ an. Am Eingang $\text{IN}_2$ liegt das LO-Signal $s_\mathrm{LO}(t)$ an. Das LO-Signal $s_\mathrm{LO}(t)$ kann folgend definiert werden.
\begin{equation*}
    s_\mathrm{LO}(t)=\hat{u}_\mathrm{LO}\cdot\cos(2\pi\cdot f_\mathrm{LO}\cdot t)
\end{equation*}
Dabei wird die Frequenz $f_\mathrm{LO}$ oft nahe der Frequenz $f_\mathrm{HF}$ des erwarteten HF-Signals gewählt. So kann das HF-Signal $s_\mathrm{HF}(t)$ möglichst weit herabgesetzt werden.\cite{Microwave_Wiley}\newline
Ist die Frequenz $f_\mathrm{HF}$ des HF-Signal gleich der Frequenz $f_\mathrm{LO}$ des lokalen Oszillator, ist das resultierende Mischprodukt $s_\mathrm{ZF}(t)$ am Ausgang eine Gleichspannung. Man spricht dabei dann auch von einem Homodynmischer\cite{Thiede_2013}. Entspricht die Frequenz $f_\mathrm{HF}$ des HF-Signal nicht der Frequenz $f_\mathrm{LO}$ des lokalen Oszillator, wird das resultierende Mischprodukt $s_\mathrm{ZF}(t)$ am Ausgang Zwischenfrequenz (ZF) genannt. Beim Mischer handelt es sich dann um einen Heterodyn-Mischer.\cite{Thiede_2013}\newline
 Nach der Gleichung \ref{eq:Multiplikation-Mischer} folgt für das Mischprodukt  $s_\mathrm{ZF}(t)$.
\begin{equation*}
    \begin{split}
    s_\mathrm{ZF}(t)
        &=s_\mathrm{HF}(t)\cdot s_\mathrm{LO}(t)\\
        &=\frac{\hat{u}_\mathrm{HF}\cdot \hat{u}_\mathrm{LO}}{2}\left(\cos(2\pi\cdot( f_\mathrm{HF}+ f_\mathrm{LO})\cdot t)+ \cos(2\pi\cdot( f_\mathrm{HF}- f_\mathrm{LO})\cdot t)\right)
    \end{split}
\end{equation*}
Das ZF-Signal besteht aus zwei Frequenzkomponenten, der Summe und der Differenz beider Eingangsfrequenzen $f_\mathrm{HF}$ und $f_\mathrm{LO}$. Die Summe $f_\mathrm{HF}+f_\mathrm{LO}$ entspricht nahezu $2\cdot f_\mathrm{HF}$, da die Frequenz des LO-Signal nahe der Frequenz des HF-Signal gewählt wird. Dafür ist aber die Frequenz  des ZF-Signals $f_\mathrm{ZF}<<f_\mathrm{HF}$. Bei der Abwärtsmischung ist nur die$f_\mathrm{HF}-f_\mathrm{LO}$ Komponente von Interesse.\cite{Microwave_Wiley}
\begin{equation*}
    f_\mathrm{ZF} = |f_\mathrm{HF} - f_\mathrm{LO}|
\end{equation*}
Die $f_\mathrm{HF}+f_\mathrm{LO}$ Komponente kann mithilfe eines Tiefpassfilters entfernt werden.\cite{Microwave_Wiley}\newline
\begin{figure}
    \centering
    \includegraphics[width=0.75\linewidth]{Bilder/Abwärtsmischer.png}
    \caption{Darstellung des Frequenzspektrums vom Mischprodukt $s_\mathrm{ZF}(t)$ nach der Abwärtsmischung}
    \label{fig:Spektrum-von-szf}
\end{figure}
In der Abbildung \ref{fig:Spektrum-von-szf} ist das Frequenzspektrum der Abwärtsmischung dargestellt. Im oberen Plot ist das HF-Signal am Eingang $\text{IN}_1$ bei $f_\mathrm{HF}=200\,\text{MHz}$ und das Signal des lokalen Oszillator bei $f_\mathrm{LO}=150\,\text{MHz}$ zu sehen. Im unteren Plot ist das Frequenzspektrum des ZF-Signal am Ausgang des Mischer dargestellt. Zu erkennen sind die beiden Frequenzkomponenten $f_\mathrm{HF}+f_\mathrm{LO}$ und $f_\mathrm{HF}-f_\mathrm{LO}$. Die $f_\mathrm{HF}+f_\mathrm{LO}$ befindet sich bei $f_\mathrm{HF}+f_\mathrm{LO}=350\,\text{MHz}$, während sich die $|f_\mathrm{HF}-f_\mathrm{LO}|$ Komponente bei $|f_\mathrm{HF}-f_\mathrm{LO}|=50\,\text{MHz}$ befindet. Von Interesse ist hier nur die Differenz $|f_\mathrm{HF}-f_\mathrm{LO}|$. Die $f_\mathrm{HF}+f_\mathrm{LO}$ Komponente kann durch einen Tiefpassfilter entfernt werden.\newline

\subsubsection*{Spiegelfrequenz}
Die Frequenzumsetzung bei Abwärtsmischer ist jedoch nicht immer eindeutig.\cite{HEUERMANN_2018}\cite{Microwave_Wiley}\newline
Um die Frequenz $f_\mathrm{LO}$ gibt es zwei Frequenzen, welche bei der Abwärtsmischung die gewünschte Zwischenfrequenz $f_\mathrm{ZF}$ ergeben.
\begin{figure}[H]
    \centering
    \includegraphics[width=0.75\linewidth]{Bilder/Spiegelfrequenz.png}
    \caption{Demonstration der Spiegelfrequenz $f_\mathrm{SP}$}
    \label{fig:Demo-Spiegelfrequenz}
\end{figure}
Die Abbildung \ref{fig:Demo-Spiegelfrequenz} zeigt den Vorgang der Abwärtsmischung im Frequenzspektrum. Jedoch liegt zusätzlich zum HF-Signal ein weiteres Signal, das SP-Signal, am Eingang des Mischers an. Für eine bessere Demonstration hat das SP-Signal einen geringfügig kleineren Pegel.\newline
Der obere Plot zeigt die beiden Eingangsignale $s_\mathrm{HF}(t)$ und $s_\mathrm{SP}(t)$ um das Signal des lokalen Oszillator. Die Frequenz der beiden Signale betragen $f_\mathrm{HF}=f_\mathrm{LO}+f_\mathrm{ZF}$ und $f_\mathrm{SP}=f_\mathrm{LO}-f_\mathrm{ZF}$. Beide Signale befinden sich also im Abstand von $f_\mathrm{ZF}$ um das LO-Signal. Bei der Abwärtsmischung des HF-Signals kommt mit Gleichung \ref{eq:Frequenz-des-Mischproduktes} das gewünschte ZF-Signal mit der Frequenz $f_\mathrm{ZF}$ raus, wie es im unteren Plot dargestellt ist. Bei der Abwärtsmischung des SP-Signal kommt es mit Gleichung \ref{eq:Frequenz-des-Mischproduktes} zu einer negative Frequenz $-f_\mathrm{ZF}$. Aufgrund der Spiegelung des Fourierspektrum um $0\,\mathrm{Hz}$ kommt es auch zu einer positiven Frequenz $f_\mathrm{ZF}$, welche das gewünschte ZF-Signal überlagert.\cite{Microwave_Wiley} 
Dieser Effekt ist im unteren Plot bei $f=50\,\text{MHz}$ zu sehen. Das herabgesetzte SP-Signal (orange) überlagert das gewünschte ZF-Signal (blau).\newline
Die Frequenz $f_\mathrm{SP}$ des SP-Signals vor dem herabsetzen wird auch Spiegelfrequenz (engl. Image Frequency) genannt. Ein an den Ausgang des Mischer angeschlossener Empfänger hat keine Möglichkeit die beiden Signale auseinander zuhalten, weshalb die Spiegelfrequenz vor dem Eingang des Mischers unterdrückt werden muss. Erreicht werden kann das mit Filtern oder sogenannten Einseitenbandmischer (engl. Image Rejection Mixer).\cite{HEUERMANN_2018}\cite{Microwave_Wiley}

\subsubsection*{Rauschen und Verluste von Mischern}
Die Ein- und Ausgänge des Mischers müssen auf die jeweilige Impedanz angepasst werden. Erschwert wird das durch die Vielzahl an Frequenzen und Oberwellen, welche während des Mischprozesses auftreten.\cite{Microwave_Wiley}\newline
Im Idealfall werden alle drei Tore des Mischers auf ihre jeweilige Frequenz $f_\mathrm{in}$, $f_\mathrm{LO}$ und $f_\mathrm{out}$ angepasst und alle weiteren Frequenzkomponenten werden mithilfe von ohmschen Lasten absorbiert oder durch reaktive Lasten geblockt. Beide Methoden bringen Verluste mit sich. Weitere Verluste treten bei der Frequenzumsetzung durch die Entstehung von Oberwellen und harmonischen Schwingungen auf.\cite{Microwave_Wiley}.\newline
In der Praxis ist für die Betrachtung der Verluste der Pfad vom LO-Signal nicht von Bedeutung. Interessant ist nur der Weg vom HF-Eingang zum ZF-Ausgang oder umgekehrt. Somit können die Verluste des Mischer wie bei einer lineare Schaltung, z.B. eines Dämpfungsglieds, über das Verhältnis der Eingangsleistung $P_\mathrm{IN}$ zu der Ausgangsleistung $P_\mathrm{OUT}$ beschrieben werden.\cite{HEUERMANN_2018}\cite{Microwave_Wiley}
\begin{equation}
    L_\mathrm{conv,dB}=10\cdot \log_{10}\left(\frac{P_\mathrm{IN}}{P_\mathrm{OUT}} \right)
    \label{eq:Mischverluste}
\end{equation}
Im Konversationsverlust $L_\mathrm{conv,dB}$ werden alle ohmschen, sowie Verluste bei Frequenzumsetzung, berücksichtigt. Die Gleichung \ref{eq:Mischverluste} kann sowohl bei Aufwärts-, als auch bei Abwärtsmischung verwendet werden.\cite{Microwave_Wiley}\newline
Das Rauschen eines Mischers kann über seine Rauschzahl $F$ ausgedrückt werden. Die Rauschzahl $F$ eines Mischers entspricht näherungsweise seiner Konversationsverluste \newline $L_\mathrm{conv,dB}$.\cite{HEUERMANN_2018}
\begin{equation}
    F_\mathrm{Mischer,dB}\approx L_\mathrm{conv,dB}
\end{equation}
Neben den Konversationsverlusten $L_\mathrm{conv}$ und der Rauschzahl $F_\mathrm{Mischer}$ kann ein Mischer auch über den Frequenzbereich, den notwendigen Pegel des lokalen Oszillator, der Isolation zwischen den Toren, seiner Linearität ($IIP_\mathrm{3}$) und seiner Impedanzanpassung seiner Tore beschrieben werden.\cite{HEUERMANN_2018}\cite{Microwave_Wiley}



















 






\subsection{Antenne}
Die Antenne ist mit der wichtigste Bestandsteil der Empfangskette an der Satellitenbodenstation. Erst mit einer geeigneten Antenne ist es mögliche die Signale vom Satelliten, welcher ebenfalls eine Antenne braucht um die Signale zu senden, zu empfangen. Die Antenne wandelt die leitungsgebundene Welle um und strahlt diese in den freien Raum ab oder empfängt die Wellen im freien Raum und gibt diese an die Leitung ab. Sie ist also das Verbindungsglied zwischen der leitungsgebundenen Welle und der Welle im freien Raum.\newline
Die IEEE definiert eine Antenne als ein passives, lineares und reziprokes Bauelement, welches Radiowellen abstrahlen, als auch empfangen kann\cite{IEEE145-1993}\cite{Balanis_2005}.\newline
Eine Antenne kann über viele verschiedene Parameter beschrieben werden. Diese Parameter helfen dabei eine geeignete Antenne für die jeweilige Anwendung zu finden.

\subsubsection*{Nah- und Fernfeld}
Der Bereich um die Antenne kann in mehrere Bereiche aufgeteilt werden. Im mittelbaren Umfeld liegt das Nahfeld, auch Fresnel-Breich genannt\cite{Radartutorial-Nahundfernfeld}, der Antenne. Neben den abgestrahlten elektromagnetische Wellen wirken hier auch starke stationäre Felder, welche ebenfalls von der Antenne ausgehen. Beschreiben lassen sich die Felder durch die maxwellschen Gleichungen. Im Nahfeld wird die Berechnung der Felder aufgrund der hohen Ordnungen der Polynome erschwert\cite{Radartutorial-Nahundfernfeld}. Aus diesem Grund werden die Strahlungscharakteristiken einer Antenne im Fernfeld bestimmt. \cite{Balanis_2005}.\newline
Das Fernfeld, auch Fraunhofer-Bereich genannt, ist geometrisch deutlich größer als das Nahfeld. Es beginnt da, wo sich die elektromagnetischen Wellen frei im Raum ausbreiten können. Der Übergang zum Fernfeld kann Näherungsweise bestimmt werden. Für Antennen, welche in ihren geometrischen Abmessung kleiner als ihre Wellenlänge $\lambda$ sind, gilt\cite{Radartutorial-Nahundfernfeld}:
\begin{equation}
    r_{fern}=2\cdot\lambda
    \label{Nahfeld}
\end{equation}
Bei größeren Antennen, zum Beispiel Parabolantennen, gilt\cite{Radartutorial-Nahundfernfeld}:
\begin{equation}
    r_{fern}=\frac{2\cdot L^2}{\lambda}
    \label{Fernfeld}
\end{equation}
Dabei gibt die Variable L die geometrische Abmessung der Antenne an. Als sichere Faustformel kann ab einem Abstand $r>5\cdot\lambda$ vom Fernfeld ausgegangen werden.\newline
Im Fernfeld existieren nur die Felder der elektromagnetische Welle, was die Berechnung der Felder deutlich vereinfacht. Die elektrische und magnetische Komponente der EM-Welle befinden sich Phase zu einander und stehen orthogonal zur Ausbreitungsrichtung. Über das Verhältnis vom elektrischen und magnetischen Feld kann der Freiraumwiderstand $\eta_0$ bestimmt werden.
\begin{equation}
    \eta_0=\frac{\left|\vec{E}\right|}{\left|\vec{H}\right|}=\sqrt{\frac{\mu_0}{\varepsilon_0}}=\mu_0\sqrt{\frac{1}{\mu_0\cdot\varepsilon_0}}=377\Omega
    \label{GleichungFreimraumwiderstand}
\end{equation}
Bis zur Entfernung $r=\frac{L^2}{2\cdot \lambda}$ um die Antenne liegt die sogenannte Rayleigh-Zone. In diesem Bereich strahlt Antenne nicht nur Energie ab, sondern nimmt auch einen Teil der abgestrahlten Energie als Blindleistung wieder auf.\cite{Radartutorial-Nahundfernfeld}

\subsubsection*{Antennen-/Richtdiagramm}
Ein Antennen- oder Richtdiagramm stellt die Strahlungscharakteristik einer Antenne grafisch dar. Die Strahlungscharakteristik einer Antenne umfasst dabei die Strahlungsleistungsdichte, die Feldstärke, Intensität, Richtfaktor, Phasenlage und Polarisation.\cite{Balanis_2005} In den meisten Fällen wird im Antennendiagramm allerdings die Intensität der abgestrahlten Energie oder ihre Feldstärke in Abhängigkeit der Richtung dargestellt\cite{Radartutorial-Antennendiagramm}. Da Antennen reziproke Elemente sind gilt ein Antennendiagramm gleichermaßen für das Senden und auch für das Empfangen mit der jeweiligen Antenne. Im Sendefall gibt das Antennendiagramm die richtungsabhängige Ausstrahlung der Antenne an und im Empfangsfall die richtungsabhängige Empfangsempfindlichkeit.\cite{Radartutorial-Antennendiagramm}\newline
Auch besteht die Möglichkeit die Strahlungscharakteristik der Antenne mithilfe einer mathematische Funktion zu definieren.\cite{Balanis_2005}\newline
\begin{figure}[H]
    \centering
    \includegraphics[width=0.5\linewidth]{Bilder/Antennendiagramm.png}
    \caption{Ein Beispiel für ein horizontales Antennendiagramm im Polarkoordinatensystem\cite{Radartutorial-Antennendiagramm}}
    \label{Antennendiagrammbeispiel}
\end{figure}
Für das Antennendiagramm kann in unterschiedlichen Formen und in verschiednen Ebenen dargestellt werden. Ein Antennendiagramm kann im 2D-Raum entlang der horizontalen (Azimuth), als auch entlang der vertikalen Ebene (Elevation) der Antenne erstellt werden. Auch kann ein Antennendiagramm im 3D-Raum erstellt werden. Die Abbildung \ref{Antennendiagrammbeispiel} zeigt ein horizontales Antennendiagramm im polaren Koordinatensystem.\newline
Neben dem polaren Koordinatensystem kann auch das kartesische Koordinatensystem verwendet werden, jedoch kann im polaren Koordinatensystem die Richtwirkung der Antenne besser dargestellt werden.
\cite{Radartutorial-Antennendiagramm}.
\subsubsection*{Haupt- und Nebenkeulen}\label{Keulen}
Im Antennendiagramm in Abbildung \ref{Antennendiagrammbeispiel} lassen sich verschiedene Muster in der Strahlungscharakteristik der Antenne erkennen, welche auch Keulen genannt werden. Dabei werden die Keulen weiter in Haupt- und Nebenkeulen unterteilt. \newline
Bei der Hauptkeule handelt es sich um den Bereich einer Antenne, in dessen Richtung am meisten Energie abgestrahlt oder, im Empfangsfall, empfangen wird.\cite{Balanis_2005}
Bei einigen Antennen können auch mehrere Hauptkeulen vorhanden sein. Ein Beispiel dafür sind Loop- oder Dipolantennen, welche zwei Hauptkeule im Antennendiagramm aufweisen. Diese Hauptkeulen sind im 180\degree versetzt zu einander. Die Hauptkeulen stellen die bevorzugte Anwendungsrichtung einer Antenne dar, egal ob die Antenne im Sende- oder Empfangsbetrieb verwendet wird.\newline
Die Nebenkeulen handelt es sich um alle Keulen, welche nicht die Hauptkeule darstellen. Diese sind jedoch deutlich kleiner und sollte auch so klein wie möglich sein. Nebenkeule sind meistens unerwünscht, da sie Enegie in ungewollte Richtungen abstrahlen und so weniger Energie durch die Hauptkeule abgestrahlt wird oder da sie im Empfangsfall dafür sorgen, dass die Antenne aus eventuell unerwünschten Richtungen Signale aufnimmt und so den Empfang stören.\cite{Balanis_2005}. Die größten beiden größten Nebenkeulen werden auch Seitenkeulen genannt.\cite{Balanis_2005}.\newline
Der Abstand von der Hauptkeule zur größten Nebenkeule ist die Nebenkeulendämpfung. Je größer der Wert ist, desto kleiner sind die Nebenkeulen. Die Nebenkeulendämpfung ist ein wichtiger Parameter für Richtantennen, da damit die Richtschärfe ausgedrückt werden kann.\newline
Die Haupt- und Nebenkeulen bilden sich bei jeder Antenne, welche kein isotropischer Kugelstrahler ist.

\subsubsection*{Strahlbreite}
Im Zusammenhang mit dem Strahlungsmuster einer Antenne kann ein weiterer Parameter hergeleitet werden. Die Stahlbreite beschreibt den Öffnungswinkel der Hauptkeule. Gemessen wird die Strahlbreite an zwei identischen Punkten auf beiden Seiten des Maximums der Hauptkeule\cite{Balanis_2005}.\newline
Oft verwendet wird die 3dB-Strahlbreite, auch Half-Power Beamwidth genannt. Diese wird von der IEEE definiert als der Winkel zwischen den zwei Punkten an der Hauptkeule, wo die abgestrahlte Leistung nur noch die Hälfte des Maximums der Hauptkeule beträgt\cite{Balanis_2005}.\newline
Es gibt auch noch andere Strahlbreite wie die First Null Beamwith (FNBW), diese findet aber in der Praxis keine große Anwendung\cite{Balanis_2005}.\newline
Die Strahlbreite ist gerade für Richtantennen ein wichtiger Parameter, da die Strahlbreite ihr Auflösungsvermögen beschreibt. Mit einer kleineren Strahlbreite kann im Empfangsfall eine größere Winkelauflösung erreicht werden. Eine größere Winkelauflösung hilft einer Antenne dabei zwischen mehreren benachbarten Strahlungsquellen zu unterscheiden. Mit einem größeren Öffnungswinkel neigt die Antenne dazu benachbarte Quellen als eine wahrzunehmen. Das kann für zum Beispiel Radaranlagen wichtig sein\cite{Balanis_2005}. Allerdings wachsen mit geringere Strahlbreite auch die Nebenkeulen, was unerwünschte Effekte, wie in \ref{Keulen} beschrieben, führt \cite{Balanis_2005}.

\subsubsection*{Antennengewinn}
Ein weiterer nützlicher Parameter, welcher für die Beschreibung von Antennen verwendet werden kann, ist der Antennengewinn $G$. Der Antennengewinn ist eng mit dem Richtfaktor und dem Wirkungsrad der Antenne verbunden\cite{Balanis_2005}.\newline
Eine reale Antenne strahlt die eingespeiste Leistung $P_S$ nicht gleichmäßig in alle Richtungen ab. Eine reale Antenne weißt bevorzugte Richtungen $(\phi,\theta)$ auf, gekennzeichnet durch die Haupt- und Nebenkeulen im Antennendiagramm, in welche sie die Leistung abstrahlt oder aus welcher sie Leistung aufnimmt.\newline
Im Sendefall entspricht der Antennengewinn $G(\varphi,\theta)$ dem Verhältnis der abgestrahlten Strahlungsleistungsdichte $S(\phi,\theta)$ der Antenne zu der abgestrahlten Strahlungsleistungsdichte $S_{ref}(\phi,\theta)$ einer Referenzantenne bei gleicher eingespeisten Leistung $P_S$, Richtung $(\phi,\theta)$ und Entfernung $r$\cite{Balanis_2005}.
\begin{equation}
    G(\phi,\theta)=\frac{S(r,\phi,\theta)}{S_{ref}(r,\phi,\theta)}
    \label{Grunddefinition Antennengewinn}
\end{equation}
Die Entfernung r kürzt sich aus der Gleichung raus. Sie ist für den Antennengewinn nicht entscheidend.\newline
Da Antennen reziproke Elemente sind gilt die Gleichung \ref{Grunddefinition Antennengewinn} gleichermaßen für den Empfangsbetrieb. Im Empfangsbetrieb entspricht der Antennengewinn $G$ dem Verhältnis der empfangenen Leistung $P_E(\phi,\theta)$ der jeweiligen Antenne zu der empfangenen Leistung $P_{Eref}(\phi,\theta)$ einer Referenzantenne bei gleicher Sendequelle mit fester Sendeleistung $P_S$ und Entfernung $r$ und gleichen Empfangswinkel $(\phi,\theta)$.\newline
\begin{equation}
    G(\phi,\theta)=\frac{P_E(\phi,\theta)}{P_{Eref}(\phi,\theta)}
    \label{Antennengewinn Empfangsfall}
\end{equation}
Als Referenzantenne in beiden Fällen eine beliebige Antenne gewählt werden. In den meisten Fällen wird als Referenzantenne der isotrope Kugelstrahler verwendet. Allerdings kann auch der einfache hertzsche Dipol verwendet werden\cite{Balanis_2005}.\newline
Der isotrope Kugelstrahler ist eine rein theoretische Antenne. Der isotrope Kugelstrahler strahlt die eingespeiste Leitung $P_S$ in alle Richtungen gleichmäßig aus und empfängt auch aus allen Richtungen die gleiche Leistung $P_E$. Aus diesem Grund eignet sich der isotrope Kugelstrahler besonders gut als Referenzantenne. Für die Strahlungsleistungsdichte eines isotrope Kugelstrahler gilt:
\begin{equation}
    S_0=\frac{P_S}{4\cdot \pi \cdot r^2 }
    \label{Strahlungsleistungsdichte isotroper Kugelstrahler}
\end{equation}
Der Gewinn wird meistens logarithmisch in dBi angegeben. Das i in dBi bedeutet, dass der Gewinn auf einen isotropen Kugelstrahler bezogen angeben wird. Aus der Gleichung \ref{Grunddefinition Antennengewinn} und \ref{Strahlungsleistungsdichte isotroper Kugelstrahler} folgt dann für die logarithmische Darstellung:
\begin{equation}
    G=10 \cdot \log_{10}\left( \frac{S(\phi,\theta)\cdot 4\cdot \pi \cdot r^2}{P_S} \right)
    \label{GewinndBi}
\end{equation}
Wird nichts weiter angegeben, kann im Datenblatt einer Antenne beim Gewinn $G(\varphi,\theta)$ vom Gewinn in Richtung der Hauptkeule ausgegangen werden, da diese auch die bevorzugte Anwendnugsrichtung der Antenne darstellt.\cite{Balanis_2005}

\subsubsection*{Richtfaktor und Wirkungsgrad}
Beim Richtfaktor $D$ einer Antenne handelt es sich um  das Verhältnis der Strahlungsintensität bei einem bestimmten Abstrahlwinkel $(\varphi,\theta)$ zu der durchschnittliche Strahlungsintensität der Antenne in alle Richtungen. Dabei wird meistens als Abstrahlwinkel $(\varphi,\theta)$ der Winkel von der maximalen Strahlungsintensität, also der Hauptkeule der Antenne, verwendet.\cite{Balanis_2005}
\begin{equation}
    D=\frac{\text{Maximale Strahlungsintensität}}{\text{Durchschnittliche Strahlungsintensität}}=\frac{\phi_\mathrm{max}}{\phi_\mathrm{\varnothing}}
    \label{RichtfakotrD}
\end{equation}
Die durchschnittliche Strahlungsintensität kann über die von der Antenne abgestrahlten Leistung $P_\mathrm{rad}$ bestimmt werden.\cite{Balanis_2005}
\begin{equation}
    \phi_\mathrm{\varnothing}=\frac{P_\mathrm{rad}}{4\pi}
\end{equation}
Mit dem Richtfaktor $D$ und mithilfe des Antennenwirkungsgrad $\eta$ kann der Gewinn $G$ einer Antenne ermittelt werden.
\begin{equation}
    G=\eta\cdot D = \eta\cdot\frac{\phi_\mathrm{max}}{\phi_\mathrm{\varnothing}}
\end{equation}
Der Antennenwirkungsgrad $\eta$ berücksichtigt Verluste, welche innerhalb der Antenne auftreten. Zusammensetzen tut sich der Antennenwirkungsgrad $\eta$ aus den ohmschen Verlusten $\epsilon_\mathrm{R}$, den Verlusten durch Reflexion $\epsilon_\mathrm{\Gamma}$ und den dielektrischen Verlusten $\epsilon_\mathrm{d}$.\cite{Balanis_2005}
\begin{equation}
    \eta=\epsilon_\mathrm{R}\cdot\epsilon_\mathrm{\Gamma}\cdot\epsilon_\mathrm{d}
\end{equation}
Bei verlustlosen Antennen gilt $G=D$, da $\eta=1$. -> Mehr zu den Verlusten raus suchen.

\subsubsection*{Äquivalente Strahlungsleistung}
Die äquivalente Strahlungsleistung oder auch effektive Strahlungsleistung (ERP) ist eine nützliche Größe um die die Auswirkung des Gewinns $G(\varphi,\theta)$ einer Antenne zu verdeutlichen.\newline
Die äquivalente Strahlungsleistung gibt die Leistung an welche eine Referenzantenne abstrahlen müsste, um die gleiche Strahlungsleistungsdichte $S(\varphi,\theta)$ der Bezugsantenne in einem bestimmten Abstrahlwinkel $(\varphi,\theta)$ zu erreichen. Beim Abstrahlwinkel $(\varphi,\theta)$ der Bezugsantenne wird in der Regel von Hauptkeule der Antenne ausgegangen.\cite{Radartutorial-ERP}\newline
Als Referenzantenne kann eine beliebige Antenne verwendet werden. In der Praxis werden für gewöhnlich eine Dipolantenne oder ein isotroper Kugelstrahler als Referenz gewählt. Wird ein Dipol als Referenz gewählt, wird die äquivalente Strahlungsleistung als ERP angegeben. Wird jedoch ein isotroper Kugelstrahler als Referenzantenne verwendet, wird die äquivalente Strahlungsleistung als EIRP angeben. EIRP steht für equivalente isotropic radiated power oder äquivalente isotropische Strahlungsleistung.\cite{Radartutorial-ERP}\newline
Das EIRP setzt sich aus der Sendeleistung $P_\mathrm{T}$ und dem Gewinn $G$ in Richtung der Hauptkeule und den Verlusten der Antenne $L_\mathrm{ANT}$ zusammen.\cite{Radartutorial-ERP}
\begin{equation}
    EIRP=P_\mathrm{T}\cdot \frac{G}{L_\mathrm{ANT}}
    \label{EIRPdBm}
\end{equation}
Das EIRP kann auch logarithmisch, z.B. in dBm, angeben werden.
\begin{equation}
    EIRP_\mathrm{dBm}=10 \cdot \log_{10} \left( \frac{P_\mathrm{T}\cdot \frac{G}{L_\mathrm{ANT} }}{1\cdot 10^{-3}}\right) = P_\mathrm{T,dBm}+G_\mathrm{dBi}-L_\mathrm{ANT,dB}
    \label{EIRPdBm}
\end{equation}
Das EIRP und ERP hängen über den Gewinn $G=1.64$ der Dipolantenne gegenüber dem isotropen Kugelstrahler miteinander zusammen.\cite{Radartutorial-ERP}
\begin{equation*}
    EIRP = 1.64\cdot ERP
\end{equation*}
Mithilfe des ERP und EIRP kann die scheinbare Leistung eines Senders quantifiziert werden. Anwendung findet das im Bereich der Telekommunikationstechnik. Die Bundesnetzagentur gibt mit dem EIRP die maximale zulässige Sendeleistung im sogenannten Frequenznutzungsplan \cite{FrequenzplanBundesnetzagentur} an. So soll eine gemeinschaftliche Nutzung der einzelnen Frequenzbänder garantiert und gegenseitige Störungen minimiert werden.\cite{Radartutorial-ERP}

\subsubsection*{Effektive Antennenfläche}
Die effektive Antennenfläche $A_\mathrm{E}$ ist ein wichtiger Parameter für Antennen, welche als Empfangsantennen betrieben werden.\newline
Die effektive Antennenfläche, auch Absoprtionsfläche oder Wirkläche genannt, ist rein theoretische Fläche. Diese kann gleich oder kleiner als die reale Fläche der jeweiligen Antenne sein.\newline
Bestimmt wird die effektive Antennenfläche $A_\mathrm{E}$, bei verlustlosen Antennen, über das Verhältnis von der am Fuße der Antenne verfügbaren Leistung $P_\mathrm{E}$ zu der Strahlungsleistungsdichte $S_\mathrm{E}$ von der auf die Antenne eintreffende Welle.\cite{Balanis_2005}
\begin{equation}
    A_\mathrm{E}=\frac{P_\mathrm{E}}{S_\mathrm{E}}
\end{equation}
Dank der Reziprozität von Antennen kann aus der effektiven Antennenfläche $A_\mathrm{E}$ auch der Gewinn $G$ der Antenne ermittelt werden und umgekehrt.\newline
-> weiter ausführen




\subsubsection*{Polarisation}



Eine Antenne fungiert als eine Schnittstelle zwischen elektrischen Signalen und elektromagnetischen Wellen im freien Raum. Sie wandelt leitungsgebunde Energie in elektromagnetischen Wellen um oder umgekehrt.\newline
Im freien Raum existieren zwei verschiedene Arten von Wellen, die Transversal- und Longitudinalwellen. Bei einer transversalen Welle erfolgen die Schwingungen senkrecht zur Ausbreitungsrichtung. Die Schwingungen einer Longitudinalwelle erfolgen in Richtung der Ausbreitung.\cite{Wellentypen}\newline
-> Bild der Wellentypen \newline
Bei elektromagnetischen Wellen handelt es sich um Transversalwellen. Im Gegensatz zu Lognitudalwellen können Transversalwellen polarisiert werden.\cite{Wellentypen}
Die Polarisierung $E$ einer elektromagnetischen Welle kann als eine Funktion der Zeit $t$ angesehen werden. Sie beschreibt die Veränderung der Richtung und relative Amplitude des E-Feld Vektors, indem sie in gleichmäßigen zeitlichen Intervallen $n$ die Extremstellen der Schwingungen entlang der Ausbreitungsrichtung der elektromagnetischen Welle im Raum darstellt.\cite{Balanis_2005}\newline
->Bild der Polarisierung einfügen\newline
Die Polarisierung einer Antenne kann mit der Polarisierung der von ihr abgestrahlten EM-Welle beschrieben werden. Jedoch kann die Bauform und Ausrichtung einer Antenne zu unterschiedlichen Polarisierung innerhalb ihrer Strahlungscharakteristik führen. Demnach ist die Polarisierung einer Antenne abhängig vom Abstrahlwinkel $(\varphi,\theta)$, kann aber mit der Polarisierung der aus diesem Winkel abgestrahlten EM-Welle beschrieben werden.\cite{Balanis_2005}
Im Fernfeld einer Antenne kann die von ihr abgestrahlte elektromagnetische Welle an jedem Punkt auf dem Ausbreitungspfad durch eine ebene Welle mit derselben Ausbreitungsrichtung und elektrischen Feldstärke $\vec{E}$
angenähert werden. Dies gilt jedoch nur für Punkte, die sich tatsächlich im Fernfeldbereich und entlang des Ausbreitungspfads der Welle befinden. Im Nahfeld wirken zusätzlich statische und induktive Feldkomponenten, weshalb die Wellenfront hier noch gekrümmt ist. Mit wachsender Entfernung zur Antenne vergrößert sich der Krümmungsradius der Wellenfront. Der Einfluss der Nahfeldanteile nimmt also mit wachsender Entfernung zur Quelle ab. Für sehr große Entfernungen wird die Wellenfront lokal praktisch eben, und die elektromagnetische Welle kann hinsichtlich Ausbreitung und Polarisationsverhalten wie eine ebene Welle betrachtet werden. Diese Eigenschaft erlaubt es, die Polarisation der abgestrahlten Welle im Fernfeld durch die Polarisationsrichtung einer ebenen elektromagnetischen Welle eindeutig zu charakterisieren.\cite{Balanis_2005}\newline
Für eine auf eine Antenne einfallende elektromagnetische Welle wird die Polarisierung als die Polarisierung einer ebenen Welle definiert, die aus einer gegebenen Richtung mit fester Leistungsflussdichte einfällt und die maximale verfügbare Leistung an den Antennenklemmen liefert.\cite{Balanis_2005}\newline
Die Polarisierung von Antennen und EM-Wellen können in drei Arten klassifiziert. Eine Antenne oder EM-Wellen kann entweder linear, zirkular/kreisförmig oder elliptisch polarisiert sein.\cite{Balanis_2005}\newline
Bei einer linearen Polarisierung bleibt die Richtung der Schwingung unverändert. Nur die relative Amplitude des E-Feld Vektors ändert periodisch ihren Betrag und Vorzeichen. Damit der E-Feld Vektor nur ein einer Ebene entland der Ausbreitungsrichtung der EM-Welle schwingt darf dieser nur aus einer Komponente oder aus zwei Komponenten bestehen, welche entweder in Phase oder ein vielfaches von $180\degree$ oder $\pi$ außer Phase zu einander sind.\cite{Balanis_2005}
\begin{figure}[H]
    \centering
    \includegraphics[width=0.5\linewidth]{Bilder/Linearly_Polarized_Wave.png}
    \caption{Arten an linearen Polarisationen\cite{linear_polarization_image}}
    \label{LinearePolarisation}
\end{figure}
Die Abbildung \ref{LinearePolarisation} zeigt die unterschiedlichen Arten der linearen Polarisation. Im Bezug zu der Erdoberfläche kann eine EM-Welle vertikal linear polarisiert (blau), horizontal linear polarisiert (grün) oder schräg linear polarisiert (rot) sein.\newline
Im Falle einer zirkularen oder kreisförmigen Polarisation bleibt die realtive Amplitude des E-Feld Vektors konstant. Jedoch rotiert der E-Feld Vektor mit konstanter Winkelgeschwindigkeit senkrecht zur Ausbreitungsrichtung. Am Ort des Beobachters zeigt E-Feld Vektor damit einen Kreis auf.\cite{Balanis_2005}
\begin{figure}[H]
    \centering
    \includegraphics[width=0.5\linewidth]{Bilder/Rising_circular.png}
    \caption{Zusammensetzung einer zirkularen Polarisation\cite{circular_polarization_image}}
    \label{ZusammensetzungZirkular}
\end{figure}
In Abbildung \ref{ZusammensetzungZirkular} sind die Komponenten des E-Feld Vektors (blauer Pfeil) dargestellt. Der E-Feld Vektor setzt sich aus zwei einzelnen E-Feld Vektoren (Rot und Blau) zusammen, welche orthogonal zu einander stehen. Die Phasendifferenz zwischen den roten und blauen E-Feld Vektoren beträgt dabei $90\degree$ oder $\frac{\pi}{2}$ oder ein ungerade vielfaches davon.\cite{Balanis_2005}\newline
Der Vektor des E-Feldes kann dabei gegen den Uhrzeigersinn oder im Uhrzeigersinn rotieren. Ist eine Welle im Uhrzeigersinn polarisiert, nennt man diese auch rechtshändig polarisiert (engl. right hand circularly polarized) oder auch RHCP. Ist die Welle gegen den Uhrzeigersinn polarisiert, ist von einer linkshändig polarisierten Welle (engl. left hand circularly polarized) oder LHCP die Rede.\cite{Balanis_2005}\newline














 

\subsection{Rauschen}
In jedem System tritt neben dem gewünschten Nutzsignal s(t) zusätzlich noch Rauschen auf. Einfach gesagt ist Rauschen eine unerwünschte Form an Energie, welches das Nutzsignal s(t) überlagert und das  Übertragen, Empfangen stört und die Weiterverarbeitung des Signals s(t) erschwert.\newline
\begin{figure}[H]
    \centering
    \includegraphics[width=1\linewidth]{Bilder/Example SNR good.png}
    \caption{Ein von Rauschen überlagertes Signal s(t) mit einer Frequenz von 50 Hz im Zeit- und Frequenzbereich}
    \label{ExampleNoise}
\end{figure}
Genauer betrachtet handelt es sich beim Rauschen um einen stochastischen Prozess. Das bedeutet, dass der Verlauf des Rauschens nicht periodisch und zufällig ist und keine brauchbaren Informationen enthält.\cite{HEUERMANN_2018}\newline
Bei der Planung und Entwicklung von Kommunikationssystemen spielt das Rauschen ein wichtige Rolle. Damit das Nutzsignal s(t) problemlos gesendet, übertragen , empfangen oder verarbeitet werden kann muss zwischen der Leistung $P_{\text{Signal}}$ des Nutzsignals s(t) und der Leistung $P_{\text{Rausch}}$ einer gewisser Abstand eingehalten werden. Diesen Abstand zwischen der Rauschleistung $P_{\text{Rausch}}$ und der Leistung $P_{\text{Signal}}$ des Nutzsignals s(t) oder das Verhältnis der beiden Leistungen zueinander nennt man Signal-zu-Rausch-Abstand (engl. Signal-to-Noise-Ratio) oder kurz SNR.\newline
\begin{equation}
    SNR=\frac{P_{\text{Signal}}}{P_{\text{Rausch}}}=\frac{S}{N}
\end{equation}
Das SNR kann auch logarithmisch in dB angeben werden.
\begin{equation}
    SNR_{\text{dB}}=10\cdot \log_{10}\left(\frac{P_{\text{Signal}}}{P_{\text{Rausch}}}\right)=S_{\text{dB}}-N_{\text{dB}}
\end{equation}
Das SNR ist ein Maß für die Qualität des Nutzsignals s(t). Auch wenn die Rauschleistung N im Vergleich zur Leistung S des Nutzsignals s(t) in den meisten Fällen sehr gering ist, ist diese letztendlich der limitierende Faktor in einem Kommunikationssystems. Um zum Beispiel ein problemloses Empfangen und die anschließende Weiterverarbeitung des Nutzsignals s(t) zu gewährleisten muss dafür das Nutzsignal s(t) gut vom Rauschen unterscheidbar sein. Mit steigendem SNR steigt auch der Abstand zwischen dem Signal s(t) und dem Rauschen. Je größer das SNR ist, desto besser kann das Nutzsignal s(t) vom Rauschen unterschieden werden. Im unteren Plot der Abbildung \ref{ExampleNoise} ist das Frequenzspektrum zu sehen. Das Nutzsignal s(t) hat einen Abstand von ca. 20 dB zum Rauschen und kann daher gut vom Rauschen unterschieden werden.\newline
Sinkt das SNR, so sinkt auch der Abstand zwischen dem Nutzsignal s(t) und dem Rauschen. Wird das SNR zu gering kann das Signal s(t) nicht mehr zuverlässig vom Rauschen unterschieden werden.\newline

In Abbildung \ref{BADSNR} ist ein Beispiel für ein sehr niedriges SNR dargestellt. Das Rauschen überlagert das Signal s(t) fast vollständig und es kann nicht mehr einwandfrei vom Rauschen unterschieden werden.\newline
\begin{figure}[H]
    \centering
    \includegraphics[width=1\linewidth]{Bilder/Example SNR bad.png}
    \caption{Das Signal s(t) verschwindet im Rauschen}
    \label{BADSNR}
\end{figure}
In Fällen von Signalen mit digitalen Modulationen steigt mit sinkendem SNR die Bitfehlerrate (engl. Bit-Error-Rate), auch BER genannt. Je kleiner also das SNR wird, desto mehr Bitfehler treten auf und erschweren die Kommunikation.\newline
-> Quelle für BER finden und passende Grafik.\newline
Die Verteilung der Rauschleistung $N$ über das Frequenzspektrum wird mit dem Leistungsdichtespektrum $S_\text{N}(f)$ angegeben.
\begin{equation}
    S_\text{N}(f)=\frac{N}{2\cdot B} = \frac{k\cdot T}{2}=\frac{n_\text{0}}{2}
    \label{PDS-Funktion}
\end{equation}
Die Einheit des Leistungsdichtespektrum ist W/Hz. Herleiten lässt sich das Leistungsdichtespektrum mithilfe der Fouriertransformation aus der Autokorrelationsfunktion (AKF) und einer Normallast von $1 \Omega$.\cite{Thiede_2013}
\subsection{Arten und Quellen von Rauschen}
Rauschen kann sehr vielfältig sein. Es gibt verschiedene Arten von Rauschen, welche in unterschiedlichen Bereich auftreten und auch unterschiedliche Rauschquellen aufweisen. Hauptsächlich kann zwischen internen und externen Rauschquellen unterschieden werden, wobei diese sich wieder auf künstliche oder natürliche Rauschquellen aufteilen lassen. Zu den internen Rauschquellen gehören unter anderem thermisches Rauschen, Schrotrauschen und 1/f-Rauschen. Zu externen Rauschquellen gehören Atmosphärisches- und Industrielles Rauschen und Hintergrundrauschen, welche in Form der Antennentemperatur ausgedrückt werden können.
\subsubsection*{Thermisches Rauschen (Thermal Noise)}
Alle Metalle und elektrische Bauteile, wie Widerstände und Halbleiter, erzeugen ab einer Temperatur $T>0\text{K}$ eigenständig eine Rauschenergie. Dieses Rauschen wird auch als thermisches oder Gaußsches Rauschen bezeichnet.\newline
Zurückführen lässt sich das thermische Rauschen auf die zufällige Bewegung von Elektronen und Löchern innerhalb der Metalle und elektrischen Bauteile. Bei einer Temperatur $T=0\text{K}$ stehen alle Ladungsträger und es wird damit auch kein Rauschen generiert. Ab einer Temperatur $T>0\text{K}$
fangen sich die Ladungsträger an in zufällige Richtungen zu Bewegen, was zum Rauschen führt.\cite{HEUERMANN_2018}\newline
-> Grafik einbinden \newline
An den Anschlüssen eines Widerstandes oder einer anderen beliebigen Impedanz liegt aufgrund der Bewegung der Ladungsträger eine gewisse Spannung $U_{\text{Rausch}}$ an.\cite{HEUERMANN_2018} Das Rauschen kann als ein Signal n(t) angesehen werden dessen Spannung $U_{\text{Rausch}}$ im Mittelwert. 
\begin{equation}
    \overline{U}_{\text{Rausch}}=\lim_{T\to\infty}\frac{1}{T}\int^T_0n(t)\space dt = 0
\end{equation}
entspricht. Über den quadratische Mittelwert oder Niquist-Gleichung allerdings lässt sich der Effektivwert der Spannung ermitteln.\cite{Thiede_2013}\newline
\begin{equation}
    U_\text{Rausch,eff}=\sqrt{4\cdot k\cdot T\cdot B\cdot R}
\end{equation}
Über den Effektivwert $U_\text{Rausch,eff}$ der Spannung , welche auch Niquistgleichung genannt wird, lässt sich die Leistung des Rauschens ermitteln
\begin{equation}
    N_\text{T}=P_\text{Rausch}= \frac{U_\text{Rausch,eff}^2}{4R} = \frac{4R \cdot T\cdot B}{4R}= k\cdot T\cdot B
    \label{ThermalNoiseGl}
\end{equation}
Die Leistung des thermischen Rauschen $N_\text{T}$ ist letztendlich unabhängig von dem Widerstand $R$ und nur noch abhängig von der Boltzmannkonstante $k=1,38\cdot 10^{-23}\frac{\text{J}}{\text{K}}$, der Temperatur T und der gewählten Bandbreite B.\cite{HEUERMANN_2018}\cite{Thiede_2013}\newline
Die Gleichung \ref{ThermalNoiseGl} kann auf verschiedene Zweitore angewendet werden, um zum Beispiel das Rauschen am Ausgang eines LNA zu bestimmen. Die Temperatur $T$ wird dabei durch die äquivalente Rauschtemperatur $T_\text{e}$ ersetzt.\cite{HEUERMANN_2018}


Die Leistung des thermischen Rauschen ist unabhängig von der Frequenz $f$ und ist gleichmäßig über das gesamte Frequenzspektrum verteilt. Somit ist das Leistungsdichtespektrum $S_\text{N}(f)$ des thermischen Rauschens konstant. Damit handelt es sich beim thermischen Rauschen um sogenanntes weißes Rauschen\cite{Thiede_2013} Die thermische Rauschleistung $N_\text{T}$ in einem System steigt mit der gewählten Bandbreite $B$.

\subsubsection*{Schrotrauschen (Shot Noise)}
Erwähnt wird das Schrotrauschen erstmals von Schottky im Jahre 1918, weshalb es auch Schottky-Rauschen genannt wird. Auftreten tut das Schrotrauschen in Halbleiterbauelemente, wie z.B. Dioden, zusätzlich zum thermischen Rauschen.\cite{HEUERMANN_2018}\newline
Seinen Ursprung hat das Schrotrauschen in der zufälligen Bewegung von Ladungsträgern zwischen dem Leitungs- und Valenzband. Die damit verbundene Fluktuation von Energie erzeugt das Rauschen. Der energetische Abstand zwischen den beiden Bändern wird Potentialschwelle genannt.\cite{HEUERMANN_2018}\newline
-> Bild einfügen von den Bändern und der Potentialschwelle
\newline
In einem Halbleiter erfolgt der Transport von Energie durch gequantelte Ladungsträger statt. Bei gequantelten Ladungsträgern handelt es sich um Teilchen, wie Elektronen oder Löcher, deren Ladung der elementar Ladung $e = 1.602 \cdot 10^{-19}\text{As}$ oder ein vielfaches davon entspricht.\cite{HEUERMANN_2018}\cite{leifiphysik-elementarladung}\newline
Das Schrotrauschen ist proportional zum mittleren fließenden Strom in dem jeweiligen Halbleiter. Da sich der mittlere fließende Strom je nach Halbleiter und Anwendung unterscheidet, muss das Schrotrauschen immer individuell betrachtet werden.\cite{HEUERMANN_2018}\cite{Thiede_2013}\newline
->Vielleicht Beispiel anhand einer Schottkydiode\newline

\subsubsection*{1/f-Rauschen (Flicker Noise)}
Eine weitere Rauschquelle in einem Halbleiter geht vom 1/f-Rauschen (engl. Flicker Noise) aus. Beim 1/f-Rauschen handelt es sich um sogenanntes pinkes Rauschen.\cite{liquid-flicker} Anders als beim weißen Rauschen, wie z.B. thermisches Rauschen, was eine gleichbleibende Leistungsdichte über das gesamte Frequenzspektrum aufweist, nimmt beim pinken Rauschen mit steigender Frequenz $f$ die Leistungsdichte ab.\cite{liquid-flicker}\newline
->Graphen von 1/f-Rauschen einfügen \newline
Daher stammt auch der Name 1/f-Rauschen. Die Leistungsdichte des 1/f-Rauschen dominiert im niedrigen Frequenzbereich bis zu einer Grenzfrequenz $f_\text{c}$ gegenüber der Leistungsdichte des thermischen Rauschen. Ab der Grenzfrequenz $f_\text{c}$ geht das 1/f-Rauschen im thermischen Rauschen unter.\cite{Thiede_2013} \cite{HEUERMANN_2018}\newline
Die Leistungsdichte und die Grenzfrequenz $f_\text{c}$ des 1/f-Rauschen unterscheiden sich je nach Halbleitermaterial, z.b. Germanium und Silizium, und Bauelement, wie z.B. Diode oder MOSFET. Im Falle von Bipolartransistoren dominiert das 1/f-Rauschen bis zu einer Grenzfrequenz $0.1\text{ Hz} \leq f_\text{c} \leq 1 \text{ kHz}$. Bei MOSFETs kann das 1/f-Rauschen in einigen Fällen bis zu einer Grenzfrequenz $f_\text{c}=10 \text{ MHz}$ das thermischen Rauschen dominieren.\cite{HEUERMANN_2018}\newline
Eine genaue Erklärung für das 1/f-Rauschen gibt es nicht.\cite{HEUERMANN_2018} Es gibt aber verschiedene Theorien zur Entstehung des 1/f-Rauschen. Eine einfache Theorie besagt, dass ein Transistor in tiefen Frequenzen mit das Rauschen mit einer Verstärkung $G=\frac{1}{f}$ verstärkt und so das Grundrauschen anhebt.\cite{HEUERMANN_2018} Eine weitere Theorie ist, das das 1/f-Rauschen aus der zufälligen Bewegung der Ladungsträger und damit verbundenen Fluktuation von Energie und Änderung der Ladungsträgerkonzentration hervorgeht. Diese Fluktuationen entstehen durch Defekte in der Gitterstruktur des Halbleiters. Diese Defekte treten überwiegend an der Oberfläche des Halbleiters, auch Interface genannt, auf. Die Ladungsträger werden von diesen Defekten "gefangen" oder "freigelassen" (engl. trapping und detrapping). Dieser Vorgang soll zum Rauschen führen.\cite{liquid-flicker}\cite{Thiede_2013}\newline
Auch wenn das 1/f-Rauschen nur niedrigen Frequenzbereich auftrifft, muss es auch für Anwendungen in höheren Frequenzbereichen berücksichtigt werden. Durch die nichtlineare Eigenschaften von nichtlinearen Bauteilen, wie z.B. Dioden, welche auch in einem Mischer eingesetzt werden um Signale in verschiedene Frequenzbereiche umzusetzen, kann auch das Rauschen durch den Prozess der Frequenzumsetzung in höheren Frequenzbereiche umgesetzt werden und den Rauschpegel anheben.\cite{HEUERMANN_2018}

\subsubsection*{Antennentemperatur}



\subsubsection*{Äquivalente Rauschtemperatur}
Bei der äquivalenten Rauschtemperatur $T_\text{e}$ handelt es sich nicht um eine physikalische Temperatur sonder um eine rein fiktive Temperatur, welche als Rechengröße verwendet wird.\newline
Mit den bisherigen Erkenntnissen lässt sich schlussfolgern, das jedes Bauteil, egal ob Widerstand oder Halbleiter, rauscht. Auf Bauteilebene kann das Rauschen in seinen einzelnen Formen mit thermischen Rauschen, 1/f-Rauschen, Schrotrauschen, etc. beschrieben werden. In komplexeren Schaltungen kann die Beschreibung des Rausches in all seinen verschiedenen Formen sehr aufwendig werden. Mithilfe einer äquivalente Rauschtemperatur $T_\text{e}$ lässt sich das Rauschen eines einzelnen Bauteils, Zweitores oder ganzen Systemen in einer Erhöhung der Temperatur ausdrücken.\newline
-> Grafik einfügen \newline
Die Gleichung \ref{xxx} zeigt ein Beispiel für die äquivalente Rauschtemperatur $T_\text{e}$. Am Eingang des Zweitors liegt die Rauschleistung $N_\text{i}$ an. Diese geht von einer angepassten Rauschquelle, wie ein Widerstand, aus und entspricht dem thermischen Rauschen. Das zusätzlichen Rauschen des Zweitors wird in der Rauschleistung $N_\text{0}$ an dessen Ausgang durch die Addition einer fiktiven Erhöhung der Temperatur $T_\text{e}$ berücksichtigt.\cite{Thiede_2013} 
\begin{equation}
    N=k \cdot T_\text{f} \cdot B = k \cdot (T_\text{e}+T_\text{0}) \cdot B \cdot G
\end{equation}
Bei dieser virtuellen Erhöhung der Temperatur handelt es um die äquivalente Rauschtemperatur $T_\text{e}$. Diese entspricht genau der Erhöhung der Temperatur, damit das Zweitor rechnerisch genau die Rauschleistung $N_\text{0}$ am Ausgang erzeugt, wie sie auch am Ausgang bei der tatsächlich Temperatur $T_\text{0}$ anliegt. Es wird die gleiche Gleichung wie für das thermische Rauschen verwendet. Ebenfalls wird auch die Verstärkung des Zweitores in der Gleichung berücksichtigt.\cite{Thiede_2013}
Die äquivalente Rauschtemperatur $T_\text{e}$ kann umgekehrt auch aus der Rauschleistung $N_\text{0}$ am Ausgang des Zweitores gewonnen werden.
\begin{equation}
    T_\text{e}=\frac{N_\text{0}}{k\cdot B \cdot G}-T_\text{0}
\end{equation}
Die äquivalente Rauschtemperatur $T_\text{e}$ ist immer größer als die eigentliche Temperatur $T_0$, da es neben dem thermischen Rauschen auch alle weiteren Rauschquellen, wie 1/f-Rauschen und Schrotrauschen, berücksichtigt.\cite{Thiede_2013}















\section{Theoretische Betrachtung des Downlinks vom Schmalbandtransponder auf Es'Hail-2 (QO-100)}
\label{chap:Theoretische-Betrachtung-des-Downlinks}
\subsection{Darstellung des Downlinks}
Beim Downlink handelt es sich um eine Datenverbindung zwischen einem Satelliten und einer Bodenstation, wobei der Datenaustausch von Satellit in Richtung der Bodenstation stattfindet.
\begin{figure}[H]
    \centering
    \includegraphics[width=0.5\linewidth]{Bilder/Skizze Downlink.drawio.png}
    \caption{Vereinfachte Darstellung des Downlinks}
    \label{SkizzeDownlink}
\end{figure}
Die Abbildung \ref{SkizzeDownlink} zeigt eine vereinfachte Darstellung des Downlinks zwischen dem Satelliten Es'Hail-2 und der Bodenstation am IAT. Einteilen lässt sich der Downlink in drei kleinere Bereiche - dem Sender, der Übertragungsstrecke und dem Empfänger.






\subsection{Sender - Schmalbandtransponder auf Es'Hail-2 (QO-100)}
Beim ersten Bereich des Downlinks handelt es sich um den Sender. In diesem Fall handelt es sich um den Schmalbandtransponder auf Es'Hail-2 (QO-100).
\begin{figure}[H]
    \centering
    \includegraphics[width=0.5\linewidth]{Bilder/Vereinfachter Sender BSB.png}
    \caption{Vereinfachte Darstellung des Schmalbandtransponders auf Es'Hail-2\cite{Satellite_Communications_Systems}}
    \label{fig:Vereinfachter Sender}
\end{figure}
Die Abbildung \ref{fig:Vereinfachter Sender} zeigt ein vereinfachtes Blockschaltbild des Schmalbandtransponders auf Es'Hail-2 (QO-100). Der Schmalbandtransponder deckt ungefähr $1/3$ der Erdoberfläche ab, siehe Abbildung \ref{fig:CoverageEsHail2Amateur}, und hat die Aufgabe die Signale, welche von Amateurfunkern über den Uplink zum Satelliten gesendet werden, wieder in Richtung Erde mit der zusenden.\newline
Der Schmalbandtransponder verwendet zur Verstärkung der über den Uplink gesendeten Signale einen TWTA (engl.Traveling-Wave Tube Amplifier) mit einer Ausgangsleistung von $P_\mathrm{TX}=100\space\text{W} = 50\space\text{dBm}$\cite{PräsiEsHail2}. Die internen Verluste des Schmalbandtransponders werden mit $L_\mathrm{SAT}=1.5\space\text{dB}$ angegeben\cite{PräsiEsHail2}. Mit den beiden Angaben und mit einer $OBO = 6\space\text{dB}$ (engl. Output Back Off)\cite{PräsiEsHail2} lässt sich die Sendeleistung $P_\text{T}$ des Schmalbandtransponders ermitteln.\newline
\begin{equation}
    \label{Sendeleistung Es'Hail-2}
    P_\mathrm{T}=P_\mathrm{TX,dB}-L_\mathrm{SAT,dB}-OBO_\mathrm{dB}=50\space\text{dBm}-1.5\space\text{dB}-6\space\text{dB}=42.5\space\text{dBm}
\end{equation}
Der Schmalbandtransponder auf Es'Hail-2  verwendet eine Hornantenne mit einem Gewinn von $G_\mathrm{T}=17\space\text{dBi}$\cite{PräsiEsHail2} und einer 3dB-Strahlungsbreite von $\theta_\mathrm{3dB}=17.4\space\degree$\cite{PräsiEsHail2}. Mit der Sendeleistung $P_\mathrm{T}$ und Gewinn $G_\mathrm{T}$ der Antenne kann dann das $EIRP$ des Satelliten über die Gleichung \ref{eq:EIRPdBm} bestimmt werden. Zu den Verlusten der verwendeten Antenne lassen sich keine Informationen finden, weshalb eine verlustlose Antenne angenommen wird.
\begin{equation}
    EIRP_\mathrm{dBm}=P_\mathrm{T,dBm}+G_\mathrm{T,dBi}-L_\mathrm{SATANT,dB}=42.5\,\mathrm{dBm}+17\,\mathrm{dBi}-0\,\mathrm{dB}=59.5\,\mathrm{dBm}
    \label{eq:EIRP_dBm_Eshail2}
\end{equation}
Das in Gleichung \ref{eq:EIRP_dBm_Eshail2} $EIRP_\mathrm{dBm}$ kann in auch in $\text{[W]}$ angegeben werden.
\begin{equation}
    EIRP=10^{\frac{EIRP_\mathrm{dBM}}{10}}\cdot0.001\,\text{W}=891.251\,\text{W}
    \label{eq:EIRP_W_EsHail2}
\end{equation}
Ebenfalls kann die Strahlungsleistungsdichte $S_\mathrm{SAT}$ der von Es'Hail-2 abgestrahlten EM-Welle bestimmt werden. Diese ist für die spätere Bestimmung der Empfangsleistung von Bedeutung.\newline
Ermittelt werden kann die Strahlungsleistungsdichte $S_\mathrm{SAT}$ mit der Gleichung \ref{eq:isotroperkugelstrahler-strahlungsleistungsdichte} unter der Berücksichtigung des Gewinns $G_\mathrm{T}=17\space\mathrm{dBi}$ der Hornantenne, sowie der Entfernung $D_\mathrm{SAT}$, welche in Gleichung \ref{eq:EntfernungEsHail2} ermittelt wird.

\begin{equation}
    S_\mathrm{SAT}=\frac{P_\mathrm{T}\cdot \frac{G_\mathrm{T}}{L_\mathrm{ANT}}}{4\pi\cdot D_\mathrm{SAT}}=\frac{EIRP}{4\pi\cdot D_\mathrm{SAT}}=\frac{891.251\,\text{W}}{4\pi\cdot 38676\,\text{km}}=1.834\cdot10^{-6}\,\frac{\text{W}}{\text{m}^2}
    \label{eq:Strahlungsleistungsdichte_EsHail2}
\end{equation}


\subsection{Übertragungsstrecke zwischen Es'Hail-2 und der Bodenstation am IAT}
Den zweiten Bereich bildet die Übertragungsstrecke zwischen dem Satelliten Es'Hail-2 und der Bodenstation am IAT. Bevor die vom Schmalbandtransponder abgestrahlten EM-Wellen von der Bodenstation am IAT empfangen werden können legen diese eine große Entfernung zurück. Auf dem Weg verlieren die abgestrahlten EM-Wellen einen großen Teil ihrer Leistung, was mit der Dämpfung $L$ ausgedrückt wird. Die gesamt Dämpfung $L$ setzt sich dabei aus mehreren einzelnen Dämpfungen zusammen, welche in unterschiedlichen Abschnitten der Übertragungsstrecke auftreten.
\subsubsection*{Freiraumdämpfung}
Die Freiraumdämpfung, auch Pfadverlust genannt, $L_\mathrm{FR}$ bildet den größten Teil der auftretenden Dämpfung $L$. Sie ist abhängig von der Entfernung $D_\mathrm{SAT}$ zwischen dem Sender und Empfänger, Es'Hail-2 und der Bodenstation am IAT, sowie die Wellenlänge $\lambda$ von der Frequenz $f$, mit welcher der Downlink betrieben wird.
\begin{equation}
    \label{eq:Freiraumdämpfung}
    L_\mathrm{FR,dB}=10\cdot\log_\mathrm{10}\left( \left( \frac{4\pi\cdot D_\mathrm{SAT}}{\lambda} \right)^2\right)=20\cdot\log_\mathrm{10} \left( \frac{4\pi\cdot D_\mathrm{SAT}}{\lambda} \right)
\end{equation}
Mit der Freiraumdämpfung $L_\mathrm{FR}$ wird die Abnahme der Strahlungsleistungsdichte $S_\mathrm{SAT}$ beschrieben. Ein isotroper Kugelstrahler strahlt die Energie, in Form von EM-Wellen, gleichmäßig in allen Richtung ab. Somit verteilt sich die Energie gleichmäßig in Form einer Kugel um die Quelle herum, wie es auch in Gleichung \ref{eq:isotroperkugelstrahler-strahlungsleistungsdichte} und \ref{eq:Strahlungsleistungsdichte_EsHail2} ausgedrückt wird. Wird die Oberfläche der Kugel in gleichgroße Bereiche aufgeteilt, weisen alle Bereiche die gleiche Strahlungsleistungsdichte $S_\mathrm{SAT,xy}$ auf\cite{RadartutorialFreiraumdämpfung}.\newline
Mit steigender Entfernung $r$ zur Quelle, welche dem Radius der Kugel entspricht, wird auch die Oberfläche der Kugel größer. Da aber die von dem isotropen Kugelstrahler abgestrahlte Energie gleichbleibend ist, hat die steigende Oberfläche der Kugel eine Abnahme der Strahlungsleistungsdichte $S_\mathrm{SAT}$ zur Folge.\cite{RadartutorialFreiraumdämpfung}
\begin{figure}[H]
    \centering
    \includesvg[width=0.5\linewidth]{Bilder/Freiraumdämpfung Beispiel.svg}
    \caption{Graphische Repräsentation der Freiraumdämpfung}
    \label{Graphische Repräsentation der Freiraumdämpfung}
\end{figure}
Mit der Wellenlänge 
\begin{equation*}
   \lambda_\mathrm{center}=\frac{c}{f_\mathrm{center}}=\frac{3\cdot 10\,\frac{\text{m}}{\text{s}}}{10489.750\,\text{MHz}}=0.0286\,\text{m} 
\end{equation*}
und der Entfernung $D_\mathrm{SAT}$ aus Gleichung \ref{eq:EntfernungEsHail2} kann die Freiraumdämpfung $L_\mathrm{FR}$ über Gleichung \ref{eq:Freiraumdämpfung} bestimmt werden.
\begin{equation}
    \label{eq:BestimmteFreiraumdämpfung}
    L_\mathrm{FR,dB}=20\cdot\log_\mathrm{10}\left( \frac{4\pi\cdot D_\mathrm{SAT}}{\lambda} \right)=20\cdot\log_\mathrm{10}\left( \frac{4\pi\cdot 38676\,\text{km}}{0.0286\,\text{m}}\right)=204.61\,\text{dB}
\end{equation}
Die Freiraumdämpfung $L_\mathrm{FR,dB}$ ist zusammen mit der Strahlungsleistungsdichte $S_\mathrm{SAT}$ oder dem $EIRP$ wichtig für Bestimmung der empfangen Leistung $P_\mathrm{R}$ an der IAT Bodenstation.\newline
Obwohl sich der größte Teil der Übertragungsstrecke im freien Raum befindet, müssen neben der Freiraumdämpfung $L_\mathrm{FR,dB}$ noch zusätzlich Dämpfungen innerhalb der Atmosphäre berücksichtigt werden. Beim durchqueren der Atmosphäre erfahren die elektromagentischen Wellen eine nicht zu vernachlässigende Dämpfung. Die Dämpfung basiert dabei hauptsächlich durch die Absorption und Entpolarisierung der elektromagnetischen Wellen, welche durch Gase, Partikel und Dämpfe innerhalb der Atmosphäre hervorgerufen werden. \cite{Satellite_Communications_Systems}\newline
\begin{figure}[H]
    \centering
    \includesvg[width=0.5\linewidth]{Bilder/Atmosphäre_Stufen}
    \caption{Aufbau der Atmosphäre\cite{Bild_Atmosphäre}}
    \label{fig:Aufbau-der-Atmosphäre}
\end{figure}
Im Frequenzbereich von $1\,\text{GHz}$ bis $30\,\text{GHz}$ haben hauptsächlich Wasser- und Sauerstoffmoleküle eine großen Einfluss auf die elektromagnetischen Wellen. Daher sind hauptsächlich zwei Schichten der Atmosphäre von Interesse. Die Troposphäre, in welcher sich das Wettergeschehen abspielt, und die Ionosphäre, in welcher die UV-Strahlung der Sonne Gasmoleküle ionisiert.\cite{Ionosphäre} \cite{Satellite_Communications_Systems}\newline

\subsubsection*{Dämpfung durch Regen und Schnee}
Das Wetter, hauptsächlich Regen und Schnee, in der Troposphäre bildet den größten Teil der Dämpfung innerhalb der Atmosphäre. Gerade bei höheren Frequenzen $(f\geq10\,\text{GHz})$ darf die Dämpfung durch Regen $L_\mathrm{Regen}$ nicht vernachlässigt werden.
Für die Dämpfung durch Regen sind Niederschlagsraten $R_\mathrm{p}$ in $\text{mm/h}$ interessant, welche nur zu einem bestimmte Prozentsatz $p$ die durchschnittliche Niederschlagsmenge $\text{mm/h}$ eines Jahres überschreiten.\cite{Satellite_Communications_Systems}\newline
Für de Betrieb des Downlinks können drei Wetterbedingungen festgelegt werden.
\begin{itemize}
    \item Klarer Himmel (Clear Sky): Den größten Teil der Zeit $(p\approx20\,\%)$ sind mit niedrigen Regenraten zu rechnen. Je kleiner die Niederschlagsrate $R_\mathrm{p}$ in $\text{mm/h}$ ist, desto geringer ist die zu erwartende Dämpfung $L_\mathrm{Regen}$. Bei der Bedingung klarer Himmel sind die Niederschlagsmengen so gering, dass die resultierende Dämpfung vernachlässigt werden können.\cite{Satellite_Communications_Systems}
    
    \item leichter Regen (light Rain): Die häufigsten zu erwartenden Regenschauer sind leichte Regenschauer. Zu leichten Regenschauern zählen Regenschauer dessen Niederschlagsrate $R_\mathrm{p}$ in $\text{mm/h}$ zu $p=5\,\%$ der Zeit den jährlichen Durchschnitt $\text{mm/h}$ überschreiten. Die zu erwartende Dämpfung $L_\mathrm{leichterRegen}$ ist überschaubar und bietet einen guten Schätzwert für durchschnittliche Regenschauer.
    
    \item Regen (Rain): Starke Niederschläge haben eine sehr große Auswirkung auf die elektromagnetischen Wellen. Die starken Niederschläge verursachen eine nicht zu vernachlässigende Dämpfung $L_\mathrm{Regen}$. Zu starken Niederschlägen zählen Regenschauer deren Niederschlagsmenge $R_\mathrm{p}$ in $\text{mm/h}$ die durchschnittliche Niederschlagsmenge $\text{mm/h}$ eines Jahres zu $p=0.01\,\%$ der Zeit überschreiten.\cite{Satellite_Communications_Systems}
\end{itemize}


Die durch starke Niederschläge verursachte Dämpfung $L_\mathrm{Regen}$ ist das Produkt aus der spezifischen Dämpfung $\gamma_\mathrm{Regen}$ in $\text{dB/km}$ und der effektiven Pfadlänge $D_\mathrm{Regen}$ $(\text{km})$, welche die elektromagnetischen Wellen durch den Regen zurücklegen müssen.\cite{Satellite_Communications_Systems}
\begin{equation}
    L_\mathrm{Regen}=\gamma_\mathrm{Regen}\cdot D_\mathrm{Regen}
    \label{eq:Dämpfung-durch-Regen}
\end{equation}
Die spezifische Dämpfung $\gamma_\mathrm{Regen}$ ist abhängig von der Frequenz $f$ und der Niederschlagsmenge $R_\mathrm{\mathrm{0.01}}$, welche die durchschnittliche Niederschlagsmenge $\text{mm/h}$ eines Jahres zu $p=0.01\,\%$
überschreitet. Diese Niederschlagsmenge ist wichtig, da dann die spezifische Dämpfung $\gamma_\mathrm{Regen}$ am größten ist und der Downlink eventuell nicht mehr aufrecht erhalten werden kann. Damit steigt dann auch die Ausfallzeit.\cite{Satellite_Communications_Systems}\newline
Die Dämpfung für andere Niederschlagsrate $R_\mathrm{p}$, welche zu $0\,\%\leq p\leq5\,\%$ der Zeit den Jahresdurchschnitt überschreiten können aus der Dämpfung $L_\mathrm{Regen}$ für $R_\mathrm{0.01}$ gewonnen werden.\cite{Satellite_Communications_Systems}\newline
\begin{figure}[H]
    \centering
    \includegraphics[width=0.75\linewidth]{Bilder/Rainrate.png}
    \caption{Karte zeigt die Niederschlagsmenge $(\text{mm/h})$ welche zu $p=0.01\,\%$ den jährlichen Durchschnitt überschreitet \cite{ITU-RP.837-8}}
    \label{fig:ITUR-Regenrate}
\end{figure}
Die Karte in Abbildung \ref{fig:ITUR-Regenrate} zeigt eine globale Übersicht über die Niederschlagsmenge $R_\mathrm{p}$ $(\text{mm/h)}$ welche zu $p=0.01\,\%$ den Jahresdurchschnitt in der jeweiligen Region überschreitet. Für den Norddeutschen Raum kann eine Niederschlagsrate $R_\mathrm{0.01}\approx35\,\text{mm/h}$ entnommen werden.\newline
Die durch starke Niederschläge verursachte Dämpfung $L_\mathrm{Regen}$ wird in mehreren Schritten bestimmt. Im ersten Schritt muss die effektive Regenhöhe $h_\mathrm{R}$ bestimmt werden. Dafür ist die Höhe $h_\mathrm{iso}$ der durchschnittliche $0\degree$ isothermische Schicht über dem Meeresspiegel wichtig. Die isothermische Höhe $h_\mathrm{iso}$ ist eine fiktive Grenze zwischen zwei Luftmassen. Oberhalb der Grenze weisen die Luftmassen eine negative Temperatur und unterhalb eine positive Temperatur auf. Sie kann aus einer Karte entnommen werden.\cite{isothermehöhe} Im Raum Europa beträgt diese $h_\mathrm{iso}=3\,\text{km}$.\cite{Satellite_Communications_Systems}
\begin{equation}
    h_\mathrm{R} = h_\mathrm{iso}+0.36\,\text{km}=3\,\text{km}+0.36\,\text{km}=3.36\,\text{km}
    \label{eq:effekitve-Regenhöhe}
\end{equation} 
Mithilfe der effektiven Regenhöhe $h_\mathrm{R}$ kann die Länge des Pfade unter den Regenwolken $D_\mathrm{S}$ bestimmt werden. Dafür ist auch der Elevationswinkel $\varepsilon$ der Antenne notwendig und die Höhe $h_\mathrm{Station}$der Bodenstation über dem Meeresspiegel. Das Gebäude der Hochschule Bremen am Flughafen, in welchem die Bodenstation errichtet wird, befindet sich $7\,\text{m ü.N.N}$\cite{höhebremen}. Die Höhe des Gebäudes kann mit $12\,\text{m}$ angenommen werden. Daraus ergibt sich für die Höhe der Station über dem Meeresspiegel $h_\mathrm{Station}=7\,\text{m}+12\text{m}=19\,\text{m}$. Für die Berechnung der Pfadlänge $D_\mathrm{S}$ wird der Elevationswinkel $\varepsilon$ der Antenne benötigt. Dieser wird in der Gleichung \ref{eq:Elevation-Antenne} mit $\varepsilon=27.36\degree$ angegeben.\cite{Satellite_Communications_Systems}
\begin{equation}
    D_\mathrm{S}=\frac{h_\mathrm{R}-h_\mathrm{Station}}{\sin(\epsilon)}=\frac{3.36\,\text{km}-0.019\,\text{km}}{\sin(27.36\degree)}=7.26\,\text{km}
    \label{eq:länge-des-schrägen-Pfads-unter-den-Wolken}
\end{equation}
Im nächsten Schritt wird die horizontale Projektion $D_\mathrm{HP}$ der Pfadlänge $D_\mathrm{S}$ bestimmt. Diese wird für die Bestimmung der spezifische Dämpfung $\gamma_\mathrm{Regen}$ benötigt.\cite{Satellite_Communications_Systems}
\begin{equation}
    D_\mathrm{HP}=D_\mathrm{S}\cdot\cos(\varepsilon)=7.26\,\text{km}\cdot{\cos(27.36\degree)}=6.44\,\text{km}
    \label{eq:horizontale-Projektion}
\end{equation}
Die spezifische Dämpfung $\gamma_\mathrm{Regen}$ ist von der bestimmten Niederschlagsrate 
$R_\mathrm{0.01}\approx35\,\frac{\text{mm}}{\text{h}}$ abhängig.\cite{Satellite_Communications_Systems}
\begin{equation}
    \gamma_\mathrm{Regen}=k\cdot(R_\mathrm{0.01})^\alpha
\end{equation}
Dabei sind $k$ und $\alpha$ frequenzabhängige Koeffizienten, welche mit\cite{Satellite_Communications_Systems} 
\begin{equation}
    k=\frac{k_\mathrm{H}+k_\mathrm{V}+(k_\mathrm{H}-k_\mathrm{V})\cos^2{\epsilon}\cdot\cos{2\tau}}{2}
\end{equation}
beziehungsweise
\begin{equation} \alpha=\frac{k_\mathrm{H}\cdot\alpha_\mathrm{H}+k_\mathrm{V}\cdot\alpha_\mathrm{V}+(k_\mathrm{H}\cdot\alpha_\mathrm{H}-k_\mathrm{V}\cdot\alpha_\mathrm{V})\cos^2{\epsilon}\cdot\cos{2\tau}}{2k}
\end{equation}
bestimmt werden können. Die Werte für $k_\mathrm{H}$, $k_\mathrm{V}$, $\alpha_\mathrm{H}$ und $\alpha_\mathrm{V}$ sind von der Frequenz des Downlinks abhängig und können aus einer Tabelle in ITU-R P.838 entnommen werden. Für $f=10\,\text{GHz}$ gilt\cite{ITU-RP.838-3}:
\begin{itemize}
    \item $k_\mathrm{H}=0.01217 $
    \item $k_\mathrm{V}=0.01129 $
    \item $\alpha_\mathrm{H}=1.2571 $
    \item $\alpha_\mathrm{V}=1.2156 $
\end{itemize}
Die weitere Berechnung der spezifischen Dämpfung $\gamma_\mathrm{Regen}$ erfolgt in Python. Das Pythonskript ist im Github-Repository und im Anhang \ref{lst:Dämpfung-durch-Regen-python} hinterlegt. Für die spezifischen Dämpfung $\gamma_\mathrm{Regen\,0.01}$ ergibt sich für eine Niederschlagsmenge $R_\mathrm{0.01}=35\,\text{mm/h}$ ein Wert von 
\begin{equation*}
    \gamma_\mathrm{Regen}=1.03\,\frac{\text{dB}}{\text{km}}
\end{equation*}
Für die effektive Pfadlänge ergibt sich ein Länge von
\begin{equation*}
    D_\mathrm{Regen}= 8.59\,\text{km}
\end{equation*}
Die mit Gleichung \ref{eq:Dämpfung-durch-Regen} bestimmte Dämpfung $L_\mathrm{Regen}$ durch Niederschläge, welche zu $p=0.01\,\%$ den Jahresdurchschnitt überschreitet, beträgt damit 
\begin{equation}
    L_\mathrm{Regen}=\gamma_\mathrm{Regen}\cdot D_\mathrm{Regen}=1.03\,\frac{\text{dB}}{\text{km}}\cdot8.59\,\text{km}=8.86\,\text{dB}
    \label{eq:bestimmte-Regendämpfung}
\end{equation}
Die Dämpfung $L_\mathrm{Regen}$ gilt für eine Niederschlagsmenge $R_\mathrm{0.01}=35\,\text{mm/h}$. Diese Niederschlagsmenge überschreitet im Bereich Norddeutschland den Jahresdurchschnitt zu $p=0.01\,\%$ der Zeit. Diese Dämpfung ist für die Wetterbedingung Regen relevant und bietet einen guten Schätzwert für die Dämpfung, welche bei stärkeren Regenschauern auftritt.\newline
\begin{figure}[H]
    \centering
    \includegraphics[width=0.75\linewidth]{Bilder/Attenuation_caused_by_rain_10_GHz.png}
    \caption{Graph zeigt die Dämpfung $L_\mathrm{Regen}$ in Abhängigkeit von der Niederschlagsrate $R_\mathrm{0.01}$ in $\text{mm/h}$ für die Frequenz $f=10\,\text{GHz}$ }
    \label{RegenDämpdungGraph}
\end{figure}
Die Abbildung \ref{RegenDämpdungGraph} zeigt die Dämpfung $L_\mathrm{Regen}$ in Abhängigkeit von der Niederschlagsrate für die Frequenz $f=10\,\text{GHz}$. Wichtig ist, dass die Niederschlagsraten $R_\mathrm{p}$ in $\text{mm/h}$ dabei die Niederschlagsraten sind, welche den Jahresdurchschnitt zu $p = 0.01\,\%$ der Zeit überschreiten.\newline
Wie bereits erwähnt können die geschätzte Dämpfungen für Niederschlagsmengen $R_\mathrm{p}$, welche den Jahresdurchschnitt zu  $0\,\%\leq p\leq5\,\%$ der Zeit überschreiten, aus der Dämpfung $L_\mathrm{Regen}$ bestimmt werden.\cite{Satellite_Communications_Systems}
\begin{equation}
    L_\mathrm{Regen\,p}=L_\mathrm{Regen}\cdot\left(\frac{p}{0.01}\right)^{-(0.655+0.033\cdot\ln(p)-0.045\ln(L_\mathrm{Regen})-\beta(1-p)\cdot\sin(\varepsilon)}
    \label{eq:Dämpfung-durch-Regen-für-0-bis-5}
\end{equation}
Der Koeffizient $\beta$ ist abhängig von der Wahrscheinlichkeit $p$, dem Längengrad $long_\mathrm{BS}$, sowie dem Elevationswinkel $\varepsilon$ der Antenne.\cite{Satellite_Communications_Systems}
\begin{equation}
 \beta = \begin{cases} 
  0 & \text{,}\, p \geq1\,\% \,\text{oder}\,|long_\mathrm{BS}|\geq36\degree \\ 
  -0.005(|long_\mathrm{BS}|-36) & \text{,}\, p <1\,\% \,\&\,|long_\mathrm{BS}|<36\degree\,\&\,\epsilon\geq27\degree \\ 
  -0.005(|long_\mathrm{BS}|-36)+1.8-4.25\cdot\sin(\varepsilon) & \text{,}\,\text{sonst}
  \label{eq:beta}
\end{cases} 
\end{equation}
Zur Bestimmung der Dämpfung $L_\mathrm{leichterRegen}$ für Niederschlagsmenge $R_\mathrm{p}$, welche zu $p=5\,\%$ der Zeit den Jahresdurchschnitt $(\text{mm/h})$ überschreiten, werden die Koordinaten der Bodenstation benötigt. Diese können aus der Karte in Abbildung \ref{fig:Koordinaten der Bodenstation} gewonnen werden. Die Bodenstation befindet sich an den Koordinaten $53.055\degree, 8.78\degree$, womit $long_\mathrm{BS} =8.78\degree$ ist. Der Elevationswinkel der Antenne wird in Gleichung \ref{eq:Elevation-Antenne} bestimmt und beträgt $e=27.88\degree$.\newline
Die Dämpfung $L_\mathrm{leichterRegen}$ für Niederschlagsmenge, welche zu $p=5\,\%$ der Zeit den Jahresdurchschnitt $(\text{mm/h})$ überschreiten, wird mithilfe der Gleichungen \ref{eq:Dämpfung-durch-Regen-für-0-bis-5} und \ref{eq:beta} in Python bestimmt.
\begin{equation}
    L_\mathrm{leichterRegen}=0.2\,\text{dB}
    \label{eq:Dämpfung-durch-leichten-Regen}
\end{equation}
Die Dämpfung $L_\mathrm{leichterRegen}$ aus Gleichung \ref{eq:Dämpfung-durch-leichten-Regen} ist deutlich geringer als die Dämpfung $L_\mathrm{Regen}$ in Gleichung \ref{eq:bestimmte-Regendämpfung}, da in $L_\mathrm{leichterRegen}$ auch niedrigere Regenrate mitberücksichtigt. Die Dämpfung $L_\mathrm{leichterRegen}$ bietet einen bessere Schätzung für die durchschnittliche Dämpfung bei normalen Regenschauern, während die die Dämpfung $L_\mathrm{Regen}$ eine gute Schätzung für starke Regenschauer ist.


\subsubsection*{Dämpfung durch Gase und Dämpfe in der Atmosphäre}
Neben Regen und Schnee haben auch Gase, Dämpfe und andere Partikel in der Atmosphäre eine dämpfende Wirkung auf die elektromagnetischen Wellen. In den Frequenzen $f\leq1000\,\text{GHz}$ tragen hauptsächlich Sauerstoff und Wasserdampf, welche in der Ionosphäre ionisiert werden, zur Dämpfung $L_\mathrm{Gas}$ bei.\cite{ITU-RP.676-13}\newline
Durch die Addition der einzelnen Spektrallinien von Wasserdampf und Sauerstoff kann, in Kombination mit einem kleinen Faktor, die spezifische Dämpfung $\gamma_\mathrm{Gas}$ präzise für verschiedene Drücke, Temperatur und Luftfeuchtigkeit bestimmt werden.\cite{ITU-RP.676-13}\cite{Satellite_Communications_Systems}
\begin{equation}
    \gamma_\mathrm{Gas}=\gamma_\mathrm{Oxygen}+\gamma_\mathrm{Waterwapor}=0.1820\cdot f(N^{\glqq}_\mathrm{0xygen}(f)+N^{\glqq}_\mathrm{Waterwapor}(f))
\end{equation}
Dabei wird die Frequenz $f$ in GHz angegeben und die $N^"_\mathrm{0xygen}(f)$und$N^"_\mathrm{Waterwapor}(f)$ sind von der Frequenzabhängige Funktionen. Ihre Werte können für eine bestimmte Frequenz in einem Graphen nachgeschaut werden.\cite{ITU-RP.676-13}
Für eine Standard Atmosphäre, bedeutet Druck am Boden $p=1013\,\text{hPa}$, Temperatur am Boden $T=19\degree\text{C}$ und Wasserdampfkonzentration am Boden $c=7.5\,\frac{\text{g}}{\text{m}³}$, lässt sich die Dämpfung $L_\mathrm{Gas}$ in Abhängigkeit der Frequenz $f$ für verschiedene Elevationswinkel $\epsilon\geq10\degree$ in einem Graphen darstellen.\cite{Satellite_Communications_Systems} 
\begin{figure}[H]
    \centering
    \includegraphics[width=0.6\linewidth]{Bilder/Dämpfung durch Gase.png}
    \caption{Die Dämpfung durch Gase und Dämpfe $L_\mathrm{Gas}$ in Abhängigkeit der Frequenz für verschiedene Elevationswinkel $\epsilon$.\cite{Satellite_Communications_Systems}}
    \label{DämpfungdurchGaseGraph}
\end{figure}
Die Abbildung \ref{DämpfungdurchGaseGraph} zeigt die Dämpfung durch Gase und Dämpfe $L_\mathrm{Gas}$ in der Atmosphäre in Abhängigkeit von der Frequenz $f$ für verschiedene Elevationswinkel $\epsilon$ der Antenne.\newline
Für Frequenzen $f\leq15\,\text{GHz}$ ist die Dämpfung eher gering. Die maximale Dämpfung im dargestellten Frequenzbereich liegt bei $L_\mathrm{Gas}=2,7\,\text{dB}$ bei $f=22.24\,\text{GHz}$. Die Dämpfung an dieser Frequenz folgt aus dem Absorptionsband von Wasserdampf.\cite{Satellite_Communications_Systems}\newline
Die Mittenfrequenz $f_\mathrm{center}$ des Schmalbandtransponders auf Es'Hail-2 (QO-100) liegt bei $f_\mathrm{center}=10489.750\,\text{MHz}\approx 10.5\,\text{GHz}$. Der Elevationswinkel der Antenne an der Bodenstation ist in Gleichung \ref{eq:Elevation-Antenne} mit $\epsilon=27.36\degree\approx30\degree$ angeben. Aus dem Graph kann für die angegeben Werte eine Dämpfung von $L_\mathrm{Gas}\approx0.1\,\text{dB}$ entnommen werden.


\subsubsection*{Verluste durch nicht optimale Ausrichtung der Antennen}
Die Antenne des Schmalbandtransponders auf Es’Hail‑2 (QO-100) und die Antenne der Bodenstation am IAT sind nicht ideal aufeinander ausgerichtet, wie in Abbildung\ref{fig:Ausrichtungsverluste} dargestellt. 
\begin{figure}[H]
    \centering
    \includesvg[width=0.5\linewidth]{Bilder/Ausrichtunsgverluste}
    \caption{Veranschaulichung der Ausrichtungsverluste. Die optimale Ausrichtung ist in schwarz dargestellt, die tatsächliche Ausrichtung in rot.}
    \label{fig:Ausrichtungsverluste}
\end{figure}
Die Antenne des Transponders ist senkrecht auf den Äquator ausgerichtet, um eine möglichst gleichmäßige Abdeckung großer Teile der Erde zu gewährleisten. Dadurch ergibt sich jedoch ein Gewinnverlust $ G_\mathrm{T} $ gegenüber dem maximal möglichen Gewinn $ G_\mathrm{T,max} $ auf der Seite des Senders. Dieser Verlust im Gewinn führt wiederum zu einer Reduzierung der empfangenen Leistung $ P_\mathrm{R} $ im Vergleich zur maximal erreichbaren Empfangsleistung $ P_\mathrm{R,max} $. Diese Verluste werden auch Ausrichtungsverluste genannt (engl. Depointing Losses).\cite{Satellite_Communications_Systems}\newline
Die Ausrichtungsverluste lassen sich auf der Seite des Senders mit\cite{Satellite_Communications_Systems}
\begin{equation}
    L_\mathrm{\theta T}=12\cdot \left(\frac{\theta_\mathrm{T}}{\theta_\mathrm{3dB}}\right)^2
\end{equation}
beziehungsweise auf der Empfängerseite mit\cite{Satellite_Communications_Systems}
\begin{equation}
    L_\mathrm{\theta R}=12\cdot \left(\frac{\theta_\mathrm{R}}{\theta_\mathrm{3dB}}\right)^2
\end{equation}
bestimmen. Dabei sind $\theta_\mathrm{T}$,beziehungsweise $\theta_\mathrm{R}$ der Winkel der Fehlausrichtung und $\theta_\mathrm{3dB}$ die 3dB-Strahlbreite der Antenne des Senders.\cite{Satellite_Communications_Systems}\newline
Die $3\,\text{dB-Strahlbreite}$ der Hornantenne auf Es'Hail-2 (QO-100) beträgt $\theta_\mathrm{3dB}=17.4\degree$.\cite{EsHail2} Zur Bestimmung der Ausrichtungsverluste auf der Seite des Senders $L_\mathrm{\theta T}$ muss zunächst der Winkel der Fehlausrichtung $\theta_\mathrm{T}$ bestimmt werden. Die Hornantenne des Schmalbandtransponders auf Es'Hail-2 (QO-100) ist für gleichmäßige Abdeckung über die Erde auf den Äquator gerichtet. Die Ausrichtung von Es'Hail-2 (QO-100) und der Bodenstation sind in Abbildung \ref{eq:EntfernungEsHail2} dargestellt. Dabei entspricht der Winkel der Fehlausrichtung $\theta_\mathrm{T}$ dem Winkel $\beta$ zwischen dem Abstand $r+r_\mathrm{02}$ von Es'Hail-2 (QO-100) zur Höhe der Bodenstation über den Äquator und dem Abstand zur Bodenstation am IAT $D_\mathrm{SAT}$.
\begin{equation*}
    \theta_\mathrm{T}=\beta= \arccos\left(\frac{r+r_\mathrm{02}}{D_\mathrm{SAT}}\right)=\arccos\left(\frac{35790\,\text{km}+2548.22\,\text{km}}{38676\,\text{km}}\right)=7.58\degree
\end{equation*}
Damit beträgt der Ausrichtungsverlust auf der Seite des Senders
\begin{equation}
     L_\mathrm{\theta T}=12\cdot \left(\frac{\theta_\mathrm{T}}{\theta_\mathrm{3dB}}\right)^2=12\cdot\left(\frac{7.58\degree}{17.4\degree}\right)^2=5.23\,\text{dB}
     \label{eq:Senderseitige-Fehlausrichtung}
\end{equation}
Auf der Empfängerseite kann mit einem maximalen Winkel der Fehlausrichtung von $\theta_R=1\degree$ ausgegangen werden. So ergibt sich ein Ausrichtungsverlust auf der Seite des Empfängers von
\begin{equation}
     L_\mathrm{\theta R}=12\cdot \left(\frac{\theta_\mathrm{R}}{\theta_\mathrm{3dB}}\right)^2=12\cdot\left(\frac{1\degree}{17.4\degree}\right)^2=0.69\,\text{dB}
     \label{eq:Empfängerseitige-Fehlausrichtung}
\end{equation}


\subsubsection*{Weitere Einflüsse und Gesamtdämpfung}
Weiterhin kann auch die Dämpfung durch Regen- und Eiswolken, sowie Nebel berücksichtigt werden. Die spezifische Dämpfung $\gamma_\mathrm{REN}$ in $\text{dB/km}$ wird folgend bestimmt.\cite{Satellite_Communications_Systems}
\begin{equation}
    \gamma_\mathrm{REN}=K\cdot M
\end{equation}
Wobei $K=1.2\cdot10^{-3}\cdot f^{1.9}\,\text{in}\frac{\text{dB/km}}{\text{g/}\text{m}^3}$ ein approximierter Wert ist. Die Frequenz $f$ wird in $\text{GHz}$ angeben. Die Variable $M$ in $\text{g/}\text{m}^3$ ist die Wasserkonzentration in den Wolken oder Nebel.\cite{Satellite_Communications_Systems}\newline
Die Dämpfung durch Regen- und Eiswolken ist im Vergleich zu der Dämpfung durch Regen sehr gering $L_\mathrm{Wolken}\approx0.2\,\text{dB}$.\cite{Satellite_Communications_Systems}\newline
Für dichten Nebel, welcher im Raum Bremen schon häufiger auftritt, kann von einer Wasserkonzentration $M=0.5\,\text{g/}\text{m}^3$ ausgegangen werden\cite{Satellite_Communications_Systems}. Damit ergibt sich für eine Frequenz $f\approx10.5\,\text{GHz}$ eine spezifische Dämpfung $\gamma_\mathrm{REN}$ von
\begin{equation*}
    \gamma_\mathrm{REN}=K\cdot M=1.2\cdot 10^{-3}\cdot (10.5\,\text{GHz})^{1.9}\frac{\text{dB/km}}{\text{g/}\text{m}^3}\cdot0.5\,\text{g/}\text{m}^3=0.052\,\text{dB/km}
\end{equation*}
Was zusammen mit der in Gleichung effektiven Pfadlängen $D_\mathrm{Regen}=8.59\,\text{km}$ zu einer Dämpfung
\begin{equation}
     L_\mathrm{Nebel}=\gamma_\mathrm{REN}\cdot D_\mathrm{Regen}=0.052\,\text{dB/km}\cdot 8.59\,\text{km}=0.447\,\text{dB}
\end{equation}
führt. Die einzelnen in der Atmosphäre bestimmten Dämpfungen $L_\mathrm{Regen}$,$L_\mathrm{leichterRegen}$, $L_\mathrm{Gas}$, $L_\mathrm{Wolken}$ und $L_\mathrm{Nebel}$ können für die jeweilige Wetterbedienung zu einer gesamten Dämpfung $L_\mathrm{At}$ zusammengefasst werden.\newline
Bei der Wetterbedingung klarer Himmel (engl. clear Sky) kann die Dämpfung durch Regen und Wolken vernachlässigt werden. Die einzigen auftretenden Dämpfungen in der Atmosphäre entstehen durch Gase und Dämpfe in der Atmosphäre und gegebenenfalls durch Nebel auf dem Boden.\newline
\begin{equation}
    L_\mathrm{ATklarerHimmel}=L_\mathrm{Gas,\,dB}+L_\mathrm{Nebel,\,dB}=0.1\,\text{dB}+0.447\,\text{dB}=0.547\,\text{dB}
    \label{eq:Dämpfung-in-der-Atmosphäre-klarer-Himmel}
\end{equation}
In der Bedingung leichter Regen wird die Dämpfung durch Niederschlagsraten $R_\mathrm{p}$ in $\text{mm/h}$ berücksichtigt, welche zu $p=5\,\%$ der Zeit den Jahresdurchschnitt $(\text{mm/h})$ überschreiten. Ebenfalls werden auch die Verluste durch Gase und Dämpfe in der Atmosphäre, durch Wolken und gegebenenfalls durch Nebel berücksichtigt.
\begin{equation}
\begin{split}
    L_\mathrm{ATleichterRegen} & =L_\mathrm{leichterRegen\,dB}+L_\mathrm{Gas,\,dB}+L_\mathrm{Wolken,\,dB}+L_\mathrm{Nebel,\,dB}\\
    & = 0.2\,\text{dB}+0.1\,\text{dB}+0.2\,\text{dB}+0.447\,\text{dB} = 0.947\,\text{dB}
\end{split} 
\label{eq:Dämpfung-in-der-Atmosphäre-leichter-Regen}
\end{equation}
In der Bedingung Regen dominiert die Dämpfung durch starke Niederschläge. Es wird die Dämpfung durch Niederschlagsraten $R_\mathrm{p}$ in $\text{mm/h}$ berücksichtigt, welche den Jahresdurchschnitt $(\text{mm/h})$ zu $p=0.01\,\%$ der Zeit überschreiten. Auch werden wieder die Dämpfungen durch Gase und Dämpfe in der Atmosphäre, durch die Wolken und gegebenenfalls auch wieder durch Nebel berücksichtigt.
\begin{equation}
\begin{split}
    L_\mathrm{ATRegen} & =L_\mathrm{Regen,\,dB}+L_\mathrm{Gas,\,dB}+L_\mathrm{Wolken,\,dB}+L_\mathrm{Nebel,\,dB}\\
    & = 8.86\,\text{dB}+0.1\,\text{dB}+0.2\,\text{dB}+0.447\,\text{dB} = 9.61\,\text{dB}
\end{split}
\label{eq:Dämpfung-in-der-Atmosphäre-Regen}
\end{equation}

\subsection{Empfänger - Bodenstation am IAT}
Um die Signale des Schmalbandtransponders von Es'Hail-2 (QO-100) empfangen und weiterverarbeiten zu können, benötigt es ein geeignetes Empfangssystem. Das Empfangssystem muss mehrere Voraussetzung erfüllen.
\begin{enumerate}
    \item Das Empfangssystem sollte technisch dazu fähig sein den Downlink von Es'Hail-2 (QO-100) im X-Band empfangen und verarbeiten zu können.
    \item Das Empfangssystem sollte eine großen Dynamikbereich besitzen. Mit einem hohen Dynamikbereich kann das Empfangssystem Signale mit sehr kleinen Pegel in der Nähe des Rauschflurs, als auch Signale mit sehr großem Pegel ohne zu übersteuern verarbeiten.
    \item Die Bandbreite des Empfangssystem sollte mindestens $B_\mathrm{min}=2.7\,\text{kHz}$ breit sein, um die maximal zulässige Einzelsignalbandbreite gemäß des Bandplans vom Schmalbandtransponders auf Es'Hail-2 (Q0-100) empfangen zu können.
    \item Das Empfangssystem sollte ein möglichst hohes $SNR$ an seinem Eingang und Ausgang aufweisen. So können Fehler in der Demodulation der Signale von Es'Hail-2 (QO-100) gering gehalten werden. Ein hohes $SNR$ am Ausgang erfordert einen möglichst rauscharmen RF-Bereich des Empfangssystem. Daher sollte die möglichen Rauschquellen im RF-Bereich des Empfangssystem und die Rauschzahl $F$ so gering wie möglich gehalten werden.
    \item Gleichzeitig muss das Empfangssystem eine gewissen Verstärkung $G_\mathrm{sys}$ aufweisen. Mit einer passenden Verstärkung $G_\mathrm{sys}$ kann der schwache Pegel der Signale von Es'Hail-2 (QO-100) so angehoben werden, dass es vom SDR optimal verarbeiten werden kann.
    \item Das Empfangssystem sollte auch bei leichten Regenschauern, spricht Niederschlagsmengen $\text{mm/h}$, welche zu $5\,\%$ der Zeit die Durchschnittsniederschlagsmenge $\text{mm/h}$ eines Jahres überschreiten, den Downlink von Es'Hail-2 (QO-100) aufrechterhalten können.
    \item Die Verluste im RF-Fronted $L_\mathrm{sys}$ des Empfangssytems sollte so gering wie möglich gehalten werden. 
    \item Die neu zu beschaffenden Komponenten sollte mit den vorhandenen Komponenten der Bodenstation vom IAT kompatibel sein und den Standards in der Telekommunikation entsprechen.
    \item Für das Software Defined Radio (SDR) muss eine geeignete SDR-Software mithilfe von GNU-Radio Companion erstellt werden.
\end{enumerate}
Damit das Empfangssystem den Downlink im X-Band empfangen kann, braucht es hierfür eine geeignete Antenne. Am Flughafen Standort der Hochschule Bremen ist bereits eine Parabolantenne, wie sie für Satellitenfernsehen üblich ist, vorhanden. Diese müsste hinsichtlich ihrer Eignung für den Empfang des Downlinks von Es'Hail-2 (QO-100) im X-Band überprüft werden.\newline
Die Größe des Dynamikbereich wird vom ADC des SDR bestimmt. In der Bodenstation vorhanden ist ein USRP X310 SDR von National Instruments (NI). Auch wird die maximale Bandbreite $B_\mathrm{max}$ mit vom SDR bestimmt. Das SDR muss ebenfalls auf die Eignung, Hardware und Softwareunterstützung, für den Einsatz im Empfangssystem für den Downlink von Es'Hail-2 (QO-100) überprüft werden.\newline
Mithilfe der Gleichungen \ref{eq:Gesamtrauschzahl-Kaskade} und \ref{eq:Gesamt-äquivalente-Rauschtemperatur-Kaskade} lassen sich mehrere Vorschriften und Bedienungen für den Aufbau der Empfangskette herleiten.
\begin{enumerate}
    \item Das erste Zweitor nach der Antenne hat den größten Einfluss auf die Gesamtrauschzahl $F_\mathrm{sys}$ der Empfangskette. Daher sollte das erste Zweitor ein niedriges Eigenrauschen und damit verbundene niedrige Rauschzahl $F_\mathrm{1}$, sowie eine hohe Verstärkung $G_\mathrm{1}$ besitzen, da die folgenden Zweitore mit dem Gewinn $G_\mathrm{1}$ des ersten Zweitores gewichtet werden.\cite{HEUERMANN_2018}
    \item Die folgenden Zweitore haben bei einem entsprechend großen Gewinn $G_\mathrm{1}$ nur noch geringe Auswirkung auf die Gesamtrauschzahl $F_\mathrm{sys}$ oder der äquivalenten Rauschtemperatur des Gesamtsystems $T_\mathrm{e,sys}$.\cite{HEUERMANN_2018}
    \item Wie in Gleichung \ref{eq:Rauschzahl-passives-Zweitor} gezeigt, ist der Verlust $L$ passiver Zweitore gleich ihrer Rauschzahl $F$. Daher sollten Koaxialleitung mit niedrigen Verlust verwendet und Lange Wege reduziert werden. Auch sollte auf Dämpfungsglieder verzichtet werden.
    \item Da jedes Zweitor zwangsläufig das $SNR$ verschlechtert, sollte die Anzahl der Zweitore im RF-Frontend auf das mögliche Minimum reduziert werden.
\end{enumerate}
Mithilfe dieser Vorschriften und Bedienungen kann ein geeigneter RF-Bereich der Empfangssystems geplant werden.
\begin{figure}[H]
    \centering
    \includesvg[width=0.75\linewidth]{Bilder/Empfangskette}
    \caption{Blockschaltbild vom RF-Bereich des Empfangssystems}
    \label{fig:geplante-Empfangskette}
\end{figure}
In der Abbildung \ref{fig:geplante-Empfangskette} ist ein Blockschaltbild vom geplanten RF-Bereich des Empfangssystems zusehen. Als erstes Zweitor wird eine Koaxialleitung mit sehr geringen Verlusten eingesetzt. Die Koaxialleitung ist notwendig, da das zweite Zweitor, ein LNC, nicht direkt an die Antenne angeschlossen werden kann. Der LNC ist rauscharmer Mischer mit integrierten Verstärker. Montiert wird der LNC in der Nähe der Antenne, damit die Länge der ersten Koaxialleitung gering gehalten werden kann. Das dritte Zweitor ist wieder eine Koaxialleitung, welche die lange Strecke vom LNC auf dem Dach bis zum Serverschrank, wo sich die Fernspeiseweiche befindet, überbrückt wird. Über die Fernspeiseweiche wird der LNC mit der nötigen Betriebsspannung $V_\mathrm{cc}$ und ggf. einem $10\,\text{MHz}$ Referenzsignal versorgt. Das fünfte Zweitor ist ebenfalls wieder eine Koaxialleitung, mit welcher die Fernspeiseweiche mit dem Patchfeld verbunden wird. Über das Patchfeld, welches das sechste Zweitor ist, kann eine einfache Verkablung im Serverschrank vorgenommen werden. Vom Patchfeld gibt es eine weitere Koaxialleitung zur Schaltmatrix (engl. RF-Switch). Die Schaltmatrix hat mehrere Eingänge und kann einen beliebigen Eingang auf seinen Ausgang durchschalten. So könnten mit einem SDR mehrere verschiedene Empfangssysteme betrieben werden. Vom Ausgang der Schaltmatrix führt anschließend noch eine Koaxialleitung zum SDR, welches das letzte Zweitor im RF-Bereich vom Empfangssystem ist. Mit dem SDR werden die Datensignale von Es'Hail-2 (Q0-100) in brauchbare Daten und Informationen umgewandelt und anschließend an einen PC weitergegeben.\newline
Als Systemimpedanz werden $50\,\Omega$ gewählt. Alle verwendeten Komponenten im RF-Frontend sind auf dieses Systemimpedanz angepasst. Die $50\,\Omega$ Impedanz hat sich als Standard in der Telekommunikationstechnik etabliert. Begründet werden kann das mit der Impedanz von Koaxialleitungen.\cite{Altium-Impedanz}
\begin{figure}[H]
    \centering
    \includegraphics[width=0.75\linewidth]{Bilder/Impedanzkompromiss.png}
    \caption{Darstellung der Verluste (blau) und der maximalen Leistungsübertragung (rot)  von luftgefüllten Koaxialleitungen ($\varepsilon_\mathrm{r}=1$) über die Impedanz\cite{Altium-Impedanz}}
    \label{fig:Kompromiss-Impedanz}
\end{figure}
Eine Koaxialleitung soll drei Kriterien erfüllen. Sie soll möglichst geringe Verluste aufweisen. Das Dielektrikum soll dabei hohen Spannungen und damit verbunden hohen Feldstärke standhalten und die Koaxialleitung soll in der Lage sein, hohe Leistungen zu übertragen.\cite{Altium-Impedanz}\newline 
In der Grafik \ref{fig:Kompromiss-Impedanz} sind zwei Graphen zu sehen. Dargestellt sind die normierten Verluste (blau) und die normierte maximale übertragbare Leistung (rot) von luftgefüllten Koaxialleitungen über deren Impedanz. Die höchste maximal übertragbare Leistung wird bei einer Impedanz von $30\,\Omega$ erreicht. Jedoch sind dann auch die Verluste vergleichsweise hoch. Am niedrigsten sind die Verluste bei einer Impedanz von $77\,\Omega$, jedoch ist hier die maximale übertragbare Leistung eher gering. Mit dieser Gegebenheit kann möglicherweise die $75\,\Omega$ Impedanz im Rundfunkbereich erklärt werden. Im TE10-Modus erreicht das elektrische Feld sein Maximum bei einer Impedanz von $60\,\Omega$.\cite{Altium-Impedanz}\newline
Die $50\,\Omega$ Impedanz ist der beste Kompromiss aus minimaler Dämpfung, maximaler Leistungsübertragung und maximaler Feldstärke. Aus diesem Grund hat sie sich als Standard Referenz Impedanz etabliert.\cite{Altium-Impedanz}

\subsubsection*{Antenne und Antennenfeed}
Bei der verwendeten Antenne handelt es sich um eine Parabolantenne, wie sie z.B. für Satellitenfernsehen üblich ist. Die Parabolantenne gehört zu der Gruppe der Reflektorantennen, welche eine Kombination aus einem Reflektor und einem Antennenfeed sind.\cite{Balanis_2005}\newline
Der Reflektor hat die Aufgabe, einfallende elektromagnetische Wellen in eine bestimmte Richtung zu reflektieren und damit die abgestrahlte Energie gezielt zu bündeln. Anders als bei flachen Reflektoren werden bei Parabolspiegeln die auftretenden EM-Wellen durch die parabolische Krümmung des Reflektors in einem gemeinsamen Punkt fokussiert. Dieser Punkt wird Brennpunkt (engl. focal point) genannt.\cite{Balanis_2005}\newline
Im Brennpunkt der Parabolantenne befindet sich der Antennenfeed. Der Antennenfeed ist nichts anderes als eine gewöhnliche Antenne, welche die fokussierten EM-Wellen empfängt oder in Richtung des Reflektors abstrahlt. Hier wird eine Hornantenne als Antennenfeed verwendet. Diese bietet gegenüber einer Dipolantenne einen höheren Richtfaktor $D$ und damit verbunden auch einen höheren Gewinn $G$.\cite{Balanis_2005}\newline
An der Hochschule ist bereits eine Parabolantenne vorhanden. Diese ist im Zuge eines Sturmes umgeknickt und soll nun wieder verwendet werden. 
\begin{figure}[H]
    \centering
    \includegraphics[width=0.75\linewidth]{Bilder/Parabolantenne.jpg}
    \caption{Vorhandene Parabolantenne -> Bild noch austauschen}
    \label{fig:vorhandene-Parabolantenne}
\end{figure}
Die Bild \ref{fig:vorhandene-Parabolantenne} zeigt die auf dem Dach der Hochschule vorhandene Parabolantenne. Beim Parabolspiegel handelt es sich um eine Kathrein CAS 90 HD ohne Logo. \newline
Dieser Parabolspiegel hat einen Durchmesser von $d_\mathrm{Antenne}=0.9\,\text{m}$ und ist für den Betrieb im X-Band und unteres Ku-Band, genauer zwischen $10.7\,\text{GHz}$ und $12.75\,\text{GHz}$, vorgesehen.\cite{KathreinCAS90HD}\newline
Aus dem Datenblatt lässt sich für den maximalen Gewinn $G_\mathrm{R,max}$ folgende Werte entnehmen.\cite{KathreinCAS90HD}
\begin{equation}
    G_\mathrm{R,max}=
    \begin{cases}
        38.6\,\mathrm{dBi}&,10.7\,\text{GHz}\leq f\leq 11.7\,\text{GHz}\\
        39.2\,\mathrm{dBi}&,11.7\,\text{GHz}\leq f\leq 12.5\,\text{GHz}\\
        39.6\,\mathrm{dBi}&,12.5\,\text{GHz}\leq f\leq 12.75\,\text{GHz}\\
    \end{cases}
    \label{eq:Gewinn-der-Empfangsantenne}
\end{equation}
Der bisherigen LNB zum Empfang von Satellitenfernsehen wird gegen neuen Feed zum Empfang von Es'Hail-2 ausgetauscht.\newline 
Als Antennenfeed wird ein Triplebandfeed von der Firma BaMaTech eingesetzt. Dieser Antennenfeed ist speziell für den Einsatz an Es'Hail-2 (QO-100) entwickelt und kann in drei verschiedenen Frequenzbänder eingesetzt werden. Daher kommt auch der folgt auch sein Name.\cite{amatech-feed}
\begin{figure}[H]
    \centering
    \includegraphics[width=0.75\linewidth]{Bilder/Triplebandfeed.jpg}
    \caption{Das Bild zeigt den ausgewählten Triplebandfeed von BaMaTech. Die weiße Halterung am Antennenfeed dient zur Montage des Antennenfeeds in einem Halter. Diese wird durch eine eigene Halterung ausgetauscht.}
    \label{fig:verwendeter Triplebandfeed}
\end{figure}
Das Abbildung \ref{fig:verwendeter Triplebandfeed} zeigt den Triplebandfeed von der Firma BaMaTech. Im Grunde handelt es sich dabei um eine Hornantenne mit drei Anschlüssen für die einzelnen Frequenzbänder. Verwendet werden kann der Triplebandfeed im S-Band bei $2.4\,\text{GHz}$, im C-Band bei $5.6\,\text{GHz}$ und im X-Band bei $10.5\,\text{GHz}$.\cite{amatech-feed}\newline
Interessant ist in dieser Arbeit nur der Anschluss für das X-Band, wo sich der Downlink von Es'Hail-2 (QO-100) befindet. Allerdings kann somit der Feed auch für einen späteren Uplink zu Es'Hail-2 (QO-100), welcher im S-Band liegt, verwendet werden.\newline 
Angeschlossen werden kann der Triplebandfeed mittels SMA-Steckverbindungen.\cite{amatech-feed}\newline
Um die Tauglichkeit der Parabolantenne und des Feed für die Anwendung an Es'Hail-2 zu überprüfen kann zu einem die Effizienz $\eta_\mathrm{ANT}$ der effektive Antennenfläche $A_\mathrm{E}$ bestimmt werden. Auch kann die Reflexion $S11$ des Antennenfeeds gemessen und anschließend das $VSWR$ werden.\newline
Die effektive Antennenfläche $A_\mathrm{E}$ der Parabolantenne kann über die Gleichung \ref{eq:effektive-Antennenfläche} bestimmt werden. Benötigt wird dafür der Gewinn $G_\mathrm{R,max}$ der Parabolantenne und die Wellenlänge $\lambda$ der betrachteten Frequenz $f$. Von Interesse ist die Frequenz des Downlink von Es'Hail-2 (QO-100). Die Mittenfrequenz beträgt $f_\mathrm{center}=10489.750\,\text{MHz}\approx 10.5\,\text{GHz}$, was zu einer Wellenlänge $\lambda_\mathrm{center}=0.0286\,\text{m}$ führt. Der Gewinn der Antenne kann aus Gleichung \ref{eq:Gewinn-der-Empfangsantenne} mit $G_\mathrm{R,max}=38.6\,\text{dBi}$ angenommen werden.
\begin{equation}
    A_\mathrm{E}=\frac{G_\mathrm{R,max}\cdot\lambda_\mathrm{center}^2}{4\cdot\pi}=\frac{38.6\,\text{dBi}\cdot(0.0286\,\text{m})^2}{4\cdot\pi}=0.472\,\text{m}^2
    \label{eq:effektive-Antennenfläche-der-Empfangsantenne}
\end{equation}
Um die Effizienz der effektiven Antennenfläche $A_\mathrm{E}$ zu bestimmen, muss zunächst die physikalische Fläche $A_\mathrm{phy}$ der Parabolantenne bestimmt werden. Diese kann mithilfe der Mantelfläche eines Rotationsparaboloid bestimmt werden.\newline 
Die Mantelfläche des Rotationsparaboloid kann mit der Höhe $h$ des Rotationsparaboloid und dessen Radius $r=\frac{d}{2}$ bestimmt werden. Die Höhe von Boden bis zur Kante der Parabolantenne beträgt $h=0.1\,\text{m}$.
\begin{equation}
\begin{split}
   A_\mathrm{phy}
   &=\frac{\pi\cdot r}{6h^2}\cdot\left((r^2+4h^2)^{\frac{3}{2}}-r^3\right)\\
   &=\frac{\pi\cdot 0.45\,\text{m}}{6\cdot(0.1\,\text{m})^2}\cdot\left(((0.45\,\text{m})^2+4\cdot(0.1\,\text{m})^2)^{\frac{3}{2}}-(0.45\,\text{m})^3\right)\\
   &=0.667\,\text{m}^2 
\end{split}
\label{eq:physikalische-Fläche-der-Empfangsantenne}
\end{equation}
Zusammen mit der effektiven Antennenfläche $A_\mathrm{E}$ in \ref{eq:effektive-Antennenfläche-der-Empfangsantenne} und der physikalischen Antennenfläche $A_\mathrm{phy}$ in \ref{eq:physikalische-Fläche-der-Empfangsantenne} kann die Effizienz $\eta_\mathrm{ANT}$ der Parabolantenne mit Gleichung \ref{eq:Effizienz-effektive-Antennenfläche} bestimmt werden.
\begin{equation}
    \eta_\mathrm{ANT}=\frac{A_\mathrm{E}}{A_\mathrm{phy}}=\frac{0.472\,\text{m}^2}{0.667\,\text{m}^2}=0.708=70.8\,\%
    \label{eq:Effizienz-der-Antennenfläche-der-Empfangsantenne}
\end{equation}
Mit $\eta_\mathrm{Ant}= 70.8\,\%$ liegt die Effizienz der effektiven Antennenfläche $A_\mathrm{E}$ im typischen Bereich $(60\,\%-80\,\%)$ für Parabolantennen. Damit ist diese Parabolantenne für den Empfang des Downlinks von Es'Hail-2 (QO-100) geeignet.\newline
Die Reflexion $S11$ des Antennenfeeds kann mithilfe eines Vektor Netzwerk Analysator (VNA) gemessen werden. Verwendet wird hierfür ein VNA von Rohde und Schwarz, genauer der R\&S ZVL. Vor der Messung wird dieser entsprechend in einem Frequenzbereich von $2\,\text{GHz}$ bis $12\,\text{GHz}$ für Offen (Open), Kurzschluss (Short) und einer Last (Load) $50\,\Omega$ kalibriert. Beim Kalibrierkit handelt es sich um ein 01 BN 533828 vom Hersteller Spinner. Der Messaufbau ist in der Abbildung \ref{fig:Messaufbau-für-S11-des-Antennenfeeds} dargestellt.
\begin{figure}[H]
    \centering
    \includesvg[width=0.5\linewidth]{Bilder/Messaufbau S11}
    \caption{Messaufbau zur Ermittlung der Reflexion $S11$ des Antennenfeeds}
    \label{fig:Messaufbau-für-S11-des-Antennenfeeds}
\end{figure}
Für die Reflexion im X-Band wird der Frequenzbereich von $9\,\text{GHz}$ bis $12\,\text{GHz}$ mit $201$ Messpunkten betrachtet. Die Leistung wird auf $0\,\text{dBm}$ gesetzt, da es sich um ein passives Element handelt. Die Auflösung wird auf $5\,\text{dB/div}$ und das Referenzlevel auf $0\,\text{dB}$ eingestellt.
\begin{figure}[H]
    \centering
    \includegraphics[width=0.75\linewidth]{Bilder/Antennenfeed 10_5_GHz.PNG}
    \caption{Gemessene Reflexion $S11$ des Antennenfeeds im X-Band}
    \label{fig:S11-Antennenfeed-X-Band}
\end{figure}
Der Graph in Abbildung \ref{fig:S11-Antennenfeed-X-Band} zeigt die gemessene Reflexion $S11$ des Antennenfeeds zwischen $9\,\text{GHz}$ und $12\,\text{GHz}$. Der Marker 1 befindet sich bei $10.498\,\text{GHz}$, was nahe der Mittenfrequenz des Downlins von $f_\mathrm{center}=10489.750\,\text{MHz}$ ist. Gemessen wird an der Stelle eine Reflexion von $S11=-24.1\,\text{dB}$. Mithilfe der Reflexion kann das $VSWR$ bestimmt werden. Das $VSWR$ ist das Stehwellenverhältnis ist ein Maß für die Impedanzanpassung zwischen der Antenne und der Koaxialleitung (Quelle) an. Es entspricht dem Verhältnis der maximalen zur minimalen Spannung einer stehenden Welle und gibt damit an, wie viel Leistung von der Antenne zurück zur Quelle reflektiert wird. Bei $VSWR=1$ wird keine Leistung zurück zur Quelle reflektiert und die gesamte Leistung wird von der Antenne abgestrahlt. Das $VSWR$ kann mithilfe des Reflexionsfaktors $\Gamma=10^{\frac{S11}{20}}$bestimmt werden.\newline
\begin{equation}
    VSWR=\frac{U_\mathrm{max}}{U_\mathrm{min}}=\frac{1+|\Gamma|}{1-|\Gamma|}=\frac{1+|10^{\frac{-24.1\,\text{dB}}{20}}|}{1-|10^{\frac{-24.1\,\text{dB}}{20}}|}=1.13
    \label{eq:VSWR-Empfangsantenne-X-Band}
\end{equation}
Mit einem $VSWR=1.13$ ist die Antenne sehr gut für den Frequenzbereich des Downlinks von Es'Hail-2 (QO-100) angepasst und ist somit für diese Anwendung geeignet.\newline
Auch kann die Reflexion des S-Band gemessen werden, falls in der Zukunft noch ein Uplink zu Es'Hail-2 (QO-100) an der Bodenstation eingerichtet werden soll. Der Messaufbau bleibt der gleiche. Betrachtet wird für die Reflexion $S11$ der Bereich von $2\,\text{GHz}$ bis $3\,\text{GHz}$. Ebenfalls mit $201$ Messpunkten und einer Leistung von $0\,\text{dBm}$. Die Auflösung wird auf $5\,\text{dB/div}$ und das Referenzlevel auf $0\,\text{dB}$ eingestellt.
\begin{figure}[H]
    \centering
    \includegraphics[width=0.75\linewidth]{Bilder/Antennenfeed 2_4_GHz_Band.PNG}
    \caption{Gemessene Reflexion $S11$ zwischen $2\,\text{GHz}$ und $3\,\text{GHz}$}
    \label{fig:S11-Antennenfeed-S-Band}
\end{figure}
Die Abbildung \ref{fig:S11-Antennenfeed-S-Band} zeigt die gemessene Reflexion $S11$ des Antennenfeeds zwischen $2\,\text{GHz}$ und $3\,\text{GHz}$. Der Marker befindet sich bei $2.4\,\text{GHz}$, was nah an der Mittenfrequenz $f_\mathrm{center}=2400.250\,\text{MHz}$ des Uplinks zu Es'Hail-2 (QO-100) ist. An diesem Punkt wird eine Reflexion von $S11=-24.9\,\text{dB}$ gemessen. Mithilfe der Gleichung \ref{eq:VSWR-Empfangsantenne-X-Band} kann das $VSWR$ bestimmt werden.
\begin{equation}
    VSWR=\frac{U_\mathrm{max}}{U_\mathrm{min}}=\frac{1+|\Gamma|}{1-|\Gamma|}=\frac{1+|10^{\frac{-24.9\,\text{dB}}{20}}|}{1-|10^{\frac{-24.9\,\text{dB}}{20}}|}=1.12
    \label{eq:VSWR-Empfangsantenne-S-Band}
\end{equation}
Das Ergebnis in Gleichung \ref{eq:VSWR-Empfangsantenne-S-Band} zeigt, dass die Antenne für Frequenzbereich des Uplinks zu Es'Hail-2 (QO-100) angepasst ist. Somit kann der Antennenfeed auch für den Uplink verwendet werden.\newline
\subsubsection*{Low Noise Converter und Bias-Tee}
Der von der Antenne empfangene Frequenzbereich wird über eine Koaxialleitung an das Empfangssystem weitergegeben. Jedoch ist ist die Leistung der gewünschte Signale zu gering und die Frequenz des Signals mit $f\approx10.5\,\text{GHz}$ viel zu hoch um direkt vom SDR verarbeitet werden zu können. Zuvor müssen die Signale also verstärkt und in niedriges Frequenzband umgesetzt werden.\newline
Für diesen Zweck wird rauscharmer Signalumsetzer (engl. low Noise Converter) LNC der Firma Kuhne Electronic. Der MKU LNC 10 QO-100 ist für die Anwendung im $3\,\text{cm-Amateurfunkband}$, speziell für die Anwendung an Es'Hail-2 (QO-100), vorgesehen.\cite{kuhne-downconverter}\newline
Dieser LNC kombiniert einen rauscharmen Verstärker (engl. Low Noise Amplifier) LNA und einen Mischer in einem Gerät und hat mehrere Vorteile gegenüber einem diskreten Aufbau aus Mischer und LNA. Gegenüber dem diskreten Aufbau benötigt der LNC weniger Platz und nur eine Stromversorgung gegenüber zwei beim diskreten Aufbau. Somit werden auch weniger Komponenten und Leitungen benötigt, was die Anzahl der Rauschquellen reduziert.\newline
\begin{figure}[H]
    \centering
    \includegraphics[width=0.5\linewidth]{Bilder/MKU LNC10 QO 100.png}
    \caption{MKU LNC 10 QO-100 von Kuhne Electronic\cite{kuhne-downconverter}}
    \label{MKU LNC 10 QO-100}
\end{figure}
Der MKU LNC 10 QO-100 verfügt über einen HF-Eingang und einen ZF-Ausgang, wobei am Eingang eine SMA-Buchse und am Ausgang eine N-Buchse verbaut ist. Er unterstützt vier verschiedene Frequenzbänder zwischen wischen $10.35\,\text{GHz}$ und $10.5\,\text{GHz}$, welche über die Frequenz des lokalen Oszillator $f_\mathrm{LO}$ ausgewählt werden können.\cite{kuhne-downconverter}\newline
Die Frequenz des lokalen Oszillator $f_\mathrm{ZF}$ kann mithilfe der Betriebsspannung $V_\mathrm{CC}$ ausgewählt werden. Die Betriebsspannung $V_\mathrm{CC}$ wird über die ZF-Buchse eingespeist.\cite{kuhne-bias-tee}\newline
Mit der Veränderung der Frequenz des lokalen Oszillator $f_\mathrm{LO}$ wird auch die Frequenz des ZF-Signals $f_\mathrm{ZF}$ am ZF-Ausgang verändert.\cite{kuhne-downconverter}
\begin{itemize}
    \item Q0-100 SSB-Betrieb: Bei einer Betriebsspannung $12\,\text{V}\leq V_\mathrm{CC} \leq 17\,\text{V}$ befindet sich der LNC im Einseitenbandbetrieb. Die Frequenz des HF-Bereiches $f_\mathrm{HF}$ befindet sich dann zwischen $10489\,\text{MHz}$ und $10490\,\text{MHz}$ und die Frequenz des lokalen Oszillator liegt bei $f_\mathrm{LO}=10056\,\text{MHz}$.\cite{kuhne-downconverter} Die Frequenz des ZF-Signals $f_\mathrm{ZF}$ kann mit Gleichung \ref{eq:Frequenz-des-Mischproduktes} bestimmt werden.
    \begin{equation}
    \label{eq:fZF-QO100-SSB}
        \begin{split}
            &f_\mathrm{ZF,min}=f_\mathrm{HF,min}-f_\mathrm{LO}=10489\,\text{MHz}-10056\,\text{MHz}=433\,\text{MHz}\\  
            &f_\mathrm{ZF,max}=f_\mathrm{HF,max}-f_\mathrm{LO}=10490\,\text{MHz}-10056\,\text{MHz}=434\,\text{MHz}
        \end{split}
    \end{equation}
    \item QO-100 ATV-Betrieb: Um den QO-100 Betriebsmodus verwenden zu können muss die Betriebsspannung $V_\mathrm{CC}$ zwischen $18\,\text{V}$ und $23\,\text{V}$ DC liegen. Die Frequenz des HF-Bereich $f_\mathrm{HF}$befindet sich dann zwischen $10490\,\text{MHz}$ und $10500\,\text{MHz}$. Der lokale Oszillator hat damit dann eine Frequenz von $f_\mathrm{LO}=9240\,\text{MHz}$.\cite{kuhne-downconverter} 
    \begin{equation}
    \label{eq:fZF-QO100-ATV}
        \begin{split}
            &f_\mathrm{ZF,min}=f_\mathrm{HF,min}-f_\mathrm{LO}=10490\,\text{MHz}-9240\,\text{MHz}=1250\,\text{MHz}\\  
            &f_\mathrm{ZF,max}=f_\mathrm{HF,max}-f_\mathrm{LO}=10500\,\text{MHz}-9240\,\text{MHz}=1260\,\text{MHz}
        \end{split}
    \end{equation}
    \item SSB-Betrieb: Bei einer Betriebsspannung $V_\mathrm{CC}$ zwischen $9\,\text{V}$ und $11\,\text{V}$ DC befindet sich der LNC im Einseitenbandbetrieb im HF-Bereich zwischen $10368\,\text{MHz}$ und $10370\,\text{MHz}$. Der lokale Oszillator hat in diesem Betriebsmodus eine Frequenz von $f_\mathrm{LO}=9936\,\text{MHz}$.\cite{kuhne-downconverter}
    \begin{equation}
    \label{eq:fZF-SSB1}
        \begin{split}
            &f_\mathrm{ZF,min}=f_\mathrm{HF,min}-f_\mathrm{LO}=10368\,\text{MHz}-9936\,\text{MHz}=433\,\text{MHz}\\  
            &f_\mathrm{ZF,max}=f_\mathrm{HF,max}-f_\mathrm{LO}=10370\,\text{MHz}-9936\,\text{MHz}=434\,\text{MHz}
        \end{split}
    \end{equation}    
    \item SSB-Betrieb: Das letzte unterstütze Frequenzband von $10450\,\text{MHz}$ bis $10452\,\text{MHz}$ im HF-Bereich, kann mit einer Betriebsspannung $V_\mathrm{CC}$ zwischen $24\,\text{V}$ und $36\,\text{V}$ DC ausgewählt werden. Der LNC befindet sich wieder in einem Einseitenbandbetrieb. Der lokale Oszillator schwingt dabei mit einer Frequenz $f_\mathrm{LO}=10016\,\text{MHz}$.\cite{kuhne-downconverter} 
    \begin{equation}
    \label{eq:fZF-SSB1}
        \begin{split}
            &f_\mathrm{ZF,min}=f_\mathrm{HF,min}-f_\mathrm{LO}=10450\text{MHz}-10016\,\text{MHz}=434\,\text{MHz}\\  
            &f_\mathrm{ZF,max}=f_\mathrm{HF,max}-f_\mathrm{LO}=10452\,\text{MHz}-10016\,\text{MHz}=436\,\text{MHz}
        \end{split}
    \end{equation}  
\end{itemize}
Der LNC hat eine typische Verstärkung von $G_\mathrm{LNC}=55\,\text{dB}$ und wird mit einer Rauschzahl von $F_\mathrm{LNC}=1.7\,\text{dB}=1.48$ bei $T_\mathrm{0}=291\,\text{K}$ angegeben, was für eine Kombination aus Mischer und Verstärker gute Werte sind.\cite{kuhne-downconverter}\newline
Mit Gleichung \ref{eq:Rauschzahl-aus-Te-und-T0} lässt sich die äquivalente Rauschtemperatur $T_\mathrm{e,LNC}$ des LNC bestimmen.
\begin{equation}
    T_\mathrm{eLNC}=(F_\mathrm{LNC}-1)T_\mathrm{0}=(1.48-1)290\,\text{K}=139.2\,\text{K}
    \label{eq:TE-LNC}
\end{equation}
Für den Betrieb des LNC ist nur die Einspeisung der Betriebsspannung $V_\mathrm{CC}$ notwendig. Die Einspeisung eines $10\,\text{MHz}$ Referenzsignals ist nicht erforderlich, kann aber zu einer besseren Frequenzstabilität beitragen.\cite{kuhne-downconverter}\newline
Die Betriebsspannung $V_\mathrm{CC}$ und ggf. das $10\,\text{MHz}$ Referenzsignal werden mithilfe einer Fernspeiseweiche (engl. Bias-Tee) in die Leitung am ZF-Ausgang eingespeist. Verwendet wird hier für die Fernspeiseweiche KU BT 10 REF. Diese stammt ebenfalls von der Firma Kuhne Electronic und ist für den MKU LNC 10 QO-100 vorgesehen.\newline
\begin{figure}[H]
    \centering
    \includegraphics[width=0.5\linewidth]{Bilder/KU BT 10 REF.png}
    \caption{Fernspeiseweiche KU BT 10 REF \cite{kuhne-bias-tee}}
    \label{fig:KU-BT-10-REF}
\end{figure}
Die Fernspeiseweiche verfügt über drei Anschlüsse. Über de beiden N-Buchsen wird die Fernspeiseweiche in den RF-Weg zwischen dem ZF-Ausgang am LNC und dem Patchfeld geschaltet. Über die SMA-Buchse kann ein sinusförmiges $10\,\text{MHz}$ Referenzsignal mit maximal $2\,\text{Vss}$ für den LNC eingespeist werden. Die Versorgungsspannung $V_\mathrm{CC}$ für den LNC wird über die beide Pins auf der linken Seite in Abbildung \ref{fig:KU-BT-10-REF} in die Leitung zum LNC eingespeist.\cite{kuhne-bias-tee}\newline
Auslegt ist die Fernspeiseweiche im RF-Weg für Frequenzen $f_\mathrm{ZF}$ zwischen $140\,\text{MHz}$ und $1500\,\text{MHz}$. Die Verluste der Fernspeiseweiche betragen typisch $L_\mathrm{BiasTee}=1\,\text{dB}$ und maximal $L_\mathrm{BiasTee}=1.5\,\text{dB}$. Die maximale Leistung des Signals im RF-Weg darf $100\,\text{mW}=20\,\text{dBm}$ nicht überschreiten, da sonst die Fernspeiseweiche beschädigt werden könnte.\cite{kuhne-bias-tee}\newline
Die eingespeiste Versorgungsspannung $V_\mathrm{CC}$ kann im Bereich von $0\,\text{V}$ bis maximal $36\,\text{V}$ DC liegen und der Strom darf nicht größer als $500\,\text{mA}$ DC werden.\cite{kuhne-bias-tee}
\subsubsection*{Patchfeld und Schaltmatrix}
Bei der verwendeten Schaltmatrix handelt es sich um das Model RC-2SP4T-A18 von der Firma Mini-Circuits. Diese kann mit einer PC-Software über USB oder Ethernet gesteuert werden kann. Auch Remotezugriffe über eine Webseite sollten möglich sein.\cite{RFSP4T_Switch}\newline
Insgesamt bietet die Schaltmatrix 10 SMA Anschlüsse, welche für Signale mit einer Frequenz $f$ von $\text{DC}$ bis $18\,\text{GHz}$ und einer Leistung bis zu $20\,\text{W}$ sind. Die Schaltgeschwindigkeit beträgt dabei typischerweise $25\,\text{ms}$.\cite{RFSP4T_Switch}\newline
Die maximalen auftretenden Verluste sind von der Frequenz $f$ abhängig.\cite{RFSP4T_Switch}
\begin{equation}
    L_\mathrm{RF-Switch,max}=
    \begin{cases}
        0.3\,\text{dB}&,\text{DC}\leq f\leq 8\text{GHz}\\
        0.4\,\text{dB}&,8\,\text{GHz}\leq f\leq 12\text{GHz}\\
        0.8\,\text{dB}&,12\,\text{GHz}\leq f\leq 20\text{GHz}\\
    \end{cases}
\label{eq:max-Verluste-RF-Switch}    
\end{equation}
Auch bei hohen Frequenzen bleibt der maximale Verlust der Schaltmatrix gering. Bei den zu erwartenden Frequenzen $f_\mathrm{ZF}\leq1.3\,\text{GHz}$ würde ein maximaler Verlust durch die RF-Switch Matrix von $L_\mathrm{RF-Switch}=0.3\,\text{dB}$ auftreten. Das macht diese Schaltmatrix geeignet für die Anwendung im RF-Bereich des Empfangssystems.\newline
\begin{figure}[H]
    \centering
    \includegraphics[width=0.75\linewidth]{Bilder/RF-Switch und Patchfeld.jpg}
    \caption{Verwendete Schaltmatrix RC-2SP4T-A18 und das Patchfeld}
    \label{fig:RF-Switch und Patchfeld}
\end{figure}
Die Abbildung \ref{fig:RF-Switch und Patchfeld} zeigt die verwendete Schaltmatrix (blau/silbernede Box in linken Mitte des Bildes) und das Patchfeld (schwarze Leiste mit den einzelnen SMA-Anschlüssen) im Serverschrank. Beim eingesetzten Patchfeld handelt es sich um ein Eigenbau. Insgesamt besteht das Patchfeld aus 7 N zu SMA Adapter Buchsen und einer N auf N Buchse. In diesem Empfangssystem wird eine N zu SMA Adapter Buchse verwendet. Der Verlust durch die Apapter Buchse kann ebenfalls mit einem VNA gemessen werden.
\begin{figure}[H]
    \centering
    \includesvg[width=0.4\linewidth]{Bilder/Messaufbau S21 Patchfeld}
    \caption{Messaufbau zum Messen des Verlustes durch Adapter Buchse}
    \label{fig:Messaufbau-S21-Patchfeld}
\end{figure}
Die Abbildung \ref{fig:Messaufbau-S21-Patchfeld} zeigt den angewendeten Messaufbau. Gemessen werden die Einspeiseverluste $S21$ zwischen $400\,\text{MHz}$ und $1.5\,\text{GHz}$, da die zu erwartenden Frequenzen $f_\mathrm{ZF}$ des ZF-Signals zwischen $433\,\text{MHz}$ und $1.26\,\text{GHz}$ liegen. Vor dem durchführen der Messung wird der VNA für den entsprechenden Frequenzbereich für Offen (Open), Kurzschluss (Short), Last (Load) von $50\,\Omega$ und Durchgang (Through) kalibriert. Das verwendete Kalibriertkit ist das 01 BN 533828 vom Hersteller Spinner.\newline
Das es sich um ein passives Element handelt, wird die Leistung auf $0\,\text{dBm}$ gestellt. Die Anzahl der Messpunkte beträgt $201$, die Auflösung $0.2\,\text{dB/div}$ und das Referenzlevel $0\,\text{dB}$.
\begin{figure}[H]
    \centering
    \includegraphics[width=0.75\linewidth]{Bilder/Verluste Patch Panel.PNG}
    \caption{Gemessene Einspeiseverluste $S21$ zwischen $400\,\text{MHz}$ und $1.5\,\text{GHz}$ des Patchfeld }
    \label{fig:S21-Patchfeld}
\end{figure}
Der Graph in Abbildung \ref{fig:S21-Patchfeld} zeigt die gemessenen Einspeiseverluste $S21$ des Patchfeld zwischen $400\,\text{MHz}$ und $1.5\,\text{GHz}$. Der Marker 1 befindet sich bei $433\,\text{MHz}$, was die niedrigste zu erwartende Frequenz $f_\mathrm{ZF}$ des ZF-Signals ist. Gemessen wird hier ein Verlust von $S21=0.008\,\text{dB}$. Der zweite Marker befindet sich bei $1.225\text{GHz}$. Der gemessene Verlust liegt hier bei $S21=0.076\,\text{dB}$. Beider Verluste sind sehr niedrig und fallen kaum ins Gewicht. Daher kann das Patchfeld problemlos im RF-Frontend des Empfangsystems verwendet werden. Für einen allgemeinen Wert wird die Dämpfung durch das Patchfeld mit dem höchsten gemessenen Dämpfungswert angegeben.
\begin{equation}
    L_\mathrm{Patchfeld}=0.076\,\text{dB}\approx0.8\,\text{dB}
\end{equation}




\subsubsection*{Wahl der Koaxialleitungen}
Die Koaxialleitungen sind notwendig um die einzelnen Komponenten mit einander zu verbinden. Es gibt viele verschiedene Arten an Koaxialleitungen, jedoch kann nicht eine beliebige genommen für jeden Abschnitt im RF-Bereich des Empfangssystems genommen werden. Jede Art an Koaxialleitung hat unterschiedliche Eigenschaften, welche es für bestimmte Anwendungsgebiete geeignet und für andere wiederum ungeeignet machen. Zum Beispiel kann je nach Art der Frequenzbereich, die Leitungsimpedanz ,die Dämpfung $\text{dB/m}$, die Anschlussmöglichkeiten, maximale Spannungsfestigkeit, sowie Umweltanforderungen und Preis stark variieren. Deshalb muss für jeden Abschnitt im RF-Frontend eine geeignete Koaxialleitung ausgewählt werden. Die jeweilige Koaxialleitung muss die folgenden Voraussetzungen erfüllen.
\begin{enumerate}
    \item Um mit der Systemimpedanz kompatibel zu sein, müssen die die Koaxialleitungen eine Impedanz von von $50\,\Omega$ aufweisen.
    \item Die jeweilige Koaxialleitung sollte für den jeweiligen Frequenzbereich im RF-Bereich des Empfangsystems geeignet sein. So können unnötigen Dämpfungen, etc. vermieden werden.
    \item Um die Rauschzahl $F_\mathrm{sys}$ des RF-Frontends und um die Verluste allgemein gering zu halten, sollte die Koaxialleitung eine niedrige Dämpfung und um das Signal gegen äußere elektromagnetische Strahlung zu schützen, ein hohes Schirmmaß aufweisen.
    \item Um das Verlegen der Koaxialleitungen zu vereinfachen und diese ebenfalls nicht zu beschädigen, sollte die Koaxialleitungen entsprechend geeignet für die Verlegung sein. Auch sollten die Koaxialleitungen für die jeweilige Umwelteinflüsse an ihrem Einsatzort geeignet sein.
    \item Die Koaxialleitung sollte die jeweiligen Steckverbindungen der Komponenten, ohne die Verwendung von Adapter, unterstützen. So können Verluste und damit verbunden mögliche Rauschquellen verringert werden.
\end{enumerate}
\begin{figure}[H]
    \centering
    \includesvg[width=0.75\linewidth]{Bilder/RF-Frontend}
    \caption{Darstellung des RF-Bereiches des Empfangssystems mit den einzelnen notwendigen Längen der Koaxialleitungen}
    \label{fig:länge-Koaxialleitungen}
\end{figure}
In der Abbildung \ref{fig:länge-Koaxialleitungen} ist eine Skizze vom RF-Bereich des Empfangsystems zu sehen. Dargestellt sind die einzelne Zweitore und die Koaxialleitungen, welche diese miteinander verbinden. Ebenfalls sind die notwendigen Längen der Koaxialleitungen eingetragen.\newline
Die erste Koaxialleitung (Koax 1) verbindet den Antennenfeed mit dem LNC. Diese Koaxialleitung ist das wichtigste Zweitor im gesamten RF-Frontend, da es das erste Zweitor in der Kette ist. Somit hat es die größte Auswirkung auf die Rauschzahl $F_\mathrm{ges}$ vom RF-Bereich des Empfangssystems. Da es eine Koaxialleitung ein passives Zweitor ist, wird die Rauschzahl $F_\mathrm{1}$, nach Gleichung \ref{eq:Rauschzahl-passives-Zweitor}, direkt aus den Verlusten der Koaxialleitung gewonnen. Daher muss eine Koaxialleitung mit besonders niedrigen Verlusten ausgewählt werden und die Länge der Koaxialleitung zu kurz wie möglich gehalten werden. Weiterhin muss die Koaxialleitung für eine Frequenz $f\approx10.5\,\text{GHz}$ ausgelegt sein, da diese Koaxialleitung das Signal des Downlink von Es'Hail-2 (QO-100) im X-Band zum LNC bringt, wo es dann in einen niedrigeren Frequenzbereich umgesetzt wird. Auch muss die Koaxialleitung für die Anwendung im Außenbereich geeignet sein, da sowohl Antenne und der LNC auf dem Dach der Hochschule montiert werden und die Leitung sollte flexibel sein, da vom Antennenfeed zum LNC enge Biegeradien zu erwarten sind. Als Steckverbindung sollte die Koaxialleitung SMA unterstützen.\newline
Für diese Verbindung wird die S\_04212\_B $50\,\Omega$ LOW-LOSS Koaxialleitung von Huber\&Suhner gewählt. Diese bestimmte Koaxialleitung ist für Frequenzen bis $18\,\text{GHz}$ geeignet und bietet mit einem Schirmmaß von $90\,\text{dB/m}$ guten Schutz vor elektromagnetischer Strahlung.\cite{S_04212_B}\newline 
Die Verluste in $\text{dB/m}$ können mithilfe einer Gleichung im Datenblatt für die gewünschte Frequenz bestimmt werden. Für eine Frequenz von $f\approx10.5\,\text{GHz}$ kann die folgende Dämpfung ermittelt werden.\cite{S_04212_B}
\begin{equation*}
    L=0.197\cdot \sqrt{f}+ 0.045\cdot f=0.197\cdot\sqrt{10.5\,\text{GHz}}+ 0.045 \cdot 10.5\,\text{GHz}=1.11\,\text{dB/m}
\end{equation*}
Das führt mit eine Länge von $l=1.5\,\text{m}$ zu einer Gesamtdämpfung von
\begin{equation}
    L_\mathrm{Koax1}=1.11\,\text{dB/m} \cdot 1.5\,\text{m}=1.655\,\text{dB}
    \label{eq:Dämpfung_Koax1}
\end{equation}
Die Dämpfung ist mit $1.655\,\text{dB}$, im Vergleich zum LL142 STR mit $\sim 1\,\text{dB}$\cite{LL142_koax24}, etwas höher. Jedoch ist die S\_04212\_B Koaxialleitung dank ihres SPE-Dielektrikum mit $50\,\%$ Luftanteil besonders flexibel und so auch führe engere statisch Biegeradien $\geq25\,\text{mm}$ geeignet\cite{S_04212_B}. Auch ist die S\_04212\_B Koaxialleitung deutlich günstiger \cite{LL142_koax24}\cite{S_04212_B_koax24} und mit ihren Mantel aus PUR (Polyurethan) besonders witterungsbeständig und für Temperaturen von $-40\degree\text{C}$ bis $+85\degree\text{C}$ geeignet.\cite{PUR_koax24}\newline
Die zweite Koaxialleitung (Koax2) verbindet den LNC mit der Fernspeiseweiche (Bias-Tee). Diese Koaxialleitung legt den größten Weg mit $l=13.5\,\text{m}$ zurück und sollte daher eine niedrige Dämpfung $\text{[db/m]}$ aufweisen. Zu erwarten sind nach der Abwärtsmischung Frequenzen $f_\mathrm{ZF}\leq1.3\,\text{GHz}$, was eine große Auswahlmöglichkeit an möglichen Koaxialleitung bietet. Zusätzlich sollte die Koaxialleitung für die Anwendung im Außenbereich und für Spannung $V_\mathrm{cc}\leq36\,\text{V}$ geeignet sein. Auch sollte die Koaxialleitung N-Stecker als Steckverbindung unterstützen.\newline
Ursprünglich ist für die zweiten Koaxialleitung eine LMR 600 Koaxialleitung vorgesehen gewesen. Diese ist bietet mit $0.109\,\text{dB/m}$ bei $f=1.5\,\text{GHz}$ eine sehr niedrige Dämpfung. Allerdings ist LMR 600 Koaxialleitung mit einem Biegeradius von $\geq38.1\,\text{mm}$ sehr starr, was zu Problemen bei der Verlegung führen könnte.\cite{LMR600}\newline
Aus diesem Grund wird die LMR 400 FR Koaxialleitung von Times Microwave verwendet. Diese hat mit $0.169\,\text{dB/m}$ bei $f=1.5\,\text{GHz}$ eine geringfügig höhere Dämpfung als das LMR 600 Kabel, ist jedoch mit einem Biegeradius $\geq25.4\,\text{mm}$ deutlich flexibler.\cite{LMR400_koax24}\cite{LMR400}\newline
Die Gesamtdämpfung $L_\mathrm{Koax2}$ dieser Koaxialleitung wird in Gleichung \ref{eq:Dämpfung_Koax2} mit einer Gesamtlänge $l=13.5\,\text{m}$ bestimmt.
\begin{equation}
    L_\mathrm{Koax2}=0.169\,\text{dB/m} \cdot 13.5\,\text{m}=2.282\,\text{dB}
    \label{eq:Dämpfung_Koax2}
\end{equation}
Dank des Mantels aus FRPE (Feuer Resistenten Polyethylen) ist die LMR 400 FR Koaxialleitung Wetterbeständig und für einen Temperaturbereich von $-40\degree\text{C}$ bis $+85\degree\text{C}$ geeignet.\cite{LMR400_koax24}\cite{FRPE_koax24}\newline
Mit einem Schirmmaß von $90\,\text{dB/m}$ schützt die LMR 400 FR Leitung gut gegen von außen einwirkende elektromagnetische Strahlung. Auch sind Spannungen bis $2500\,\text{V}$ und N-Stecker als Steckverbindung unterstützt.\cite{LMR400_koax24}\newline
Für die dritte Koaxialleitung (Koax3), welche die Fernspeiseweiche mit dem Patchfeld verbindet, wird ebenfalls die LMR 400 FR Koaxialleitung verwendet. Die Gesamtlänge dieser Leitung beträgt $l=1.5\,\text{m}$.
\begin{equation}
    L_\mathrm{Koax3}=0.169\,\text{dB/m} \cdot 1.5\,\text{m}=0.254\,\text{dB}
    \label{eq:Dämpfung_Koax3}
\end{equation}
Damit beträgt die Dämpfung der dritten Koaxialleitung $L_\mathrm{Koax3}=0.254\,\text{dB}$.\newline
Für die vierte (Koax4) und fünfte (Koax5) Koaxialleitung, welche das Patchfeld mit der Schaltmatrix und anschließend mit dem SDR verbinden, braucht es flexible Leitungen. Die flexiblen Leitung würden die Verkabelung im Serverschrank vereinfachen. Die Koaxialleitungen sollten zudem eine möglichst geringe Dämpfung und hohes Schirmmaß aufweisen, sowie für Frequenzen $f_\mathrm{ZF}\leq1.3\,\text{GHz}$ geeignet sein und SMA-Steckverbindung unterstützen.\newline
Verwendet werden für die beiden Koaxialleitungen die Hyperflex 5/ $50\,\Omega$ LOW-LOSS Koaxialleitung von Messi\&Paoloni. Diese Koaxialleitung ist für Frequenzen bis $6\,\text{GHz}$ geeignet und ist mit einem wiederholbaren Biegeradius von $50\,\text{mm}$ flexibel genug für diese Anwendung.\cite{Hyperflex5}\newline
Bei einer Frequenz $f=1.296\,\text{GHz}$ hat die Hyperflex 5 Koaxialleitung eine Dämpfung von $0.305\,\text{dB/m}$ und hat damit eine geringfügig bessere Dämpfung als eine Aircell 5 Koaxialleitung. Bei einer Länge von jeweils $l=0.35\,\text{m}$ führt das zu folgender Dämpfung je Koaxialleitung.\cite{Hyperflex5_koax24}
\begin{equation}
    L_\mathrm{Koax4}=L_\mathrm{Koax5}=0.309\,\text{dB/m} \cdot 0.35\,\text{m}=0.108\,\text{dB}
    \label{eq:Dämpfung_Koax4 und Koax5}
\end{equation}
Auch bietet die Hyperflex 5 Koaxialleitung mit $105\,\text{dB/m}$ ein sehr hohes Schirmmaß und es werden SMA-Steckverbindungen unterstützt.\cite{Hyperflex5_koax24}\newline
Damit die Kabel nicht unter ihren zulässigen Biegeradius gebogen werden können werden die Hyperflex 5 Koaxialleitungen mit einem Knickschutz versehen.\newline


\subsubsection*{Software Defined Radio und SDR Software}
Bei dem vorhandenen Software Defined Radio (SDR) handelt es sich um einen USRP X310 von Nationale Instruments. Der USRP X310 ist bestandteil einer skalierbaren SDR-Plattform, welche für die Entwicklung, Testung und Einsatz von Kommunikationsequipment vorgesehen ist.\cite{USRP-X310}
\begin{figure}[H]
    \centering
    \includegraphics[width=0.75\linewidth]{Bilder/NI-Ettus-X310.jpeg}
    \caption{Der URSP X310 von National Instruments\cite{USRP-X310}}
    \label{fig:USRP-X310}
\end{figure}
Der USRP X310 basiert auf einem  XC7K410T FPGA, welcher eine schnelle Verbindung zu den Erweiterungskarten, dem $1\,\text{GB}$ DDR3 Arbeitsspeicher und verschiedenen Schnittstellen bietet, über welche das SDR mit einem PC verbunden werden kann.\cite{USRP-X310}\newline
Das SDR bietet die Möglichkeit über PCIe, über zwei $10\,\text{Gig}$ Ethernet- oder über zwei $1\,\text{Gig}$ Ethernetschnittstellen mit einem PC verbunden zu werden \cite{USRP-X310}. In diesem Fall ist der USRP X310 über eine $10\,\text{Gig}$ Ethernetschnittstelle mit dem PC verbunden.\newline
Die beiden Herzstücke des USRP X310 sind die modulare Erweitertungskarten. Insgesamt stehen 10 verschiedene Erweiterungskarten zur Auswahl mit denen ein Frequenzbereich von $0\,\mathrm{Hz}$ bis zu $6\,\text{GHz}$ abgedeckt werden kann. Insgesamt stehen dem USRP X310 zwei Kanäle, welche Voll-Duplex fähig sind, mit einer Bandbreite bis zu $160\,\text{MHz}$ zu Verfügung. Die Bandbreite ist von den jeweiligen Erweiterungskarten abhängig.\cite{USRP-X310}\cite{USRP-X310-Doku}\newline
Je nach Erweiterungskarte ist eine maximale Verstärkung zwischen$G_\mathrm{SDR}=31.5\,\text{dB}$\cite{USRP-X310-UBX-Doku} und $G_\mathrm{SDR}=93\,\text{dB}$\cite{USRP-X310-TwinRX-Doku} möglich.\cite{USRP-X310-Doku} 
Da eine Verstärkung von $G_\mathrm{SDR}=93\,\text{dB}$ viel zu hoch für die geplante Anwendung ist, wird folgend mit einer maximalen Verstärkung von $G_\mathrm{SDR,max}=30\,\text{dB}$ gerechnet.\newline
Jeder Kanal verfügt über einen 14-Bit ADC und einen 16-Bit DAC, wobei die maximale Abtastrate des ADC $200\,\text{MS/s}$. Die maximale Abtastrate des DAC beträgt $800\,\text{MS/s}$.\cite{USRP-X310}\newline
Mit der Bitgröße des ADC $n=14$ kann der Dynamikumfang des SDR ermittelt werden.\cite{DynamicRange}
\begin{equation}
\mathrm{DR} = 20 \cdot \log_{10}\left( 2^n \sqrt{\frac{3}{2}} \right)
= 20 \cdot \log_{10}\left( 2^{14} \sqrt{\frac{3}{2}} \right)
\approx 86\,\text{dB}.
\label{eq:dynamic-Range-USRP-X310}
\end{equation}
Mit dem Dynamikumfang wird das Verhältnis vom stärksten und schwächsten Signal angegeben, welches der SDR verarbeiten kann. Das stärkste Signal ist dabei das Signal, welches der SDR verarbeiten kann, ohne dabei zu übersteuern und das schwächste Signal stellt das Grundrauschen des SDR da. Der Dynamikumfang ist unter anderem wichtig für die zu wählende Verstärkung. Ist das verstärkte Signal zu groß für den Dynamikumfang kommt es zu einer Übersteuerung.\cite{DynamicRange}\newline
Mit der Gleichung \ref{eq:Rauschleistung-Ausgang-Zweitor} kann das Grundrauschen des SDR bestimmt werden. Die entschiedenen Faktoren für das thermische Grundrauschen ist die betrachtete Bandbreite $B$, die physikalische Temperatur $T_\mathrm{0}$ des SDR und die äquivalente Rauschtemperatur $T_\mathrm{eSDR}$ des SDR. Die Bandbreite $B$ wird mit der maximalen zulässigen Bandbreite $B=2.7\,\text{kHz}$ eines Signals über den Schmalbandtransponder angenommen. Bei physikalischen Temperatur wird von der Raumtemperatur $T_\mathrm{0}=290\,\text{K}$ ausgegangen. Die äquivalente Rauschtemperatur des SDR kann über die Umstellung der Gleichung \ref{eq:Rauschzahl-aus-Te-und-T0} aus der Rauschzahl $F_\mathrm{SDR}$ ermittelt werden. Diese wird mit typisch $F_\mathrm{SDR,dB}=8\,\text{dB}=6.31$ angegeben \cite{USRP-X310}.\newline
\begin{equation}
    T_\mathrm{eSDR}=(F_\mathrm{SDR}-1)\cdot T_\mathrm{0}=(6.31-1)\cdot290\,\text{K}=1539,9\,\text{K}
    \label{eq:äquivalente-Rauschtemperatur-SDR}
\end{equation}
Mit der äquivalente Rauschzahl aus Gleichung \ref{eq:äquivalente-Rauschtemperatur-SDR} kann dann das Grundrauschen vom SDR bestimmt werden.
\begin{equation}
\begin{split}
     N_\mathrm{oSDR}&=k\cdot(T_\mathrm{o}+T_\mathrm{eSDR})\cdot B=1.38\cdot10^{-23}\,\frac{\text{J}}{\text{K}}\cdot (290\,\text{K}+1539.9\,\text{K})\cdot2.7\,\text{kHz}\\
     &=6.82\cdot10^{-17}\,\text{W}=-131.66\,\text{dBm}
\end{split}
 \label{eq:Grundrauschen-SDR}  
\end{equation}
Der Pegel eines Siganls am Eingang des SDR müsste also größer als $-131.66\,\text{dBm}$ sein, um nicht im Grundrauschen des SDR zu verschwinden.\newline
Der maximale Eingangspegel am SDR darf $-15\,\mathrm{dBm}$ nicht überschreiten.\cite{USRP-X310-Doku}
Dank des Open-Source Software Support bietet der USRP X310 UHD Treiberunterstützung für verschiedene Plattformen, wie Windows und Linux Betriebssysteme, und ist mit C++ und Python APIs, sowie GNU Radio und anderen Frameworks und Programmen kompatibel.\cite{USRP-X310}\newline
Der Open-Source Software Support und die Kompatibilität mit GNU Radio bieten die Möglichkeit eine eigene geeignete SDR Software, speziell für den Einsatz an Es'Hail-2 (QO-100), zu erstellen. Bei GNU Radio handelt es sich um eine freie Software-Toolkit-Sammlung zur Implementierung von Software Defined Radio, kurz SDR. GNU Radio bietet eine umfangreiche Bibliothek an Signalverarbeitungsblöcken, welche zu einem gemeinsamen Flussgraphen einfach zusammengefügt werden können. Neben der Realisierung von Software Defined Radios, kann GNU Radio auch ohne Hardware für Simulation verwendet werden.\cite{GNU-Radio}\newline
Die in GNU Radio erstelle SDR Software muss mehrere Voraussetzungen erfüllen, um für die Anwendung an Es'Hail-2 (QO-100) geeignet zu sein.
\begin{enumerate}
    \item Die Software muss in der Lage sein den USRP X310 ansteuern zu können.
    \item Innerhalb der Software sollte es verschiedene Optionen für die Demodulation geben, um die Signale von Es'Hail-2 (QO-100) richtig demodulieren zu können. Gewünscht sind Einseitenband-AM (LSB/USB),CW und FM. Obwohl FM nicht auf Es'Hail-2 (Q0-100) verwendet werden darf,sollte diese Option für andere Anwendungszwecke der Software vorhanden sein.
    \item Die demodulierten Signale sollten als Audio ausgegeben werden. Auch sollten die demodulierten Signale als Audio abgespeichert werden können.
    \item Das empfangene Frequenzspektrum sollte als FFT und Wasserfalldiagramm korrekt dargestellt werden können.
    \item Die Frequenz, Filterbandbreite, Art der Modulation und Lautstärke sollte im Betrieb verändert werden können.
\end{enumerate}
Die Erstellung der SDR-Software erfolgt im Kapitel \ref{kap:SDR Software}.




\subsection{Bewertung des Empfangssystems}
Um die Eignung des zusammengestellten Empfangssystems für den geplanten Einsatz zu überprüfen, muss dessen Leistungsfähigkeit überprüft werden. Für die Bewertung der Leistungsfähigkeit stehen mehrere Möglichkeiten zur Verfügung. 
\subsubsection*{Rauschen des Empfangssystems}
Zwei wichtige Größen für das Empfangssystem sind seine äquivalente Rauschtemperatur $T_\mathrm{e,sys}$
und seine Rauschzahl $F_\mathrm{sys}$. Die beiden Größen drücken das eigen Rauschen des Systems und die Verschlechterung des $SNR$ vom Eingang des Empfangssystems bis zum Eingang des SDR aus.\newline
\begin{figure}[H]
    \centering
    \includesvg[width=0.9\linewidth]{Bilder/Rauschen der Empfangskette}
    \caption{Blockschaltbild des Empfangssystems mit den einzelnen äquivalenten Rauschtemperaturen $T_\mathrm{e}$, Rauschzahlen $F$ und Verstärkungen $G$ }
    \label{fig:Rauschen-des-Empfangssystems}
\end{figure}
In der Abbildung \ref{fig:Rauschen-des-Empfangssystems} ist ein Blockschaltbild des Empfangssystems zu sehen. Eingetragen sind neben den Namen der einzelnen Zweitore auch ihre äquivalente Rauschtemperatur $T_\mathrm{e}$, Rauschzahl $F$ und ihre Verstärkung $G$.\newline 
Die äquivalente Rauschtemperatur $T_\mathrm{e,sys}$ setzt sich aus äquivalenten Rauschtemperaturen $T_\mathrm{e}$ der einzelnen Zweitore im Empfangssystem zusammen. Deren äquivalente Rauschzahl $T_\mathrm{e}$ kann mithilfe der Gleichung \ref{eq:Rauschzahl-aus-Te-und-T0} mit ihrer Rauschzahl $F$ bestimmt werden. Die Rauschzahl passiver Zweitore, wie den Koaxialleitungen, der Fernspeiseweiche, dem Patchfeld und dem RF-Switch können mit der Gleichung \ref{eq:Rauschzahl-passives-Zweitor} aus ihrem Verlust $L$ bestimmt werden.
\begin{table}[H]
    \centering
    \begin{tabular}{c|c|c|c}
      Name   & Rauschzahl $F\,\text{in Absolut}$& $T_\mathrm{e}\,\text{in K}$ & Gewinn $G\,\text{in Absolut}$\\
      \hline
      Koax1   & $F_\mathrm{1}=L_\mathrm{Koax1}=1.46$& $T_\mathrm{e1}=133.4$ & $G_\mathrm{1}=\frac{1}{1.46}=0.685$\\
      LNC   & $F_\mathrm{LNC}=1.48$& $T_\mathrm{eLNC}=139.2$ & $G_\mathrm{LNC}=316227.77$\\
      Koax2   & $F_\mathrm{2}=L_\mathrm{Koax2}=1.69$& $T_\mathrm{e2}=200.1$ & $G_\mathrm{2}=\frac{1}{1.69}=0.59$\\
      BiasTee   & $F_\mathrm{BiasTee}=L_\mathrm{BiasTee}=1.41$& $T_\mathrm{eBiasTee}=118.9$ & $G_\mathrm{BiasTee}=\frac{1}{1.41}=0.71$\\
      Koax3   & $F_\mathrm{3}=L_\mathrm{Koax3}=1.06$& $T_\mathrm{e3}=17.4$ & $G_\mathrm{3}=\frac{1}{1.06}=0.94$\\
      Patchfeld   & $F_\mathrm{Patchfeld}=L_\mathrm{Patchfeld}=1.2$& $T_\mathrm{ePatchfeld}=58$ & $G_\mathrm{Patchfeld}=\frac{1}{1.2}=0.83$\\
      Koax4   & $F_\mathrm{4}=L_\mathrm{Koax4}=1.03$& $T_\mathrm{e4}=8.7$ & $G_\mathrm{4}=\frac{1}{1.03}=0.97$\\
      RF-Switch & $F_\mathrm{RF-Switch}=L_\mathrm{RF-Switch}=1.07$& $T_\mathrm{eRF-Switch}=20.3$ & $G_\mathrm{RF-Switch}=\frac{1}{1.97}=0.93$\\
      Koax5   & $F_\mathrm{5}=L_\mathrm{Koax5}=1.03$& $T_\mathrm{e5}=8.7$ & $G_\mathrm{5}=\frac{1}{1.03}=0.97$\\
      SDR  & $F_\mathrm{SDR}=6.31$& $T_\mathrm{eSDR}=1539.9$ & $G_\mathrm{SDR}=30\,\text{dB}$\\
    \end{tabular}
    \caption{Bestimmte Rauschzahl $F$, äquivalente Rauschtemperatur $T_\mathrm{e}$ und Verstärkung $G$ der einzelnen Zweitore}
    \label{tab:Bestimmte-Werte-der-Zweitore-für-Te}
\end{table}
Mit den Werten in Tabelle \ref{tab:Bestimmte-Werte-der-Zweitore-für-Te} kann die äquivalente Rauschtemperatur $T_\mathrm{sys}$ des Empfangssystems mithilfe der Gleichung \ref{eq:Gesamt-äquivalente-Rauschtemperatur-Kaskade} bestimmt werden. Der für die Berechnung verwendete Python Code ist im Github-Repository und im Anhang \ref{lst:Link-Budget-python} hinterlegt.
\begin{equation}
\begin{split}
        T_\mathrm{e,sys}=&T_\mathrm{e1}+\frac{T_\mathrm{eLNC}}{G_\mathrm{1}}+\frac{T_\mathrm{e2}}{G_\mathrm{1}\cdot G_\mathrm{LNC}}+\frac{T_\mathrm{eBiasTee}}{G_\mathrm{1}\cdot G_\mathrm{LNC}\cdot G_\mathrm{2}}+\frac{T_\mathrm{e3}}{G_\mathrm{1}\cdot G_\mathrm{LNC}\cdot G_\mathrm{2}\cdot G_\mathrm{BiasTee}}\\
        &+\frac{T_\mathrm{ePatchfeld}}{G_\mathrm{1}\cdot G_\mathrm{LNC}\cdot G_\mathrm{2}\cdot G_\mathrm{BiasTee}\cdot G_\mathrm{3}}\\&+\frac{T_\mathrm{e4}}{G_\mathrm{1}\cdot G_\mathrm{LNC}\cdot G_\mathrm{2}\cdot G_\mathrm{BiasTee}\cdot G_\mathrm{3}\cdot G_\mathrm{Patchfeld}}\\
        &+\frac{T_\mathrm{eRF-Switch}}{G_\mathrm{1}\cdot G_\mathrm{LNC}\cdot G_\mathrm{2}\cdot G_\mathrm{BiasTee}\cdot G_\mathrm{3}\cdot G_\mathrm{Patchfeld}\cdot G_\mathrm{4}}\\
        &+\frac{T_\mathrm{e5}}{G_\mathrm{1}\cdot G_\mathrm{LNC}\cdot G_\mathrm{2}\cdot G_\mathrm{BiasTee}\cdot G_\mathrm{3}\cdot G_\mathrm{Patchfeld}\cdot G_\mathrm{4}\cdot G_\mathrm{RF-Switch}}\\
        &+\frac{T_\mathrm{eSDR}}{G_\mathrm{1}\cdot G_\mathrm{LNC}\cdot G_\mathrm{2}\cdot G_\mathrm{BiasTee}\cdot G_\mathrm{3}\cdot G_\mathrm{Patchfeld}\cdot G_\mathrm{4}\cdot G_\mathrm{RF-Switch}\cdot G_\mathrm{5}}\\
        &=336.63\,\text{K}
\end{split}
\label{eq:äquivalente-Rauschtemperatur-Empfangsystem}
\end{equation}
Mit der bestimmten äquivalenten Rauschtemperatur $T_\mathrm{e,sys}$ des Empfangssystems kann die Rauschzahl $F_\mathrm{sys}$ des Empfangssystems mit Gleichung \ref{eq:Rauschzahl-aus-Te-und-T0}.
\begin{equation}
    F_\mathrm{sys}=1+\frac{T_\mathrm{e,sys}}{T_\mathrm{0}}=1+\frac{336.63\,\text{K}}{290\,\text{K}}=2.161=3.34\,\text{dB}
\end{equation}
 Die größte Auswirkung auf $T_\mathrm{e,sys}$ und $F_\mathrm{sys}$ haben die erste Koaxialleitung und der LNC. Dank der hohen Verstärkung des LNC mit $G_\mathrm{LNC}=55\,\text{dB}=316227.66$ als zweites Element werden die Rauschzahlen $F$, bzw. die äquivalenten Rauschtemperaturen $T_\mathrm{e}$ der folgenden Zweitore stark reduziert, sodass diese keine große Auswirkung mehr auf den Gesamtwert haben. Jedoch wird die äquivalente Rauschtemperatur des LNC $T_\mathrm{eLNC}$ durch die Dämpfung der ersten Koaxialleitung stärker gewichtet und nicht reduziert wird. Das erklärt die  Rauschzahl $F_\mathrm{sys}=3.34\,\text{dB}$ und die äquivalente Rauschtemperatur des Empfangssystems von $ T_\mathrm{e,sys}=336.63\,\text{K}$.

\subsubsection*{Verstärkungen und Dämpfungen im Empfangssystem}
Im Empfangssystem treten verschiedene Verstärkungen und Dämpfungen auf. Diese können in einer Größe, der Verstärkung $G_\mathrm{sys}$ des Empfangssystems, zusammengefasst werden. Mit dieser Größe kann übersichtliche Angabe zur Verstärkung des Empfangssystems gemacht werden.\newline
Im Empfangssystem sind nur zwei Verstärkenden Komponenten vorhanden. Die erste Komponente ist der LNC, welcher mit einer typischen Verstärkung von $G_\mathrm{LNC}=55\,\text{dB}$ angeben wird. Die zweite Komponente ist der USRP X310, welcher eine variable Verstärkung besitzt. Für Rechenzwecke wird hier von einer Verstärkung von $G_\mathrm{SDR}=30\,\text{dB}$ angegeben. Diese beiden Verstärkungen können in eine Größe $G$ zusammengefasst werden.
\begin{equation*}
    G=G_\mathrm{LNC,dB}+G_\mathrm{SDR,dB}=55\,\text{dB}+30\,\text{dB}=85\,\text{dB}
\end{equation*}
Dämpfungen treten im Empfangssystem an mehreren Stellen auf. Die größte Dämpfung wird durch Koaxialleitungen verursacht. Ihre Dämpfung ist in den Gleichung \ref{eq:Dämpfung_Koax1} bis \ref{eq:Dämpfung_Koax4 und Koax5} angegeben und kann in einer gemeinsamen Größe $L_\mathrm{Koax}$ zusammengefasst werden.
\begin{equation*}
    \begin{split}
        L_\mathrm{Koax}&=L_\mathrm{Koax1,dB}+L_\mathrm{Koax2,dB}L_\mathrm{Koax3,dB}L_\mathrm{Koax4,dB}L_\mathrm{Koax5,dB}\\
        &=1.655\,\text{dB}+2.282\,\text{dB}+0.254\,\text{dB}+0.108\,\text{dB}+0.108\,\text{dB}=4.407\,\text{dB}
    \end{split}
\end{equation*}
Eine weitere dämpfende Komponente ist die Fernspeiseweiche. Ihre Dämpfung wird mit $L_\mathrm{BiasTee}=1.5\,\text{dB}$ angegeben. Weitere Dämpfungen treten am Patchfeld $L_\mathrm{Patchfeld}=0.8\,\text{dB}$ und an der Schaltmatrix auf. Bei einer Frequenz $f_\mathrm{ZF}\leq1.3\,\text{GHz}$ tritt durch die Schaltmatrix eine maximale Dämpfung von $L_\mathrm{RF-Switch}=0.3\,\text{dB}$ auf. Die Dämpfungen $L_\mathrm{Koax}$, $L_\mathrm{BiasTee}$, $L_\mathrm{Patchfeld}$ und $L_\mathrm{RF-Switch}$
können zur besseren Übersicht in einer Größe $L_\mathrm{sys}$ zusammengefasst werden.
\begin{equation}
    \begin{split}
        L_\mathrm{sys}&=L_\mathrm{Koax,dB}+L_\mathrm{BiasTee,dB}+L_\mathrm{Patchfeld,dB}+L_\mathrm{RF-switch,dB}\\
        &=4.407\,\text{dB}+1.5\,\text{dB}+0.8\,\text{dB}+0.3\,\text{dB}=7.01\,\text{dB}
    \end{split}
    \label{eq:Verluste-Empfangsystem}
\end{equation}
Mit der Verstärkung $G$ und der Dämpfung $L_\mathrm{sys}$ kann die Gesamtverstärkung des Empfangssystems in einer übersichtlichen Größe $G_\mathrm{sys}$ dargestellt werden.
\begin{equation}
    G_\mathrm{sys}=G_\mathrm{LNC,dB}-L_\mathrm{sys,dB}=85\,\text{dB}-7.01\,\text{dB}=77.99\,\text{dB}
    \label{eq:Gesamtverstärkung-Empfangssystems}
\end{equation}
Die Gesamtverstärkung des Empfangssystems beträgt maximal $G_\mathrm{sys}=77.99\,\text{dB}$. Mit dieser Verstärkung ist es problemlos möglich die schwache Signale von Es'Hail-2 (QO-100) zu verstärken. 

\subsubsection*{Empfangsgüte $G/T$}
Mit der Empfangsgüte $G/T$ kann die Empfindlichkeit des Empfangssystems angegeben werden. Sie ist ein Maß für die Qualität des Empfangssystems, einschließlich der Antenne. Sie entspricht dem Verhältnis des Antennengewinn $G_\mathrm{R,max}$ und des Rauschens $T_\mathrm{A}+T_\mathrm{e,sys}$ des Empfangssystems.\cite{Satellite_Communications_Systems}
\begin{equation}
    G/T=\frac{G_\mathrm{R,max}}{T_\mathrm{e,sys}+T_\mathrm{A}}
    \label{eq:Empfangsgüte}
\end{equation}
Die Einheit der Empfangsgüte ist dabei $\text{1/K}$ oder $\text{dB/K}$. Je höher der Wert $(\text{dB/K})$ ist, desto besser ist die Empfangsgüte, da das Empfangssystem mehr Gewinn pro Rauschtemperatur hat.\newline
Da sich die Antennentemperatur $T_\mathrm{A}$ je nach Bedingung unterscheidet, muss diese zuerst für jede Bestimmung ermittelt werden.\newline
Für die Bedingung klarer Himmel kann die Antennentemperatur mit der Gleichung \ref{eq:Antennentemperatur-klarer-Himmel} bestimmt werden. Da für die verwendete Parabolantenne kein 
Antennendiagramm vorhanden ist, wird diese als eine ideale Antenne ohne Nebenkeulen Richtung Boden ausgegangen. Damit wird $T_\mathrm{Ground}=0\,\text{K}$. Die Helligkeitstemperatur kann für den jeweiligen Elevationswinkel der Antenne aus der Abbildung \ref{fig:Temperatur-Himmel} entnommen werden. Der Elevationswinkel $\varepsilon$ der Antenne ist in Gleichung \ref{eq:Elevation-Antenne} mit $\varepsilon=27.36\degree\approx30°$ angegeben. Zusammen mit einer Frequenz $f\approx10.5\,\text{GHz}$ kann eine Helligkeitstemperatur von $T_\mathrm{Sky}\approx6.5\,\text{K}$ ermittelt werden.\newline
\begin{equation}
    T_\mathrm{A,klarerHimmel}=T_\mathrm{Sky}+T_\mathrm{Ground}=6.5\,\text{K}+0\,\text{K}=6.5\,\text{K}
    \label{eq:Antennentemperatur-Bedingung-klarer-Himmel}
\end{equation}
Mit der Gleichung \ref{eq:Empfangsgüte} kann die Empfangsgüte $G/T$ für die Bedingung klarer Himmel bestimmt werden. Die äquivalente Rauschtemperatur ist in Gleichung \ref{eq:äquivalente-Rauschtemperatur-Empfangsystem} angegeben. Der Gewinn der Empfangsantenne ist in Gleichung \ref{eq:Gewinn-der-Empfangsantenne} angegeben und wird für eine Frequenz $f=10.5\,\text{GHz}$ mit $G_\mathrm{R,max}=38.6\,\text{dBi}=7244.36$ angenommen.
\begin{equation}
    G/T=\frac{G_\mathrm{R,max}}{T_\mathrm{e,sys}+T_\mathrm{A,klarerHimmel}}=\frac{7244.36}{336.63\,\text{K}+6.5\,\text{K}}=21.11\,\text{1/K}=13.24\,\text{dB/K}
    \label{eq:Empfangsgüte-Bedingung-klarer-Himmel}
\end{equation}
Eine Empfangsgüte von $13.24\,\text{dB/K}$ ist ein solider Wert für die geplante Anwendung, kann aber nicht mit der Empfangsgüte $G/T=26.5\,\text{dB/K}$ professionellen Empfangssystemen mithalten.\cite{Vergleich-GT}
Für die Bedingung leichter Regen muss bei der Antennentemperatur $T_\mathrm{A}$ die Dämpfung durch leichte Regenschauer berücksichtigt werden. Die Dämpfung ist in Gleichung     \ref{eq:Dämpfung-durch-leichten-Regen} mit $L_\mathrm{leichterRegen}=0.2\,\text{dB}=1.05$.Auch die Temperatur der Wolken $T_\mathrm{m}=275\,\text{K}$ spielt dabei eine Rolle. Die Antennentemperatur für die Bedingung leichter Regen kann mit der Gleichung \ref{eq:Antennentemperatur-bei-Regen} bestimmt werden.
\begin{equation}
\begin{split}
        T_\mathrm{A,leichterRegen}&=\frac{T_\mathrm{Sky}}{L_\mathrm{leichterRegen}}+T_\mathrm{m}\left(1-\frac{1}{L_\mathrm{leichterRegen}}\right)+T_\mathrm{Ground}\\
        &=\frac{6.5\,\text{K}}{1.05}+275\,\text{K}\left(1-\frac{1}{1.05}\right)+0\,\text{K}=19.29\,\text{K}
\end{split}
\label{eq:Antennentemperatur-leichterRegen}
\end{equation}
Im Vergleich zur Antennentemperatur bei klaren Himmel $T_\mathrm{A,klarerHimmel}=6.5\,\text{K}$ ist die Antennentemperatur bei leichten Regenfälle mit $T_\mathrm{A,leicherRegen}=19.29\,\text{K}$ fast dreimal so groß. Das zeigt auf, wie wichtig die Berücksichtigung der Dämpfung durch Regenschauer in der Antennentemperatur ist. Die höhere Antennentemperatur bedeutet mehr Rauschen am Eingang des Empfangssystem, was zur einer Verschlechterung des $G/T$ führt.
\begin{equation}
    G/T=\frac{G_\mathrm{R,max}}{T_\mathrm{e,sys}+T_\mathrm{A,leichterRegen}}=\frac{7244.36}{336.63\,\text{K}+19.29\,\text{K}}=20.35\,\text{1/K}=13.12\,\text{dB/K}
    \label{eq:Empfangsgüte-Bedingung-leichter-Regen}
\end{equation}
Im Vergleich zur Empfangsgüte bei klaren Himmel $G/T=13.24\,\text{dB/K}$ sinkt die Empfangsgüte bei leichten Regenschauern auf $G/T=13.12\,\text{dB/K}$. Das entspricht einem Verlust von $0.12\,\text{dB/K}$, was im ersten Moment nicht viel wirkt. In Absoluten Zahlen entspricht ein Verlust von $0.12\,\text{dB}$ einem Verlust von ca. $3.6\,\%$\newline
Die Antennentemperatur für die Bedingung Regen kann wird auch mit der Gleichung \ref{eq:Antennentemperatur-bei-Regen} bestimmt. Die Dämpfung durch stärkere Niederschläge ist in Gleichung \ref{eq:bestimmte-Regendämpfung} mit $L_\mathrm{Regen}=8.86\,\text{dB}=7.69$ angegeben.
\begin{equation}
\begin{split}
        T_\mathrm{A,Regen}&=\frac{T_\mathrm{Sky}}{L_\mathrm{Regen}}+T_\mathrm{m}\left(1-\frac{1}{L_\mathrm{Regen}}\right)+T_\mathrm{Ground}\\
        &=\frac{6.5\,\text{K}}{7.69}+275\,\text{K}\left(1-\frac{1}{7.69}\right)+0\,\text{K}=240.1\,\text{K}
\end{split}
\label{eq:Antennentemperatur-Regen}
\end{equation}
Verglichen mit der Antennentemperatur bei leichten Regen $T_\mathrm{A,leichterRegen}=19.29\,\text{K}$ ist die Antennentemperatur bei stärkeren Niederschläge $T_\mathrm{A,Regen}=240.1\,\text{K}$ deutlich höher. Die hohe Antennentemperatur ist den stärkeren Niederschlägen geschuldet. Die höhere Antennentemperatur führt zu einer deutlichen Erhöhung des Rausches im System, was sich negativ auf die Empfangsgüte $G/T$ und das $SNR$ auswirken wird.
\begin{equation}
    G/T=\frac{G_\mathrm{R,max}}{T_\mathrm{e,sys}+T_\mathrm{A,Regen}}=\frac{7244.36}{336.63\,\text{K}+240.1\,\text{K}}=12.56\,\text{1/K}=10.99\,\text{dB/K}
    \label{eq:Empfangsgüte-Bedingung-Regen}
\end{equation}
Wie vermutet wirkt sich die deutlich höhere Antennentemperatur negativ auf die Empfangsgüte $G/T$ aus. Im Vergleich zur Empfangsgüte bei klaren Himmel $G/T=13.24\,\text{dB/K}$ sinkt die Empfangsgüte bei starken Niederschlägen auf $G/T=10.99\,\text{dB/K}$. Das entspricht einem Verlust von $2.25\,\text{dB}$ oder ca.$40.5\,\%$. Das wird sich deutlich negativ auf das $SNR$ des Empfängers auswirken und eventuell zu Ausfällen des Downlinks führen.



\subsection{Link Budget und Link Qualität}
Die Bestimmung des Link Budget ist ein wichtiger Schritt in der Planung von Satelliten Kommunikationssystemen. Mit dem Link Budget wird die Leistungsbilanz des jeweiligen Satellitenlink angeben. Es setzt sich aus der eingespeisten Leistung $P_\mathrm{T}$ des  und allen auftretenden Verlusten $L$ und Verstärkungen $G$ vom Sender bis zum Empfänger zusammen.\cite{Satellite_Communications_Systems}\cite{Link-Budget} 
\begin{equation}
    P_\mathrm{RX,dB}=P_\mathrm{T,dB}-L_\mathrm{dB}+G_\mathrm{dB}
    \label{eq:Linkbudget}
\end{equation}
Die Gleichung \ref{eq:Linkbudget} ist vereinfachte Form der Bestimmung des Link Budgets. Die Leistung $P_\mathrm{RX}$ aus Gleichung \ref{eq:Linkbudget} ist dann die zu erwartende Leistung am Ausgang des RF-Bereiches vom Empfangssystem.\cite{Satellite_Communications_Systems}\cite{Link-Budget}\newline
\begin{figure}[H]
    \centering
    \includesvg[width=0.75\linewidth]{Bilder/Link Budget}
    \caption{Grafische Darstellung des Link Budgets}
    \label{fig:Grafische Darstellung des Link Budgets}
\end{figure}
Die Abbildung \ref{fig:Grafische Darstellung des Link Budgets} zeigt eine grafische Darstellung des Link Budgets. Das Link Budget kann in drei Bereich aus Kapitel \ref{chap:Theoretische-Betrachtung-des-Downlinks} eingeteilt werden.\newline
Im ersten Abschnitt befindet sich der Sender, in diesem Fall der Satellit Es'Hail-2 (QO-100). In diesem Abschnitt sind für das Link Budget die Sendeleistung $P_\mathrm{T}$, der Gewinn der Sendeantenne $G_\mathrm{T}$ und das daraus resultierende $EIRP$. Auch die Bandbreite $B$ ist für die spätere Bestimmung der Rauschleistung im Empfangssystem von Bedeutung.\newline 
Die Sendeleistung von Es'Hail-2 (QO-100) ist in Gleichung \ref{Sendeleistung Es'Hail-2} angegeben. Der Gewinn der Sendeantenne beträgt $G_\mathrm{T}=17\,\text{dBi}$ und das $EIRP$ ist in Gleichung \ref{eq:EIRP_dBm_Eshail2} in $\text{dB}$ und in Gleichung \ref{eq:EIRP_W_EsHail2} in $\text{W}$ angegeben. Diese Werte können in einer Tabelle zusammengefasst werden. 
\begin{table}[H]
    \centering
    \begin{tabular}{c|c|c|c}
        Name & Variable & Wert & Einheit\\
        \hline
         Sendeleistung & $P_\mathrm{T}$ & $42.5$ & $\text{dBm}$\\
                       & $P_\mathrm{T}$ & $17.78$ & $\text{W}$\\
         Gewinn& $G_\mathrm{T}$ & $17$ & $\text{dBi}$\\
                & $G_\mathrm{T}$ & $50.12$ & \\
        EIRP & $EIRP$ & $59.5$ & $\text{dBm}$\\
            & $EIRP$ & $891.25$ & $\text{W}$\\
        Bandbreite & $B$ & $500$ $\text{kHz}$
    \end{tabular}
    \caption{Bestimmte Parameter des Schmalbandtransponder auf Es'Hail-2 (QO-100)}
    \label{tab:LinkBudet-EsHail-2}
\end{table}
Den zweiten Abschnitt bildet die Übertragungsstrecke zwischen Es'Hail-2 (QO-100) und der Bodenstation am IAT. In diesem Abschnitt treten nur Dämpfungen auf. Die Dämpfungen setzen sich aus der Freiraumdämpfung $L_\mathrm{FR}$ aus Gleichung \ref{eq:BestimmteFreiraumdämpfung}, der Verluste durch nicht optimale Ausrichtung $L_\mathrm{\theta T}$ aus Gleichung \ref{eq:Senderseitige-Fehlausrichtung} auf der Sendeseite und $L_\mathrm{\theta R}$ aus Gleichung \ref{eq:Empfängerseitige-Fehlausrichtung} auf der Empfangsseite. Die einzige nicht feste Dämpfung ist die Dämpfung durch die Atmosphäre. Diese unterscheidet sich je nach den festgelegten Bedingung klarer Himmel, leicher Regen und Regen. Die Dämpfung durch die Atmosphäre bei klaren Himmel $L_\mathrm{ATklarerHimmel}$ist in Gleichung \ref{eq:Dämpfung-in-der-Atmosphäre-klarer-Himmel} angegeben. Für die Bedingung leichter Regen $L_\mathrm{ATleicherRegen}$ in \ref{eq:Dämpfung-in-der-Atmosphäre-leichter-Regen} und für stärkere Niederschläge $L_\mathrm{ATRegen}$ in \ref{eq:Dämpfung-in-der-Atmosphäre-Regen}.
\begin{table}[H]
    \centering
    \begin{tabular}{c|c|c|c}
        Name & Variable & Wert & Einheit\\
        \hline
         Freiraumdämpfung& $L_\mathrm{FR}$ & $204.61$ & $\text{dB}$\\
        & $L_\mathrm{FR}$ & $2.9\cdot10^{20}$ & \\
         Sendeseite Fehlausrichtung& $L_\mathrm{\theta T}$ & $5.23$ & $\text{dB}$\\
        & $L_\mathrm{\theta T}$ & $3.33$ & \\
        Empfangsseitige Fehlausrichtung & $L_\mathrm{\theta R}$ & $0.69$ & $\text{dB}$\\
         & $L_\mathrm{\theta R}$ & $1.17$ & \\
         Dämpfung klarer Himmel & $L_\mathrm{ATklarerHimmel}$ & $0.547$ & $\text{dB}$\\
         & $L_\mathrm{ATklarerHimmel}$ & $1.13$ & \\
        Dämpfung leichter Regen & $L_\mathrm{ATleicherRegen}$ & $0.947$ & $\text{dB}$\\
         & $L_\mathrm{ATleicherRegen}$ & $1.24$ & \\
        Dämpfung Regen & $L_\mathrm{ATRegen}$ & $9.61$ & $\text{dB}$\\
         & $L_\mathrm{ATRegen}$ & $9.14$ & \\
    \end{tabular}
    \caption{Bestimmte Parameter des Abschnittes Übertragungsstrecke zwischen Es'Hail-2 (QO-100) und der Bodenstation am IAT}
    \label{tab:LinkBudet-Übertragungsstrecke}
\end{table}
Den letzten Abschnitt bildet die Bodenstation. Diese ist der Empfänger der Signale des Schmalbandtransponders von Es'Hail-2 (QO-100). Wichtige Parameter des Empfangssystems sind die empfangene Leistung $P_\mathrm{R}$ am Eingang des Empfangssystems, sowie die Leistung am Ausgang des Empfangssystems $P_\mathrm{RX}$. Die Leistung am Eingang des Empfangssystems $P_\mathrm{R}$ kann mit einer Ergänzung der Gleichung \ref{eq:Linkbudget} durch die Werte in den Tabellen \ref{tab:LinkBudet-EsHail-2} und \ref{tab:LinkBudet-Übertragungsstrecke}, sowie dem Gewinn der Empfangsantenne $G_\mathrm{R,max}$ ermittelt werden.
\begin{equation}
    P_\mathrm{R}=P_\mathrm{T}\cdot G_\mathrm{T}\cdot G_\mathrm{R}\cdot\frac{1}{L_\mathrm{FR}}\cdot\frac{1}{L_\mathrm{\theta T}}\cdot\frac{1}{L_\mathrm{\theta R}}\cdot\frac{1}{L_\mathrm{ATx}}
    \label{eq:empfangene-Leistung}
\end{equation}
Dabei ist die Dämpfung $L_\mathrm{ATx}$ die Dämpfung durch die Atmosphäre für die jeweilige Wetterbedingung. Diese Gleichung ist eine Erweiterung der Friis'sche Übertragungsgleichung, welche nur die Freiraumdämpfung berücksichtigt.\newline 
Die Leistung am Ausgang des Empfangssystems kann mit einer Erweiterung der Gleichung \ref{eq:empfangene-Leistung} um die Verstärkung des Empfangssystems ermittelt werden.
\begin{equation}
    P_\mathrm{RX}=P_\mathrm{T}\cdot G_\mathrm{T}\cdot G_\mathrm{R}\cdot\frac{1}{L_\mathrm{FR}}\cdot\frac{1}{L_\mathrm{\theta T}}\cdot\frac{1}{L_\mathrm{\theta R}}\cdot\frac{1}{L_\mathrm{ATx}}
    \label{eq:Ausgang-Leistung}
\end{equation}
Auch hier ist die Dämpfung $L_\mathrm{ATx}$ die Dämpfung durch die Atmosphäre für die jeweilige Wetterbedingung.\newline
Ebenfalls ist es wichtig die Qualität des Downlinks zu bestimmen. Die Qualität des Downlinks wird über $C/N_\mathrm{0}$ angegeben. Je größer $C/N_\mathrm{0}$ wird, desto mehr Leistung wird pro $\degree\text{K}$ Rauschen empfangen.\cite{Satellite_Communications_Systems}\newline
Dabei ist $C$ die Ausgangsleistung des Empfangssystems $P_\mathrm{RX}$ und $N_\mathrm{0}$ die Rauschleistungsdichte. Die Rauschleistungsdichte entspricht $n_\mathrm{0}$ aus Gleichung \ref{eq:PDS-Funktion}. Nur wird die Temperatur $T_\mathrm{0}$ durch die Rauschtemperatur $T_\mathrm{S}$ des Empfangssystems ersetzt.\cite{Satellite_Communications_Systems}\newline
\begin{equation}
    C/N_\mathrm{0}=\frac{P_\mathrm{RX}}{k\cdot T_\mathrm{S}}
    \label{eq:Qualität-Downlink}
\end{equation}
Die Rauschtemperatur des Empfangssystems $T_\mathrm{S}$ setzt sich aus der jeweiligen Antennentemperatur $T_\mathrm{A}$, der physikalischen Temperatur $T_\mathrm{0}$, der äquivalenten Rauschtemperatur des Empfangssystems $T_\mathrm{e,sys}$ und den Verlusten des Empfangssystems zusammen.\cite{Satellite_Communications_Systems}
\begin{equation}
    T_\mathrm{S}=\frac{T_\mathrm{A}}{L_\mathrm{sys}}+T_\mathrm{0}\left( 1-\frac{1}{L_\mathrm{sys}}\right)+T_\mathrm{e,sys}
    \label{eq:Rauschen-Temperatur-System}
\end{equation}
Auch ist es wichtig das $SNR$ am Eingang und Ausgang des Empfangssystems zu bestimmen. Mit dem $SNR$ können Aussagen zur Qualität des Empfangssystems und zur Qualität des Ausgangssignals getroffen und mögliche Fehlerrate in der Demodulation bestimmt werden.
\begin{figure}[H]
    \centering
    \includesvg[width=0.5\linewidth]{Bilder/SNR-Empfangssytem}
    \caption{Das $SNR$ am Eingang und Ausgang des Empfangssystems}
    \label{fig:SNR-Empfangssystem}
\end{figure}
Die Abbildung \ref{fig:SNR-Empfangssystem} zeigt das $SNR_\mathrm{i}$ am Eingang und das $SNR_\mathrm{0}$ am Ausgang des Empfangssystems.\newline 
Das $SNR_i$ am Eingang des Empfangssystems kann mit der Gleichung \ref{eq:SNR} bestimmt werden. Die Eingangsleistung $S_\mathrm{i}$ entspricht dabei der von der Antenne empfangenen Leistung $P_\mathrm{R}$. Die Rauschleistung am Eingang $N_\mathrm{i}$ ist von der jeweiligen Antennentemperatur $T_\mathrm{A}$ und der Bandbreite $B$ abhängig.
\begin{equation}
    SNR_\mathrm{i}=\frac{S_\mathrm{i}}{N_\mathrm{i}}=\frac{P_\mathrm{R}}{k\cdot T_\mathrm{A}\cdot B}
    \label{eq:SNR-Eingang-Empfangsystem}
\end{equation}
Das am $SNR_\mathrm{o}$ am Ausgang des Empfangssystems kann auch mit der Gleichung \ref{eq:SNR} bestimmt werden. Die Leistung am Ausgang $S_\mathrm{o}$ ist abhängig von Eingangsleistung $S_\mathrm{i}$ und der Verstärkung des Empfangssystems $G_\mathrm{sys}$ aus Gleichung \ref{eq:Gesamtverstärkung-Empfangssystems}. Die Rauschleistung am Ausgang $N_\mathrm{o}$kann mit der Gleichung \ref{eq:Rauschleistung-Ausgang-Zweitor} bestimmt werden. Diese setzt sich aus der jeweiligen Antennentemperatur $T_\mathrm{A}$, der Bandbreite $B$, der äquivalenten Rauschleistung des Empfangssystems $T_\mathrm{e,sys}$, sowie der Verstärkung des Empfangssystems $G_\mathrm{sys}$ aus Gleichung \ref{eq:Gesamtverstärkung-Empfangssystems} zusammen.
\begin{equation}
    SNR_\mathrm{o}=\frac{S_\mathrm{o}}{N_\mathrm{o}}=\frac{P_\mathrm{R}\cdot G_\mathrm{sys}}{k\cdot (T_\mathrm{A}+T_\mathrm{e,sys})\cdot B \cdot G_\mathrm{sys}}=\frac{P_\mathrm{R}}{k\cdot (T_\mathrm{A}+T_\mathrm{e,sys})\cdot B}
    \label{eq:SNR-Ausgang-Empfangsystem}
\end{equation}
Anhand von Gleichung \ref{eq:SNR-Ausgang-Empfangsystem} lässt sich zeigen, dass das $SNR_\mathrm{o}$ unabhängig von der Verstärkung $G_\mathrm{sys}$ des Empfangssystem ist. Es kann also nicht durch verstärken verbessert werden. Sollte das $SNR$ verbessert werden sollen, muss das über die Verringerung der Rauschleistung $N_\mathrm{o}$ geschehen. Je höher das $SNR_\mathrm{o}$ ist, desto besser können Signale von Es'Hail-2 (QO-100) vom Rauschen unterschieden und demoduliert werden. Im Falle von digitalen Modulationen sinkt die Bitfehlerrate $BER$, wie in Abbildung \ref{fig:BeispielBER} zu erkennen.\newline
In den folgenden Abschnitten wird das Link Budget, sowie die Qualität des Downlinks und das $SNR$ für die jeweiligen Wetterbedingungen klarer Himmel, leichter Regen und Regen betrachtet.\newline
Der für die Bestimmung der einzelnen Link Budgets verwendete Python Code ist im Github-Repository und im Anhang \ref{lst:Link-Budget-python} hinterlegt.
\subsubsection*{Link Budget und Link Qualität für die Bedingung klarer Himmel}
Die empfangene Leistung $P_\mathrm{R}$ wird mit der Gleichung \ref{eq:empfangene-Leistung} bestimmt und den Werten in den Tabellen \ref{tab:LinkBudet-EsHail-2} und \ref{tab:LinkBudet-Übertragungsstrecke} bestimmt. Für die Dämpfung durch die Atmosphäre $L_\mathrm{ATx}$ wird für die Wetterbedingung die in Gleichung \ref{eq:Dämpfung-in-der-Atmosphäre-klarer-Himmel}
bestimmten Dämpfung $L_\mathrm{ATklarerHimmel}=0.544\,\text{dB}=1.13$ eingesetzt. Der Gewinn der Empfangsantenne ist in Gleichung \ref{eq:Gewinn-der-Empfangsantenne} zu finden und beträgt $G_\mathrm{R,max}=38.6\,\text{dBi}=7244.36$.
\begin{equation}
\begin{split}
        P_\mathrm{R}&=EIRP\cdot G_\mathrm{R,max}\cdot\frac{1}{L_\mathrm{FR}}\cdot\frac{1}{L_\mathrm{\theta T}}\cdot\frac{1}{L_\mathrm{\theta R}}\cdot\frac{1}{L_\mathrm{ATklarerHimmel}}\\
        &=891.25\,\text{W}\cdot 7244.36\cdot\frac{1}{2.9\cdot10^{20}}\cdot\frac{1}{3.33}\cdot\frac{1}{0.69}\cdot\frac{1}{1.13}\\
        &=8.57\cdot 10^{-15}\,\text{W} =-110.67\,\text{dBm}
\end{split}
    \label{eq:empfangene-Leistung-klarer-Himmel}
\end{equation}
Der Pegel des von der Antenne empfangenen Signals ist mit $P_\mathrm{R}=-110.67\,\text{dBm}$ sehr schwach und liegt nur ca. $21\,\text{dB}$ über dem Grundrauschen des SDR. Das ist in Gleichung \ref{eq:Grundrauschen-SDR} mit $N_\mathrm{oSDR}=-131.66\,\text{dBm}$ angegeben.\newline
 Aus diesem Grund verstärkt der RF-Bereich des Empfangssystem das empfangene Signal weiter. Mit der Gleichung \ref{eq:Ausgang-Leistung} kann die Leistung am Ausgang des RF-Bereiches vom Empfangssystem bestimmt werden. Die Gesamtverstärkung des Systems ist in Gleichung \ref{eq:Gesamtverstärkung-Empfangssystems} mit $G_\mathrm{sys}=77.99\,\text{dB}=62.99\cdot10^{6}$ angegeben.
\begin{equation}
\begin{split}
        P_\mathrm{RX}&=EIRP\cdot G_\mathrm{R,max}\cdot G_\mathrm{sys}\cdot\frac{1}{L_\mathrm{FR}}\cdot\frac{1}{L_\mathrm{\theta T}}\cdot\frac{1}{L_\mathrm{\theta R}}\cdot\frac{1}{L_\mathrm{ATklarerHimmel}}\\
        &=891.25\,\text{W}\cdot 7244.36\cdot62.99\cdot10^{6}\cdot\frac{1}{2.9\cdot10^{20}}\cdot\frac{1}{3.33}\cdot\frac{1}{0.69}\cdot\frac{1}{1.13}\\
        &=5.4\cdot 10^{-7}\,\text{W} =-32.68\,\text{dBm}
\end{split}
    \label{eq:Ausgang-Leistung-klarer-Himmel}
\end{equation}
Durch die Verstärkung des Eingangssignals um $77.99\,\text{dB}$ beträgt die Leistung am Ausgang des RF-Bereiches $P_\mathrm{RX}=-32.68\,\text{dBm}$. Der Pegel des Signals liegt deutlich über dem Grundrauschen des SDR. Allerdings kommt durch das restlichen Empfangssystem noch zusätzliches Rauschen in den SDR, weshalb noch keine Aussage auf die mögliche Verarbeitung des Signals getroffen werden kann. Dafür muss noch das $SNR$ bestimmt werden.\newline
Das $SNR_\mathrm{i}$ am Eingang des Empfangssystems kann mit der Gleichung \ref{eq:SNR-Eingang-Empfangsystem} bestimmt werden. Die Antennentemperatur bei klaren Himmel ist in \ref{eq:Antennentemperatur-Bedingung-klarer-Himmel} mit $T_\mathrm{A,klarerHimmel}=6.5\,\text{K}$ angegeben. Am Anfang wird die Bandbreite $B$ mit der Bandbreite des Downlinks von $B=500\,\text{kHz}$ angenommen
\begin{equation}
\begin{split}
    SNR_\mathrm{i,klarerHimmel}&=\frac{P_\mathrm{R}}{k\cdot T_\mathrm{A,klarerHimmel}\cdot B}\\&=\frac{8.57\cdot 10^{-15}\,\text{W}}{1.38\cdot10^{-23}\,\frac{\text{J}}{\text{K}}\cdot6.5\,\text{K}\cdot500\,\text{kHz}}=195.09=22.9\,\text{dB}
\end{split}
    \label{eq:SNRi-klarer-Himmel-B500}
\end{equation}
Bei einer Bandbreite von $B=500\,\text{kHz}$ weißt das Empfangssystem am Eingang ein $SNR_\mathrm{i,klarerHimmel}=22.9\,\text{dB}$ auf. Das ist ein guter Wert, welcher noch Puffer für das zusätzliche Rauschen des RF-Bereiches vom Empfangssystem bieten sollte.\newline
 Das $SNR_\mathrm{o,klarerHimmel}$ kann über die Gleichung \ref{eq:SNR-Ausgang-Empfangsystem} bestimmt werden. Die äquivalente Rauschtemperatur des Empfangssystems ist in \ref{eq:äquivalente-Rauschtemperatur-Empfangsystem} mit $T_\mathrm{e,sys}=336.63\,\text{K}$ angegeben.
\begin{equation}
\begin{split}
    SNR_\mathrm{o,klarerHimmel}&=\frac{P_\mathrm{R}}{k\cdot (T_\mathrm{A,klarerHimmel}+T_\mathrm{e,sys})\cdot B}\\&=\frac{8.57\cdot 10^{-15}\,\text{W}}{1.38\cdot10^{-23}\,\frac{\text{J}}{\text{K}}\cdot(6.5\,\text{K}+336.63\,\text{K})\cdot500\,\text{kHz}}=3.67=5.68\,\text{dB}
\end{split}
    \label{eq:SNRo-klarer-Himmel-B500}
\end{equation}
Das $SNR_\mathrm{o,klarerHimmel}=5.68\,\text{dB}$ am Ausgang des RF-Bereiches ist sehr gering. Die empfangenen Signale von Es'Hail-2 (QO-100) könnten nur schwer vom Rauschen unterschieden werden. Der Grund dafür ist die hohe Rauschleistung im RF-Bereich des Empfangssystems, welches hauptsächlich durch die äquivalente Rauschtemperatur $T_\mathrm{e,sys}=336.63\,\text{K}$ und damit vom RF-Bereich des Empfangssystem selbst verursacht wird. Die einzige Möglichkeit das Rauschen zu reduzieren ist die Reduzierung der Bandbreite $B$. Diese wird im nächsten Schritt auf $B=25\,\text{kHz}$ reduziert.
\begin{equation}
\begin{split}
    SNR_\mathrm{i,klarerHimmel}&=\frac{P_\mathrm{R}}{k\cdot T_\mathrm{A,klarerHimmel}\cdot B}\\&=\frac{8.57\cdot 10^{-15}\,\text{W}}{1.38\cdot10^{-23}\,\frac{\text{J}}{\text{K}}\cdot6.5\,\text{K}\cdot25\,\text{kHz}}=3901,9=35.9\,\text{dB}
\end{split}
    \label{eq:SNRi-klarer-Himmel-B25}
\end{equation}
Durch die deutliche Reduzierung der Bandbreite $B$ steigt wie erwartet der Signal-zu-Rauschabstand. Am Eingang liegt ein $SNR_\mathrm{i,klarerHimmel}=35.9\,\text{dB}$ an, was einer Steigerung von $13\,\text{dB}$ entspricht. 
\begin{equation}
\begin{split}
    SNR_\mathrm{o,klarerHimmel}&=\frac{P_\mathrm{R}}{k\cdot (T_\mathrm{A,klarerHimmel}+T_\mathrm{e,sys})\cdot B}\\&=\frac{8.57\cdot 10^{-15}\,\text{W}}{1.38\cdot10^{-23}\,\frac{\text{J}}{\text{K}}\cdot(6.5\,\text{K}+336.63\,\text{K})\cdot25\,\text{kHz}}=73.91=18.69\,\text{dB}
\end{split}
    \label{eq:SNRo-klarer-Himmel-B25}
\end{equation}
Am Ausgang liegt ein $SNR_\mathrm{o,klarerHimmel}=18.69\,\text{dB}$ an. Im Vergleich zum $SNR_\mathrm{o,klarerHimmel}=5.68\,\text{dB}$ bei $B=500\,\text{kHz}$ entspricht das einer Steigerung von $13.01\,\text{dB}$. Durch diese Steigerung können die Signale im empfangenen Downlink von Es'Hail-2 (QO-100) vom Rauschen zu unterscheiden.\newline
Soll nur eine einzigen Übertragung empfangen werden, kann die Bandbreite $B$ auf die maximale Bandbreite eines Signals reduziert werden. Diese beträgt beim Schmalbandtransponder auf Es'Hail-2 (QO-100) $B=2.7\,\text{kHz}$.
\begin{equation}
\begin{split}
    SNR_\mathrm{i,klarerHimmel}&=\frac{P_\mathrm{R}}{k\cdot T_\mathrm{A,klarerHimmel}\cdot B}\\&=\frac{8.57\cdot 10^{-15}\,\text{W}}{1.38\cdot10^{-23}\,\frac{\text{J}}{\text{K}}\cdot6.5\,\text{K}\cdot2.7\,\text{kHz}}=35401.11=45.49\,\text{dB}
\end{split}
    \label{eq:SNRi-klarer-Himmel-B2.7}
\end{equation}
Durch die weitere Reduzierung der Bandbreite auf $B=2.7\,\text{kHz}$ steigt der Signal-zu-Rauschabstand auf $SNR_\mathrm{i,klarerHimmel}=45.49\,\text{dB}$. Verglichen mit mit dem $SNR_\mathrm{i,klarerHimmel}=22.9\,\text{dB}$ bei $B=500\,\text{kHz}$ ist das ein Anstieg um $22.59\,\text{dB}$. Gegenüber dem $SNR_\mathrm{i,klarerHimmel}=35.9\,\text{dB}$ bei $B=25\,\text{kHz}$ ist es ein Anstieg von $9.59\,\text{dB}$.\newline
\begin{equation}
\begin{split}
    SNR_\mathrm{o,klarerHimmel}&=\frac{P_\mathrm{R}}{k\cdot (T_\mathrm{A,klarerHimmel}+T_\mathrm{e,sys})\cdot B}\\&=\frac{8.57\cdot 10^{-15}\,\text{W}}{1.38\cdot10^{-23}\,\frac{\text{J}}{\text{K}}\cdot(6.5\,\text{K}+336.63\,\text{K})\cdot2.7\,\text{kHz}}=670.61=28.26\,\text{dB}
\end{split}
    \label{eq:SNRo-klarer-Himmel-B2.7}
\end{equation}
Durch das hohe $SNR_\mathrm{i,klarerHimmel}=45.49\,\text{dB}$ ist genug Puffer für das zusätzlichen Rauschen des RF-Bereiches vom Empfangssystems. Das $SNR_\mathrm{o,klarerHimmel}$ am Ausgang des RF-Bereiches beträgt $SNR_\mathrm{o,klarerHimmel}=28.26\,\text{dB}$. Das ist ein Anstieg von $22.58\,\text{dB}$ gegenüber $SNR_\mathrm{o,klarerHimmel}=5.68\,\text{dB}$ bei $B=500\,\text{kHz}$, bzw. ein Anstieg von $9.57\,\text{dB}$ gegenüber $SNR_\mathrm{o,klarerHimmel}=18.69\,\text{dB}$ bei $B=25\,\text{kHz}$.\newline
Mit einem Signal-zu-Rauschabstand von $ SNR_\mathrm{o,klarerHimmel}=28.26\,\text{dB}$ ist es problemlos möglich, das empfangene Signal zu demodulieren. Bei digitalen Modulationen bleibt die Bitfehlerrate $BER$ gering, wie es an Beispiel an einer n-QAM aus Abbildung \ref{fig:BeispielBER} entnommen werden kann.\newline
Die Qualität des Downlinks bei klaren Himmel kann mit der Gleichung \ref{eq:Qualität-Downlink} bestimmt werden. Zuvor muss noch die Rauschtemperatur $T_\mathrm{S}$ mit Gleichung \ref{eq:Rauschen-Temperatur-System} bestimmt werden. Die Verluste des Empfangssystems betragen $L_\mathrm{sys}=7.01\,\text{dB}=5.02$. Die Antennentemperatur beträgt $T_\mathrm{A,klarerHimmel}=6.5\,\text{K}$, die physikalische Temperatur $T_\mathrm{0}=290\,\text{K}$ und die äquivalente Rauschtemperatur $T_\mathrm{e,sys}=336.63\,\text{K}$. 
\begin{equation}
\begin{split}
     T_\mathrm{S}&=\frac{T_\mathrm{A,klarerHimmel}}{L_\mathrm{sys}}+T_\mathrm{0}\left(1-\frac{1}{L_\mathrm{sys}}\right)+T_\mathrm{e,sys}\\
     &=\frac{6.5\,\text{K}}{5.02}+290\,\text{K}\left(1-\frac{1}{5.02}\right)+336.63\,\text{K}\\
     &=570.16\,\text{K}
\end{split}
\label{eq:Rauschen-Temperatur-System-klarer-Himmel}
\end{equation}
Zusammen mit der Leistung am Ausgang des RF-Bereiches $P_\mathrm{RX}$ aus Gleichung \ref{eq:Ausgang-Leistung-klarer-Himmel} kann die Qualität des Downlinks bei einem klaren Himmel bestimmt werden.
\begin{equation}
C/N_\mathrm{o}=\frac{P_\mathrm{RX}}{k\cdot T_\mathrm{S}}=\frac{5.4\cdot 10^{-7}\,\text{W}}{1.38\cdot10^{-23}\,\frac{\text{J}}{\text{K}}\cdot570.16\,\text{K}}=6.86\cdot10^{13}\,\text{Hz}=138.37\,\text{dBHz}
 \label{eq:Qualität-Downlink-klarer-Himmel}
\end{equation}
Je höher der Wert der von $C/N_\mathrm{o}$ ist, desto besser ist die Qualität des Downlinks. Mit einem Wert von $C/N_\mathrm{o}=138.37\,\text{dBHz}$ ist die Qualität des Downlinks sehr gut. Verantwortlich für die hohe Qualität des Downlinks hauptsächlich der hohe Gewinn $G_\mathrm{R,max}$ der Empfangsantenne und die Verstärkung $G_\mathrm{sys}$ des Empfangssystems. Die Empfangsantenne hat dabei einen großen Einfluss auf die Qualität des Downlinks. Während die äquivalente Rauschtemperatur des Empfangssystems $T_\mathrm{e,sys}$ eher fest ist, kann je nach Elevationswinkel $\varepsilon$ kann das eingefangene Rauschen in Form der Antennentemperatur $T_\mathrm{A}$ stark variieren, was zu einer drastischen Verschlechterung der Qualität des Downlinks führen kann. In diesem Fall ist die Antennentemperatur $T_\mathrm{A,klarerHimmel}=6.5\,\text{K}$ im Vergleich zur äquivalenten Rauschtemperatur $T_\mathrm{e,sys}=336.63\,\text{K}$ des Empfangssystem sehr gering.
\begin{table}[H]
    \centering
    \begin{tabular}{c|c|c|c}
        Name & Variable & Wert & Einheit\\
        \hline
         Durchmesser Antenne& $d$                   & $0.9$             & $\text{m}$\\
        Physikalische Fläche& $A_\mathrm{phy}$      & $0.667$ &\text{m}^2 \\
         Effektive Fläche   & $A_\mathrm{E}$        & $0.472$            & $\text{m}^2$\\
                Effizienz   & $\eta_\mathrm{ANT}$   & $0.708$            & \\
                 Gewinn     & $G_\mathrm{R,max}$    & $38.6$            & $\text{dBi}$\\
                            & $G_\mathrm{R,max}$    & $7244.36$         & \\
    Antennentemperatur      & $T_\mathrm{A,klarerHimmel}$ & $6.5$       & $\text{K}$\\
    Empfangsgüte            & $G/T$                 & $21.11$       & $\text{1/K}$\\
                            & $G/T$                 & $13.24$       & $\text{dB/K}$ \\
    Empfangene Leistung     & $P_\mathrm{R}$        & $8.57\cdot 10^{-15}$     & $\text{W}$\\
                            & $P_\mathrm{R}$         & $-110.67$   & $\text{dBm}$ \\
    Verstärkung des Empfangssystem  & $G_\mathrm{sys}$        & $77.99$ & $\text{dB}$\\
                            & $G_\mathrm{sys}$      & $62.99\cdot 10^{6}$ \\
    Leistung am Ausgang     & $P_\mathrm{RX}$        & $5.4\cdot 10^{-7}$  & $\text{W}$\\
                            & $P_\mathrm{RX}$        & $-32.68$        & $\text{dBm}$\\
    Äquivalente Rauschzahl  & $T_\mathrm{e,sys}$    & $336.63$          & $\text{K}$ \\
    Bei $B=500\,\text{kHz}$ & & & \\
    $SNR$ am Eingang        & $SNR_\mathrm{i}$        &  $22.9$ & \text{dB}\\
    $SNR$ am Ausgang        & $SNR_\mathrm{o}$ & $5.68$         & \text{dB} \\
    Bei $B=25\,\text{kHz}$ & & & \\
    $SNR$ am Eingang        & $SNR_\mathrm{i}$        &  $35.9$ & \text{dB}\\
    $SNR$ am Ausgang        & $SNR_\mathrm{o}$ & $18.69$         & \text{dB} \\
    Bei $B=2.7\,\text{kHz}$ & & & \\
    $SNR$ am Eingang        & $SNR_\mathrm{i}$        &  $45.49$ & \text{dB}\\
    $SNR$ am Ausgang        & $SNR_\mathrm{o}$ & $28.26$         & \text{dB} \\
    Link Qualität           & $C/N_\mathrm{o}$ & $138.37$ & \text{dB/Hz} \\
    \end{tabular}
    \caption{Bestimmte Parameter der Bodenstation für das Link Budget bei klaren Himmel}
    \label{tab:LinkBudet-Bodenstation-klarer-Himmel}
\end{table}
Die Tabelle \ref{tab:LinkBudet-Bodenstation-klarer-Himmel} ist eine Übersicht der bestimmte Parameter der Bodenstation für das Link Budget bei klaren Himmel. Die physikalische Fläche $A_\mathrm{phy}$, die effektive Fläche $A_\mathrm{E}$ und die Effizienz der Antenne sind in den Gleichungen \ref{eq:physikalische-Fläche-der-Empfangsantenne},\ref{eq:effektive-Antennenfläche-der-Empfangsantenne} und \ref{eq:Effizienz-der-Antennenfläche-der-Empfangsantenne} angegeben.\newline
Die Empfangsgüte findet sich in Gleichung \ref{eq:Empfangsgüte-Bedingung-klarer-Himmel} und die Verstärkung des Empfangssystem ist in \ref{eq:Gesamtverstärkung-Empfangssystems} angegeben.
\begin{figure}[H]
    \centering
    \includegraphics[width=0.75\linewidth]{Bilder/LinkBudget_clear_Sky.png}
    \caption{Grafische Darstellung des Link Budgets bei klaren Himmel}
    \label{fig:Link-Budget-klarer-Himmel}
\end{figure}
Die Abbildung \ref{fig:Link-Budget-klarer-Himmel} repräsentiert das Link Budget bei klaren Himmel in übersichtlich in Form eines Graphen. Mit dem Link Budget wird eine Übersicht über die Verteilung der Dämpfungen und Verstärkungen in den unterschiedlichen Abschnitten des Downlinks gezeigt. Die größten Einfluss stellt dabei die Entfernung $D_\mathrm{SAT}$ zwischen Es'Hail-2 (QO-100) und der Bodenstation dar. Im Graphen wird diese durch die Freiraumdämpfung $L_\mathrm{FR}$ repräsentiert. Die Verluste durch die nicht optimale Ausrichtung von Sender und Empfänger zu einandere $L_\mathrm{\theta T}$ und $L_\mathrm{\theta R}$, sowie die Dämpfung durch Atmosphäre bei klaren Himmel $L_\mathrm{ATklarerHimmel}$ fallen, gegenüber der Freiraumdämpfung, eher weniger ins Gewicht.\newline
Die einzigen auftretenden Verstärkungen im Downlink sind auf der Sendeseite der Gewinn der Sendeantenne $G_\mathrm{T}$ und auf der Empfangsseite der Gewinn der Empfangsantenne $G_\mathrm{R,max}$ und die Verstärkung des Empfangssystems $G_\mathrm{sys}$.\newline
Das bestimmte Link Budget gilt für den größten Teil der Zeit. Es repräsentiert das best möglichste Link Budget für den betrachteten Downlink.







\subsubsection*{Link Budget und Link Qualität für die Bedingung leichter Regen}
Die bei der Wetterbedingung leichter Regen von der Antenne empfangene Leistung $P_\mathrm{R}$ lässt sich mithilfe der Gleichung \ref{eq:empfangene-Leistung} bestimmen. Die Dämpfung in der Atmosphäre bei leichten Regen ist in Gleichung \ref{eq:Dämpfung-in-der-Atmosphäre-leichter-Regen} mit $L_\mathrm{ATleichterRegen}=0.947\,\text{dB}=1.24$ angegeben.
\begin{equation}
\begin{split}
        P_\mathrm{R}&=EIRP\cdot G_\mathrm{R,max}\cdot\frac{1}{L_\mathrm{FR}}\cdot\frac{1}{L_\mathrm{\theta T}}\cdot\frac{1}{L_\mathrm{\theta R}}\cdot\frac{1}{L_\mathrm{leichterRegen}}\\
        &=891.25\,\text{W}\cdot 7244.36\cdot\frac{1}{2.9\cdot10^{20}}\cdot\frac{1}{3.33}\cdot\frac{1}{0.69}\cdot\frac{1}{1.24}\\
        &=7.81\cdot 10^{-15}\,\text{W} =-111.07\,\text{dBm}
\end{split}
    \label{eq:empfangene-Leistung-leichter-Regen}
\end{equation}
Im Vergleich zur empfangenen Leistung bei klaren Himmel $P_\mathrm{R}=-110.668\,\text{dBm}$ sinkt die empfangene Leistung bei leichten Regen auf $P_\mathrm{R}=-111.07\,\text{dBm}$, was einem Verlust von $0.402\,\text{dBm}$ oder $8.87\,\%$ entspricht. Das zeigt, dass auch kleinere Regenschauer eine deutliche Auswirkungen auf die empfangene Leistung haben.\newline
Da der Pegel des empfangene Signal mit $P_\mathrm{R}=-110.668\,\text{dBm}$ schwächer ist, als bei der Bedingung klarer Himmel, muss auch dieses verstärkt werden. Die Leistung am Ausgang des RF-Bereich des Empfangssystems kann mit der Gleichung \ref{eq:Ausgang-Leistung} bestimmt werden. Die Verstärkung des RF-Bereich ist in Gleichung \ref{eq:Gesamtverstärkung-Empfangssystems} mit $G_\mathrm{sys}=77.99\,\text{dB}=62.99\cdot10^{6}$ angegeben.
\begin{equation}
\begin{split}
        P_\mathrm{RX}&=EIRP\cdot G_\mathrm{R,max}\cdot G_\mathrm{sys}\cdot\frac{1}{L_\mathrm{FR}}\cdot\frac{1}{L_\mathrm{\theta T}}\cdot\frac{1}{L_\mathrm{\theta R}}\cdot\frac{1}{L_\mathrm{ATleichterRegen}}\\
        &=891.25\,\text{W}\cdot 7244.36\cdot62.99\cdot10^{6}\cdot\frac{1}{2.9\cdot10^{20}}\cdot\frac{1}{3.33}\cdot\frac{1}{0.69}\cdot\frac{1}{1.24}\\
        &=4.92\cdot 10^{-7}\,\text{W} =-33.07\,\text{dBm}
\end{split}
    \label{eq:Ausgang-Leistung-leichter-Regen}
\end{equation}
Nach der Verstärkung beträgt die Leistung am Ausgang des RF-Bereiches $P_\mathrm{RX}=-33.07\,\text{dBm}$. Der Verlust gegenüber der Ausgangsleistung bei klaren Himmel $P_\mathrm{RX}=-32.68\,\text{dBm}$ beträgt weiterhin ca. $0.4\,\text{dBm}$ oder $8.9\,\%$. Das ist bei gleichbleibender Verstärkung des Empfangssystems $G_\mathrm{sys}=77.99\,\text{dB}$ auch zu erwarten gewesen.\newline
Um die mögliche Demodulation und Verarbeitung der Signale im empfangenen Downlink zu überprüfen, muss der Signal-zu-Rauschabstand am Ein- und Ausgang des Empfangssystems bestimmt werden. Mit der Gleichung \ref{eq:SNR-Eingang-Empfangsystem} wird der Signal-zu-Rauschabstand $SNR_\mathrm{i}$ am Eingang des Empfangssystems bestimmt. Die Antennentemperatur für die Bedienung leichter Regen ist in Gleichung \ref{eq:Antennentemperatur-leichterRegen} mit $T_\mathrm{A,leichterRegen}=19.29\,\text{K}$ angegeben. Die Bandbreite wird am Anfang wieder mit $B=500\,\text{kHz}$ angenommen.
\begin{equation}
\begin{split}
    SNR_\mathrm{i,leichterRegen}&=\frac{P_\mathrm{R}}{k\cdot T_\mathrm{A,leichterRegen}\cdot B}\\&=\frac{7.81\cdot 10^{-15}\,\text{W}}{1.38\cdot10^{-23}\,\frac{\text{J}}{\text{K}}\cdot19.29\,\text{K}\cdot500\,\text{kHz}}\\
    &=58.68=17.68\,\text{dB}
\end{split}
    \label{eq:SNRi-leichter-Regen-B500}
\end{equation}
Im Vergleich zum $SNR_\mathrm{i,klarerHimmel}=22.9\,\text{dB}$ bei $B=500\,\text{kHz}$ ist der Signal-zu-Rauschabstand am Eingang des Empfangssystems um $5.22\,\text{dB}$ auf $SNR_\mathrm{i,leicherRegen}=17.68\,\text{dB}$ gesunken. Gründe dafür sind die, durch leichte Regenschauer, gedämpfte Empfangsleistung $P_\mathrm{R}$ und die, durch die leichten Regenschauer, erhöhte eingefangene Rauschleistung. Diese wird durch die höhere Antennentemperatur $T_\mathrm{A,leichterRegen}=19.29\,\text{K}$ repräsentiert.\newline
Ein Signal-zu-Rauschabstand von $SNR_\mathrm{i,leichterRegen}=17.69\,\text{dB}$ ist im ersten Moment ein zufriedenstellender Wert. Jedoch kann es mit der erhöhten Rauschleistung vom RF-Bereiches des Empfangssystem, repräsentiert durch die äquivalente Rauschleistung $T_\mathrm{e,sys}$, eng werden. Um Aussage um die Unterscheidbarkeit zwischen Signal und Rauschen treffen zu können, muss der Signal-zu-Rauschabstand $SNR_\mathrm{o,leichterRegen}$ am Ausgang des RF-Bereiches vom Empfangssystem bestimmt werden. Dieses kann mit der Gleichung \ref{eq:SNR-Ausgang-Empfangsystem} bestimmt werden. Die äquivalente Rauschtemperatur $T_\mathrm{e,sys}$ ist in Gleichung \ref{eq:äquivalente-Rauschtemperatur-Empfangsystem} angegeben.
\begin{equation}
\begin{split}
    SNR_\mathrm{o,leichterRegen}&=\frac{P_\mathrm{R}}{k\cdot (T_\mathrm{A,leichterRegen}+T_\mathrm{e,sys})\cdot B}\\&=\frac{7.81\cdot 10^{-15}\,\text{W}}{1.38\cdot10^{-23}\,\frac{\text{J}}{\text{K}}\cdot(19.29\,\text{K}+336.63\,\text{K})\cdot500\,\text{kHz}}\\
    &=3.18=5.02\,\text{dB}
\end{split}
    \label{eq:SNRo-leichter-Regen-B500}
\end{equation}
Wie bereits vermutet verschlechtert sich der Signal-zu-Rauschabstand durch den RF-Bereich um $12.66\,\text{dB}$ auf $SNR_\mathrm{o,leichterRegen}=5.02\,\text{dB}$. Der Grund für die Verschlechterung des Signal-zu-Rauschabstand ist die Rauschleistung des RF-Bereiches vom Empfangssystem, welche mit der äquivalenten Rauschtemperatur $T_\mathrm{e,sys}=336.63\,\text{K}$ angegeben wird.\newline
Der Signal-zu-Rauschabstand ist, ähnlich wie bei der Bedienung klarer Himmel, mit$SNR_\mathrm{o,leichterRegen}=5.02\,\text{dB}$ zu niedrig, um verlässlich die Signale vom Rauschen zu unterscheiden. Aus diesem Grund kann die Bandbreite $B$ auf $B=25\,\text{kHz}$ reduziert werden, um so die Rauschleistung im Empfangssystem zu verringern.
\begin{equation}
\begin{split}
    SNR_\mathrm{i,leichterRegen}&=\frac{P_\mathrm{R}}{k\cdot T_\mathrm{A,leichterRegen}\cdot B}\\&=\frac{7.81\cdot 10^{-15}\,\text{W}}{1.38\cdot10^{-23}\,\frac{\text{J}}{\text{K}}\cdot19.29\,\text{K}\cdot25\,\text{kHz}}\\
    &=1136.42=30.56\,\text{dB}
\end{split}
    \label{eq:SNRi-leichter-Regen-B25}
\end{equation}
Durch die Reduzierung der Bandbreite $B$ auf $B=25\,\text{kHz}$ sinkt das eingefangene Rauschen, wodurch der Signal-zu-Rauschabstand um $12.88\,\text{dB}$ auf $SNR_\mathrm{i,leichterRegen}=30.56\,\text{dB}$ steigt.\newline
Verglichen mit dem Signal-zu-Rauschabstand bei $B=25\,\text{kHz}$ und klaren Himmel $SNR_\mathrm{i,klarerHimmel}=35.9\,\text{dB}$ ist das $SNR_\mathrm{i,leichterRegen}=30.56\,\text{dB}$ um $5.34\,\text{dB}$ geringer, was einem Verlust des Singal-zu-Rauschabstand von $70.87\,\%$ bedeutet. Dennoch sollte der Signal-zu-Rauschabstand von $SNR_\mathrm{i,leichterRegen}=30.56\,\text{dB}$ bei $B=25\,\text{kHz}$ jetzt genug Puffer für die hohe Eigenrauschleistung des RF-Bereiches vom Empfangssystem bieten.
\begin{equation}
\begin{split}
    SNR_\mathrm{o,leichterRegen}&=\frac{P_\mathrm{R}}{k\cdot (T_\mathrm{A,leichterRegen}+T_\mathrm{e,sys})\cdot B}\\&=\frac{7.81\cdot 10^{-15}\,\text{W}}{1.38\cdot10^{-23}\,\frac{\text{J}}{\text{K}}\cdot(19.29\,\text{K}+336.63\,\text{K})\cdot25\,\text{kHz}}\\
    &=63.6=18.03\,\text{dB}
\end{split}
    \label{eq:SNRo-leichter-Regen-B25}
\end{equation}
Mit einem Signal-zu-Rauschabstand von $SNR_\mathrm{o,leichterRegen}=18.03\,\text{dB}$ können die Signale im empfangenen Downlink von Es'Hail-2 (QO-100) verlässlich vom Rauschen unterschieden werden. Im Vergleich zum Signal-zu-Rauschabstand bei $B=25\,\text{kHz}$ und klaren Himmel $SNR_\mathrm{o,klarerHimmel}=18.69\,\text{dB}$ ist dieser um $0.6\,\text{dB}$ gesunken, was einem Verlust von $13.9\,\%$ entspricht.\newline
Für den Empfang eines einzelnen Signals vom Schmalbandtransponder von Es'Hail-2 (QO-100) wird die Bandbreite $B$ auf $B=2.7\,\text{kHz}$ reduziert. Dadurch wird der Signal-zu-Rauschabstand am Ein- und Ausgang des Empfangssystem weiter ansteigen.
\begin{equation}
\begin{split}
    SNR_\mathrm{i,leichterRegen}&=\frac{P_\mathrm{R}}{k\cdot T_\mathrm{A,leichterRegen}\cdot B}\\&=\frac{7.81\cdot 10^{-15}\,\text{W}}{1.38\cdot10^{-23}\,\frac{\text{J}}{\text{K}}\cdot19.29\,\text{K}\cdot2.7\,\text{kHz}}=10870.63=40.36\,\text{dB}
\end{split}
    \label{eq:SNRi-klarer-Himmel-B2.7}
\end{equation}
Wie erwartet steigt durch die verringerter Bandbreite $B=2.7\,\text{kHz}$ der Signal-zu-Rauschabstand am Eingang des Empfangssystems auf $SNR_\mathrm{i,leichterRegen}=17.68\,\text{dB}$ an. Verglichen mit dem Signal-zu-Rauschabstand von $SNR_\mathrm{i,leichterRegen}=17.68\,\text{dB}$ bei $B=500\,\text{kHz}$ ist es um $22.68\,\text{dB}$ angestiegen. Gegenüber dem Signal-zu-Rauschabstand  $SNR_\mathrm{i,leichterRegen}=30.56\,\text{dB}$ bei $B=25\,\text{kHz}$ ist es um $9.8\,\text{dB}$ angestiegen.\newline
Vergleicht man den Signal-zu-Rauschabstand mit dem vom klaren Himmel $SNR_\mathrm{i,klarerHimmel}=45.49\,\text{dB}$ ist es jedoch um $5.13\,\text{dB}$ gesunken. Das entspricht einem Verlust des Signal-zu-Rauschabstand von $69.29\,\%$.\newline
Durch den deutlich höheren Signal-zu-Rauschabstand am Eingang des Empfangssystems sollte auch der Signal-zu-Rauschabstand am Ausgang vom RF-Bereich deutlich ansteigen.
\begin{equation}
\begin{split}
    SNR_\mathrm{o,leichterRegen}&=\frac{P_\mathrm{R}}{k\cdot (T_\mathrm{A,leichterRegen}+T_\mathrm{e,sys})\cdot B}\\&=\frac{7.81\cdot 10^{-15}\,\text{W}}{1.38\cdot10^{-23}\,\frac{\text{J}}{\text{K}}\cdot(19.29\,\text{K}+336.63\,\text{K})\cdot2.7\,\text{kHz}}\\
    &=589.16=27.7\,\text{dB}
\end{split}
    \label{eq:SNRo-leichter-Regen-B2.7}
\end{equation}
Wie erwartet ist der Signal-zu-Rauschabstand am Ausgang ebenfalls angestiegen. Dieser beträgt $SNR_\mathrm{o,leichterRegen}=27.7\,\text{dB}$, was einer Steigerung von $24.52\,\text{dB}$ gegenüber dem Signal-zu-Rauschabstand $SNR_\mathrm{o,leichterRegen}=3.18\,\text{dB}$ bei $B=500\,\text{kHz}$, bzw. einem Anstieg um $9.67\,\text{dB}$ gegenüber $SNR_\mathrm{o,leichterRegen}=18.03\,\text{dB}$ bei $B=25\,\text{kHz}$ entspricht.\newline
Verglichen mit dem Signal-zu-Rauschabstand bei klaren Himmel $SNR_\mathrm{o,klarerHimmel}=28.26\,\text{dB}$ ist es jedoch ein Verlust von $0.56\,\text{dB}$ oder $12.2\,\%$.\newline 
Der leichte Regen wird auch einen negativen Einfluss auf die Qualität des Downlinks haben. Bestimmt werden kann diese mithilfe der Gleichung \ref{eq:Qualität-Downlink}. Die dafür benötigte Rauschtemperatur $T_\mathrm{S}$ kann mithilfe \ref{eq:Rauschen-Temperatur-System} ermittelt werden. Die Antennentemperatur bei leichten Regen ist in Gleichung \ref{eq:Antennentemperatur-leichterRegen} mit $T_\mathrm{A,leichterRegen}=19.29\,\text{K}$ angegeben. Die Verluste des Empfangssystems betragen $L_\mathrm{sys}=7.01\,\text{dB}=5.02$. Die physikalische Temperatur beträgt $T_\mathrm{0}=290\,\text{K}$ und die äquivalente Rauschtemperatur des Empfangssystems kann aus Gleichung \ref{eq:äquivalente-Rauschtemperatur-Empfangsystem} mit $T_\mathrm{e,sys}=336.63\,\text{K}$ entnommen werden.
\begin{equation}
\begin{split}
     T_\mathrm{S}&=\frac{T_\mathrm{A,leichterRegen}}{L_\mathrm{sys}}+T_\mathrm{0}\left(1-\frac{1}{L_\mathrm{sys}}\right)+T_\mathrm{e,sys}\\
     &=\frac{19.29\,\text{K}}{5.02}+290\,\text{K}\left(1-\frac{1}{5.02}\right)+336.63\,\text{K}\\
     &=572.71\,\text{K}
\end{split}
\label{eq:Rauschen-Temperatur-System-leichter-Regen}
\end{equation}
Die Rauschtemperatur $T_\mathrm{S}$ ist im Vergleich zur Rauschtemperatur bei klaren Himmel aus Gleichung \ref{eq:Rauschen-Temperatur-System-klarer-Himmel} um $2.55\,\text{K}$ gestiegen. Dieser Anstieg der Rauschtemperatur wird sich negativ auf die Qualität des Downlinks auswirken. Die Leistung am Ausgang des RF-Bereiches vom Empfangssystem wird mit $P_\mathrm{RX}=4.92\cdot 10^{-7}\,\text{W}$ in Gleichung \ref{eq:Ausgang-Leistung-leichter-Regen} angegeben.
\begin{equation}
C/N_\mathrm{o}=\frac{P_\mathrm{RX}}{k\cdot T_\mathrm{S}}=\frac{4.92\cdot 10^{-7}\,\text{W}}{1.38\cdot10^{-23}\,\frac{\text{J}}{\text{K}}\cdot572.71\,\text{K}}=6.23\cdot10^{13}\,\text{Hz}=137.94\,\text{dBHz}
 \label{eq:Qualität-Downlink-leichter-Regen}
\end{equation}
Wie erwartet ist die Qualität des Downlinks bei leichten Regen in Gleichung \ref{eq:Qualität-Downlink-leichter-Regen} gegenüber der Qualität des Downlinks bei klaren Himmel in Gleichung \ref{eq:Qualität-Downlink-klarer-Himmel} um $0.43\,\text{dBHz}$ gesunken. Dafür ist die erhöhte Rauschtemperatur $T_\mathrm{S}$ verantwortlich. Trotz allem ist die Qualität des Downlinks, auch bei leichten Regenschauern, sehr zufriedenstellend.
\begin{table}[H]
    \centering
    \begin{tabular}{c|c|c|c}
        Name & Variable & Wert & Einheit\\
        \hline
         Durchmesser Antenne& $d$                   & $0.9$             & $\text{m}$\\
        Physikalische Fläche& $A_\mathrm{phy}$      & $0.667$ &\text{m}^2 \\
         Effektive Fläche   & $A_\mathrm{E}$        & $0.472$            & $\text{m}^2$\\
                Effizienz   & $\eta_\mathrm{ANT}$   & $0.708$            & \\
                 Gewinn     & $G_\mathrm{R,max}$    & $38.6$            & $\text{dBi}$\\
                            & $G_\mathrm{R,max}$    & $7244.36$         & \\
    Antennentemperatur      & $T_\mathrm{A,leichterRegen}$ & $19.29$       & $\text{K}$\\
    Empfangsgüte            & $G/T$                 & $20.35$       & $\text{1/K}$\\
                            & $G/T$                 & $13.12$       & $\text{dB/K}$ \\
    Empfangene Leistung     & $P_\mathrm{R}$        & $7.81\cdot 10^{-15}$     & $\text{W}$\\
                            & $P_\mathrm{R}$         & $-111.07$   & $\text{dBm}$ \\
    Verstärkung des Empfangssystem  & $G_\mathrm{sys}$        & $77.99$ & $\text{dB}$\\
                            & $G_\mathrm{sys}$      & $62.99\cdot 10^{6}$ \\
    Leistung am Ausgang     & $P_\mathrm{RX}$        & $4.92\cdot 10^{-7}$  & $\text{W}$\\
                            & $P_\mathrm{RX}$        & $-33.07$        & $\text{dBm}$\\
    Äquivalente Rauschzahl  & $T_\mathrm{e,sys}$    & $336.63$          & $\text{K}$ \\
    Bei $B=500\,\text{kHz}$ & & & \\
    $SNR$ am Eingang        & $SNR_\mathrm{i}$        &  $17.68$ & \text{dB}\\
    $SNR$ am Ausgang        & $SNR_\mathrm{o}$ & $3.18$         & \text{dB} \\
    Bei $B=25\,\text{kHz}$ & & & \\
    $SNR$ am Eingang        & $SNR_\mathrm{i}$        &  $30.56$ & \text{dB}\\
    $SNR$ am Ausgang        & $SNR_\mathrm{o}$ & $18.03$         & \text{dB} \\
    Bei $B=2.7\,\text{kHz}$ & & & \\
    $SNR$ am Eingang        & $SNR_\mathrm{i}$        &  $40.36$ & \text{dB}\\
    $SNR$ am Ausgang        & $SNR_\mathrm{o}$ & $27.7$         & \text{dB} \\
    Link Qualität           & $C/N_\mathrm{o}$ & $137.94$ & \text{dB/Hz} \\
    \end{tabular}
    \caption{Bestimmte Parameter der Bodenstation für das Link Budget bei leichten Regen}
    \label{tab:LinkBudet-Bodenstation-leichter-Regen}
\end{table}
In der Tabelle \ref{tab:LinkBudet-Bodenstation-leichter-Regen} sind die Parameter der Bodenstation für das Link Budget bei leichten Regenschauern zur besseren Übersicht dargestellt. 
\begin{figure}[H]
    \centering
    \includegraphics[width=0.75\linewidth]{Bilder/LinkBudget_light_Rain.png}
    \caption{Grafische Darstellung des Link Budgets bei klaren Himmel}
    \label{fig:Link-Budget-leichter_Regen}
\end{figure}
Der Graph in Abbildung \ref{fig:Link-Budget-leichter_Regen} repräsentiert das Link Budget für den Downlink bei leichten Regenschauern. Es gilt für Regenschauer mit einer Niederschlagsmenge $R_\mathrm{p}$ $(\text{mm/h})$, welche den Jahresdurchschnitt $(\text{mm/h})$ zu $p\leq5\,\%$ der Zeit überschreitet.\newline
Gegenüber dem Link Budget bei einem klarer Himmel hat sich im Abschnitt der Übertragungsstrecke zwischen Es'Hail-2 (QO-100) die Dämpfung in der Atmosphäre $L_\mathrm{ATx}$ geändert. Durch die höhere Dämpfung $L_\mathrm{ATleichterRegen}=0.947\,\text{dB}$ verringert sich die empfangene Leistung $P_\mathrm{R}$ am Eingang des Empfangssystems. Dadurch ist auch die Leistung am Ausgang des RF-Bereiches $P_\mathrm{RX}$ geringer. \newline 
Ebenfalls hat sich auch der Signal-zu-Rauschabstand am Ein- und Ausgang des Empfangssystems leicht reduziert. Bei einer Bandbreite von $B=500\,\text{kHz}$ ist es am Eingang von $SNR_\mathrm{i,klarerHimmel}=22.9\,\text{dB}$ auf $SNR_\mathrm{i,leichterRegen}=17.68\,\text{dB}$ gesunken. Am Ausgang hat sich der Signal-zu-Rauschabstand von $SNR_\mathrm{o,klarerHimmel}=5.68\,\text{dB}$ auf $SNR_\mathrm{o,klarerHimmel}=5.02\,\text{dB}$ reduziert. Eine Aufrechterhaltung des Downlinks ist bei beiden Wetterbedingung mit einer Bandbreite von $B=500\,\text{kHz}$ nicht möglich. Bei einer Bandbreite $B=25\,\text{kHz}$ reduziert sich der Signal-zu-Rauschabstand am Eingang von $SNR_\mathrm{i,klarerHimmel}=35.9\,\text{dB}$ auf $SNR_\mathrm{i,leichterRegen}=30.56\,\text{dB}$. Am Ausgang wiederum von $SNR_\mathrm{o,klarerHimmel}=18.69\,\text{dB}$ auf $SNR_\mathrm{o,leichterRegen}=18.03\,\text{dB}$ verschlechtert. Damit ist eine Aufrechterhaltung des Downlinks auch bei leichten Regenschauern möglich.\newline
Beim empfangen eines einzigen Signals beträgt die Bandbreite $B=2.7\,\text{kHz}$. Bei leichten Regenfälle reduziert sich dabei der Signal-zu-Rauschabstand am Eingang des Empfangssystems von $SNR_\mathrm{i,klarerHimmel}=45.49\,\text{dB}$ auf $SNR_\mathrm{i,leichterRegen}=40.36\,\text{dB}$. Am Ausgang des RF-Bereiches reduziert es sich von $SNR_\mathrm{o,klarerHimmel}=28.26\,\text{dB}$ auf $SNR_\mathrm{o,leichterRegen}=27.7\,\text{dB}$









\subsubsection*{Link Budget und Link Qualität für die Bedingung Regen}
Die letzte betrachtete Bedingung ist die Wetterbedingung Regen. In dieser wird die Dämpfung $L_\mathrm{Regen}$ durch starke Niederschläge berücksichtigt.\newline
Die unter dieser Bedingung von der Antenne empfangene Leistung $P_\mathrm{R}$ kann mithilfe der Gleichung \ref{eq:empfangene-Leistung} bestimmt werden. Die Dämpfung in der Atmosphäre ist in Gleichung \ref{eq:Dämpfung-in-der-Atmosphäre-Regen} mit $L_\mathrm{ATRegen}=9.61\,\text{dB}=9.14$ angegeben.
\begin{equation}
\begin{split}
        P_\mathrm{R}&=EIRP\cdot G_\mathrm{R,max}\cdot\frac{1}{L_\mathrm{FR}}\cdot\frac{1}{L_\mathrm{\theta T}}\cdot\frac{1}{L_\mathrm{\theta R}}\cdot\frac{1}{L_\mathrm{Regen}}\\
        &=891.25\,\text{W}\cdot 7244.36\cdot\frac{1}{2.9\cdot10^{20}}\cdot\frac{1}{3.33}\cdot\frac{1}{0.69}\cdot\frac{1}{9.14}\\
        &=1.06\cdot 10^{-15}\,\text{W} =-119.74\,\text{dBm}
\end{split}
    \label{eq:empfangene-Leistung-Regen}
\end{equation}
Anhand des Ergebnisses in Gleichung \ref{eq:empfangene-Leistung-Regen} kann gesagt werden, dass starke Niederschläge einen großen Einfluss auf die empfangene Leistung $P_\mathrm{R}$ haben. Im Vergleich zur empfangenen Leistung bei klaren Himmel $P_\mathrm{R}=-110.67\,\text{dBm}$ ist die empfangene Leistung bei starken Niederschlägen um $87.63\,\%$ auf $P_\mathrm{R}=-119.74\,\text{dBm}$
gesunken. Im Vergleich zur empfangenen Leistung bei leichten Niederschlägen $P_\mathrm{R}=-111.07\,\text{dBm}$ ist diese um $86.4\,\%$ geringer.\newline
Der Pegel $P_\mathrm{RX}$ des Signals am Ausgang des RF-Bereiches kann mit der Gleichung \ref{eq:Ausgang-Leistung} bestimmt werden. Die Verstärkung des Systems ist in Gleichung \ref{eq:Gesamtverstärkung-Empfangssystems} angegeben und beträgt $G_\mathrm{sys}=77.99\,\text{dB}=62.99\cdot 10^{6}$.
\begin{equation}
\begin{split}
        P_\mathrm{RX}&=EIRP\cdot G_\mathrm{R,max}\cdot G_\mathrm{sys}\cdot\frac{1}{L_\mathrm{FR}}\cdot\frac{1}{L_\mathrm{\theta T}}\cdot\frac{1}{L_\mathrm{\theta R}}\cdot\frac{1}{L_\mathrm{ATRegen}}\\
        &=891.25\,\text{W}\cdot 7244.36\cdot62.99\cdot10^{6}\cdot\frac{1}{2.9\cdot10^{20}}\cdot\frac{1}{3.33}\cdot\frac{1}{0.69}\cdot\frac{1}{9.14}\\
        &=6.68\cdot 10^{-8}\,\text{W} =-41.75\,\text{dBm}
\end{split}
    \label{eq:Ausgang-Leistung-Regen}
\end{equation}
Auch nach der Verstärkung des sehr schwachen Signals bleibt der Pegel mit $P_\mathrm{RX}=-41.75\,\text{dBm}$ niedrig. Bei einem klaren Himmel kann eine Ausgang des RF-Bereiches eine Leistung von $P_\mathrm{RX}=-32.68\,\text{dBm}$, bzw. $P_\mathrm{RX}=-33.07\,\text{dBm}$ bei leichten Regenschauern, erreicht.\newline
Als nächstes kann der Signal-zu-Rauschabstand und damit auch die Rauschleistung im RF-Bereich des Empfangssystem ermittelt werden. So kann eine Aussage darüber getroffen werden, ob der Downlink von Es'Hail-2 (QO-100) Aufrechterhalten werden kann. Der Signal-zu-Rauschabstand $SNR_\mathrm{i,Regen}$ am Eingang des Empfangssystem kann mit der Gleichung \ref{eq:SNR-Eingang-Empfangsystem} bestimmt werden. Die Antennentemperatur bei starken Niederschlägen ist in \ref{eq:Antennentemperatur-bei-Regen} mit $T_\mathrm{A,Regen}=240.1\,\text{K}$. Aufgrund der hohen Antennentemperatur ist mit einem deutlich Anstieg der eingefangenen Rauschleistung zu rechnen. Die Bandbreite $B$ wird wie bei den anderen Bedienungen zuvor auch mit der Breite des Downlinks angenommen, bedeutet $B=500\,\text{kHz}$
\begin{equation}
\begin{split}
    SNR_\mathrm{i,Regen}&=\frac{P_\mathrm{R}}{k\cdot T_\mathrm{A,Regen}\cdot B}\\&=\frac{1.06\cdot 10^{-15}\,\text{W}}{1.38\cdot10^{-23}\,\frac{\text{J}}{\text{K}}\cdot240.1\,\text{K}\cdot500\,\text{kHz}}\\
    &=0.64=-1.94\,\text{dB}
\end{split}
    \label{eq:SNRi-Regen-B500}
\end{equation}
Wie erwartet hat sich die eingefangene Rauschleistung deutlich erhöht, was zu einer deutlichen Reduzierung des Signal-zu-Rauschabstand führt. Verglichen mit dem Signal-zu-Rauschabstand bei klaren Himmel und  $B=500\,\text{kHz}$ Bandbreite $SNR_\mathrm{i,klarerHimmel}=22.9\,\text{dB}$ ist der Signal-zu-Rauschabstand um $24.84\,\text{dB}$ abgefallen auf $SNR_\mathrm{i,Regen}=-1,94\,\text{dB}$. Gegenüber dem Signal-zu-Rauschabstand bei leichten Regen $SNR_\mathrm{i,leichterRegen}=17.68\,\text{dB}$ um $19.62\,\text{dB}$. Ein negativer Signal-zu-Rauschabstand bedeutet, dass die Signale im empfangenen Downlink komplett vom Rauschen überlagert werden und darin untergehen. Eine Aufrechterhaltung des Downlink ist so nicht möglich. Die Gründe für den schlechten Signal-zu-Rauschabstand sind die deutlich geringere empfangene Leistung $P_\mathrm{R}=1.06\cdot10^{-15}\,\text{W}$ und das deutlich höhere empfangene Rauschen, repräsentiert durch die Antennentemperatur $T_\mathrm{A,Regen}=240.1\,\text{K}$. Beides wird durch die hohe Dämpfung durch starke Niederschläge
$L_\mathrm{Regen}=8.86\,\text{dB}$ hervorgerufen.\newline
Auch wenn das Signal-zu-Rauschabstand am Ausgang des RF-Bereich vom Emfpangssystem bei $B=500\,\text{kHz}$ kein positives Ergebnis bringen wird, wird es der Vollständigkeit zu gute trotzdem bestimmt. Dieses kann mit der Gleichung \ref{eq:SNR-Ausgang-Empfangsystem} bestimmt werden. Die äquivalente Rauschtemperatur des Empfangssystems ist in \ref{eq:äquivalente-Rauschtemperatur-Empfangsystem} mit $T_\mathrm{e,sys}=336.63\,\text{K}$ angegeben. 
\begin{equation}
\begin{split}
    SNR_\mathrm{o,Regen}&=\frac{P_\mathrm{R}}{k\cdot (T_\mathrm{A,Regen}+T_\mathrm{e,sys})\cdot B}\\&=\frac{1.06\cdot 10^{-15}\,\text{W}}{1.38\cdot10^{-23}\,\frac{\text{J}}{\text{K}}\cdot(240.1\,\text{K}+336.63\,\text{K})\cdot500\,\text{kHz}}\\
    &=0.27=-5.69\,\text{dB}
\end{split}
    \label{eq:SNRo-Regen-B500}
\end{equation}
Wie erwartet hat sich der Signal-zu-Rauschabstand am Ausgang des RF-Bereiches weiter verschlechtert. Im Vergleich zum Signal-zu-Rauschabstand bei klaren Himmel und $500\,\text{kHz}$ Bandbreite $SNR_\mathrm{o,klarerHimmel}=5.68\,\text{dB}$ hat sich es um $11.37\,\text{dB}$ verschlechtert auf 
$SNR_\mathrm{o,Regen}=-5.69\,\text{dB}$. Gegenüber dem Signal-zu-Rauschabstand bei leichten Regenfällen $SNR_\mathrm{o,leichterRegen}=3.18\,\text{dB}$ hat es sich um $8.87\,\text{dB}$ verschlechtert.\newline
Wie zuvor auch kann zur Reduzierung der Rauschleistung die Bandbreite $B$ auf $B=25\,\text{dB}$ reduziert werden. 
\begin{equation}
\begin{split}
    SNR_\mathrm{i,Regen}&=\frac{P_\mathrm{R}}{k\cdot T_\mathrm{A,Regen}\cdot B}\\&=\frac{1.06\cdot 10^{-15}\,\text{W}}{1.38\cdot10^{-23}\,\frac{\text{J}}{\text{K}}\cdot240.1\,\text{K}\cdot25\,\text{kHz}}\\
    &=12.8=11.07\,\text{dB}
\end{split}
    \label{eq:SNRi-Regen-B25}
\end{equation}
Eine Reduzierung der Bandbreite auf $B=25\,\text{kHz}$ hebt den Signal-zu-Rauschabstand aif $SNR_\mathrm{i,Regen}=11.07\,\text{dB}$ an. Das entspricht einem Anstieg von $9.13\,\text{dB}$.
Verglichen mit dem Signal-zu-Rauschabstand bei klaren Himmel und einer Bandbreite von $B=25\,\text{kHz}$ $SNR_\mathrm{i,klarerHimmel}=35.9\,\text{dB}$, ist es um $24.83\,\text{dB}$ geringer. Das entspricht einem Verlust des Signal-zu-Rauschabstand von $99.67\,\%$. Gegenüber dem Signal-zu-Rauschabstand bei leichten Regen $SNR_\mathrm{i,leichterRegen}=30.56\,\text{dB}$ ist es um 
 $19.49\,\text{dB}$ gesunken, was einem Verlust von $98.87\,\%$ gleichkommt. Der Signal-zu-Rauschabstand ist mit $SNR_\mathrm{i,Regen}=11.07\,\text{dB}$ sehr knapp und könnte möglicherweise nicht mehr genug Puffer für das Rauschen des Empfangssystem bieten. Eine Aussage über eine Mögliche Verarbeitung der empfangenen Signale kann mit dem Signal-zu-Rauschabstand am Ausgang des RF-Bereiches getätigt werden.
\begin{equation}
\begin{split}
    SNR_\mathrm{o,Regen}&=\frac{P_\mathrm{R}}{k\cdot (T_\mathrm{A,Regen}+T_\mathrm{e,sys})\cdot B}\\&=\frac{1.06\cdot 10^{-15}\,\text{W}}{1.38\cdot10^{-23}\,\frac{\text{J}}{\text{K}}\cdot(240.1\,\text{K}+336.63\,\text{K})\cdot25\,\text{kHz}}\\
    &=5.33=7.26\,\text{dB}
\end{split}
    \label{eq:SNRo-Regen-B25}
\end{equation}
Verglichen mit dem Signal-zu-Rauschabstand bei klaren Himmel $SNR_\mathrm{o,klarerHimmel}=18.69\,\text{dB}$ hat sich es um  $11.4\,\text{dB}$ auf $SNR_\mathrm{o,Regen}=7.26\,\text{dB}$ reduziert. 
Das entspricht einem Verlust von $92.7\,\%$. Gegenüber dem Signal-zu-Rauschabstand am Ausgang des RF-Bereiches bei leichten Niederschlägen $SNR_\mathrm{o,leichterRegen}=18.03\,\text{dB}$ ist es ein Verlust von $10.77\,\text{dB}$ oder $91.6\,\%$.\newline
Auch wenn der Signal-zu-Rauschabstand am Ausgang des RF-Bereiches mit $SNR_\mathrm{o,Regen}=7.26\,\text{dB}$ positiv ist, könnte es trotzdem zu wenig sein, um die empfangenen Signale verlässlich demodulieren zu können. Signale mit analogen Modulationen (AM, CW, FM) würden stark rauschen und bei digital modulierten Signalen (BPSK,n-QAM) würde die Bitfehlerrate $BER$ sehr hoch sein, wie in Abbildung \ref{fig:BeispielBER} zu sehen. Es davon auszugehen, dass der Downlink bei starken Niederschlägen so nicht aufrechterhalten werden kann.\newline
Beim Empfang eines einzelnen Signals wiederum könnte der Downlink auch bei stärkeren Niederschläge aufrechterhalten werden. Die Bandbreite reduziert sich in diesem Fall auf $B=2.7\,\text{kHz}$.
\begin{equation}
\begin{split}
    SNR_\mathrm{i,Regen}&=\frac{P_\mathrm{R}}{k\cdot T_\mathrm{A,Regen}\cdot B}\\&=\frac{1.06\cdot 10^{-15}\,\text{W}}{1.38\cdot10^{-23}\,\frac{\text{J}}{\text{K}}\cdot240.1\,\text{K}\cdot2.7\,\text{kHz}}\\
    &=118.49=20.74\,\text{dB}
\end{split}
    \label{eq:SNRi-Regen-B2.7}
\end{equation}
Durch die Reduzierung der Bandbreite auf $B=2.7\,\text{kHz}$ steigt der Signal-zu-Rauschabstand am Eingang auf $SNR_\mathrm{i,Regen}=20.74\,\text{dB}$ an. Verglichen mit dem $SNR_\mathrm{i,Regen}=-1.94\,\text{dB}$ bei $B=500\,\text{kHz}$, bzw. $SNR_\mathrm{i,Regen}=11.07\,\text{dB}$ bei $B=25\,\text{kHz}$ entspricht das einem Anstieg von $22.68\,\text{dB}$ und $9.67\,\text{dB}$. Um eine Aussage über die mögliche Aufrechterhaltung treffen zu können, muss der Signal-zu-Rauschabstand am Ausgang des RF-Bereiches bestimmt werden. 
\begin{equation}
\begin{split}
    SNR_\mathrm{o,Regen}&=\frac{P_\mathrm{R}}{k\cdot (T_\mathrm{A,Regen}+T_\mathrm{e,sys})\cdot B}\\&=\frac{1.06\cdot 10^{-15}\,\text{W}}{1.38\cdot10^{-23}\,\frac{\text{J}}{\text{K}}\cdot(240.1\,\text{K}+336.63\,\text{K})\cdot2.7\,\text{kHz}}\\
    &=49.33=16.93\,\text{dB}
\end{split}
    \label{eq:SNRo-Regen-B2.7}
\end{equation}
Durch das empfangen einer einzelnen Übertragung steigt der Signal-zu-Rauschabstand am Ausgang des RF-Bereiches auf $SNR_\mathrm{o,Regen}=16.93\,\text{dB}$ an. Im Vergleich zum $SNR_\mathrm{o,Regen}=-5.69\,\text{dB}$ bei $B=500\,\text{kHz}$ entspricht das einem Anstieg von $22.62\,\text{dB}$. Gegenüber dem $SNR_\mathrm{o,Regen}=7.26\,\text{dB}$ bei $B=25\,\text{kHz}$ ist das $SNR_\mathrm{o,Regen}=16.93\,\text{dB}$ bei $B=2.7\,\text{kHz}$ um $9.67\,\text{dB}$ größer.\newline
Verglichen zu den Signal-zu-Rauschabständen $SNR_\mathrm{o,klarerHimmel}=28.26\,\text{dB}$ und $SNR_\mathrm{o,leichterRegen}=27.7\,\text{dB}$ bei $B=2.7\,\text{kHz}$ ist der Signal-zu-Rauschabstand bei stärkeren Niederschlägen $SNR_\mathrm{o,Regen}=16.93\,\text{dB}$ um $11.33\,\text{dB}$, bzw. um $10.77\,\text{dB}$ geringer. Das entspricht einem Verlust von $92.6\,\%$ und $91.6\,\%$.\newline
Mit dem Anstieg des Signal-zu-Rauschabstand auf $SNR_\mathrm{o,Regen}=16.93\,\text{dB}$ kann das empfangene Signal problemlos demoduliert werden. \newline
Zum Schluss kann noch die Qualität des Downlinks bestimmt werden. Wie auch bei den anderen Wetterbedingungen, muss zuerst die Rauschtemperatur $T_\mathrm{S}$ mit Gleichung \ref{eq:Rauschen-Temperatur-System} ermittelt werden. Aufgrund der hohen Antennentemperatur $T_\mathrm{A,Regen}=240.1\,\text{K}$ ist eine deutliche Erhöhung von der Rauschtemperatur $T_\mathrm{S}$ zu erwarten.
\begin{equation}
\begin{split}
     T_\mathrm{S}&=\frac{T_\mathrm{A,Regen}}{L_\mathrm{sys}}+T_\mathrm{0}\left(1-\frac{1}{L_\mathrm{sys}}\right)+T_\mathrm{e,sys}\\
     &=\frac{240.1\,\text{K}}{5.02}+290\,\text{K}\left(1-\frac{1}{5.02}\right)+336.63\,\text{K}\\
     &=616.69\,\text{K}
\end{split}
\label{eq:Rauschen-Temperatur-System-Regen}
\end{equation}
Wie erwartet ist die Rauschtemperatur $T_\mathrm{S}$ bei Regen deutlich höher als bei einem klaren Himmel $T_\mathrm{S}=570.15\,\text{K}$ oder bei leichten Regen $T_\mathrm{S}=572.71\,\text{K}$. Die deutlich höhere Rauschtemperatur wird sich negativ auf die Qualtiät des Downlinks auswirken. Diese kann mit der Gleichung \ref{eq:Qualität-Downlink} und mit $P_\mathrm{RX}=6.79\cdot 10^{-8}\,\text{W}$ aus Gleichung \ref{eq:Ausgang-Leistung-Regen} bestimmt werden.
\begin{equation}
C/N_\mathrm{o}=\frac{P_\mathrm{RX}}{k\cdot T_\mathrm{S}}=\frac{6.86\cdot 10^{-8}\,\text{W}}{1.38\cdot10^{-23}\,\frac{\text{J}}{\text{K}}\cdot616.69\,\text{K}}=7.85\cdot10^{12}\,\text{Hz}=128.95\,\text{dBHz}
 \label{eq:Qualität-Downlink-Regen}
\end{equation}
Im Vergleich zur Qualität des Downlinks bei einem klaren Himmel $C/N_\mathrm{o}=138.37\,\text{dBHz}$, wird die Qualität bei starken Niederschlägen um $9.42\,\text{dB}$ auf $C/N_\mathrm{o}=128.95\,\text{dBHz}$ gesenkt. Das entspricht einer Verschlechterung der Qualität um $88.4\,\%$. Verglichen zur Qualität bei leichten Regenschauern $C/N_\mathrm{o}=137.94\,\text{dBHz}$ verschlechtert sich die Qualität um $87.2\,\%$.
\begin{table}[H]
    \centering
    \begin{tabular}{c|c|c|c}
        Name & Variable & Wert & Einheit\\
        \hline
         Durchmesser Antenne& $d$                   & $0.9$             & $\text{m}$\\
        Physikalische Fläche& $A_\mathrm{phy}$      & $0.667$ &\text{m}^2 \\
         Effektive Fläche   & $A_\mathrm{E}$        & $0.472$            & $\text{m}^2$\\
                Effizienz   & $\eta_\mathrm{ANT}$   & $0.708$            & \\
                 Gewinn     & $G_\mathrm{R,max}$    & $38.6$            & $\text{dBi}$\\
                            & $G_\mathrm{R,max}$    & $7244.36$         & \\
    Antennentemperatur      & $T_\mathrm{A,Regen}$ & $240.1$       & $\text{K}$\\
    Empfangsgüte            & $G/T$                 & $12.56$       & $\text{1/K}$\\
                            & $G/T$                 & $10.99$       & $\text{dB/K}$ \\
    Empfangene Leistung     & $P_\mathrm{R}$        & $1.06\cdot 10^{-15}$     & $\text{W}$\\
                            & $P_\mathrm{R}$         & $-119.74$   & $\text{dBm}$ \\
    Verstärkung des Empfangssystem  & $G_\mathrm{sys}$        & $77.99$ & $\text{dB}$\\
                            & $G_\mathrm{sys}$      & $62.99\cdot 10^{6}$ \\
    Leistung am Ausgang     & $P_\mathrm{RX}$        & $6.68\cdot 10^{-8}$  & $\text{W}$\\
                            & $P_\mathrm{RX}$        & $-41.75$        & $\text{dBm}$\\
    Äquivalente Rauschzahl  & $T_\mathrm{e,sys}$    & $336.63$          & $\text{K}$ \\
    Bei $B=500\,\text{kHz}$ & & &\\
    $SNR$ am Eingang        & $SNR_\mathrm{i}$        &  $-1.94$ & \text{dB}\\
    $SNR$ am Ausgang        & $SNR_\mathrm{o}$ & $-5.69$         & \text{dB} \\
    Bei $B=25\,\text{kHz}$ & & &\\
    $SNR$ am Eingang        & $SNR_\mathrm{i}$        &  $11.07$ & \text{dB}\\
    $SNR$ am Ausgang        & $SNR_\mathrm{o}$ & $7.26$         & \text{dB} \\
    Bei $B=2.7\,\text{kHz}$ & & &\\
    $SNR$ am Eingang        & $SNR_\mathrm{i}$        &  $20.74$ & \text{dB}\\
    $SNR$ am Ausgang        & $SNR_\mathrm{o}$ & $16.93$         & \text{dB} \\
    Link Qualität           & $C/N_\mathrm{o}$ & $128.95$ & \text{dB/Hz} \\
    \end{tabular}
    \caption{Bestimmte Parameter der Bodenstation für das Link Budget bei stärkeren Niederschlägen}
    \label{tab:LinkBudet-Bodenstation-Regen}
\end{table}
Die Tabelle \ref{tab:LinkBudet-Bodenstation-Regen} stellt die für die Bedingung Regen geltenden Parameter der Bodenstation übersichtlich da.
\begin{figure}[H]
    \centering
    \includegraphics[width=0.75\linewidth]{Bilder/LinkBudget_Rain.png}
    \caption{Grafische Darstellung des Link Budgets bei starken Niederschlägen}
    \label{fig:Link-Budget-Regen}
\end{figure}
Die Abbildung \ref{fig:Link-Budget-Regen} zeigt einen Graphen, welcher das Link Budget für den Downlink bei stärkeren Niederschläge repräsentiert. Gelten tut es für Niederschlagsmengen $R_\mathrm{p}$ $(\text{mm/h})$, welche den Jahresdurchschnitt $(\text{mm/h)}$ zu $p=0.01\,\%$ der Zeit überschreiten.\newline
Im Vergleich zum Link Budget bei einem klaren Himmel, dargestellt in Abbildung \ref{fig:Link-Budget-klarer-Himmel}, und zum Link Budget bei leichten Regenschauern, zu sehen in Abbildung \ref{fig:Link-Budget-leichter_Regen}, tritt in der Atmosphäre eine deutlich höhere Dämpfung auf. Die Dämpfung in der Atmosphäre beträgt für diese Bedingung $L_\mathrm{ATRegen}=9.61\,\text{dB}$, wovon ein Großteil durch die Dämpfung der starken Niederschläge $L_\mathrm{Regen}=8.86\,\text{dB}$ hervorgerufen wird. Diese deutlich höhere Dämpfung führt zu einer deutlichen Reduzierung der empfangenen Leistung $P_\mathrm{R}$ und damit auch zu einer Reduzierung der Leistung $P_\mathrm{RX}$ am Ausgang des RF-Bereiches.\newline 
Zusätzlich zur reduzierten empfangenen Leistung $P_\mathrm{R}$ steigt noch die von der Antenne eingefangene Rauschleistung. Die Antennentemperatur mit $T_\mathrm{A,Regen}=240.1\,\text{K}$ um ein vielfaches höher als bei einem klaren Himmel $T_\mathrm{A,klarerHimmel}=6.5\,\text{K}$ oder bei leichten Regenschauern mit $T_\mathrm{A,leichterRegen}=19.29\,\text{K}$.\newline 
Durch die hohe Rauschleistung und niedrigere empfangene Leistung $P_\mathrm{R}$ kann der Downlink bei einer Bandbreite von $B=500\,\text{kHz}$ nicht Aufrecht erhalten werden. Das $SNR_\mathrm{i,Regen}=-1.94\,\text{dB}$ und $SNR_\mathrm{o,Regen}=-5.69\,\text{dB}$ bedeuten, dass die empfangenen Signale komplett vom Rauschen überlagert werden. Bei einer Reduzierung der Bandbreite auf $B=25\,\text{kHz}$ verbessern sich beide Signal-zu-Rauschabstände auf $SNR_\mathrm{i,Regen}=11.07\,\text{dB}$, bzw. $SNR_\mathrm{o,Regen}=7.26\,\text{dB}$. Die demodulierten Signale werden trotz der Verbesserung um $9.13\,\text{dB}$, bzw. $12.95\,\text{dB}$
vom starken Rauschen und hohen Bitfehlerraten bei digital modulierten Signalen betroffen sein. Eine Aufrechterhaltung des Downlinks ist auch dann schwer möglich.\newline
Durch die weitere Reduzierung auf $B=2.7\,\text{kHz}$ kann der Signal-zu-Rauschabstand weiter erhöht werden. Am Eingang steigt es auf $SNR_\mathrm{i,Regen}=20.74\,\text{dB}$ und am Ausgang auf $SNR_\mathrm{o,Regen}=16.93\,\text{dB}$. Dieser Signal-zu-Rauschabstand ermöglicht schlussendlich die Demodulation des empfangenen Signals. Bei digitalen Modulationen ist jedoch die Bitfehlerrate, im Vergleich zu den anderen beiden Wetterbedingung, höher, wie es aus Abbildung \ref{fig:BeispielBER} entnommen werden kann. Zwar kann nur noch ein einzelnes Signal des Downlinks empfangen werden, jedoch kann dieser dafür aufrechterhalten werden.\newline
Diese Ergebnisse zeigen sehr gut, dass die mit starken Niederschlägen verbundene Dämpfung $L_\mathrm{Regen}=8.86\,\text{dB}$ und das höhere empfange Rauschen $T_\mathrm{A,Regen}=240.1\,\text{K}$ deutliche Auswirkung auf die Ausfallzeiten des Downlinks haben und somit nicht vernachlässigt werden dürfen.



\section{Aufbau des Empfangssystems}




\subsection{Aufstellen und Ausrichten der Antenne}\label{Ausrichten der Antenne}
Die Antenne soll auf einem eigenen Antennenstandfuß auf dem Dach der Hochschule montiert werden. Der neue Antennenfeed wird mithilfe des alten Feedhalters im Fokuspunkt der Parabolschüssel montiert. Für die Anbringung des neuen Antennenfeeds wird ein Adapter entwickelt, welcher mithilfe eines 3D-Druckers angefertigt wird. Als Material wird PETG-HF von Bambu Lab verwendet. Die Druckdatei ist im GitHub-Repository hinterlegt.\newline
\begin{figure}[H]
    \centering
    \includesvg[width=0.5\linewidth]{Bilder/Aufbau Antennenmast}
    \caption{Skizze vom Aufbau der Antenne auf dem Dach der Hochschule}
    \label{fig:Aufbau-der-Antenne}
\end{figure}
Die Abbildung \ref{fig:Aufbau-der-Antenne} zeigt den geplanten Aufbau der Antenne auf dem Dach der Hochschule. Dieser ist zum größtenteils auch umgesetzt. Der vorgesehene Antennenmast und Antennenfuß ist nicht bis zum Abschluss der Arbeit angekommen, weshalb ein provisorischer Antennenmast gebaut wird.
\begin{figure}[H]
    \centering
    \includegraphics[width=0.4\linewidth]{Bilder/Aufgebaute Antenne.jpg}
    \caption{Umgesetzter Aufbau der Antenne auf dem Dach der Hochschule. Im Schwarzen Gehäuse am Mast ist der LNC untergebracht.}
    \label{fig:Umgesetzter-Aufbau-Antenne}
\end{figure}
Für die Ausrichtung der Antenne auf Es'Hail-2 (QO-100) muss der Azimut $\varphi$ und die Elevation $\varepsilon$ bestimmt werden. Mit Gleichung \ref{eq:Azimut} kann der Azimut $\varphi$ bestimmt werden. Die Koordinaten der Bodenstation $53.055\degree,8.78\degree$ können aus der Abbildung \ref{fig:Koordinaten der Bodenstation} entnommen werden. Der Satellit Es'Hail-2 (QO-100) befindet sich bei  $25.9\degree \text{E}$\cite{EsHail2}. Damit beträgt die Differenzen zwischen den Längengraden $\Delta long=53.055\degree-25.8\degree=27.255\degree$.
\begin{equation}
    \varphi=\arctan\left(\frac{\tan(\Delta long)}{\sin(lat_\mathrm{BS})} \right)=\arctan\left(\frac{\tan(27.255\degree)}{\sin(8.78\degree)} \right)=159.04\degree
    \label{eq:Azimut-Antenne}
\end{equation}
Für die Ermittlung der Elevation $\varepsilon$ der Antenne wird die Gleichung \ref{eq:Elevation} verwendet. Der Radius der Erde beträgt $r_\mathrm{0}=6378\,\text{km}$ und die Flughöhe von Es'Hail-2 (QO-100) kann mit $r=35790\,\text{km}$ angenommen werden.\cite{Satellitenkommunikation}
\begin{equation}
\begin{split}
    \varepsilon&=\arctan\left(\frac{\cos(lat_\mathrm{BS})\cdot\cos(\Delta long)-\frac{r_\mathrm{0}}{r_\mathrm{0}+r}}{\sqrt{1-\cos^2(lat_\mathrm{BS})\cdot\cos^2(\Delta long)}} \right)\\
    &=\arctan\left(\frac{\cos(8.78\degree)\cdot\cos(27.255\degree)-\frac{6378\,\text{km}}{6378\,\text{km}+35790\,\text{km}}}{\sqrt{1-\cos^2(8.78\degree)\cdot\cos^2(27.255\degree)}} \right)\\
    &=27.36\degree
\end{split}
    \label{eq:Elevation-Antenne}
\end{equation}
Zum Schluss muss noch die Drehung des Antennenfeeds um seine eigene Achse, der $Skew$, ermittelt werden. Dieser kann mit der Gleichung \ref{eq:Skew} bestimmt werden. Der Offset des Antennenfeeds beträgt $0\degree$.
\begin{equation}
    Skew = \arctan\left(\frac{\sin(\Delta long)}{\tan (lat_\mathrm{BS})}\right)-Offset=\arctan\left(\frac{\sin(27.255\degree)}{\tan (8.78\degree)}\right)=-12.4\degree
\end{equation}
Das Python-Skript zur Berechnung der Azimut $\varphi$, Elevation $\varepsilon$ und Drehung $Skew$ der Antenne ist im GitHub-Repository und im Anhang \ref{lst:Antenne-berechnung-python} hinterlegt. Mithilfe eines Kompass wird der Azimut $\varphi=159.04\degree$ der Antenne eingestellt. Die Elevation $\varepsilon=27.36\degree$ kann mithilfe einer Skala an der Antenne eingestellt werden. Die Neigung des Antennenfeeds um die eigene Achse $Skew=-12.4\degree$ wird mithilfe einer Winkellehre eingestellt. Als Ursprungpunkt wird dabei der SMA-Anschluss des X-Band Feeds gewählt.

\subsection{Abwärtsmischer, Fernspeiseweiche und Verkabelung im Serverschrank}
Der rauscharme Abwärtsmischer (LNC) wird neben der Antenne montiert. Um diesen vor Wettereinflüssen, wie Regen und Schnee, zu schützen, wird dieser in einem wasserdichten Gehäuse untergebracht (schwarzer Kasten in Abbildung \ref{fig:Umgesetzter-Aufbau-Antenne}).
\begin{figure}[H]
    \centering
    \includegraphics[width=0.6\linewidth]{Bilder/LNC im Gehäuse.jpg}
    \caption{Unterbringung des LNC im Gehäuse. Dank der Schiene kann dieser leicht ein- und ausgebaut werden. Die 50-Cent Münze dient als Größenvergleich.}
    \label{fig:LNC-im-Gehäuse}
\end{figure}
Der LNC wird, wie in Abbildung \ref{fig:LNC-im-Gehäuse} zu sehen, auf einer Schiene montiert. Die Druckdatei für die Halterung auf der Schiene sind ebenfalls im GitHub-Repositroy hinterlegt. Durch diese Schiene kann der LNC einfach montiert und bei Bedarf auch wieder demontiert werden. Das Gehäuse bietet theoretisch noch Platz für zusätzliche Komponenten, z.B. einen Aufwärtsmischer oder Leistungsverstärker für die Einrichtung eines Uplink zu Es'Hail-2 (QO-100). Ebenfalls bietet das Gehäuse mehrere Durchführungsmöglichkeiten für die Koaxialleitungen. Mithilfe der S\_04212\_B Koaxialleitung soll der Antennenfeed an den HF-Eingang des Abwärtsmischers angeschlossen werden. Über die LMR 400 FR Koaxialleitung wird das resultierende ZF-Signal zur Fernspeiseweiche im Serverschrank gebracht.
\begin{figure}[H]
    \centering
    \includesvg[width=0.5\linewidth]{Bilder/Verkabelung Serverschrank}
    \caption{Skizze der geplanten Verschaltung im Serverschrank}
    \label{fig:geplante-Verschaltung}
\end{figure}
Die Skizze in Abbildung \ref{fig:geplante-Verschaltung} zeigt die geplante Verschaltung im Serverschrank. Die Fernspeiseweiche wird an ein Labornetzteil angeschlossen, welches die Versorgungsspannung $V_\mathrm{CC}$ für den Abwärtsmischer liefert. Dieser soll im QO-100 SSB Betriebsmodus betrieben werden, wofür eine Spannung von $V_\mathrm{CC}=12-17\,\text{V}$ über die Fernspeiseweiche eingespeist werden muss. In diesem Modus liegt der HF-Bereich des Abwärtsmischer zwischen $10489\,\text{MHz}$ und $10490\,\text{MHz}$. Der Frequenzbereich des resultierende ZF-Singals liegt zwischen $433\,\text{MHz}$ und $434\,\text{MHz}$.Über die beiden Anschlusspins an der Fernspeiseweiche wird mit dem Labornetzteil eine Spannung von $V_\mathrm{CC}=15\,\text{V}$ angelegt.\newline
Nach der Fernspeiseweiche wird das ZF-Signal über eine LMR 400 FR Koaxialleitung an das Patchfeld angeschlossen. Ab hier kann die Verkabelung mithilfe der beiden Hyperflex 5 Koaxialleitungen leicht vorgenommen und bei Bedarf auch verändert werden. Mit dieser Koaxialleitung wird das ZF-Signal zur RF-Schaltmatrix und und anschließend an den USRP X310 angeschlossen.\newline
\begin{figure}[H]
    \centering
    \includegraphics[width=0.4\linewidth]{Bilder/Aufbau-Serverschrank.jpg}
    \caption{Umgesetzter Aufbau der Komponenten im Serverschrank. Die Fernspeiseweiche ist nicht auf dem Bild zu sehen, da sie sich am hinteren Ende des Serverschranks befindet.}
    \label{fig:Aufbau-Serverschrank}
\end{figure}
Die Abbildung \ref{fig:Aufbau-Serverschrank} zeigt den umgesetzten Aufbau des Empfangssystem im Serverschrank. Der Aufbau konnte fast vollständig umgesetzt werden. Nur eine Hyperflex 5 Koaxialleitung ist nicht angekommen, weshalb eine kurze LMR 400 Leitung als Ersatz verwendet wird.


\subsection{Erstellen einer SDR-Software mit GNU Radio Companion}\label{kap:SDR Software}
Um den USRP X310 zu steuern und den Downlink von Es'Hail-2 (QO-100) verarbeiten zu können, wird mithilfe von GNU Radio Companion eine, für diese Anwendung spezifische, SDR-Software erstellt.\newline
GNU Radio ist ein freies und Open-Source Framework, welches Nutzern die Möglichkeit bietet Radio Systeme zu entwickeln, simulieren und anzuwenden. Der GNU Radio Companion ist das zugehörige Programm, mit welchen einfache Signalflussgraphen zur digitalen Signalverarbeitung erstellt werden können.\cite{GNU-Radio}

\subsubsection*{Herstellen der Verbindung zum USRP X310, einstellen der Frequenz und Ausgabe des Frequenzspektrums}
\begin{figure}[H]
    \centering
    \includegraphics[width=0.65\linewidth]{Bilder/Eingang.png}
    \caption{Verwendeter Quellblock und variable Frequenzeinstellung in GNU Radio}
    \label{fig:Eingang}
\end{figure}
Die Abbildung \ref{fig:Eingang} zeigt den Anfangsbereich des Signalflussgraphen der erstellten SDR-Software. Den Anfang des Signalflussgraphen bildet ein Quellblock (engl. Source Block). Dieser wird benötigt, um eine Verbindung mit dem USRP X310 herzustellen. In GNU Radio gibt es mehrere verschiedene Quellblöcke zur Auswahl. Welcher verwendet werden kann hängt vom verwendeten SDR ab. Der verwendete USRP X310 bietet Unterstützung für den UHD-Treiber, weshalb als Quellblock ein UHD: URSP Quellblock verwendet wird.\cite{USRP-X310-Doku}\cite{GNU-Radio-USRP-Source}\newline
Über den Quellblock können viele verschiedene Einstellung des USRP X310 angepasst werden. Allerdings werden nicht alle Einstellmöglichkeiten benötigt. 
\begin{table}[H]
    \centering
    \begin{tabular}{c|c|p{7cm}}
       Einstellung  & Wert & Beschreibung\\
       \hline
        Abtastrate & $1\,\text{MS/s}$ & Setzen der Abtastrate vom SDR und damit auch die Bandbreite\cite{GNU-Radio-USRP-Source} \\
        Anzahl Kanäle & $1$ & Es wird nur ein Kanal für die geplante Anwendung benötigt. \\
        Mittenfrequenz & variabel & Die Mittenfrequenz kann variabel über eine Eingabe oder einen Schieber eingestellt werden. Als Standardwert wird $433.5\,\text{MHz}$ gewählt.\\
        RF Verstärkung & variabel & Die Verstärkung durch den USRP X310 kann variabel mit einem Schieber oder über eine Eingabe zwischen $0$ und $31.5\,\text{dB}$ angepasst werden. Als Standardverstärkung wird $0\,\text{dB}$ gewählt.\\
        Bandbreite & $0$ & Mit der Einstellung $0$ wird die Standardbreite vom Antialiasing Filter des USRP X310 verwendet.\cite{GNU-Radio-USRP-Source}\\
        Geräte Adresse & - & Hier kann die Geräte Adresse (z.B. IP-Adresse) des jeweilige USRP eingetragen werden. Das Feld wird leer gelassen. Damit wird der erste vom Programm gefundene USRP verwendet.\cite{GNU-Radio-USRP-Source}
    \end{tabular}
    \caption{Einstellungen im UHD: USRP Quellblock}
    \label{tab:Einstellungen-USRP-Quellblock}
\end{table}
Die Tabelle \ref{tab:Einstellungen-USRP-Quellblock} zeigt die getroffenen Einstellung im UHD: USRP Quellblock und damit auch die Einstellung des USRP X310. Die Mittenfrequenz wird Standardmäßig auf $433.5\,\text{MHz}$ gesetzt, da die ZF-Frequenz des Abwärtsmischer im SSB QO-100 Betrieb zwischen $433\,\text{MHz}$ und $434\,\text{MHz}$ liegt. Der Downlink des Schmalbandtransponder auf Es'Hail-2 (OQ-100) liegt dabei zwischen $433.5\,\text{MHz}$ und $434\,\text{MHz}$\newline
Das empfangene Frequenzspektrum soll mithilfe eines Wasserfalldiagramm und eines FFT-Spektrum dargestellt werden. Dafür gibt der Quellblock die Abtastwerte an einen Wasserfallblock (QT GUI Waterfall Sink) und einen FFT-Spektrumblock (QT GUI Frequency Sink) weiter\cite{GNU-Radio-Frequency-Sink}\cite{GNU-Radio-Waterfall-Sink}. Die Mittenfrequenz in beiden Spektren entspricht dabei der Mittenfrequenz des vom Quellblock ausgegeben Spektrums. Bei der Veränderung der Variable Mittenfrequenz verändert sich dementsprechend auch die dargestellte Mittenfrequenz.\newline
Der Multipliziererblock am Ausgang des Quellblocks wird als Mischer eingesetzt. Die Signalquelle fungiert dabei als lokaler Oszillator. Mit diesen beiden Blöcken besteht die Möglichkeit eine beliebige Frequenz im, vom Quellblock ausgegeben, Frequenzspektrum auszuwählen. Diese Frequenz wird Kanalfrequenz bezeichnet und kann über eine Eingabe oder durch einen Schieber im Betrieb verändert werden. Die Frequenz des lokalen Oszillator entspricht der Differenz zwischen der Mittenfrequenz und der Kanalfrequenz.\newline
Über einen folgenden Auswahlblock kann die Art der Demodulation ausgewählt werden. Zur Verfügung stehen dabei Einseitenband Amplitudenmodulation (SSB) mit oberen (USB) Seitenband und unteren (LSB) Seitenband, Frequenzmodulation (FM) und kontinuierliche Welle (engl. Continious Wave) CW.
\begin{figure}[H]
    \centering
    \includegraphics[width=0.65\linewidth]{Bilder/Ausgabebereich.png}
    \caption{Ausgabe des Audiosignals in GNU Radio}
    \label{fig:Ausgabe}
\end{figure}
Die Abbildung \ref{fig:Ausgabe} zeigt die möglichen Ausgabemöglichkeiten des Audiosignals. Das Audiosignal stammt dabei von dem jeweiligen angewendeten Demodulationsverfahren und wird über einen Auswahlblock an die Audio Ausgabe weitergegeben.\newline
Über einen FFT-Frequenzspektrum Block wird das Frequenzspektrum des Audiosignals angezeigt, wobei nur die positiven Frequenzen ab $0\,\text{Hz}$ dargestellt werden.\cite{GNU-Radio-Frequency-Sink}\newline
Das Audiosignal hat eine Abtastrate von $48\,\text{kS/s}$ und wird über eine Audio Senke an die Standard Ausgabe des Systems weitergegeben.\cite{GNU-Radio-Audiosink}\newline 
Die Lautstärke wird über den Multipliziererblock festgelegt. Sein Wert kann über einen Schieber im Betrieb zwischen $0$ und $1$ angepasst werden. Dabei entspricht $0$ stumm geschaltet und $1$ volle Lautstärke.\newline
Auch besteht die Möglichkeit das Audio Signal als eine 16-Bit .wav abzuspeichern. Verwendet wird dafür eine .wav Datei Senke (Wav File Sink). Der Speicherort, sowie der Name der .wav-Datei muss vor dem Start der Software in diesem Block angeben werden. Die Abtastrate des Audio Signals wird auf $48\,\text{kS/s}$ gestellt. Über einen Auswahlblock kann im Betrieb der Software die Aufnahme dann gestartet oder gestoppt werden.\cite{GNU-Radio-WAV}
\subsubsection*{FM-Demodulaton}
Bei der Frequenzmodulation (FM) werden die Informationen aus dem Basisband der Phase des Trägersignals auf moduliert.\cite{Nachrichtentechnik}\newline
\begin{equation*}
\begin{split}
        s_\mathrm{FM}(t)&=\hat{u}_\mathrm{T}\cdot \cos(\Psi_\mathrm{FM}(t))\\
        &=\hat{u}_\mathrm{T}\cdot \cos\left(\omega_\mathrm{T}+2\pi\cdot\Delta F\cdot\int u( \tau)d\tau\right)
\end{split}
\end{equation*}
Das Spektrum des FM-Signal kann nicht einfach angegeben werden, da es sich um ein nichtlineares Modulationsverfahren handelt. Die Anzahl der Nebenschwingungen neben dem Träger resultieren aus der Besselfunktion 0-ter Ordnung.\cite{Nachrichtentechnik}\newline
Für die Demodulation der FM-Signale wird eine erweiterte Version einer Beispielschaltung aus GNU Radio Dokumentation verwendet.\cite{GNU-Radio-FM}
\begin{figure}[H]
    \centering
    \includegraphics[width=0.75\linewidth]{Bilder/FM-Demodulator.png}
    \caption{Angewendeter FM Demodulator in GNU Radio}
    \label{fig:FM-Demodulator}
\end{figure}
Im ersten Schritt wird mit einem Tiefpassfilter das Frequenzspektrum begrenzt. So wird nur das gewünschte FM-Signal an den Demodulator weitergegeben. Die Grenzfrequenz des Tiefpasses kann über die Variable Filterbreite mit einem Schieber oder einer Eingabe verändert werden. Bei einer Radiostation beträgt die Bandbreite des FM-Signals ca. $120\,\text{kHz}$.\newline
Im nächsten Schritt wird mit einem rationalen Resampler die Abtastrate von $1\,\text{MS/s}$ auf $480\,\text{kS/s}$ reduziert.\cite{GNU-Radio-FM}\cite{GNU-Radio-Resampler}
\begin{equation*}
    \text{Faktor}=\frac{480000}{1000000}=\frac{12}{25}
\end{equation*}
Die Abtastrate wird erst um den Faktor $12$ erhöht und anschließend um den Faktor $10$ reduziert.\newline
Anschließend wird das FM-Signal mit dem FM-Demod Block von GNU Radio demoduliert.\cite{GNU-Radio-FM}. Die Kanalrate (engl. Channel Rate) wird auf die $480\,\text{kS/s}$ gestellt und die Audio Reduzierung auf $10$. Damit beträgt die Abtastrate des Audiosignals am Ausgang $48\,\text{kS/s}$. Somit kann das Audiosignal an die Audio Ausgabe weitergegeben werden.\cite{GNU-Radio-FM}\newline
Der FM-Demodulator wird mit in die SDR-Software aufgenommen, da mit diesem der Umgang mit dem SDR erprobt werden kann. Dieser kann verwendet werden um Radiostation zu empfangen und so z.B. neue Funktionen der Software zu testen.

\subsubsection*{Einseitenband-Demodulation}
Bei der Amplitudenmodulation (AM), wozu die Einseitenbandmodulation gehört, werden Information aus dem Basisband mit einem Mischer auf die Amplitude eines sinusförmigen Trägersignals auf moduliert. Das Trägersignal liegt dabei in einem für die Übertragung geeigneten Frequenzbereich.\cite{Nachrichtentechnik}\newline
\begin{equation*}
    \begin{split}
         s_\mathrm{AM}(t)&=s_\mathrm{BB}(t)\cdot s_\mathrm{T}(t)\\
         &=\frac{\hat{u}_\mathrm{BB}\cdot \hat{u}_\mathrm{T}}{2}\cdot (\cos((\omega_\mathrm{T}-\omega_\mathrm{BB})\cdot t)+\cos((\omega_\mathrm{T}+\omega_\mathrm{BB})\cdot t)
    \end{split}
\end{equation*}
\begin{figure}[H]
    \centering
    \includesvg[width=0.4\linewidth]{Bilder/AM-Signal}
    \caption{Spektrum eines AM-Signals}
    \label{fig:AM-Signal}
\end{figure}
Durch den Mischvorgang entstehen zwei Frequenzkomponenten links uns rechts neben dem Trägersignal, wie in Abbildung \ref{fig:AM-Signal} zu sehen. Diese beiden Frequenzkomponenten werden oberes und unteres Seitenband (engl. Upper- and Lower Sideband) genannt. Dabei befindet sich das obere Seitenband in der Regellage und das untere Seitenband in der Kehrlage\cite{Nachrichtentechnik}\newline
Beide Frequenzkomponenten enthalten dabei die gleichen Informationen aus dem Basisband, sprich sie sind identisch zu einander. Aus diesem Grund kann bei der Übertragung oder bei der Demodulation auf eins der beiden Seitenbänder verzichtet werden.\cite{Nachrichtentechnik}\newline
Um ein Einseitenbandsignal im Downlink von Es'Hail-2 (QO-100) demodulieren zu können, muss dieses zurück in das Basisband gebracht werden. Das resultierende komplexe Basisbandsignal muss dann nur noch mit einem Tief- oder Bandpass begrenzt werden. Für die Umsetzung des Einseitenband-Demodulators in GNU Radio wird eine angepasste Version eines Einseitenband-Demodulators aus der GNU Radio Dokumentation verwendet.\cite{GNU-Radio-SSB}
\begin{figure}[H]
    \centering
    \includegraphics[width=0.8\linewidth]{Bilder/SSB-Demodulator.png}
    \caption{Signalflussgraph des umgesetzten Einseitenband-Demodulator in GNU Radio}
    \label{fig:SBB-Demodulator}
\end{figure}
Die Abbildung \ref{fig:SBB-Demodulator} zeigt den in GNU Radio umgesetzten Einseitenband-Demodulator. Das Herzstück des Demodulators ist ein Frequenz-umsetzender FIR Filter (Frequency Xlating FIR Filter). Dieser FIR Filter kombiniert mehrere Funktionen in einem Block. Das am Eingang anliegende Signal wird von diesem Filter Block in das Basisband verschoben. In den Eigenschaften des Blocks kann mit der Mittenfrequenz auch ein Offset zum gewünschten Signal eingestellt werden. Die bei der Frequenzverschiebung entstehenden hochfrequenten Anteile bei $\pm2\omega_\mathrm{T}$ können mit einem Filter, welcher über die Einstellung Taps definiert werden kann, entfernt werden. Auch kann über einen Faktor die Abtastrate des Signals dezimiert werden.\cite{GNU-Radio-SSB}
\begin{table}[H]
    \centering
    \begin{tabular}{c|c|p{4cm}}
       Einstellung  & Eingestellter Wert & Beschreibung\\
       \hline
       Dezimierung & $1$ & Keine Verringerung der Abtastrate\\
       Taps & $20$ & kein Filter, 20-fache Verstärkung\\
       Mittenfrequenz  & $0$ & kein Offset\\
       Abtastrate & $1\,\text{MS/s}$ & Abtastrate des eingehenden Signals
    \end{tabular}
    \caption{Getroffene Einstellungen des Frequenz-umsetzenden FIR Filter }
    \label{tab:Einstellungen-FIR-Filter}
\end{table}
Die Tabelle \ref{tab:Einstellungen-FIR-Filter} zeigt die verwendeten Einstellungen für den 
Frequenz-umsetzenden FIR Filter.\newline
Nachdem Frequenz-umsetzenden FIR Filter wird das Basisbandsignal in einen rationalen Resampler geführt, welcher die Abtastrate von $1\,\text{MS/s}$ auf $48\,\text{kS/s}$ reduziert. Die Abtastrate von $48\,\text{kS/s}$ entspricht der Standard Abtastrate von Audiosignalen.\cite{GNU-Radio-Resampler}
\begin{equation*}
    \text{Faktor}=\frac{48000}{1000000}=\frac{6}{125}
\end{equation*}
Die Abtastrate muss erst um den Faktor $6$ interpoliert und anschließend um den Faktor $125$ dezimiert werden.\newline
Bei dem Signal handelt es sich bisher um ein komplexes Signal im Basisband, welches ein oberes und unteres Seitenband enthalten kann. Das obere Seitenband liegt nach der Verschiebung in das Basisband im positiven Frequenzbereich und bildet die In-Phase Komponente. Das untere Seitenband liegt im negativen Frequenzbereich und bildet die Quadratur Komponente. Um das untere Seitenband nutzen zu können, muss die In-Phase Komponente mit der Quadratur Komponete getauscht werden. In GNU Radio kann diese Operation mit einem IQ-Tauschblock (Swap IQ) umgesetzt werden.\cite{GNU-Radio-SwapIQ}\newline
Mit einem nachfolgende Bandpassfilter wird das jeweiligen Seitenband in seiner Bandbreite begrenzt. Die untere Grenze des Bandpassfilters liegt bei $f_\mathrm{g}=200\,\text{Hz}$. Die obere kann während des Betriebes über eine Eingabe oder einen Schieber verändert werden. Die Wahl eines Bandpasses anstelle eines Tiefpasses kann mit der engen Bandbreite von $B=2.7\,\text{kHz}$ über den Schmalbandtransponder und dem Hauptsprachbereich der menschlichen Stimme erklärt werden. Dieser liegt zwischen $200\,\text{Hz}$ und $\approx3000\,\text{Hz}$\cite{Sprachbereich}. Durch den Einsatz des Bandpasses kann die volle Bandbreite von $B=2.7\,\text{kHz}$ auf den natürlichen Sprachbereich eines Menschen angepasst werden. Der Übergangsbereich des Bandpasses wird auf $150\,\text{Hz}$ gesetzt. Die Abtastrate bleibt bei $1\,\text{MS/s}$.\newline
Nach dem jeweiligen Bandpass kann über einen Auswahlblock das jeweiligen Seitenband ausgewählt werden. Das kann auch während der Benutzung der Software verwendet werden. Nach dem Auswahlblock wird das ausgewählte Seitenband von automatischen Verstärkungseinheit auf ein Referenzlevel angehoben oder reduziert. Die Angriffsrate (engl. Attack Rate) und die Abfallrate (engl. Decay Rate) bestimmen dabei die Rate, mit der das Audiosignal auf das Referenzlevel pro Abtastwert angehoben oder abgesenkt wird. Die Angriffsrate wird auf $0.01$ und die Abfallrate auf $0.1$ gesetzt. Die Abfallrate wird höher gewählt, um zu große Signale schneller abzuschwächen. Die Verstärkung oder Dämpfung pro Abtastwert liegt bei $0.1$. Zum Schluss wird das komplexe Audiosignal in ein reales Audiosignal umgewandelt. Anschließend wird das Audiosignal an den Ausgabebereich weitergegeben.  
\subsubsection*{CW-Demodulation}
Bei der CW-Modulation handelt es sich um eine Sonderform der Amplitudenmodulation. Bei einer herkömmlichen Amplitudenmodulationen werden Informationen (z.B. Audio) in Form von Basisbandsignalen mit einem kontinuierlichen Träger übertragen. Die Basisbandsignale bilden dabei die beiden Seitenbänder. Bei der CW-Modulation erfolgt die Übermittlung von Informationen durch das Ein- und Ausschalten des Trägers. Es werden keine Basisbandsignale selbst auf den Träger auf moduliert.\cite{CW}
\begin{figure}[H]
    \centering
    \includesvg[width=0.4\linewidth]{Bilder/AM zu CWsvg}
    \caption{Vergleich zwischen einem Signal mit AM (oben) und einem Signal mit CW (unten) im Zeitbereich}
    \label{fig:Vegleich-AM-CW}
\end{figure}
Eingesetzt wird diese Art der Kommunikation bei der Telegrafie (Morse Code).\cite{CW}\newline
Eine umfangreiche Demodulation ist bei Signalen mit CW Modulation nicht erforderlich. Sie müssen jegendlich in das Basisband verschoben werden und mit einem Tiefpass in ihrer Bandbreite begrenzt werden. Der Durchbruchsbereich des Tiefpasses ist möglichst eng zu wählen $(f_\mathrm{g}\leq2\,\text{kHz})$, um Rauschen und Einflüsse durch andere Signale zu minimieren.\newline
\begin{figure}[H]
    \centering
    \includegraphics[width=0.8\linewidth]{Bilder/CW Demodulator.png}
    \caption{Signalflussgraph des in GNU Radio umgesetzten CW-Demodulators}
    \label{fig:CW-Demodulator}
\end{figure}
Die Abbildung \ref{fig:CW-Demodulator} zeigt den in GNU Radio umgesetzten Demodulator für CW-Signale. Ähnlich zum Demodulator für Einseitenband, wird für die für die Verschiebung des CW-Signals in das Basisband ein Frequenzumsetzender FIR Filter verwendet.\cite{GNU-Radio-SSB} Es wird kein Filter definiert, aber eine 20-fache Verstärkung eingestellt. Ebenso wird keine Dezimierung der Abtastrate mit dem Frequenzumsetzender FIR Filter vorgenommen.\newline
Die Reduzierung der Abtastrate wird mit einem folgenden rationalen Resampler durchgeführt. Mit diesem wird die Abtastrate von $1\,\text{MS/s}$ auf $48\,\text{kS/s}$, was der Abtastrate eines Audiosignals entspricht, herabgesetzt.\cite{GNU-Radio-Resampler}
\begin{equation*}
    \text{Faktor}=\frac{48000}{1000000}=\frac{6}{125}
\end{equation*}
Der Abtastrate wird erst um den Faktor $6$ interpoliert und anschließend um den Faktor $125$ dezimiert.\newline
Mit einem nachfolgenden Tiefpass wird das Basisband in seiner Breite begrenzt. Seine Grenzfrequenz $f_\mathrm{g}$ kann über die Bandbreite mit einer Eingabe oder Schieber während des Betriebes verändert werden. Bei CW-Signalen sollte ein möglichst schmaler Durchbruchsbereich  $(f_\mathrm{g}\leq2\,\text{kHz})$ gewählt werden. Der Übergangsbereich ist $150\,\text{Hz}$ breit.\newline
Die automatische Verstärkungseinheit hält den Pegel des Audiosignals auf einem gleichbleibenden Referenzlevel. In diesem Fall $0.5$. Die Angriffsrate wird auf $0.01$ und die Verfallsrate auf $0.1$ gesetzt. Dadurch werden Audiosignale mit zu großen Pegel schneller gedämpft. Die maximale Verstärkung pro Abtastwert liegt bei $0.1$. Anschließend wird das komplexe Audiosignal in ein reales umgewandelt. Damit kann es an den Ausgabebereich weitergegeben werden.


\subsection{Testen der Software}
Bevor die SDR-Software eingesetzt werden kann, müssen die einzelnen Demodulatoren und Funktionen der Software auf ihre Tauglichkeit überprüft werden. Für die Überprüfung des FM-Demodulator eignet sich der Empfang einer UKW-Radiostation.
\begin{figure}[H]
    \centering
    \includesvg[width=0.5\linewidth]{Bilder/Testaufbau FM}
    \caption{Angewendeter Testaufbau für den FM-Demodulator}
    \label{fig:Testaufbau FM-Demodulator}
\end{figure}
In der Abbildung \ref{fig:Testaufbau FM-Demodulator} ist der angewendete Aufbau zum testen des FM-Demodulators dargestellt. Zum testen wird eine Teleskop UKW Radioantenne über eine Koaxialleitung an den RX 2 SMA-Anschluss des RF A Erweiterungsbord vom USRP X310 angeschlossen. In der SDR-Software wird der FM-Demodulator ausgewählt. Die Mittenfrequenz wird auf $97\,\text{MHz}$ und die Kanalfrequenz auf $96.7\,\text{MHz}$ gestellt. An dieser Frequenz sollte sich  der Radiosender Bremen Next befinden. Als Filterbreite wird $B=57.423\,\text{kHz}$ eingestellt. Die Verstärkung bleibt bei $0\,\text{dB}$.
\begin{figure}[H]
    \centering
    \includegraphics[width=0.75\linewidth]{Bilder/FM Radio HF Spektrum.png}
    \caption{HF-Spektrum beim Empfang von FM Radio}
    \label{fig:FM-HF-Spektrum}
\end{figure}
Die Abbildung \ref{fig:FM-HF-Spektrum} zeigt das empfangene HF-Spektrum beim Test des FM-Demodulators. Im oberen Plot ist ein Wasserfalldiagramm und im unteren Plot ein FFT-Spektrum. Die Mittenfrequenz liegt bei $97\,\text{MHz}$. In beiden Darstellungen kann gut das FM-Signal der Radiostation bei $96.7\,\text{MHz}$ erkannt werden. Der maximale Pegel liegt bei ca. $-82\,\text{dB}$. Der Rauschpegel liegt bei ca. $-112\,\text{dB}$. Das entspricht einem Signal-zu-Rauschabstand von $SNR=30\,\text{dB}$, was ein sehr zufriedenstellender Wert ist.
\begin{figure}[H]
    \centering
    \includegraphics[width=0.75\linewidth]{Bilder/FM Radio Basisband.png}
    \caption{Spektrum vom Audiosignal des demodulierten FM-Signal}
    \label{fig:FM-Basisband}
\end{figure}
Das FFT-Spektrum in Abbildung \ref{fig:FM-Basisband} zeigt das Audiosignal des demodulierten FM-Signal. Durch den im FM-Demod Block integrierten Tiefpass wird das Audiosignal bei $15.5\,\text{kHz}$ abgegrenzt. Beim Testen des FM-Demodulators konnte ohne Probleme das Radioprogramm verfolgt werden. Es kann also von der vollen Funktionsfähigkeit des FM-Demodulators ausgegangen werden.\newline
Zum Testen des Einseitenband- und des CW-Demodulators wird ein anderer Testaufbau verwendet.
\begin{figure}[H]
    \centering
    \includesvg[width=0.7\linewidth]{Bilder/Testaufbau SSB CW}
    \caption{Angewendeter Testaufbau für den Einseitenband- und CW-Demodulator}
    \label{fig:Testaufbau-SSB-CW}
\end{figure}
In der Abbildung \ref{fig:Testaufbau-SSB-CW} ist der Aufbau für den Test des Einseitenband- und CW-Demodulator skizziert. Mit einem Hack RF One wird das jeweilige Signal generiert und über eine Koaxialleitung an der USRP X 310 angeschlossen. Verwendet wird dabei der Bereich RF A und der Anschluss RX2. Für die Generierung der Einseitenbandsignale wird ein einfacher Einseitenbandmodulator in GNU-Radio erstellt. Als Testübertragung wird die Audiospur eine .wav-Datei als oberes oder unteres Seitenband übertragen. Der verwendete Signalflussgraph ist im GitHub-Repository hinterlegt. \newline
Für den ersten Test des Einseitenband-Demodulator wird die Audiospur bei $433.05\,\text{MHz}$ als ein oberes Seitenband übertragen. In der SDR-Software wird die Mittenfrequenz auf $433.2\,\text{MHz}$ und die Kanalfrequenz auf $433.05\,\text{MHz}$ eingestellt. Die Filterbandbreite beträgt $B=2.7\,\text{kHz}$ und es wird eine Verstärkung von $G_\mathrm{SDR}=25\,\text{dB}$ verwendet.
\begin{figure}[H]
    \centering
    \includegraphics[width=0.75\linewidth]{Bilder/HF USB.png}
    \caption{Aufgezeichnetes HF-Spektrum bei der Übertragung des oberen Seitenbandes}
    \label{fig:USB-HF-Spektrum}
\end{figure}
In der Abbildung \ref{fig:USB-HF-Spektrum} ist das mit dem USRP X310 aufgezeichnete HF-Spektrum bei der Übertragung des oberen Seitenbandes zu sehen. Im oberen Plot ist das Spektrum in Form eines Wasserfalldiagramms dargestellt. Das obere Seitenband (rote Linie) liegt bei der gewünschten Frequenz von $433.05\,\text{MHz}$ und kann deutlich vom Rauschen unterschieden werden. Im unteren Plot ist das Frequenzspektrum als FFT-Spektrum dargestellt. Auch hier sticht das obere Seitenband deutlich heraus. Erreicht wird ein Pegel von $-45\,\text{dB}$. Der Rauschpegel liegt bei ca. $-90\,\text{dB}$. Damit wird ein Signal-zu-Rauschabstand von $SNR = 45\,\text{dB}$ erreicht, was für die Demodulation des oberen Seitenbandes mehr als ausreichend ist.
\begin{figure}[H]
    \centering
    \includegraphics[width=0.75\linewidth]{Bilder/Basisband USB.png}
    \caption{Frequenzspektrum des demodulierten oberen Seitenbandes}
    \label{fig:USB-Basisband}
\end{figure}
Die Abbildung \ref{fig:USB-Basisband} zeigt das Spektrum des demodulierten Signals im Basisband. Erkennbar ist der Durchlassbereich von $300\,\text{Hz}$ bis $3\,\text{kHz}$ des Bandpassfilters. Durch die automatische Verstärkungseinheit wird der Pegel des Audiosignals auf ca. $-20\,\text{dB}$ angehoben. Dadurch ist das übertragende Audiosignal deutlich hörbar. Die aufgenommen Audiospur des übertragenden und demodulierten Testsignals ist im GitHub-Repository hinterlegt.\newine
Im zweiten Test des Einseitenband-Demodulators wird das Audiosignal als unteres Seitenband übertragen. Als Frequenz wird wieder $433.05\,\text{MHz}$ gewählt. Die Einstellung in der SDR-Software sind die gleichen wie im ersten Test. Jedoch wird als Seitenband jetzt das untere ausgewählt.
\begin{figure}[H]
    \centering
    \includegraphics[width=0.75\linewidth]{Bilder/HF LSB.png}
    \caption{Aufgezeichnetes HF-Spektrum bei der Übertragung des unteren Seitenbandes}
    \label{fig:LSB-HF-Spektrum}
\end{figure}
Die Abbildung \ref{fig:LSB-HF-Spektrum} zeigt das mit dem USRP X310 augenommene HF-Spektrum als Wasserfalldiagramm (oberer Plot) und als FFT-Spektrum (unterer Plot). In beiden Darstellungen kann das unteren Seitenband an der gewünschten Frequenz von $433.05\,\text{MHz}$ angetroffen werden. Ähnlich wie beim oberen Seitenband wird ein Pegel von $-45\,\text{dB}$, sowie ein Signal-zu-Rauschabstand von $SNR=45\,\text{dB}$ erreicht, welcher mehr als ausreichend für die Demodulation ist.
\begin{figure}[H]
    \centering
    \includegraphics[width=0.75\linewidth]{Bilder/Basisband LSB.png}
    \caption{Frequenzspektrum des demodulierten unteren Seitenbandes}
    \label{fig:LSB-Basisband}
\end{figure}
In der Abbildung \ref{fig:LSB-Basisband} ist das Frequenzspektrum des demodulierten unteren Seitenbandes dargestellt. Ebenfalls ist wieder der Durchlassbereich des Bandpassfilter zwischen $300\,\text{Hz}$ und $3\,\text{kHz}$ erkennbar. Auch hier wird der Pegel des Audiosignals durch die automatische Verstärkungseinheit auf ca. $-25\,\text{dB}$ angehoben. Dadurch liegt der Pegel des Audiosignals im wahrnehmbaren Bereich. Die aufgezeichnete Audiospur des demodulierten unteren Seitenbandes ist im GitHub-Repository hinterlegt.\newline
Zum Testen des CW-Demodulators wird als .wav-Datei ein Morsecode mit einem $700\,\text{Hz}$ Ton als oberes Seitenband bei $433.05\,\text{MHz}$ übertragen. In der SDR-Software wird die Mittenfrequenz auf $433.2\,\text{MHz}$ und die Kanalfrequenz auf $433.05\,\text{MHz}$ eingestellt. Die Grenzfrequenz des Tiefpassfilters wird auf $1.2\,\text{kHz}$ gesetzt. Die Verstärkung beträgt $G_\mathrm{SDR}=25\,\text{dB}$.
\begin{figure}[H]
    \centering
    \includegraphics[width=0.75\linewidth]{Bilder/HF CW.png}
    \caption{Aufgezeichnetes HF-Spektrum bei der Übertragung des CW-Signals}
    \label{fig:HF-CW}
\end{figure}
Die Abbildung \ref{fig:HF-CW} zeigt das aufgezeichnete HF-Spektrum bei der Übertragung des CW-Signals als Wasserfalldiagramm (oberer Plot) und als FFT-Spektrum (unterer Plot). Das CW-Signal kann in beiden Darstellungsarten an der gewünschten Frequenz von $433.05\,\text{MHz}$ angetroffen werden. Der maximale Pegel liegt dabei bei ca. $-42\,\text{dB}$. Der Rauschpegel liegt wieder bei ungefähr $-90\,\text{dB}$, was zu eine Signal-zu-Rauschabstand von $SNR=47\,\text{dB}$ führt. Dieser ist ebenfalls mehr als ausreichend für die Demodulierung der CW-Signale.
\begin{figure}[H]
    \centering
    \includegraphics[width=0.75\linewidth]{Bilder/Basisband CW.png}
    \caption{Frequenzspektrum des demodulierten CW-Signals}
    \label{fig:CW-Basisband}
\end{figure}
Bei dem Spektrum in Abbildung \ref{fig:CW-Basisband} handelt es sich das Frequenzspektrum des demodulierten CW-Signals. Die Grenzfrequenz des Tiefpassfilters bei $1.2\,\text{kHz}$ ist gut zu erkennen. Wie beim Einseitenbanddemodulator wird auch beim CW-Demodulator der Pegel des demodulierten Signals auf ca. $-25\,\text{dB}$ angehoben. Dadurch konnte der Morsecode währrend des Test deutlic wahrgenommen werden. Eine Audioaufnahme des demodulierten Morsecode ist im GitHub-Repository hinterlegt. Aufgrund der kurzen Pulse bei Morsecode Übertragungen ist der $700\,\text{Hz}$ Trägerton nicht sichtbar. 


\section{Auswerten und Vergleichen der Ergebnisse}
\subsection{Vergleich von Theorie und Praxis}

\subsection{Vergleich mit anderen Bodenstationen}

\section{Zusammenfassung}
\input{Fazit}

\section{Anhang}
\subsection{Literaturverzeichnis}
\printbibliography
\subsection{Abbildungsverzeichnis}
\listoffigures
\subsection{Tabellenverzeichnis}
\listoftables


\subsection{Python Skripts}
\begin{lstlisting}[caption={Simulation eines Mischers in Python}, label={lst:mischer-python}]
import numpy as np
import matplotlib.pyplot as plt
import scipy.signal as sig
#Define Samplingtime and Sampling Frequency
ts = 0.001              #Samplingtime 0.001s
t = np.arange(0,1,ts)   #time axis
#Define the signals and noise
A_s = 1*10**(-3)
A_r = 1*10**(-3)
A_i = 0.1*10**(-3)
f_s_t = 200                                 #Input Frequency
f_i_t = 100                                 #Image Frequency
f_n_t = 150                                 #LO Frequency
s_t = A_s * np.cos(2*np.pi*f_s_t*t)         #Input Signal
i_t = A_i * np.cos(2*np.pi*f_i_t*t)         #Image Signal
n_t = A_r * np.cos(2*np.pi*f_n_t*t)         #LO Signal
x_t = s_t * n_t                             #Mixing
xsp_t = i_t * n_t                             #Mixing
#Using the FFT
n = len(t)
fs = int(1/ts)                              #Defining the Sampling Frequency and those the sampling points
x_t_fft = np.fft.fft(x_t,fs)                #Doing the FFT
s_t_fft = np.fft.fft(s_t,fs)                #Doing the FFT
n_t_fft = np.fft.fft(n_t,fs)                #Doing the FFT
i_t_fft = np.fft.fft(i_t,fs)                #Doing the FFT
xsp_t_fft = np.fft.fft(xsp_t,fs)                #Doing the FFT
shifted_x_t_fft = np.fft.fftshift(x_t_fft)  #bringing the 0 Hz into the center
shifted_s_t_fft = np.fft.fftshift(s_t_fft)  #bringing the 0 Hz into the center
shifted_n_t_fft = np.fft.fftshift(n_t_fft)  #bringing the 0 Hz into the center
shifted_i_t_fft = np.fft.fftshift(i_t_fft)  #bringing the 0 Hz into the center
shifted_xsp_t_fft = np.fft.fftshift(xsp_t_fft)  #bringing the 0 Hz into the center
mag_x = abs(shifted_x_t_fft)/n                          #calculating the magnitude
mag_s = abs(shifted_s_t_fft)/n                          #calculating the magnitude
mag_n = abs(shifted_n_t_fft)/n                          #calculating the magnitude
mag_i = abs(shifted_i_t_fft)/n                          #calculating the magnitude
mag_xsp = abs(shifted_xsp_t_fft)/n                          #calculating the magnitude
mag_x_dB = 10*np.log10(mag_x[int(fs/2):]/0.001)
mag_s_dB = 10*np.log10(mag_s[int(fs/2):]/0.001)
mag_n_dB = 10*np.log10(mag_n[int(fs/2):]/0.001)
mag_i_dB = 10*np.log10(mag_i[int(fs/2):]/0.001)
mag_xsp_dB = 10*np.log10(mag_xsp[int(fs/2):]/0.001)
freq = np.arange(0,500,len(t)/fs)
#Plotting
plt.figure()
plt.subplot(211)
plt.plot(freq,mag_s_dB, label = "HF-Signal")
plt.plot(freq,mag_n_dB, label = "LO-Signal")
plt.plot(freq,mag_i_dB, label = "SP-Signal")
plt.ylabel("Signalstrength in [dBm]")
plt.xlabel("Frequency in [MHz]")
plt.grid()
plt.legend(loc='upper right')
plt.tight_layout(h_pad=1.5, w_pad=0.8) 
plt.subplot(212)
plt.plot(freq,mag_x_dB, label = "ZF-Signal")
plt.plot(freq,mag_xsp_dB, label = "SP-Signal")
plt.ylabel("Signalstrength in [dBm]")
plt.xlabel("Frequency in [MHz]")
plt.grid()
plt.legend(loc='upper right')
plt.tight_layout(h_pad=1.5, w_pad=0.8)  
\end{lstlisting}

\begin{lstlisting}[caption={Simulation von Rauschen und SNR in Python}, label={lst:Rauschen-SNR-python}]
import numpy as np
import matplotlib.pyplot as plt
import scipy.signal as sig
#Define Samplingtime and Sampling Frequency
ts = 0.001              #Samplingtime 0.001s
t = np.arange(0,1,ts)   #time axis
#Define the signals and noise
A_s = 1*10**(-3)
A_r = 3*10**(-3)
f_s_t = 50
s_t = A_s * np.sin(2*np.pi*f_s_t*t)         #wanted signal
n_t = A_r * np.random.randn(len(t))         #Generating Noise with the length of the time axis
x_t = s_t + n_t
#Using the FFT
n = len(t)
fs = int(1/ts)                              #Defining the Sampling Frequency and those the sampling points
s_t_fft = np.fft.fft(x_t,fs)                #Doing the FFT
shifted_s_t_fft = np.fft.fftshift(s_t_fft)  #bringing the 0 Hz into the center
mag = abs(shifted_s_t_fft)/n                          #calculating the magnitude
freq = np.arange(-500,500,len(t)/fs)
mag = 10*np.log10(mag[int(fs/2):]/0.001)
freq = np.arange(0,500,len(t)/fs)
#Plotting
plt.figure()
plt.subplot(311)
plt.plot(t,s_t*1000,label = "Wanted signal")
plt.plot(t,n_t*1000,label = "Noise")
plt.ylabel("Voltage in [mV]")
plt.xlabel("Time in [s]")
plt.grid()
plt.legend(loc='upper right')
plt.subplot(312)
plt.plot(t,x_t*1000,label = "Mixed signal")
plt.ylabel("Voltage in [mV]")
plt.xlabel("Time in [s]")
plt.grid()
plt.legend(loc='upper right')
plt.subplot(313)
plt.plot(freq,mag, label = "FFT of the mixed signal")
plt.ylabel("Signalstrength in [dBm]")
plt.xlabel("Frequency in [Hz]")
plt.grid()
plt.legend(loc='upper right')
plt.tight_layout(h_pad=1.5, w_pad=0.8)    
\end{lstlisting}

\begin{lstlisting}[caption={Berechnung der Dämpfung durch Regen}, label={lst:Dämpfung-durch-Regen-python}]
import numpy as np
import matplotlib.pyplot as plt
elevation = np.deg2rad(27.36)  #elevation of the Antenna [°]
tau = np.deg2rad(-12.412)       #skew of the Antenna
h_0 = 3                         #Height of the isothermic barrier of the region from the groundstation[km]
h_Station = 0.023               #Height of the groundstation above Sealevel [km]
freq = 10.5                    #Frequency at which the link operates [GHz]
cordiantes_station = (53.055, 8.78,0.0)
#determination of the effective rain height [km]
h_R = h_0 + 0.36
#determination of path length the EM-Wave has to travel below the 
D_s = (h_R-h_Station)/(np.sin(elevation))
#determination of the horizontal projection caused by the path length of the EM-Wave
D_HP = D_s*np.cos(elevation)
#determination of the Rain intensity [mm/h] which exceeding the anual mean by 0.01% of the time
R_001 = 35                    #Mean in the northern part of Germany 35 to 40 mm/h 
R_graph = np.arange(0,40, 0.1 ) #For Graph Plotting 0 mm/H to 40mm/h
#determination of the frequency dependend coeffizients. Equations from the Book
# k_H = 3.949*10**(-6)*freq**(3.4078)
# k_V = 2.785*10**(-6)*freq**(3.5032)
# alpha_H = -0.7451*np.log10(freq)+2.0211
# alpha_V = -0.8083*np.log10(freq)+2.0723
#von ITU-R P.838-3
k_H = 0.01217
k_V = 0.01129
alpha_H = 1.2571
alpha_V = 1.2156
#determination of the specific rain attenuatuion
k = (k_H + k_V + (k_H-k_V) * (np.cos(elevation))**2 * np.cos(2*tau))/2
alpha = (k_H * alpha_H + k_V * alpha_V + (k_H * alpha_H - k_V * alpha_V) * (np.cos(elevation))**2 * np.cos(2*tau)) / (2*k)
gamma_R001 = k*(R_001)**alpha
print("y_R001:",gamma_R001,"dB/km")
#calculation horizontal reduction factor
r001 = 1 / (1 + 0.78 * np.sqrt((D_HP*gamma_R001) / freq) - 0.38 * (1 - np.exp(-2 * D_HP)))
#calculation vertical adjustment factor
cc001 = np.rad2deg(np.arctan((h_R-h_Station)/(D_HP*r001)))
if cc001 > np.rad2deg(elevation):
    D_R001 = (D_HP*r001)/np.cos(elevation)
else:
    D_R001 = (h_R-h_Station)/np.sin(elevation)
if abs(cordiantes_station[0]) > 36:
    X = 36 - abs(cordiantes_station[0])
else:
    X = 0
v001 = 1 / (1 + np.sqrt(np.sin(elevation)) * (31 * (1 - np.exp( (-1)* ( elevation/(1+X))))*(np.sqrt(D_R001 *gamma_R001))/(freq**2)-0.45))
#for graph 
A_graph = np.empty_like(R_graph)
gamma_Rgraph = np.empty_like(R_graph)
for i, R in enumerate(R_graph): 
    gamma_Rgraph[i] = k * (R)**alpha
    rgraph = 1 / (1 + 0.78 * np.sqrt((D_HP * gamma_Rgraph[i]) / freq) - 0.38 * (1 - np.exp(-2 * D_HP)))
    ccgraph = np.rad2deg(np.arctan((h_R - h_Station) / (D_HP * rgraph)))
    if ccgraph > np.rad2deg(elevation):
        D_Rgraph = (D_HP * rgraph) / np.cos(elevation)
    else:
        D_Rgraph = (h_R - h_Station) / np.sin(elevation)
    if abs(cordiantes_station[0]) > 36:
        X = 36 - abs(cordiantes_station[0])
    else:
        X = 0
    vgraph = 1 / (1 + np.sqrt(np.sin(elevation)) * (31 * (1 - np.exp(-1 * (elevation / (1 + X)))) * 
                      np.sqrt(D_Rgraph * gamma_Rgraph[i]) / (freq**2) - 0.45))
    D_Regengraph = D_Rgraph * vgraph
    A_graph[i] = gamma_Rgraph[i] * D_Regengraph
print("Vertikal adjusmentfaktor 0.01:",v001)
#calculation effective path length D_Regen
D_Regen001 = D_R001*v001
print("Effectiv Path lenght through the rain 0.01:",D_Regen001,"km")
#calculation worst case attenuation for rain exceeded for 0.01% of an avarage year
A_001 = gamma_R001*D_Regen001
print("worst case attenuation caused by rain exceeded for 0.01% of an avarage year:",A_001,"dB")
plt.figure("Attenuation caused by rain 10 GHz")
plt.title("Dämpfung durch starke Niederschläge bei f = 10 GHz")
plt.plot(R_graph,A_graph)
plt.ylabel("Dämpfung in dB")
plt.xlabel("Niederschlagsmenge in mm/h")
plt.grid()
# Attenuatuion for other percanteges
unit = 1
p = 5*unit
long = 8.78
if p >= 1*unit or abs(long) >= 36:
    beta = 0
elif p < 1*unit and abs(long) < 36 and elevation >= 25:
    beta = (-1)*0.005*(abs(long)-36)
else:
    beta = (-1)*0.005*(abs(long)-36)+1.8-4.25*np.sin(elevation)
exponent = (-1)*(0.655+0.033*np.log(p)-0.045*np.log(A_001)-beta*(1-p)*np.sin(elevation))
A_other = A_001*(p/0.01)**exponent
print("Attenuation for Rainrates exceeding",p,"% of the time the anual avarage:",A_other,"dB")   
\end{lstlisting}


\begin{lstlisting}[caption={Berechnung des Link Budgets}, label={lst:Link-Budget-python}]
import numpy as np
import matplotlib.pyplot as plt
#Allgemeine Parameter
k = 1.38*10**(-23)
P_T = 17.78                         #Sendeleistung Es'Hail-2 in W
P_T_dBm = 42.5
G_T = 50.12                         #Gewinn der Sendeantenne
EIRP = P_T * G_T                    #EIRP von Es'Hail-2
EIRP_dBm = 59.5
B = 2.7*10**3                       #Bandbreite des Downlinks
L_FR = 2.9*10**(20)                 #Freiraumdämpfung
L_FR_dB = 204.61
L_OT = 3.33                         #Senderseitige Fehlausrichtung
L_OT_dB = 5.23
L_OR = 0.69                         #Empfangsseitige Fehlasusrichtung
L_OR_dB = 0.69
G_R = 7244.36
G_LNC = 316227.76
G_SDR = 1000
L_sys = 5.02  
G_sys = G_LNC*G_SDR*(1/L_sys)
T0 = 290
#Äquivalente Rauschtemperatur
Te1 = 133.4                #133.4
TeLNC = 139.2
Te2 = 200.1
TeBiasTee = 118.9
Te3 = 17.4
TePatchfeld = 58
Te4 = 8.7
TeRFSwitch = 20.3
Te5 = 8.7
TeSDR = 1539.9
G1 = 0.685
GLNC = 316227.77
G2 = 0.59
GBiasTee = 0.71
G3 = 0.94
GPatchfeld = 0.83
G4 = 0.97
GRFSwitch = 0.93
G5 = 0.97
T_esys = Te1 + (TeLNC/(G1) ) + (Te2/(G1*GLNC) ) + (TeBiasTee/(G1*GLNC*G2) ) + (Te3/(G1*GLNC*G2*GBiasTee) ) + (TePatchfeld/(G1*GLNC*G2*G3) ) + (Te4/(G1*GLNC*G2*G3*GPatchfeld) ) + (TeRFSwitch /(G1*GLNC*G2*G3*GPatchfeld*G4) )+ (Te5 /(G1*GLNC*G2*G3*GPatchfeld*G4*GRFSwitch) )+ (TeSDR /(G1*GLNC*G2*G3*GPatchfeld*G4*GRFSwitch*G5) )
print("Äquivalente Rauschtemperatur Te,sys:",T_esys,"K")
#klarer Himmel
print("Für Bedingung klarer Himmel:")
L_ATklarerHimmel = 1.13             #Dämpfung in der Atmosphäre bei klarem Himmel
L_ATklarerHimmel_dB = 0.547
T_AklarerHimmel = 6.5               #Antennentemperatur bei klaren Himmel
P_R_klarer_Himmel = EIRP*G_R*(1/L_FR)*(1/L_OT)*(1/L_OR)*(1/L_ATklarerHimmel)
P_R_klarer_Himmel_dB = 10*np.log10(P_R_klarer_Himmel/(0.001))
print("empfangene Leistung $P_R$:",P_R_klarer_Himmel,"W")
print("empfangene Leistung $P_R$",P_R_klarer_Himmel_dB,"dBm")
P_RX_klarer_Himmel = EIRP*G_R*(1/L_FR)*(1/L_OT)*(1/L_OR)*(1/L_ATklarerHimmel)*G_sys
P_RX_klarer_Himmel_dB = 10*np.log10(P_RX_klarer_Himmel/(0.001))
print("Ausgangsleistung des Empfangssystems $P_{RX}$:",P_RX_klarer_Himmel,"W")
print("Ausgangsleistung des Empfangssystems $P_{RX}$",P_RX_klarer_Himmel_dB,"dBm")
N_i = k*T_AklarerHimmel*B
SNR_i_klarerHimmel = P_R_klarer_Himmel/N_i
SNR_i_klarerHimmel_dB = 10*np.log10(SNR_i_klarerHimmel)
print("SNR am Eingang bei klaren Himmel:",SNR_i_klarerHimmel_dB,"dB")
N_o = k*(T_AklarerHimmel+T_esys)*B
SNR_o_klarer_Himmel = P_R_klarer_Himmel/N_o
SNR_o_klarer_Himmel_dB = 10*np.log10(SNR_o_klarer_Himmel)
print("SNR am Ausgang des Empfangssystems:",SNR_o_klarer_Himmel_dB,"dB")
T_S_klarer_Himmel = (T_AklarerHimmel/L_sys)+T0*(1-(1/L_sys))+T_esys
CN0_klarer_Himmel = P_RX_klarer_Himmel/(k*T_S_klarer_Himmel)
CN0_klarer_Himmel_dBHz = 10*np.log10(CN0_klarer_Himmel)
print("Qualität des Downlinks:",CN0_klarer_Himmel_dBHz,"dBHz")
Link_Budget_klarer_Himmel_label = np.array(["Sende-\nLeistung","EIRP","Freiraum-\nDämpfung","Ausrichtungs-\nVerluste","Dämpfung\n Atmosphäre","Empfangene\nLeistung","Leistung\n am Ausgang"])
Link_Budget_klarer_Himmel = np.array([P_T,EIRP_dBm,EIRP_dBm-L_FR_dB,EIRP_dBm-L_FR_dB-L_OR_dB-L_OT_dB,EIRP_dBm-L_FR_dB-L_OR_dB-L_OT_dB-L_ATklarerHimmel_dB,P_R_klarer_Himmel_dB,P_RX_klarer_Himmel_dB])
plt.figure("LinkBudget clear Sky")
plt.title("Link Budget bei klaren Himmel")
for i, val in enumerate(Link_Budget_klarer_Himmel):
    plt.annotate(f'{val:.2f}', (i, val), textcoords="offset points",
                 xytext=(0,10), ha='center')  # 10 Pkt über dem Punkt
plt.plot(Link_Budget_klarer_Himmel_label,Link_Budget_klarer_Himmel,'o-')
plt.ylabel("Leistung in dBm")
plt.grid()
plt.ylim([-180,90])
plt.xticks(range(len(Link_Budget_klarer_Himmel_label)), Link_Budget_klarer_Himmel_label, rotation=45)
plt.tight_layout(pad=0.5)
plt.show()
#leichter Regen
print("Für die Bedingung leichter Regen:")
L_ATleichterRegen = 1.24
L_ATleichterRegen_dB = 0.947
T_AleichterRegen = 19.29
P_R_leichter_Regen = EIRP*G_R*(1/L_FR)*(1/L_OT)*(1/L_OR)*(1/L_ATleichterRegen)
P_R_leichter_Regen_dB = 10*np.log10(P_R_leichter_Regen/(0.001))
print("empfangene Leistung $P_R$:",P_R_leichter_Regen,"W")
print("empfangene Leistung $P_R$",P_R_leichter_Regen_dB,"dBm")
P_RX_leichter_Regen = EIRP*G_R*(1/L_FR)*(1/L_OT)*(1/L_OR)*(1/L_ATleichterRegen)*G_sys
P_RX_leichter_Regen_dB = 10*np.log10(P_RX_leichter_Regen/(0.001))
print("Ausgangsleistung des Empfangssystems $P_{RX}$:",P_RX_leichter_Regen,"W")
print("Ausgangsleistung des Empfangssystems $P_{RX}$",P_RX_leichter_Regen_dB,"dBm")
N_i = k*T_AleichterRegen*B
SNR_i_leichter_Regen = P_R_leichter_Regen/N_i
SNR_i_leichter_Regen_dB = 10*np.log10(SNR_i_leichter_Regen)
print("SNR am Eingang bei klaren Himmel:",SNR_i_leichter_Regen_dB,"dB")
N_o = k*(T_AleichterRegen+T_esys)*B
SNR_o_leichter_Regen = P_R_leichter_Regen/N_o
SNR_o_leichter_Regen_dB = 10*np.log10(SNR_o_leichter_Regen)
print("SNR am Ausgang des Empfangssystems:",SNR_o_leichter_Regen_dB,"dB")
T_S_leichter_Regen = (T_AleichterRegen/L_sys)+T0*(1-(1/L_sys))+T_esys
CN0_leichter_Regen = P_RX_leichter_Regen/(k*T_S_leichter_Regen)
CN0_leichter_Regen_dBHz = 10*np.log10(CN0_leichter_Regen)
print("Qualität des Downlinks:",CN0_leichter_Regen_dBHz,"dBHz")
Link_Budget_leichter_Regen_label = np.array(["Sende-\nLeistung","EIRP","Freiraum-\nDämpfung","Ausrichtungs-\nVerluste","Dämpfung\n Atmosphäre","Empfangene\nLeistung","Leistung\n am Ausgang"])
Link_Budget_leichter_Regen = np.array([P_T,EIRP_dBm,EIRP_dBm-L_FR_dB,EIRP_dBm-L_FR_dB-L_OR_dB-L_OT_dB,EIRP_dBm-L_FR_dB-L_OR_dB-L_OT_dB-L_ATleichterRegen_dB,P_R_leichter_Regen_dB,P_RX_leichter_Regen_dB])
plt.figure("LinkBudget light Rain")
plt.title("Link Budget bei leichten Regen")
for i, val in enumerate(Link_Budget_leichter_Regen):
    plt.annotate(f'{val:.2f}', (i, val), textcoords="offset points",
                 xytext=(0,10), ha='center')  # 10 Pkt über dem Punkt
plt.plot(Link_Budget_leichter_Regen_label,Link_Budget_leichter_Regen,'o-')
plt.ylabel("Leistung in dBm")
plt.grid()
plt.ylim([-180,90])
plt.xticks(range(len(Link_Budget_leichter_Regen_label)), Link_Budget_leichter_Regen_label, rotation=45)
plt.tight_layout(pad=0.5)
plt.show()
#Regen
print("Für die Bedingung Regen:")
L_ATRegen = 9.14
L_ATRegen_dB = 9.61
T_ARegen = 240.1
P_R_Regen = EIRP*G_R*(1/L_FR)*(1/L_OT)*(1/L_OR)*(1/L_ATRegen)
P_R_Regen_dB = 10*np.log10(P_R_Regen/(0.001))
print("empfangene Leistung $P_R$:",P_R_Regen,"W")
print("empfangene Leistung $P_R$",P_R_Regen_dB,"dBm")
P_RX_Regen = EIRP*G_R*(1/L_FR)*(1/L_OT)*(1/L_OR)*(1/L_ATRegen)*G_sys
P_RX_Regen_dB = 10*np.log10(P_RX_Regen/(0.001))
print("Ausgangsleistung des Empfangssystems $P_{RX}$:",P_RX_Regen,"W")
print("Ausgangsleistung des Empfangssystems $P_{RX}$",P_RX_Regen_dB,"dBm")
N_i = k*T_ARegen*B
SNR_i_Regen = P_R_Regen/N_i
SNR_i_Regen_dB = 10*np.log10(SNR_i_Regen)
print("SNR am Eingang bei klaren Himmel:",SNR_i_Regen_dB,"dB")
N_o = k*(T_ARegen+T_esys)*B
SNR_o_Regen = P_R_Regen/N_o
SNR_o_Regen_dB = 10*np.log10(SNR_o_Regen)
print("SNR am Ausgang des Empfangssystems:",SNR_o_Regen_dB,"dB")
T_S_Regen = (T_ARegen/L_sys)+T0*(1-(1/L_sys))+T_esys
CN0_Regen = P_RX_Regen/(k*T_S_Regen)
CN0_Regen_dBHz = 10*np.log10(CN0_Regen)
print("Qualität des Downlinks:",CN0_Regen_dBHz,"dBHz")
Link_Budget_Regen_label = np.array(["Sende-\nLeistung","EIRP","Freiraum-\nDämpfung","Ausrichtungs-\nVerluste","Dämpfung\n Atmosphäre","Empfangene\nLeistung","Leistung\n am Ausgang"])
Link_Budget_Regen = np.array([P_T,EIRP_dBm,EIRP_dBm-L_FR_dB,EIRP_dBm-L_FR_dB-L_OR_dB-L_OT_dB,EIRP_dBm-L_FR_dB-L_OR_dB-L_OT_dB-L_ATRegen_dB,P_R_Regen_dB,P_RX_Regen_dB])
plt.figure("LinkBudget Rain")
plt.title("Link Budget bei Regen")
for i, val in enumerate(Link_Budget_Regen):
    plt.annotate(f'{val:.2f}', (i, val), textcoords="offset points",
                 xytext=(0,10), ha='center')  # 10 Pkt über dem Punkt
plt.plot(Link_Budget_Regen_label,Link_Budget_Regen,'o-')
plt.ylabel("Leistung in dBm")
plt.grid()
plt.ylim([-180,90])
plt.xticks(range(len(Link_Budget_Regen_label)), Link_Budget_Regen_label, rotation=45)
plt.tight_layout(pad=0.5)
plt.show()
\end{lstlisting}

\begin{lstlisting}[caption={Berechnung von Azimut, Elevatin und Skew der Antenne}, label={lst:Antenne-berechnung-python}]
import numpy as np
lat_ant = np.deg2rad(53.055)
long_ant = np.deg2rad(8.78)
lat_sat = np.deg2rad(0)
long_sat = np.deg2rad(25.8)
r_geo = 35790
r_earth = 6378
d_long = long_ant-long_sat
#Brechnung der Azimut
azimut = np.rad2deg(np.arctan((np.tan(d_long))/(np.sin(lat_ant))))+180
print("Azimut:",azimut,"°")
#Berechnung der Elevation
ratio = r_earth/(r_earth+r_geo)
elevation = np.rad2deg(np.arctan((np.cos(lat_ant)*np.cos(d_long)-ratio)/(np.sqrt(1-(np.cos(lat_ant)*np.cos(d_long))**2))))
print("Elevation:",elevation,"°")
#Berechnung Skew
offset = 0
skew = np.rad2deg(np.arctan((np.sin(d_long))/(np.tan(lat_ant))))-offset
print("Skew:",skew,"°")
\end{lstlisting}

\begin{lstlisting}[caption={Vergleich der Link Budgets},label={lst:Vergleich-Link-Budget}]
import numpy as np
import matplotlib.pyplot as plt
k = 1.38*10**(-23)
P_T = 17.78                         #Sendeleistung Es'Hail-2 in W
P_T_dBm = 42.5
G_T = 50.12                         #Gewinn der Sendeantenne
EIRP = P_T * G_T                    #EIRP von Es'Hail-2
EIRP_dBm = 59.5
B = 2.7*10**3                       #Bandbreite des Downlinks
L_FR = 2.9*10**(20)                 #Freiraumdämpfung
L_FR_dB = 204.61
L_OT = 3.33                         #Senderseitige Fehlausrichtung
L_OT_dB = 5.23
L_OR = 0.69                         #Empfangsseitige Fehlasusrichtung
L_OR_dB = 0.69
L_ATklarerHimmel = 1.13             #Dämpfung in der Atmosphäre bei klarem Himmel
L_ATklarerHimmel_dB = 0.547
T_AklarerHimmel = 6.5               #Antennentemperatur bei klaren Himmel
G_R = 7244.36
G_LNC = 316227.76
G_SDR = 1412.54
L_sys = 5.02  
G_sys = G_LNC*G_SDR*(1/L_sys)
P_R_klarer_Himmel = EIRP*G_R*(1/L_FR)*(1/L_OT)*(1/L_OR)*(1/L_ATklarerHimmel)
P_R_klarer_Himmel_dB = 10*np.log10(P_R_klarer_Himmel/(0.001))
P_RX_klarer_Himmel = EIRP*G_R*(1/L_FR)*(1/L_OT)*(1/L_OR)*(1/L_ATklarerHimmel)*G_sys
P_RX_klarer_Himmel_dB = 10*np.log10(P_RX_klarer_Himmel/(0.001))
Link_Budget_klarer_Himmel_label = np.array(["Sende-\nLeistung","EIRP","Freiraum-\nDämpfung","Ausrichtungs-\nVerluste","Dämpfung\n Atmosphäre","Empfangene\nLeistung","Leistung\n am Ausgang"])
Link_Budget_klarer_Himmel = np.array([P_T,EIRP_dBm,EIRP_dBm-L_FR_dB,EIRP_dBm-L_FR_dB-L_OR_dB-L_OT_dB,EIRP_dBm-L_FR_dB-L_OR_dB-L_OT_dB-L_ATklarerHimmel_dB,P_R_klarer_Himmel_dB,P_RX_klarer_Himmel_dB])
G_Goon_dB = 65
L_FR_G_dB = 204.34
L_OT_G_dB = 1.91
P_R_G = -107.77
P_RX_G = -42.77
Link_Budgter_klarer_Himmel_Goonhilly = np.array([P_T,EIRP_dBm,EIRP_dBm-L_FR_G_dB,EIRP_dBm-L_FR_G_dB-L_OR_dB-L_OT_G_dB,EIRP_dBm-L_FR_G_dB-L_OR_dB-L_OT_G_dB-L_ATklarerHimmel_dB,P_R_G,P_RX_G])
plt.figure("LinkBudget clear Sky")
plt.title("Vergleich des Link Budgets bei klaren Himmel")
for i, val in enumerate(Link_Budget_klarer_Himmel):
    plt.annotate(f'{val:.2f}', (i, val), textcoords="offset points",
                 xytext=(0,10), ha='center')  # 10 Pkt über dem Punkt
    for i, val in enumerate(Link_Budgter_klarer_Himmel_Goonhilly):
        plt.annotate(f'{val:.2f}', (i, val), textcoords="offset points",
                     xytext=(0,-15), ha='center')  # 10 Pkt über dem Punkt
plt.plot(Link_Budget_klarer_Himmel_label,Link_Budget_klarer_Himmel,'o-',label="IAT")
plt.plot(Link_Budget_klarer_Himmel_label,Link_Budgter_klarer_Himmel_Goonhilly,'o-',label="Goonhilly")
plt.ylabel("Leistung in dBm")
plt.grid()
plt.legend()
plt.ylim([-180,90])
plt.xticks(range(len(Link_Budget_klarer_Himmel_label)), Link_Budget_klarer_Himmel_label, rotation=45)
plt.tight_layout(pad=0.5)
plt.show()    
\end{lstlisting}

\end{document}