Im heutigen Zeitalter sind Satelliten nicht mehr aus unseren Leben wegzudenken. Sie werden für die verschiedenste Dinge benötigt, wie Navigation zur Arbeit oder nächsten Urlaubsziel, für Rundfunk und Fernsehn, für Internet und für vieles mehr. Damit aus den Satelliten auch ein Nutzen gezogen werden kann, muss eine Kommunikationsmöglichkeit mit diesen bestehen. Hier für werden sogenannte Bodenstationen eingesetzt. Bodenstation werden zur Steuerung und Verfolgen der Satelliten, sowie zum Senden und Empfangen von Daten und Informationen von Satelliten eingesetzt.\newline
Zurzeit entsteht am Institut für Aerospace Technologie (IAT) eine Satellitenbodenstation. Für die ersten Versuche und Test eignen sich der Empfang von Satelliten in einer geostationären Umlaufbahn. Eine akktraktive Möglichkeit bietet dabei der Kommunikations- und Amateurfunksatellit Es'Hail-2, auch bekannt unter den Rufname Qatar Oscar 100 (QO-100). Neben dem Equipment zur kommerziellen Nutzung des Satelliten, befinden sich ebenfalls zwei Amateurfunktransponder an Bord von Es'Hail-2 (QO-100). Bei den beiden Transponder handelt es sich um einen Schmal- und Breitbandtransponder, welche von Amateurfunkern aus verschiedenen Länder für verschiedene Zwecke frei verwendet werden können.
\subsection{Motivation}
Schon seit längerer Zeit interessiere ich mich privat für Kommunikations- und Hochfrequenztechnik, sowie für Satelliten und Raumfahrt im Allgemeinen. In den letzten zwei Jahren hat sich dieses Interesse zunehmend auf den Bereich der Satellitenkommunikation konzentriert. Erste eigene Versuche, Signale und Bilder der NOAA-Wettersatelliten zu empfangen, haben mein technisches Interesse weiter vertieft.\newline
Daher war es mir ein besonderes Anliegen, für meine Bachelorarbeit ein Thema im Bereich der Kommunikations- und Hochfrequenztechnik beziehungsweise der Satellitenkommunikation zu wählen. Über Prof. Dr. Peik und Prof. Dr. García ergab sich schließlich die Möglichkeit, im Rahmen der entstehenden Bodenstation des IAT ein Empfangssystem für geostationäre Satelliten im X‑Band zu planen, zu entwickeln und aufzubauen.

\subsection{Zielsetzung und Vorgehen}
Diese Arbeit beschäftigt sich mit der Planung, Entwicklung und dem Aufbau einer Empfangsstation für den Downlink des Schmalbandtransponder auf Es'Hail-2 (QO-100), der sich im X-Band befindet. Ziel ist der Aufbau und die Inbetriebnahme eines funktionsfähigen Empfangssystems, sowie das erfolgreiche Empfangen und Dekodieren von Amateurfunksignalen.\newline
Zur Erreichung des Ziels werden im ersten Schritt der Schmalbandtransponder und der 
Satellit Es'Hail-2 (QO-100) genauer untersucht. Dabei werden die technischen Eigenschaften des Satelliten und des Downlink, sowie relevante Empfangsparameter recherchiert.\newline
Im zweiten Schritt wird der Downlink theoretisch betrachtet. Besonderer Fokus liegt dabei auf der Übertragungsstrecke zwischen dem Satelliten und der Bodenstation am IAT. Untersucht werden die verschiedenen Dämpfungen, welche in den unterschiedlichen Abschnitten der Übertragungsstrecke auftreten. Von besonderem Interesse sind dabei die Dämpfungen in der Atmosphäre bei verschiedenen Wetterbedingungen und deren Auswirkungen auf den Downlink. Auf Basis dieser Informationen kann dann anschließend ein geeignetes Empfangssystem geplant und entwickelt werden.\newline 
Hierzu wird zunächst eine an der Hochschule bereits vorhandene Satellitenschüssel auf ihre Eignung für den Empfang des Downlinks überprüft. Weiterhin muss ein geeigneter Antennenfeed zum empfangen des Downlink, ein Verstärker zur Anhebung der empfangenen Signale auf ein verarbeitbaren Pegel, ein Abwärtsmischer zur Umsetzung der empfangenen Signale vom X-Band in einen niedrigeren Frequenzbereich, sowie geeignete Koaxialleitungen für die unterschiedlichen Abschnitte des Empfangssystems gewählt werden. Die ausgewählten Komponenten müssen dabei mit dem bereits vorhandenen Equipment - einer RF-Schaltmatrix von Mini-Circuits, einem Patchfeld und einem USRP X310 Software Defined Radio - kompatibel sein.\newline
Im nächsten Schritt werden die Eigenschaften des geplanten Empfangssystems ermittelt. Dazu gehören die im RF-Bereich des Empfangssystems auftretenden Dämpfungen und Verluste, die erzielten Verstärkungen, sowie das Eigenrauschen der Komponenten und damit des gesamten Empfangssystems. Mit diesen Werten kann die Empfangsgüte und die theoretischen Empfangsparameter, wie empfangene Leistung, empfangenes Rauschen, Rauschleistung im Empfangssystem, Leistung am Ausgang, das Link Budget und die Qualität des Downlinks für unterschiedliche Wetterbedingungen ermittelt und miteinander verglichen werden. Auf dieser Grundlage kann eine Aussage über die Leistungsfähigkeit des entwickelten Empfangssystems getroffen werden.\newline
Abschließend wird der Aufbau des geplanten und entwickelten Empfangssystems dokumentiert und eine Software zur Steuerung des Software Defined Radio, sowie zur Dekodierung der unterschiedlichen Signale im Downlink mithilfe von GNU Radio entwickelt. Zu dem werden die in der Praxis erreichten mit den in der Theorie bestimmten Empfangsparameter verglichen, sowie das Link Budget des Empfangssystem am IAT mit dem Link Budget des Empfangssystem an der Goonhilly in Cornwall (England) Bodenstation verglichen.



