



\subsection{Aufstellen und Ausrichten der Antenne}\label{Ausrichten der Antenne}
Die Antenne soll auf einem eigenen Antennenstandfuß auf dem Dach der Hochschule montiert werden. Der neue Antennenfeed wird mithilfe des alten Feedhalters im Fokuspunkt der Parabolschüssel montiert. Für die Anbringung des neuen Antennenfeeds wird ein Adapter entwickelt, welcher mithilfe eines 3D-Druckers angefertigt wird. Als Material wird PETG-HF von Bambu Lab verwendet. Die Druckdatei ist im GitHub-Repository hinterlegt.\newline
\begin{figure}[H]
    \centering
    \includesvg[width=0.5\linewidth]{Bilder/Aufbau Antennenmast}
    \caption{Skizze vom Aufbau der Antenne auf dem Dach der Hochschule}
    \label{fig:Aufbau-der-Antenne}
\end{figure}
Die Abbildung \ref{fig:Aufbau-der-Antenne} zeigt den geplanten Aufbau der Antenne auf dem Dach der Hochschule. Dieser ist zum größtenteils auch umgesetzt. Der vorgesehene Antennenmast und Antennenfuß ist nicht bis zum Abschluss der Arbeit angekommen, weshalb ein provisorischer Antennenmast gebaut wird.
\begin{figure}[H]
    \centering
    \includegraphics[width=0.4\linewidth]{Bilder/Aufgebaute Antenne.jpg}
    \caption{Umgesetzter Aufbau der Antenne auf dem Dach der Hochschule. Im Schwarzen Gehäuse am Mast ist der LNC untergebracht.}
    \label{fig:Umgesetzter-Aufbau-Antenne}
\end{figure}
Für die Ausrichtung der Antenne auf Es'Hail-2 (QO-100) muss der Azimut $\varphi$ und die Elevation $\varepsilon$ bestimmt werden. Mit Gleichung \ref{eq:Azimut} kann der Azimut $\varphi$ bestimmt werden. Die Koordinaten der Bodenstation $53.055\degree,8.78\degree$ können aus der Abbildung \ref{fig:Koordinaten der Bodenstation} entnommen werden. Der Satellit Es'Hail-2 (QO-100) befindet sich bei  $25.9\degree \text{E}$\cite{EsHail2}. Damit beträgt die Differenzen zwischen den Längengraden $\Delta long=53.055\degree-25.8\degree=27.255\degree$.
\begin{equation}
    \varphi=\arctan\left(\frac{\tan(\Delta long)}{\sin(lat_\mathrm{BS})} \right)=\arctan\left(\frac{\tan(27.255\degree)}{\sin(8.78\degree)} \right)=159.04\degree
    \label{eq:Azimut-Antenne}
\end{equation}
Für die Ermittlung der Elevation $\varepsilon$ der Antenne wird die Gleichung \ref{eq:Elevation} verwendet. Der Radius der Erde beträgt $r_\mathrm{0}=6378\,\text{km}$ und die Flughöhe von Es'Hail-2 (QO-100) kann mit $r=35790\,\text{km}$ angenommen werden.\cite{Satellitenkommunikation}
\begin{equation}
\begin{split}
    \varepsilon&=\arctan\left(\frac{\cos(lat_\mathrm{BS})\cdot\cos(\Delta long)-\frac{r_\mathrm{0}}{r_\mathrm{0}+r}}{\sqrt{1-\cos^2(lat_\mathrm{BS})\cdot\cos^2(\Delta long)}} \right)\\
    &=\arctan\left(\frac{\cos(8.78\degree)\cdot\cos(27.255\degree)-\frac{6378\,\text{km}}{6378\,\text{km}+35790\,\text{km}}}{\sqrt{1-\cos^2(8.78\degree)\cdot\cos^2(27.255\degree)}} \right)\\
    &=27.36\degree
\end{split}
    \label{eq:Elevation-Antenne}
\end{equation}
Zum Schluss muss noch die Drehung des Antennenfeeds um seine eigene Achse, der $Skew$, ermittelt werden. Dieser kann mit der Gleichung \ref{eq:Skew} bestimmt werden. Der Offset des Antennenfeeds beträgt $0\degree$.
\begin{equation}
    Skew = \arctan\left(\frac{\sin(\Delta long)}{\tan (lat_\mathrm{BS})}\right)-Offset=\arctan\left(\frac{\sin(27.255\degree)}{\tan (8.78\degree)}\right)=-12.4\degree
\end{equation}
Das Python-Skript zur Berechnung der Azimut $\varphi$, Elevation $\varepsilon$ und Drehung $Skew$ der Antenne ist im GitHub-Repository und im Anhang \ref{lst:Antenne-berechnung-python} hinterlegt. Mithilfe eines Kompass wird der Azimut $\varphi=159.04\degree$ der Antenne eingestellt. Die Elevation $\varepsilon=27.36\degree$ kann mithilfe einer Skala an der Antenne eingestellt werden. Die Neigung des Antennenfeeds um die eigene Achse $Skew=-12.4\degree$ wird mithilfe einer Winkellehre eingestellt. Als Ursprungpunkt wird dabei der SMA-Anschluss des X-Band Feeds gewählt.

\subsection{Abwärtsmischer, Fernspeiseweiche und Verkabelung im Serverschrank}
Der rauscharme Abwärtsmischer (LNC) wird neben der Antenne montiert. Um diesen vor Wettereinflüssen, wie Regen und Schnee, zu schützen, wird dieser in einem wasserdichten Gehäuse untergebracht (schwarzer Kasten in Abbildung \ref{fig:Umgesetzter-Aufbau-Antenne}).
\begin{figure}[H]
    \centering
    \includegraphics[width=0.6\linewidth]{Bilder/LNC im Gehäuse.jpg}
    \caption{Unterbringung des LNC im Gehäuse. Dank der Schiene kann dieser leicht ein- und ausgebaut werden. Die 50-Cent Münze dient als Größenvergleich.}
    \label{fig:LNC-im-Gehäuse}
\end{figure}
Der LNC wird, wie in Abbildung \ref{fig:LNC-im-Gehäuse} zu sehen, auf einer Schiene montiert. Die Druckdatei für die Halterung auf der Schiene sind ebenfalls im GitHub-Repositroy hinterlegt. Durch diese Schiene kann der LNC einfach montiert und bei Bedarf auch wieder demontiert werden. Das Gehäuse bietet theoretisch noch Platz für zusätzliche Komponenten, z.B. einen Aufwärtsmischer oder Leistungsverstärker für die Einrichtung eines Uplink zu Es'Hail-2 (QO-100). Ebenfalls bietet das Gehäuse mehrere Durchführungsmöglichkeiten für die Koaxialleitungen. Mithilfe der S\_04212\_B Koaxialleitung soll der Antennenfeed an den HF-Eingang des Abwärtsmischers angeschlossen werden. Über die LMR 400 FR Koaxialleitung wird das resultierende ZF-Signal zur Fernspeiseweiche im Serverschrank gebracht.
\begin{figure}[H]
    \centering
    \includesvg[width=0.5\linewidth]{Bilder/Verkabelung Serverschrank}
    \caption{Skizze der geplanten Verschaltung im Serverschrank}
    \label{fig:geplante-Verschaltung}
\end{figure}
Die Skizze in Abbildung \ref{fig:geplante-Verschaltung} zeigt die geplante Verschaltung im Serverschrank. Die Fernspeiseweiche wird an ein Labornetzteil angeschlossen, welches die Versorgungsspannung $V_\mathrm{CC}$ für den Abwärtsmischer liefert. Dieser soll im QO-100 SSB Betriebsmodus betrieben werden, wofür eine Spannung von $V_\mathrm{CC}=12-17\,\text{V}$ über die Fernspeiseweiche eingespeist werden muss. In diesem Modus liegt der HF-Bereich des Abwärtsmischer zwischen $10489\,\text{MHz}$ und $10490\,\text{MHz}$. Der Frequenzbereich des resultierende ZF-Singals liegt zwischen $433\,\text{MHz}$ und $434\,\text{MHz}$.Über die beiden Anschlusspins an der Fernspeiseweiche wird mit dem Labornetzteil eine Spannung von $V_\mathrm{CC}=15\,\text{V}$ angelegt.\newline
Nach der Fernspeiseweiche wird das ZF-Signal über eine LMR 400 FR Koaxialleitung an das Patchfeld angeschlossen. Ab hier kann die Verkabelung mithilfe der beiden Hyperflex 5 Koaxialleitungen leicht vorgenommen und bei Bedarf auch verändert werden. Mit dieser Koaxialleitung wird das ZF-Signal zur RF-Schaltmatrix und und anschließend an den USRP X310 angeschlossen.\newline
\begin{figure}[H]
    \centering
    \includegraphics[width=0.4\linewidth]{Bilder/Aufbau-Serverschrank.jpg}
    \caption{Umgesetzter Aufbau der Komponenten im Serverschrank. Die Fernspeiseweiche ist nicht auf dem Bild zu sehen, da sie sich am hinteren Ende des Serverschranks befindet.}
    \label{fig:Aufbau-Serverschrank}
\end{figure}
Die Abbildung \ref{fig:Aufbau-Serverschrank} zeigt den umgesetzten Aufbau des Empfangssystem im Serverschrank. Der Aufbau konnte fast vollständig umgesetzt werden. Nur eine Hyperflex 5 Koaxialleitung ist nicht angekommen, weshalb eine kurze LMR 400 Leitung als Ersatz verwendet wird.


\subsection{Erstellen einer SDR-Software mit GNU Radio Companion}\label{kap:SDR Software}
Um den USRP X310 zu steuern und den Downlink von Es'Hail-2 (QO-100) verarbeiten zu können, wird mithilfe von GNU Radio Companion eine, für diese Anwendung spezifische, SDR-Software erstellt.\newline
GNU Radio ist ein freies und Open-Source Framework, welches Nutzern die Möglichkeit bietet Radio Systeme zu entwickeln, simulieren und anzuwenden. Der GNU Radio Companion ist das zugehörige Programm, mit welchen einfache Signalflussgraphen zur digitalen Signalverarbeitung erstellt werden können.\cite{GNU-Radio}

\subsubsection*{Herstellen der Verbindung zum USRP X310, einstellen der Frequenz und Ausgabe des Frequenzspektrums}
\begin{figure}[H]
    \centering
    \includegraphics[width=0.65\linewidth]{Bilder/Eingang.png}
    \caption{Verwendeter Quellblock und variable Frequenzeinstellung in GNU Radio}
    \label{fig:Eingang}
\end{figure}
Die Abbildung \ref{fig:Eingang} zeigt den Anfangsbereich des Signalflussgraphen der erstellten SDR-Software. Den Anfang des Signalflussgraphen bildet ein Quellblock (engl. Source Block). Dieser wird benötigt, um eine Verbindung mit dem USRP X310 herzustellen. In GNU Radio gibt es mehrere verschiedene Quellblöcke zur Auswahl. Welcher verwendet werden kann hängt vom verwendeten SDR ab. Der verwendete USRP X310 bietet Unterstützung für den UHD-Treiber, weshalb als Quellblock ein UHD: URSP Quellblock verwendet wird.\cite{USRP-X310-Doku}\cite{GNU-Radio-USRP-Source}\newline
Über den Quellblock können viele verschiedene Einstellung des USRP X310 angepasst werden. Allerdings werden nicht alle Einstellmöglichkeiten benötigt. 
\begin{table}[H]
    \centering
    \begin{tabular}{c|c|p{7cm}}
       Einstellung  & Wert & Beschreibung\\
       \hline
        Abtastrate & $1\,\text{MS/s}$ & Setzen der Abtastrate vom SDR und damit auch die Bandbreite\cite{GNU-Radio-USRP-Source} \\
        Anzahl Kanäle & $1$ & Es wird nur ein Kanal für die geplante Anwendung benötigt. \\
        Mittenfrequenz & variabel & Die Mittenfrequenz kann variabel über eine Eingabe oder einen Schieber eingestellt werden. Als Standardwert wird $433.5\,\text{MHz}$ gewählt.\\
        RF Verstärkung & variabel & Die Verstärkung durch den USRP X310 kann variabel mit einem Schieber oder über eine Eingabe zwischen $0$ und $31.5\,\text{dB}$ angepasst werden. Als Standardverstärkung wird $0\,\text{dB}$ gewählt.\\
        Bandbreite & $0$ & Mit der Einstellung $0$ wird die Standardbreite vom Antialiasing Filter des USRP X310 verwendet.\cite{GNU-Radio-USRP-Source}\\
        Geräte Adresse & - & Hier kann die Geräte Adresse (z.B. IP-Adresse) des jeweilige USRP eingetragen werden. Das Feld wird leer gelassen. Damit wird der erste vom Programm gefundene USRP verwendet.\cite{GNU-Radio-USRP-Source}
    \end{tabular}
    \caption{Einstellungen im UHD: USRP Quellblock}
    \label{tab:Einstellungen-USRP-Quellblock}
\end{table}
Die Tabelle \ref{tab:Einstellungen-USRP-Quellblock} zeigt die getroffenen Einstellung im UHD: USRP Quellblock und damit auch die Einstellung des USRP X310. Die Mittenfrequenz wird Standardmäßig auf $433.5\,\text{MHz}$ gesetzt, da die ZF-Frequenz des Abwärtsmischer im SSB QO-100 Betrieb zwischen $433\,\text{MHz}$ und $434\,\text{MHz}$ liegt. Der Downlink des Schmalbandtransponder auf Es'Hail-2 (OQ-100) liegt dabei zwischen $433.5\,\text{MHz}$ und $434\,\text{MHz}$\newline
Das empfangene Frequenzspektrum soll mithilfe eines Wasserfalldiagramm und eines FFT-Spektrum dargestellt werden. Dafür gibt der Quellblock die Abtastwerte an einen Wasserfallblock (QT GUI Waterfall Sink) und einen FFT-Spektrumblock (QT GUI Frequency Sink) weiter\cite{GNU-Radio-Frequency-Sink}\cite{GNU-Radio-Waterfall-Sink}. Die Mittenfrequenz in beiden Spektren entspricht dabei der Mittenfrequenz des vom Quellblock ausgegeben Spektrums. Bei der Veränderung der Variable Mittenfrequenz verändert sich dementsprechend auch die dargestellte Mittenfrequenz.\newline
Der Multipliziererblock am Ausgang des Quellblocks wird als Mischer eingesetzt. Die Signalquelle fungiert dabei als lokaler Oszillator. Mit diesen beiden Blöcken besteht die Möglichkeit eine beliebige Frequenz im, vom Quellblock ausgegeben, Frequenzspektrum auszuwählen. Diese Frequenz wird Kanalfrequenz bezeichnet und kann über eine Eingabe oder durch einen Schieber im Betrieb verändert werden. Die Frequenz des lokalen Oszillator entspricht der Differenz zwischen der Mittenfrequenz und der Kanalfrequenz.\newline
Über einen folgenden Auswahlblock kann die Art der Demodulation ausgewählt werden. Zur Verfügung stehen dabei Einseitenband Amplitudenmodulation (SSB) mit oberen (USB) Seitenband und unteren (LSB) Seitenband, Frequenzmodulation (FM) und kontinuierliche Welle (engl. Continious Wave) CW.
\begin{figure}[H]
    \centering
    \includegraphics[width=0.65\linewidth]{Bilder/Ausgabebereich.png}
    \caption{Ausgabe des Audiosignals in GNU Radio}
    \label{fig:Ausgabe}
\end{figure}
Die Abbildung \ref{fig:Ausgabe} zeigt die möglichen Ausgabemöglichkeiten des Audiosignals. Das Audiosignal stammt dabei von dem jeweiligen angewendeten Demodulationsverfahren und wird über einen Auswahlblock an die Audio Ausgabe weitergegeben.\newline
Über einen FFT-Frequenzspektrum Block wird das Frequenzspektrum des Audiosignals angezeigt, wobei nur die positiven Frequenzen ab $0\,\text{Hz}$ dargestellt werden.\cite{GNU-Radio-Frequency-Sink}\newline
Das Audiosignal hat eine Abtastrate von $48\,\text{kS/s}$ und wird über eine Audio Senke an die Standard Ausgabe des Systems weitergegeben.\cite{GNU-Radio-Audiosink}\newline 
Die Lautstärke wird über den Multipliziererblock festgelegt. Sein Wert kann über einen Schieber im Betrieb zwischen $0$ und $1$ angepasst werden. Dabei entspricht $0$ stumm geschaltet und $1$ volle Lautstärke.\newline
Auch besteht die Möglichkeit das Audio Signal als eine 16-Bit .wav abzuspeichern. Verwendet wird dafür eine .wav Datei Senke (Wav File Sink). Der Speicherort, sowie der Name der .wav-Datei muss vor dem Start der Software in diesem Block angeben werden. Die Abtastrate des Audio Signals wird auf $48\,\text{kS/s}$ gestellt. Über einen Auswahlblock kann im Betrieb der Software die Aufnahme dann gestartet oder gestoppt werden.\cite{GNU-Radio-WAV}
\subsubsection*{FM-Demodulaton}
Bei der Frequenzmodulation (FM) werden die Informationen aus dem Basisband der Phase des Trägersignals auf moduliert.\cite{Nachrichtentechnik}\newline
\begin{equation*}
\begin{split}
        s_\mathrm{FM}(t)&=\hat{u}_\mathrm{T}\cdot \cos(\Psi_\mathrm{FM}(t))\\
        &=\hat{u}_\mathrm{T}\cdot \cos\left(\omega_\mathrm{T}+2\pi\cdot\Delta F\cdot\int u( \tau)d\tau\right)
\end{split}
\end{equation*}
Das Spektrum des FM-Signal kann nicht einfach angegeben werden, da es sich um ein nichtlineares Modulationsverfahren handelt. Die Anzahl der Nebenschwingungen neben dem Träger resultieren aus der Besselfunktion 0-ter Ordnung.\cite{Nachrichtentechnik}\newline
Für die Demodulation der FM-Signale wird eine erweiterte Version einer Beispielschaltung aus GNU Radio Dokumentation verwendet.\cite{GNU-Radio-FM}
\begin{figure}[H]
    \centering
    \includegraphics[width=0.75\linewidth]{Bilder/FM-Demodulator.png}
    \caption{Angewendeter FM Demodulator in GNU Radio}
    \label{fig:FM-Demodulator}
\end{figure}
Im ersten Schritt wird mit einem Tiefpassfilter das Frequenzspektrum begrenzt. So wird nur das gewünschte FM-Signal an den Demodulator weitergegeben. Die Grenzfrequenz des Tiefpasses kann über die Variable Filterbreite mit einem Schieber oder einer Eingabe verändert werden. Bei einer Radiostation beträgt die Bandbreite des FM-Signals ca. $120\,\text{kHz}$.\newline
Im nächsten Schritt wird mit einem rationalen Resampler die Abtastrate von $1\,\text{MS/s}$ auf $480\,\text{kS/s}$ reduziert.\cite{GNU-Radio-FM}\cite{GNU-Radio-Resampler}
\begin{equation*}
    \text{Faktor}=\frac{480000}{1000000}=\frac{12}{25}
\end{equation*}
Die Abtastrate wird erst um den Faktor $12$ erhöht und anschließend um den Faktor $10$ reduziert.\newline
Anschließend wird das FM-Signal mit dem FM-Demod Block von GNU Radio demoduliert.\cite{GNU-Radio-FM}. Die Kanalrate (engl. Channel Rate) wird auf die $480\,\text{kS/s}$ gestellt und die Audio Reduzierung auf $10$. Damit beträgt die Abtastrate des Audiosignals am Ausgang $48\,\text{kS/s}$. Somit kann das Audiosignal an die Audio Ausgabe weitergegeben werden.\cite{GNU-Radio-FM}\newline
Der FM-Demodulator wird mit in die SDR-Software aufgenommen, da mit diesem der Umgang mit dem SDR erprobt werden kann. Dieser kann verwendet werden um Radiostation zu empfangen und so z.B. neue Funktionen der Software zu testen.

\subsubsection*{Einseitenband-Demodulation}
Bei der Amplitudenmodulation (AM), wozu die Einseitenbandmodulation gehört, werden Information aus dem Basisband mit einem Mischer auf die Amplitude eines sinusförmigen Trägersignals auf moduliert. Das Trägersignal liegt dabei in einem für die Übertragung geeigneten Frequenzbereich.\cite{Nachrichtentechnik}\newline
\begin{equation*}
    \begin{split}
         s_\mathrm{AM}(t)&=s_\mathrm{BB}(t)\cdot s_\mathrm{T}(t)\\
         &=\frac{\hat{u}_\mathrm{BB}\cdot \hat{u}_\mathrm{T}}{2}\cdot (\cos((\omega_\mathrm{T}-\omega_\mathrm{BB})\cdot t)+\cos((\omega_\mathrm{T}+\omega_\mathrm{BB})\cdot t)
    \end{split}
\end{equation*}
\begin{figure}[H]
    \centering
    \includesvg[width=0.4\linewidth]{Bilder/AM-Signal}
    \caption{Spektrum eines AM-Signals}
    \label{fig:AM-Signal}
\end{figure}
Durch den Mischvorgang entstehen zwei Frequenzkomponenten links uns rechts neben dem Trägersignal, wie in Abbildung \ref{fig:AM-Signal} zu sehen. Diese beiden Frequenzkomponenten werden oberes und unteres Seitenband (engl. Upper- and Lower Sideband) genannt. Dabei befindet sich das obere Seitenband in der Regellage und das untere Seitenband in der Kehrlage\cite{Nachrichtentechnik}\newline
Beide Frequenzkomponenten enthalten dabei die gleichen Informationen aus dem Basisband, sprich sie sind identisch zu einander. Aus diesem Grund kann bei der Übertragung oder bei der Demodulation auf eins der beiden Seitenbänder verzichtet werden.\cite{Nachrichtentechnik}\newline
Um ein Einseitenbandsignal im Downlink von Es'Hail-2 (QO-100) demodulieren zu können, muss dieses zurück in das Basisband gebracht werden. Das resultierende komplexe Basisbandsignal muss dann nur noch mit einem Tief- oder Bandpass begrenzt werden. Für die Umsetzung des Einseitenband-Demodulators in GNU Radio wird eine angepasste Version eines Einseitenband-Demodulators aus der GNU Radio Dokumentation verwendet.\cite{GNU-Radio-SSB}
\begin{figure}[H]
    \centering
    \includegraphics[width=0.8\linewidth]{Bilder/SSB-Demodulator.png}
    \caption{Signalflussgraph des umgesetzten Einseitenband-Demodulator in GNU Radio}
    \label{fig:SBB-Demodulator}
\end{figure}
Die Abbildung \ref{fig:SBB-Demodulator} zeigt den in GNU Radio umgesetzten Einseitenband-Demodulator. Das Herzstück des Demodulators ist ein Frequenz-umsetzender FIR Filter (Frequency Xlating FIR Filter). Dieser FIR Filter kombiniert mehrere Funktionen in einem Block. Das am Eingang anliegende Signal wird von diesem Filter Block in das Basisband verschoben. In den Eigenschaften des Blocks kann mit der Mittenfrequenz auch ein Offset zum gewünschten Signal eingestellt werden. Die bei der Frequenzverschiebung entstehenden hochfrequenten Anteile bei $\pm2\omega_\mathrm{T}$ können mit einem Filter, welcher über die Einstellung Taps definiert werden kann, entfernt werden. Auch kann über einen Faktor die Abtastrate des Signals dezimiert werden.\cite{GNU-Radio-SSB}
\begin{table}[H]
    \centering
    \begin{tabular}{c|c|p{4cm}}
       Einstellung  & Eingestellter Wert & Beschreibung\\
       \hline
       Dezimierung & $1$ & Keine Verringerung der Abtastrate\\
       Taps & $20$ & kein Filter, 20-fache Verstärkung\\
       Mittenfrequenz  & $0$ & kein Offset\\
       Abtastrate & $1\,\text{MS/s}$ & Abtastrate des eingehenden Signals
    \end{tabular}
    \caption{Getroffene Einstellungen des Frequenz-umsetzenden FIR Filter }
    \label{tab:Einstellungen-FIR-Filter}
\end{table}
Die Tabelle \ref{tab:Einstellungen-FIR-Filter} zeigt die verwendeten Einstellungen für den 
Frequenz-umsetzenden FIR Filter.\newline
Nachdem Frequenz-umsetzenden FIR Filter wird das Basisbandsignal in einen rationalen Resampler geführt, welcher die Abtastrate von $1\,\text{MS/s}$ auf $48\,\text{kS/s}$ reduziert. Die Abtastrate von $48\,\text{kS/s}$ entspricht der Standard Abtastrate von Audiosignalen.\cite{GNU-Radio-Resampler}
\begin{equation*}
    \text{Faktor}=\frac{48000}{1000000}=\frac{6}{125}
\end{equation*}
Die Abtastrate muss erst um den Faktor $6$ interpoliert und anschließend um den Faktor $125$ dezimiert werden.\newline
Bei dem Signal handelt es sich bisher um ein komplexes Signal im Basisband, welches ein oberes und unteres Seitenband enthalten kann. Das obere Seitenband liegt nach der Verschiebung in das Basisband im positiven Frequenzbereich und bildet die In-Phase Komponente. Das untere Seitenband liegt im negativen Frequenzbereich und bildet die Quadratur Komponente. Um das untere Seitenband nutzen zu können, muss die In-Phase Komponente mit der Quadratur Komponete getauscht werden. In GNU Radio kann diese Operation mit einem IQ-Tauschblock (Swap IQ) umgesetzt werden.\cite{GNU-Radio-SwapIQ}\newline
Mit einem nachfolgende Bandpassfilter wird das jeweiligen Seitenband in seiner Bandbreite begrenzt. Die untere Grenze des Bandpassfilters liegt bei $f_\mathrm{g}=200\,\text{Hz}$. Die obere kann während des Betriebes über eine Eingabe oder einen Schieber verändert werden. Die Wahl eines Bandpasses anstelle eines Tiefpasses kann mit der engen Bandbreite von $B=2.7\,\text{kHz}$ über den Schmalbandtransponder und dem Hauptsprachbereich der menschlichen Stimme erklärt werden. Dieser liegt zwischen $200\,\text{Hz}$ und $\approx3000\,\text{Hz}$\cite{Sprachbereich}. Durch den Einsatz des Bandpasses kann die volle Bandbreite von $B=2.7\,\text{kHz}$ auf den natürlichen Sprachbereich eines Menschen angepasst werden. Der Übergangsbereich des Bandpasses wird auf $150\,\text{Hz}$ gesetzt. Die Abtastrate bleibt bei $1\,\text{MS/s}$.\newline
Nach dem jeweiligen Bandpass kann über einen Auswahlblock das jeweiligen Seitenband ausgewählt werden. Das kann auch während der Benutzung der Software verwendet werden. Nach dem Auswahlblock wird das ausgewählte Seitenband von automatischen Verstärkungseinheit auf ein Referenzlevel angehoben oder reduziert. Die Angriffsrate (engl. Attack Rate) und die Abfallrate (engl. Decay Rate) bestimmen dabei die Rate, mit der das Audiosignal auf das Referenzlevel pro Abtastwert angehoben oder abgesenkt wird. Die Angriffsrate wird auf $0.01$ und die Abfallrate auf $0.1$ gesetzt. Die Abfallrate wird höher gewählt, um zu große Signale schneller abzuschwächen. Die Verstärkung oder Dämpfung pro Abtastwert liegt bei $0.1$. Zum Schluss wird das komplexe Audiosignal in ein reales Audiosignal umgewandelt. Anschließend wird das Audiosignal an den Ausgabebereich weitergegeben.  
\subsubsection*{CW-Demodulation}
Bei der CW-Modulation handelt es sich um eine Sonderform der Amplitudenmodulation. Bei einer herkömmlichen Amplitudenmodulationen werden Informationen (z.B. Audio) in Form von Basisbandsignalen mit einem kontinuierlichen Träger übertragen. Die Basisbandsignale bilden dabei die beiden Seitenbänder. Bei der CW-Modulation erfolgt die Übermittlung von Informationen durch das Ein- und Ausschalten des Trägers. Es werden keine Basisbandsignale selbst auf den Träger auf moduliert.\cite{CW}
\begin{figure}[H]
    \centering
    \includesvg[width=0.4\linewidth]{Bilder/AM zu CWsvg}
    \caption{Vergleich zwischen einem Signal mit AM (oben) und einem Signal mit CW (unten) im Zeitbereich}
    \label{fig:Vegleich-AM-CW}
\end{figure}
Eingesetzt wird diese Art der Kommunikation bei der Telegrafie (Morse Code).\cite{CW}\newline
Eine umfangreiche Demodulation ist bei Signalen mit CW Modulation nicht erforderlich. Sie müssen jegendlich in das Basisband verschoben werden und mit einem Tiefpass in ihrer Bandbreite begrenzt werden. Der Durchbruchsbereich des Tiefpasses ist möglichst eng zu wählen $(f_\mathrm{g}\leq2\,\text{kHz})$, um Rauschen und Einflüsse durch andere Signale zu minimieren.\newline
\begin{figure}[H]
    \centering
    \includegraphics[width=0.8\linewidth]{Bilder/CW Demodulator.png}
    \caption{Signalflussgraph des in GNU Radio umgesetzten CW-Demodulators}
    \label{fig:CW-Demodulator}
\end{figure}
Die Abbildung \ref{fig:CW-Demodulator} zeigt den in GNU Radio umgesetzten Demodulator für CW-Signale. Ähnlich zum Demodulator für Einseitenband, wird für die für die Verschiebung des CW-Signals in das Basisband ein Frequenzumsetzender FIR Filter verwendet.\cite{GNU-Radio-SSB} Es wird kein Filter definiert, aber eine 20-fache Verstärkung eingestellt. Ebenso wird keine Dezimierung der Abtastrate mit dem Frequenzumsetzender FIR Filter vorgenommen.\newline
Die Reduzierung der Abtastrate wird mit einem folgenden rationalen Resampler durchgeführt. Mit diesem wird die Abtastrate von $1\,\text{MS/s}$ auf $48\,\text{kS/s}$, was der Abtastrate eines Audiosignals entspricht, herabgesetzt.\cite{GNU-Radio-Resampler}
\begin{equation*}
    \text{Faktor}=\frac{48000}{1000000}=\frac{6}{125}
\end{equation*}
Der Abtastrate wird erst um den Faktor $6$ interpoliert und anschließend um den Faktor $125$ dezimiert.\newline
Mit einem nachfolgenden Tiefpass wird das Basisband in seiner Breite begrenzt. Seine Grenzfrequenz $f_\mathrm{g}$ kann über die Bandbreite mit einer Eingabe oder Schieber während des Betriebes verändert werden. Bei CW-Signalen sollte ein möglichst schmaler Durchbruchsbereich  $(f_\mathrm{g}\leq2\,\text{kHz})$ gewählt werden. Der Übergangsbereich ist $150\,\text{Hz}$ breit.\newline
Die automatische Verstärkungseinheit hält den Pegel des Audiosignals auf einem gleichbleibenden Referenzlevel. In diesem Fall $0.5$. Die Angriffsrate wird auf $0.01$ und die Verfallsrate auf $0.1$ gesetzt. Dadurch werden Audiosignale mit zu großen Pegel schneller gedämpft. Die maximale Verstärkung pro Abtastwert liegt bei $0.1$. Anschließend wird das komplexe Audiosignal in ein reales umgewandelt. Damit kann es an den Ausgabebereich weitergegeben werden.


\subsection{Testen der Software}
Bevor die SDR-Software eingesetzt werden kann, müssen die einzelnen Demodulatoren und Funktionen der Software auf ihre Tauglichkeit überprüft werden. Für die Überprüfung des FM-Demodulator eignet sich der Empfang einer UKW-Radiostation.
\begin{figure}[H]
    \centering
    \includesvg[width=0.5\linewidth]{Bilder/Testaufbau FM}
    \caption{Angewendeter Testaufbau für den FM-Demodulator}
    \label{fig:Testaufbau FM-Demodulator}
\end{figure}
In der Abbildung \ref{fig:Testaufbau FM-Demodulator} ist der angewendete Aufbau zum testen des FM-Demodulators dargestellt. Zum testen wird eine Teleskop UKW Radioantenne über eine Koaxialleitung an den RX 2 SMA-Anschluss des RF A Erweiterungsbord vom USRP X310 angeschlossen. In der SDR-Software wird der FM-Demodulator ausgewählt. Die Mittenfrequenz wird auf $97\,\text{MHz}$ und die Kanalfrequenz auf $96.7\,\text{MHz}$ gestellt. An dieser Frequenz sollte sich  der Radiosender Bremen Next befinden. Als Filterbreite wird $B=57.423\,\text{kHz}$ eingestellt. Die Verstärkung bleibt bei $0\,\text{dB}$.
\begin{figure}[H]
    \centering
    \includegraphics[width=0.75\linewidth]{Bilder/FM Radio HF Spektrum.png}
    \caption{HF-Spektrum beim Empfang von FM Radio}
    \label{fig:FM-HF-Spektrum}
\end{figure}
Die Abbildung \ref{fig:FM-HF-Spektrum} zeigt das empfangene HF-Spektrum beim Test des FM-Demodulators. Im oberen Plot ist ein Wasserfalldiagramm und im unteren Plot ein FFT-Spektrum. Die Mittenfrequenz liegt bei $97\,\text{MHz}$. In beiden Darstellungen kann gut das FM-Signal der Radiostation bei $96.7\,\text{MHz}$ erkannt werden. Der maximale Pegel liegt bei ca. $-82\,\text{dB}$. Der Rauschpegel liegt bei ca. $-112\,\text{dB}$. Das entspricht einem Signal-zu-Rauschabstand von $SNR=30\,\text{dB}$, was ein sehr zufriedenstellender Wert ist.
\begin{figure}[H]
    \centering
    \includegraphics[width=0.75\linewidth]{Bilder/FM Radio Basisband.png}
    \caption{Spektrum vom Audiosignal des demodulierten FM-Signal}
    \label{fig:FM-Basisband}
\end{figure}
Das FFT-Spektrum in Abbildung \ref{fig:FM-Basisband} zeigt das Audiosignal des demodulierten FM-Signal. Durch den im FM-Demod Block integrierten Tiefpass wird das Audiosignal bei $15.5\,\text{kHz}$ abgegrenzt. Beim Testen des FM-Demodulators konnte ohne Probleme das Radioprogramm verfolgt werden. Es kann also von der vollen Funktionsfähigkeit des FM-Demodulators ausgegangen werden.\newline
Zum Testen des Einseitenband- und des CW-Demodulators wird ein anderer Testaufbau verwendet.
\begin{figure}[H]
    \centering
    \includesvg[width=0.7\linewidth]{Bilder/Testaufbau SSB CW}
    \caption{Angewendeter Testaufbau für den Einseitenband- und CW-Demodulator}
    \label{fig:Testaufbau-SSB-CW}
\end{figure}
In der Abbildung \ref{fig:Testaufbau-SSB-CW} ist der Aufbau für den Test des Einseitenband- und CW-Demodulator skizziert. Mit einem Hack RF One wird das jeweilige Signal generiert und über eine Koaxialleitung an der USRP X 310 angeschlossen. Verwendet wird dabei der Bereich RF A und der Anschluss RX2. Für die Generierung der Einseitenbandsignale wird ein einfacher Einseitenbandmodulator in GNU-Radio erstellt. Als Testübertragung wird die Audiospur eine .wav-Datei als oberes oder unteres Seitenband übertragen. Der verwendete Signalflussgraph ist im GitHub-Repository hinterlegt. \newline
Für den ersten Test des Einseitenband-Demodulator wird die Audiospur bei $433.05\,\text{MHz}$ als ein oberes Seitenband übertragen. In der SDR-Software wird die Mittenfrequenz auf $433.2\,\text{MHz}$ und die Kanalfrequenz auf $433.05\,\text{MHz}$ eingestellt. Die Filterbandbreite beträgt $B=2.7\,\text{kHz}$ und es wird eine Verstärkung von $G_\mathrm{SDR}=25\,\text{dB}$ verwendet.
\begin{figure}[H]
    \centering
    \includegraphics[width=0.75\linewidth]{Bilder/HF USB.png}
    \caption{Aufgezeichnetes HF-Spektrum bei der Übertragung des oberen Seitenbandes}
    \label{fig:USB-HF-Spektrum}
\end{figure}
In der Abbildung \ref{fig:USB-HF-Spektrum} ist das mit dem USRP X310 aufgezeichnete HF-Spektrum bei der Übertragung des oberen Seitenbandes zu sehen. Im oberen Plot ist das Spektrum in Form eines Wasserfalldiagramms dargestellt. Das obere Seitenband (rote Linie) liegt bei der gewünschten Frequenz von $433.05\,\text{MHz}$ und kann deutlich vom Rauschen unterschieden werden. Im unteren Plot ist das Frequenzspektrum als FFT-Spektrum dargestellt. Auch hier sticht das obere Seitenband deutlich heraus. Erreicht wird ein Pegel von $-45\,\text{dB}$. Der Rauschpegel liegt bei ca. $-90\,\text{dB}$. Damit wird ein Signal-zu-Rauschabstand von $SNR = 45\,\text{dB}$ erreicht, was für die Demodulation des oberen Seitenbandes mehr als ausreichend ist.
\begin{figure}[H]
    \centering
    \includegraphics[width=0.75\linewidth]{Bilder/Basisband USB.png}
    \caption{Frequenzspektrum des demodulierten oberen Seitenbandes}
    \label{fig:USB-Basisband}
\end{figure}
Die Abbildung \ref{fig:USB-Basisband} zeigt das Spektrum des demodulierten Signals im Basisband. Erkennbar ist der Durchlassbereich von $300\,\text{Hz}$ bis $3\,\text{kHz}$ des Bandpassfilters. Durch die automatische Verstärkungseinheit wird der Pegel des Audiosignals auf ca. $-20\,\text{dB}$ angehoben. Dadurch ist das übertragende Audiosignal deutlich hörbar. Die aufgenommen Audiospur des übertragenden und demodulierten Testsignals ist im GitHub-Repository hinterlegt.\newine
Im zweiten Test des Einseitenband-Demodulators wird das Audiosignal als unteres Seitenband übertragen. Als Frequenz wird wieder $433.05\,\text{MHz}$ gewählt. Die Einstellung in der SDR-Software sind die gleichen wie im ersten Test. Jedoch wird als Seitenband jetzt das untere ausgewählt.
\begin{figure}[H]
    \centering
    \includegraphics[width=0.75\linewidth]{Bilder/HF LSB.png}
    \caption{Aufgezeichnetes HF-Spektrum bei der Übertragung des unteren Seitenbandes}
    \label{fig:LSB-HF-Spektrum}
\end{figure}
Die Abbildung \ref{fig:LSB-HF-Spektrum} zeigt das mit dem USRP X310 augenommene HF-Spektrum als Wasserfalldiagramm (oberer Plot) und als FFT-Spektrum (unterer Plot). In beiden Darstellungen kann das unteren Seitenband an der gewünschten Frequenz von $433.05\,\text{MHz}$ angetroffen werden. Ähnlich wie beim oberen Seitenband wird ein Pegel von $-45\,\text{dB}$, sowie ein Signal-zu-Rauschabstand von $SNR=45\,\text{dB}$ erreicht, welcher mehr als ausreichend für die Demodulation ist.
\begin{figure}[H]
    \centering
    \includegraphics[width=0.75\linewidth]{Bilder/Basisband LSB.png}
    \caption{Frequenzspektrum des demodulierten unteren Seitenbandes}
    \label{fig:LSB-Basisband}
\end{figure}
In der Abbildung \ref{fig:LSB-Basisband} ist das Frequenzspektrum des demodulierten unteren Seitenbandes dargestellt. Ebenfalls ist wieder der Durchlassbereich des Bandpassfilter zwischen $300\,\text{Hz}$ und $3\,\text{kHz}$ erkennbar. Auch hier wird der Pegel des Audiosignals durch die automatische Verstärkungseinheit auf ca. $-25\,\text{dB}$ angehoben. Dadurch liegt der Pegel des Audiosignals im wahrnehmbaren Bereich. Die aufgezeichnete Audiospur des demodulierten unteren Seitenbandes ist im GitHub-Repository hinterlegt.\newline
Zum Testen des CW-Demodulators wird als .wav-Datei ein Morsecode mit einem $700\,\text{Hz}$ Ton als oberes Seitenband bei $433.05\,\text{MHz}$ übertragen. In der SDR-Software wird die Mittenfrequenz auf $433.2\,\text{MHz}$ und die Kanalfrequenz auf $433.05\,\text{MHz}$ eingestellt. Die Grenzfrequenz des Tiefpassfilters wird auf $1.2\,\text{kHz}$ gesetzt. Die Verstärkung beträgt $G_\mathrm{SDR}=25\,\text{dB}$.
\begin{figure}[H]
    \centering
    \includegraphics[width=0.75\linewidth]{Bilder/HF CW.png}
    \caption{Aufgezeichnetes HF-Spektrum bei der Übertragung des CW-Signals}
    \label{fig:HF-CW}
\end{figure}
Die Abbildung \ref{fig:HF-CW} zeigt das aufgezeichnete HF-Spektrum bei der Übertragung des CW-Signals als Wasserfalldiagramm (oberer Plot) und als FFT-Spektrum (unterer Plot). Das CW-Signal kann in beiden Darstellungsarten an der gewünschten Frequenz von $433.05\,\text{MHz}$ angetroffen werden. Der maximale Pegel liegt dabei bei ca. $-42\,\text{dB}$. Der Rauschpegel liegt wieder bei ungefähr $-90\,\text{dB}$, was zu eine Signal-zu-Rauschabstand von $SNR=47\,\text{dB}$ führt. Dieser ist ebenfalls mehr als ausreichend für die Demodulierung der CW-Signale.
\begin{figure}[H]
    \centering
    \includegraphics[width=0.75\linewidth]{Bilder/Basisband CW.png}
    \caption{Frequenzspektrum des demodulierten CW-Signals}
    \label{fig:CW-Basisband}
\end{figure}
Bei dem Spektrum in Abbildung \ref{fig:CW-Basisband} handelt es sich das Frequenzspektrum des demodulierten CW-Signals. Die Grenzfrequenz des Tiefpassfilters bei $1.2\,\text{kHz}$ ist gut zu erkennen. Wie beim Einseitenbanddemodulator wird auch beim CW-Demodulator der Pegel des demodulierten Signals auf ca. $-25\,\text{dB}$ angehoben. Dadurch konnte der Morsecode währrend des Test deutlic wahrgenommen werden. Eine Audioaufnahme des demodulierten Morsecode ist im GitHub-Repository hinterlegt. Aufgrund der kurzen Pulse bei Morsecode Übertragungen ist der $700\,\text{Hz}$ Trägerton nicht sichtbar. 
