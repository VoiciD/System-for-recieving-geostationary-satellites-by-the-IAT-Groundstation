\subsection{Was ist ein Satellit}
Bei einem Satelliten handelt es sich im allgemeinen Verständnis um ein Objekt, welches sich in einer Umlaufbahn um einen Himmelskörper, wie z.B ein Planet, Mond, Stern oder ähnliches befindet. Dabei kann der Satellit natürlichem oder künstlichen Ursprung sein.\cite{Satellite_Technology}\newline
Im weiteren Verlauf handelt es sich bei einem Satelliten um ein künstliches Objekt, welches sich in einer Umlaufbahn um die Erde befindet.\newline
Die erste Idee für einen Satelliten in einer geostationären Umlaufbahn stammt aus dem Jahren 1945. In diesem Jahr veröffentlichte der Autor Arthur C. Clarke im Magazin Wireless World einen Artikel, in welchem er die Bedeutung des geostationären Orbits beschreibt und die Idee eines Kommunikationssatelliten im geostationären Orbit vorstellt. Mit dem richtigen Equipment könnte solch ein Satellit interkontinentalen Datenaustausch ermöglichen.\cite{Satellite_Technology}\newline
\begin{figure}[H]
    \centering
    \includegraphics[width=0.5\linewidth]{Bilder/Sputnik 1.jpeg}
    \caption{Das Bild zeigt den ersten Satelliten Sputnik 1 vor dem Start. Sputnik 1 ist eine Aluminiumkugel mit einem Durchmesser $0.58 \text{ m}$ und einem Gewicht von $58 \text{ kg}$.\cite{Sputnik_1}\cite{Bild_Sputnik}}
    \label{Sputnik 1}
\end{figure}
Der erste Satellit starte am 04. Oktober 1957 von der damaligen UdSSR. Der Satellit mit Namen Sputnik 1, was so viel wie Begleiter oder Trabant bedeutet, umkreiste die Erde alle 98 min. Ausgerüstet war Sputnik 1 mit zwei Antennenpaaren und Telekommunikationsequipment, mit welchem er über $20.005$ MHz und $40.002$ MHz kurze Signale aussendete. Diese Signale konnten auf der ganzen Welt empfangen werden. Nach etwa 92 Tage verglühte Sputnik 1 beim Wiedereintritt in die Atmosphäre. \cite{Sputnik_1}\newline
Der erste amerikanische Satellit starte am 31. Januar 1958 mit dem Namen Explorer 1. Explorer 1 ist der erste Satellit gewesen, welcher wissenschaftliches Equipment ab Bord hatte.\cite{NASA_Explorer1}
\begin{figure}[H]
    \centering
    \includegraphics[width=0.5\linewidth]{Bilder/1958_january_explorer_01_team_0.jpg}
    \caption{Vorführung des ersten amerikanischen Satelliten Explorer 1\cite{NASA_Explorer1} }
    \label{Explorer 1}
\end{figure}
An Bord von Explorer 1 befanden sich wissenschaftliches Equipment, unter anderem auch ein Messgerät für kosmische Strahlung. Mit diesem Messgerät sollte die Strahlung in der Atmosphäre der Erde gemessen werden. Explorer 1 erbrachte den Nachweis für des Van-Allen Strahlungsgürtels. Der Satellit umrundete die Erde alle 114 min in einer kreisförmigen Umlaufbahn, wobei diese den Satelliten bis auf 354 km an die Erde ran und 2515 km entfernt brachte. Explorer 1 war 203 cm lang, hatte einen Durchmesser von 15.9 cm und wog 14 kg. Am 23. Mai 1958 machte die Explorer 1 ihre letzte Übertragung und verglühte am 31. März 1970 beim Wiedereintritt in der Atmosphäre.\cite{NASA_Explorer1}\newline
In der heutigen Zeit gibt es viele verschiedene Arten an Satelliten, welche sich in ihrem Verwendungszweck und ihrem damit verbunden Equipment und Umlaufbahn unterscheiden.\newline
Ein paar Beispiele wären dabei:
\begin{itemize}
    \item Erdbeobachtungssatelliten: Diese werden zur Beobachtung und Analyse der Erdoberfläche und Atmosphäre eingesetzt. Zu dieser Gruppe an Satelliten gehören unter anderem Wettersatelliten. Ausgerüstet mit verschiedenen Kameras und Messequipment nehmen sie Bilder von Wolkenformationen und Daten der Atmosphäre auf. Diese Daten und Bilder bilden dann die Grundlage für die Wettervorhersage. Einige Beispiele für Wettersatelliten sind unter andrem die NOAA und GOES Reihe der Amerikaner, die METOP und METOSAT Reihe der Europäer und die METEOR und Electro-L Reihe der Russen.\cite{DWD}
    \item Kommunikations- und Rundfunksatelliten: Dieser Art der Satelliten stellen verschiedenste Service im Bereich der Telekommunikation und Rundfunk bereit. Sie werden unter anderem zur Übertragung von Fernsehsignalen, Telefonie und Internet verwendet. Sie sind in den unterschiedlichsten Umlaufbahnen anzutreffen. Einige Beispiele wäre dabei die Starlink Reihe von SpaceX, Iridium von Iridium und Inmarsat von Inmarsat oder Astra von SES S.A. ASTRA.\cite{Satellite_Technology}\cite{Satellitenkommunikation}\cite{N2YO_SATLIST}
    
    \item Navigationssatelliten: Diese werden zur genauen Positionsbestimmung verwendet. Dafür bilden diese ganze Satellitenkonstellationen, welche die gesamte Erde umspannen können. Die Positionsbestimmung basiert dabei auf der der Triangulierung und Einwegentfernungsmessung. Zur Bestimmung werden Signale von drei oder mehreren Satelliten empfangen. Die Signale enthalten neben den genauen Koordinaten des Satelliten auch den genauen Zeitpunkt, an welchem die Signale versendet werden. Grundlage für den genauen Zeitbestimmung bilden Atomuhren, welche sich auf den Satelliten befinden. Beispiele für solche Sternenkonstellationen sind das GPS der Amerikaner, das russische GLONASS und das europäische Galileo.\cite{Satellitenkommunikation}
    
    \item Amateurfunksatelliten: Amateurfunksatelliten bilden eine besondere Untergruppe der Kommunikationssatelliten. Sie werden meistens von Universitäten, Vereinigungen von Amateurfunkern oder ähnlichen Vereinen geplant, entwickelt, gebaut und betrieben. Dabei sind die engen Budgets und technologischen Innovationen bewundernswert.\cite{Satellitenkommunikation}\newline
    Eine solche Vereinigung ist AMSAT, welche mehrere Ableger weltweit hat. In Deutschland gibt es die AMSAT-DL, welche sich aus Funkamateuren, Ingenieuren, Wissenschaftlern, Studenten und Raumfahrtenthusiasten zusammensetzt. Seit über 50 Jahren plant, entwickelt, baut und betreibt die AMSAT-DL verschiedenste Satelliten, welche von Funkamateuren frei verwendet werden dürfen.\cite{AMSAT-DL}\newline
    Der erste Amateurfunksatellit OSCAR-I (Orbital Satellite Carrying Amateur Radio) starte am 12. Dezember 1961, vier Jahre vor dem ersten kommerziellen Kommunikationssatelliten "Early Bird". Die ersten OSCAR-I,-II und -III Satelliten funktionierten nur wenige Tage. Erst OSCAR-VI von der deutschen AMSAT (AMSAT-DL) schaffte es 4,5 Jahren lang zu arbeiten. Es folgten  weitere deutsche OSCARS, weltweit insgesamt mehr als 50 Stück.\cite{Satellitenkommunikation}\newline
    Weitere Meilensteine von AMSAT-DL sind die sogenannten Phase 3 Satelliten. Die Entwicklung dieser Satelliten begann in den 1970er Jahren. Das Ziel der Phase 3 Satelliten ist eine Generation von Erdsatelliten in einer hoch elliptischen Umlaufbahn zu erschaffen. Gegenüber der bisherigen Satelliten würden diese einen weltweiten Benutzerkreis erschließen. Von den bisher vier gestarteten Phase 3 Satelliten, mit der Bezeichnung P3-A bis P3-D, sind nur noch P3-B und P3-D im Orbit. \cite{AMSAT-DL}\cite{Satellitenkommunikation}\newline
    Ein weiterer Meilenstein ist der erste Phase 4 Satellit. Bei dem Satelliten handelt es sich um den katarischen Rundfunk- und Kommunikationssatelliten Es'Hail-2. Dieser trägt den Rufname QO-100 und hat neben dem Equipment zur kommerziellen Nutzung auch zwei Amateurfunktransponder an Bord hat, welche die ersten im geostationären Orbit sind.\cite{AMSAT-DL}\newline
\end{itemize}


