\subsection{Antenne}
Die Antenne ist mit der wichtigste Bestandsteil der Empfangskette an der Satellitenbodenstation. Erst mit einer geeigneten Antenne ist es mögliche die Signale vom Satelliten, welcher ebenfalls eine Antenne braucht um die Signale zu senden, zu empfangen. Die Antenne wandelt die leitungsgebundene Welle um und strahlt diese in den freien Raum ab oder empfängt die Wellen im freien Raum und gibt diese an die Leitung ab. Sie ist also das Verbindungsglied zwischen der leitungsgebundenen Welle und der Welle im freien Raum.\newline
Die IEEE definiert eine Antenne als ein passives, lineares und reziprokes Bauelement, welches Radiowellen abstrahlen, als auch empfangen kann\cite{IEEE145-1993}\cite{Balanis_2005}.\newline
Eine Antenne kann über viele verschiedene Parameter beschrieben werden. Diese Parameter helfen dabei eine geeignete Antenne für die jeweilige Anwendung zu finden.

\subsubsection*{Nah- und Fernfeld}
Der Bereich um die Antenne kann in mehrere Bereiche aufgeteilt werden. Im mittelbaren Umfeld liegt das Nahfeld, auch Fresnel-Breich genannt\cite{Radartutorial-Nahundfernfeld}, der Antenne. Neben den abgestrahlten elektromagnetische Wellen wirken hier auch starke stationäre Felder, welche ebenfalls von der Antenne ausgehen. Beschreiben lassen sich die Felder durch die maxwellschen Gleichungen. Im Nahfeld wird die Berechnung der Felder aufgrund der hohen Ordnungen der Polynome erschwert\cite{Radartutorial-Nahundfernfeld}. Aus diesem Grund werden die Strahlungscharakteristiken einer Antenne im Fernfeld bestimmt. \cite{Balanis_2005}.\newline
Das Fernfeld, auch Fraunhofer-Bereich genannt, ist geometrisch deutlich größer als das Nahfeld. Es beginnt da, wo sich die elektromagnetischen Wellen frei im Raum ausbreiten können. Der Übergang zum Fernfeld kann Näherungsweise bestimmt werden. Für Antennen, welche in ihren geometrischen Abmessung kleiner als ihre Wellenlänge $\lambda$ sind, gilt\cite{Radartutorial-Nahundfernfeld}:
\begin{equation}
    r_{fern}=2\cdot\lambda
    \label{Nahfeld}
\end{equation}
Bei größeren Antennen, zum Beispiel Parabolantennen, gilt\cite{Radartutorial-Nahundfernfeld}:
\begin{equation}
    r_{fern}=\frac{2\cdot L^2}{\lambda}
    \label{Fernfeld}
\end{equation}
Dabei gibt die Variable L die geometrische Abmessung der Antenne an. Als sichere Faustformel kann ab einem Abstand $r>5\cdot\lambda$ vom Fernfeld ausgegangen werden.\newline
Im Fernfeld existieren nur die Felder der elektromagnetische Welle, was die Berechnung der Felder deutlich vereinfacht. Die elektrische und magnetische Komponente der EM-Welle befinden sich Phase zu einander und stehen orthogonal zur Ausbreitungsrichtung. Über das Verhältnis vom elektrischen und magnetischen Feld kann der Freiraumwiderstand $\eta_0$ bestimmt werden.
\begin{equation}
    \eta_0=\frac{\left|\vec{E}\right|}{\left|\vec{H}\right|}=\sqrt{\frac{\mu_0}{\varepsilon_0}}=\mu_0\sqrt{\frac{1}{\mu_0\cdot\varepsilon_0}}=377\Omega
    \label{GleichungFreimraumwiderstand}
\end{equation}
Bis zur Entfernung $r=\frac{L^2}{2\cdot \lambda}$ um die Antenne liegt die sogenannte Rayleigh-Zone. In diesem Bereich strahlt Antenne nicht nur Energie ab, sondern nimmt auch einen Teil der abgestrahlten Energie als Blindleistung wieder auf.\cite{Radartutorial-Nahundfernfeld}

\subsubsection*{Antennen-/Richtdiagramm}
Ein Antennen- oder Richtdiagramm stellt die Strahlungscharakteristik einer Antenne grafisch dar. Die Strahlungscharakteristik einer Antenne umfasst dabei die Strahlungsleistungsdichte, die Feldstärke, Intensität, Richtfaktor, Phasenlage und Polarisation.\cite{Balanis_2005} In den meisten Fällen wird im Antennendiagramm allerdings die Intensität der abgestrahlten Energie oder ihre Feldstärke in Abhängigkeit der Richtung dargestellt\cite{Radartutorial-Antennendiagramm}. Da Antennen reziproke Elemente sind gilt ein Antennendiagramm gleichermaßen für das Senden und auch für das Empfangen mit der jeweiligen Antenne. Im Sendefall gibt das Antennendiagramm die richtungsabhängige Ausstrahlung der Antenne an und im Empfangsfall die richtungsabhängige Empfangsempfindlichkeit.\cite{Radartutorial-Antennendiagramm}\newline
Auch besteht die Möglichkeit die Strahlungscharakteristik der Antenne mithilfe einer mathematische Funktion zu definieren.\cite{Balanis_2005}\newline
\begin{figure}[H]
    \centering
    \includegraphics[width=0.5\linewidth]{Bilder/Antennendiagramm.png}
    \caption{Ein Beispiel für ein horizontales Antennendiagramm im Polarkoordinatensystem\cite{Radartutorial-Antennendiagramm}}
    \label{Antennendiagrammbeispiel}
\end{figure}
Für das Antennendiagramm kann in unterschiedlichen Formen und in verschiednen Ebenen dargestellt werden. Ein Antennendiagramm kann im 2D-Raum entlang der horizontalen (Azimuth), als auch entlang der vertikalen Ebene (Elevation) der Antenne erstellt werden. Auch kann ein Antennendiagramm im 3D-Raum erstellt werden. Die Abbildung \ref{Antennendiagrammbeispiel} zeigt ein horizontales Antennendiagramm im polaren Koordinatensystem.\newline
Neben dem polaren Koordinatensystem kann auch das kartesische Koordinatensystem verwendet werden, jedoch kann im polaren Koordinatensystem die Richtwirkung der Antenne besser dargestellt werden.
\cite{Radartutorial-Antennendiagramm}.
\subsubsection*{Haupt- und Nebenkeulen}\label{Keulen}
Im Antennendiagramm in Abbildung \ref{Antennendiagrammbeispiel} lassen sich verschiedene Muster in der Strahlungscharakteristik der Antenne erkennen, welche auch Keulen genannt werden. Dabei werden die Keulen weiter in Haupt- und Nebenkeulen unterteilt. \newline
Bei der Hauptkeule handelt es sich um den Bereich einer Antenne, in dessen Richtung am meisten Energie abgestrahlt oder, im Empfangsfall, empfangen wird.\cite{Balanis_2005}
Bei einigen Antennen können auch mehrere Hauptkeulen vorhanden sein. Ein Beispiel dafür sind Loop- oder Dipolantennen, welche zwei Hauptkeule im Antennendiagramm aufweisen. Diese Hauptkeulen sind im 180\degree versetzt zu einander. Die Hauptkeulen stellen die bevorzugte Anwendungsrichtung einer Antenne dar, egal ob die Antenne im Sende- oder Empfangsbetrieb verwendet wird.\newline
Die Nebenkeulen handelt es sich um alle Keulen, welche nicht die Hauptkeule darstellen. Diese sind jedoch deutlich kleiner und sollte auch so klein wie möglich sein. Nebenkeule sind meistens unerwünscht, da sie Enegie in ungewollte Richtungen abstrahlen und so weniger Energie durch die Hauptkeule abgestrahlt wird oder da sie im Empfangsfall dafür sorgen, dass die Antenne aus eventuell unerwünschten Richtungen Signale aufnimmt und so den Empfang stören.\cite{Balanis_2005}. Die größten beiden größten Nebenkeulen werden auch Seitenkeulen genannt.\cite{Balanis_2005}.\newline
Der Abstand von der Hauptkeule zur größten Nebenkeule ist die Nebenkeulendämpfung. Je größer der Wert ist, desto kleiner sind die Nebenkeulen. Die Nebenkeulendämpfung ist ein wichtiger Parameter für Richtantennen, da damit die Richtschärfe ausgedrückt werden kann.\newline
Die Haupt- und Nebenkeulen bilden sich bei jeder Antenne, welche kein isotropischer Kugelstrahler ist.

\subsubsection*{Strahlbreite}
Im Zusammenhang mit dem Strahlungsmuster einer Antenne kann ein weiterer Parameter hergeleitet werden. Die Stahlbreite beschreibt den Öffnungswinkel der Hauptkeule. Gemessen wird die Strahlbreite an zwei identischen Punkten auf beiden Seiten des Maximums der Hauptkeule\cite{Balanis_2005}.\newline
Oft verwendet wird die 3dB-Strahlbreite, auch Half-Power Beamwidth genannt. Diese wird von der IEEE definiert als der Winkel zwischen den zwei Punkten an der Hauptkeule, wo die abgestrahlte Leistung nur noch die Hälfte des Maximums der Hauptkeule beträgt\cite{Balanis_2005}.\newline
Es gibt auch noch andere Strahlbreite wie die First Null Beamwith (FNBW), diese findet aber in der Praxis keine große Anwendung\cite{Balanis_2005}.\newline
Die Strahlbreite ist gerade für Richtantennen ein wichtiger Parameter, da die Strahlbreite ihr Auflösungsvermögen beschreibt. Mit einer kleineren Strahlbreite kann im Empfangsfall eine größere Winkelauflösung erreicht werden. Eine größere Winkelauflösung hilft einer Antenne dabei zwischen mehreren benachbarten Strahlungsquellen zu unterscheiden. Mit einem größeren Öffnungswinkel neigt die Antenne dazu benachbarte Quellen als eine wahrzunehmen. Das kann für zum Beispiel Radaranlagen wichtig sein\cite{Balanis_2005}. Allerdings wachsen mit geringere Strahlbreite auch die Nebenkeulen, was unerwünschte Effekte, wie in \ref{Keulen} beschrieben, führt \cite{Balanis_2005}.

\subsubsection*{Antennengewinn}
Ein weiterer nützlicher Parameter, welcher für die Beschreibung von Antennen verwendet werden kann, ist der Antennengewinn $G$. Der Antennengewinn ist eng mit dem Richtfaktor und dem Wirkungsrad der Antenne verbunden\cite{Balanis_2005}.\newline
Eine reale Antenne strahlt die eingespeiste Leistung $P_S$ nicht gleichmäßig in alle Richtungen ab. Eine reale Antenne weißt bevorzugte Richtungen $(\phi,\theta)$ auf, gekennzeichnet durch die Haupt- und Nebenkeulen im Antennendiagramm, in welche sie die Leistung abstrahlt oder aus welcher sie Leistung aufnimmt.\newline
Im Sendefall entspricht der Antennengewinn $G(\varphi,\theta)$ dem Verhältnis der abgestrahlten Strahlungsleistungsdichte $S(\phi,\theta)$ der Antenne zu der abgestrahlten Strahlungsleistungsdichte $S_{ref}(\phi,\theta)$ einer Referenzantenne bei gleicher eingespeisten Leistung $P_S$, Richtung $(\phi,\theta)$ und Entfernung $r$\cite{Balanis_2005}.
\begin{equation}
    G(\phi,\theta)=\frac{S(r,\phi,\theta)}{S_{ref}(r,\phi,\theta)}
    \label{Grunddefinition Antennengewinn}
\end{equation}
Die Entfernung r kürzt sich aus der Gleichung raus. Sie ist für den Antennengewinn nicht entscheidend.\newline
Da Antennen reziproke Elemente sind gilt die Gleichung \ref{Grunddefinition Antennengewinn} gleichermaßen für den Empfangsbetrieb. Im Empfangsbetrieb entspricht der Antennengewinn $G$ dem Verhältnis der empfangenen Leistung $P_E(\phi,\theta)$ der jeweiligen Antenne zu der empfangenen Leistung $P_{Eref}(\phi,\theta)$ einer Referenzantenne bei gleicher Sendequelle mit fester Sendeleistung $P_S$ und Entfernung $r$ und gleichen Empfangswinkel $(\phi,\theta)$.\newline
\begin{equation}
    G(\phi,\theta)=\frac{P_E(\phi,\theta)}{P_{Eref}(\phi,\theta)}
    \label{Antennengewinn Empfangsfall}
\end{equation}
Als Referenzantenne in beiden Fällen eine beliebige Antenne gewählt werden. In den meisten Fällen wird als Referenzantenne der isotrope Kugelstrahler verwendet. Allerdings kann auch der einfache hertzsche Dipol verwendet werden\cite{Balanis_2005}.\newline
Der isotrope Kugelstrahler ist eine rein theoretische Antenne. Der isotrope Kugelstrahler strahlt die eingespeiste Leitung $P_S$ in alle Richtungen gleichmäßig aus und empfängt auch aus allen Richtungen die gleiche Leistung $P_E$. Aus diesem Grund eignet sich der isotrope Kugelstrahler besonders gut als Referenzantenne. Für die Strahlungsleistungsdichte eines isotrope Kugelstrahler gilt:
\begin{equation}
    S_0=\frac{P_S}{4\cdot \pi \cdot r^2 }
    \label{Strahlungsleistungsdichte isotroper Kugelstrahler}
\end{equation}
Der Gewinn wird meistens logarithmisch in dBi angegeben. Das i in dBi bedeutet, dass der Gewinn auf einen isotropen Kugelstrahler bezogen angeben wird. Aus der Gleichung \ref{Grunddefinition Antennengewinn} und \ref{Strahlungsleistungsdichte isotroper Kugelstrahler} folgt dann für die logarithmische Darstellung:
\begin{equation}
    G=10 \cdot \log_{10}\left( \frac{S(\phi,\theta)\cdot 4\cdot \pi \cdot r^2}{P_S} \right)
    \label{GewinndBi}
\end{equation}
Wird nichts weiter angegeben, kann im Datenblatt einer Antenne beim Gewinn $G(\varphi,\theta)$ vom Gewinn in Richtung der Hauptkeule ausgegangen werden, da diese auch die bevorzugte Anwendnugsrichtung der Antenne darstellt.\cite{Balanis_2005}

\subsubsection*{Richtfaktor und Wirkungsgrad}
Beim Richtfaktor $D$ einer Antenne handelt es sich um  das Verhältnis der Strahlungsintensität bei einem bestimmten Abstrahlwinkel $(\varphi,\theta)$ zu der durchschnittliche Strahlungsintensität der Antenne in alle Richtungen. Dabei wird meistens als Abstrahlwinkel $(\varphi,\theta)$ der Winkel von der maximalen Strahlungsintensität, also der Hauptkeule der Antenne, verwendet.\cite{Balanis_2005}
\begin{equation}
    D=\frac{\text{Maximale Strahlungsintensität}}{\text{Durchschnittliche Strahlungsintensität}}=\frac{\phi_\mathrm{max}}{\phi_\mathrm{\varnothing}}
    \label{RichtfakotrD}
\end{equation}
Die durchschnittliche Strahlungsintensität kann über die von der Antenne abgestrahlten Leistung $P_\mathrm{rad}$ bestimmt werden.\cite{Balanis_2005}
\begin{equation}
    \phi_\mathrm{\varnothing}=\frac{P_\mathrm{rad}}{4\pi}
\end{equation}
Mit dem Richtfaktor $D$ und mithilfe des Antennenwirkungsgrad $\eta$ kann der Gewinn $G$ einer Antenne ermittelt werden.
\begin{equation}
    G=\eta\cdot D = \eta\cdot\frac{\phi_\mathrm{max}}{\phi_\mathrm{\varnothing}}
\end{equation}
Der Antennenwirkungsgrad $\eta$ berücksichtigt Verluste, welche innerhalb der Antenne auftreten. Zusammensetzen tut sich der Antennenwirkungsgrad $\eta$ aus den ohmschen Verlusten $\epsilon_\mathrm{R}$, den Verlusten durch Reflexion $\epsilon_\mathrm{\Gamma}$ und den dielektrischen Verlusten $\epsilon_\mathrm{d}$.\cite{Balanis_2005}
\begin{equation}
    \eta=\epsilon_\mathrm{R}\cdot\epsilon_\mathrm{\Gamma}\cdot\epsilon_\mathrm{d}
\end{equation}
Bei verlustlosen Antennen gilt $G=D$, da $\eta=1$. -> Mehr zu den Verlusten raus suchen.

\subsubsection*{Äquivalente Strahlungsleistung}
Die äquivalente Strahlungsleistung oder auch effektive Strahlungsleistung (ERP) ist eine nützliche Größe um die die Auswirkung des Gewinns $G(\varphi,\theta)$ einer Antenne zu verdeutlichen.\newline
Die äquivalente Strahlungsleistung gibt die Leistung an welche eine Referenzantenne abstrahlen müsste, um die gleiche Strahlungsleistungsdichte $S(\varphi,\theta)$ der Bezugsantenne in einem bestimmten Abstrahlwinkel $(\varphi,\theta)$ zu erreichen. Beim Abstrahlwinkel $(\varphi,\theta)$ der Bezugsantenne wird in der Regel von Hauptkeule der Antenne ausgegangen.\cite{Radartutorial-ERP}\newline
Als Referenzantenne kann eine beliebige Antenne verwendet werden. In der Praxis werden für gewöhnlich eine Dipolantenne oder ein isotroper Kugelstrahler als Referenz gewählt. Wird ein Dipol als Referenz gewählt, wird die äquivalente Strahlungsleistung als ERP angegeben. Wird jedoch ein isotroper Kugelstrahler als Referenzantenne verwendet, wird die äquivalente Strahlungsleistung als EIRP angeben. EIRP steht für equivalente isotropic radiated power oder äquivalente isotropische Strahlungsleistung.\cite{Radartutorial-ERP}\newline
Das EIRP setzt sich aus der Sendeleistung $P_\mathrm{T}$ und dem Gewinn $G$ in Richtung der Hauptkeule und den Verlusten der Antenne $L_\mathrm{ANT}$ zusammen.\cite{Radartutorial-ERP}
\begin{equation}
    EIRP=P_\mathrm{T}\cdot \frac{G}{L_\mathrm{ANT}}
    \label{EIRPdBm}
\end{equation}
Das EIRP kann auch logarithmisch, z.B. in dBm, angeben werden.
\begin{equation}
    EIRP_\mathrm{dBm}=10 \cdot \log_{10} \left( \frac{P_\mathrm{T}\cdot \frac{G}{L_\mathrm{ANT} }}{1\cdot 10^{-3}}\right) = P_\mathrm{T,dBm}+G_\mathrm{dBi}-L_\mathrm{ANT,dB}
    \label{EIRPdBm}
\end{equation}
Das EIRP und ERP hängen über den Gewinn $G=1.64$ der Dipolantenne gegenüber dem isotropen Kugelstrahler miteinander zusammen.\cite{Radartutorial-ERP}
\begin{equation*}
    EIRP = 1.64\cdot ERP
\end{equation*}
Mithilfe des ERP und EIRP kann die scheinbare Leistung eines Senders quantifiziert werden. Anwendung findet das im Bereich der Telekommunikationstechnik. Die Bundesnetzagentur gibt mit dem EIRP die maximale zulässige Sendeleistung im sogenannten Frequenznutzungsplan \cite{FrequenzplanBundesnetzagentur} an. So soll eine gemeinschaftliche Nutzung der einzelnen Frequenzbänder garantiert und gegenseitige Störungen minimiert werden.\cite{Radartutorial-ERP}

\subsubsection*{Effektive Antennenfläche}
Die effektive Antennenfläche $A_\mathrm{E}$ ist ein wichtiger Parameter für Antennen, welche als Empfangsantennen betrieben werden.\newline
Die effektive Antennenfläche, auch Absoprtionsfläche oder Wirkläche genannt, ist rein theoretische Fläche. Diese kann gleich oder kleiner als die reale Fläche der jeweiligen Antenne sein.\newline
Bestimmt wird die effektive Antennenfläche $A_\mathrm{E}$, bei verlustlosen Antennen, über das Verhältnis von der am Fuße der Antenne verfügbaren Leistung $P_\mathrm{E}$ zu der Strahlungsleistungsdichte $S_\mathrm{E}$ von der auf die Antenne eintreffende Welle.\cite{Balanis_2005}
\begin{equation}
    A_\mathrm{E}=\frac{P_\mathrm{E}}{S_\mathrm{E}}
\end{equation}
Dank der Reziprozität von Antennen kann aus der effektiven Antennenfläche $A_\mathrm{E}$ auch der Gewinn $G$ der Antenne ermittelt werden und umgekehrt.\newline
-> weiter ausführen




\subsubsection*{Polarisation}



Eine Antenne fungiert als eine Schnittstelle zwischen elektrischen Signalen und elektromagnetischen Wellen im freien Raum. Sie wandelt leitungsgebunde Energie in elektromagnetischen Wellen um oder umgekehrt.\newline
Im freien Raum existieren zwei verschiedene Arten von Wellen, die Transversal- und Longitudinalwellen. Bei einer transversalen Welle erfolgen die Schwingungen senkrecht zur Ausbreitungsrichtung. Die Schwingungen einer Longitudinalwelle erfolgen in Richtung der Ausbreitung.\cite{Wellentypen}\newline
-> Bild der Wellentypen \newline
Bei elektromagnetischen Wellen handelt es sich um Transversalwellen. Im Gegensatz zu Lognitudalwellen können Transversalwellen polarisiert werden.\cite{Wellentypen}
Die Polarisierung $E$ einer elektromagnetischen Welle kann als eine Funktion der Zeit $t$ angesehen werden. Sie beschreibt die Veränderung der Richtung und relative Amplitude des E-Feld Vektors, indem sie in gleichmäßigen zeitlichen Intervallen $n$ die Extremstellen der Schwingungen entlang der Ausbreitungsrichtung der elektromagnetischen Welle im Raum darstellt.\cite{Balanis_2005}\newline
->Bild der Polarisierung einfügen\newline
Die Polarisierung einer Antenne kann mit der Polarisierung der von ihr abgestrahlten EM-Welle beschrieben werden. Jedoch kann die Bauform und Ausrichtung einer Antenne zu unterschiedlichen Polarisierung innerhalb ihrer Strahlungscharakteristik führen. Demnach ist die Polarisierung einer Antenne abhängig vom Abstrahlwinkel $(\varphi,\theta)$, kann aber mit der Polarisierung der aus diesem Winkel abgestrahlten EM-Welle beschrieben werden.\cite{Balanis_2005}
Im Fernfeld einer Antenne kann die von ihr abgestrahlte elektromagnetische Welle an jedem Punkt auf dem Ausbreitungspfad durch eine ebene Welle mit derselben Ausbreitungsrichtung und elektrischen Feldstärke $\vec{E}$
angenähert werden. Dies gilt jedoch nur für Punkte, die sich tatsächlich im Fernfeldbereich und entlang des Ausbreitungspfads der Welle befinden. Im Nahfeld wirken zusätzlich statische und induktive Feldkomponenten, weshalb die Wellenfront hier noch gekrümmt ist. Mit wachsender Entfernung zur Antenne vergrößert sich der Krümmungsradius der Wellenfront. Der Einfluss der Nahfeldanteile nimmt also mit wachsender Entfernung zur Quelle ab. Für sehr große Entfernungen wird die Wellenfront lokal praktisch eben, und die elektromagnetische Welle kann hinsichtlich Ausbreitung und Polarisationsverhalten wie eine ebene Welle betrachtet werden. Diese Eigenschaft erlaubt es, die Polarisation der abgestrahlten Welle im Fernfeld durch die Polarisationsrichtung einer ebenen elektromagnetischen Welle eindeutig zu charakterisieren.\cite{Balanis_2005}\newline
Für eine auf eine Antenne einfallende elektromagnetische Welle wird die Polarisierung als die Polarisierung einer ebenen Welle definiert, die aus einer gegebenen Richtung mit fester Leistungsflussdichte einfällt und die maximale verfügbare Leistung an den Antennenklemmen liefert.\cite{Balanis_2005}\newline
Die Polarisierung von Antennen und EM-Wellen können in drei Arten klassifiziert. Eine Antenne oder EM-Wellen kann entweder linear, zirkular/kreisförmig oder elliptisch polarisiert sein.\cite{Balanis_2005}\newline
Bei einer linearen Polarisierung bleibt die Richtung der Schwingung unverändert. Nur die relative Amplitude des E-Feld Vektors ändert periodisch ihren Betrag und Vorzeichen. Damit der E-Feld Vektor nur ein einer Ebene entland der Ausbreitungsrichtung der EM-Welle schwingt darf dieser nur aus einer Komponente oder aus zwei Komponenten bestehen, welche entweder in Phase oder ein vielfaches von $180\degree$ oder $\pi$ außer Phase zu einander sind.\cite{Balanis_2005}
\begin{figure}[H]
    \centering
    \includegraphics[width=0.5\linewidth]{Bilder/Linearly_Polarized_Wave.png}
    \caption{Arten an linearen Polarisationen\cite{linear_polarization_image}}
    \label{LinearePolarisation}
\end{figure}
Die Abbildung \ref{LinearePolarisation} zeigt die unterschiedlichen Arten der linearen Polarisation. Im Bezug zu der Erdoberfläche kann eine EM-Welle vertikal linear polarisiert (blau), horizontal linear polarisiert (grün) oder schräg linear polarisiert (rot) sein.\newline
Im Falle einer zirkularen oder kreisförmigen Polarisation bleibt die realtive Amplitude des E-Feld Vektors konstant. Jedoch rotiert der E-Feld Vektor mit konstanter Winkelgeschwindigkeit senkrecht zur Ausbreitungsrichtung. Am Ort des Beobachters zeigt E-Feld Vektor damit einen Kreis auf.\cite{Balanis_2005}
\begin{figure}[H]
    \centering
    \includegraphics[width=0.5\linewidth]{Bilder/Rising_circular.png}
    \caption{Zusammensetzung einer zirkularen Polarisation\cite{circular_polarization_image}}
    \label{ZusammensetzungZirkular}
\end{figure}
In Abbildung \ref{ZusammensetzungZirkular} sind die Komponenten des E-Feld Vektors (blauer Pfeil) dargestellt. Der E-Feld Vektor setzt sich aus zwei einzelnen E-Feld Vektoren (Rot und Blau) zusammen, welche orthogonal zu einander stehen. Die Phasendifferenz zwischen den roten und blauen E-Feld Vektoren beträgt dabei $90\degree$ oder $\frac{\pi}{2}$ oder ein ungerade vielfaches davon.\cite{Balanis_2005}\newline
Der Vektor des E-Feldes kann dabei gegen den Uhrzeigersinn oder im Uhrzeigersinn rotieren. Ist eine Welle im Uhrzeigersinn polarisiert, nennt man diese auch rechtshändig polarisiert (engl. right hand circularly polarized) oder auch RHCP. Ist die Welle gegen den Uhrzeigersinn polarisiert, ist von einer linkshändig polarisierten Welle (engl. left hand circularly polarized) oder LHCP die Rede.\cite{Balanis_2005}\newline














 
