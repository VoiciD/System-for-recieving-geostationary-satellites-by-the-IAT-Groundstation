\subsection{Darstellung des Downlinks}
Beim Downlink handelt es sich um eine Datenverbindung zwischen einem Satelliten und einer Bodenstation, wobei der Datenaustausch von Satellit in Richtung der Bodenstation stattfindet.
\begin{figure}[H]
    \centering
    \includegraphics[width=0.5\linewidth]{Bilder/Skizze Downlink.drawio.png}
    \caption{Vereinfachte Darstellung des Downlinks}
    \label{SkizzeDownlink}
\end{figure}
Die Abbdildung \ref{SkizzeDownlink} zeigt die eine vereinfachte Darstellung des Downlinks zwischen dem Schmallbandtransponder auf dem geostationären Satelliten Es'Hail-2 und der Bodenstation am IAT. Einteilen lässt sich der Downlink in drei kleinere Bereiche.\newline
Den ersten Bereich des Downlinks bildet der Sender, welche in diesem Fall der Schmallbandtransponder auf Es'Hail-2. Der Satellit sendet die Informationstragenden Signale $s(t)$ mit der Sendeleistung $P_\text{T}$ über eine Atennne mit dem Gewinn $G_\text{T}(\varphi,\theta)$ in Richtung Erde. Zusammen mit der Gleichung \ref{}



Das EIRP unterscheidet sich je nach Region auf der Erde, da sich der Gewinn $G_\text{T}(\varphi,\theta)$ der Sendeantenne abhängig vom Abstrahlwinkel ist.




